\section{Verbale della riunione}
Si espone al proponente il meccanismo di compravendita di beni e servizi attraverso
una "conferma d'ordine\glo" per mezzo della quale il cliente conferma o meno un acquisto precedentemente effettuato.
Tutto ciò viene implementato attraverso smart contract\glo{}. Inoltre il proponente
conferma il meccanismo di \textit{Escrow\glo{}} adottato dal gruppo. \\
Viene richiesta una delucidazione sulla dicitura "put on hold" ed il proponente conferma
che si intende il pagamento dilazionato del saldo IVA, ad esempio nel caso in cui non 
ci siano abbastanza fondi nel wallet\glo{} dell'azienda per effettuare il saldo. 
Il proponente specifica che non è necessario permettere la dilazione dei pagamenti durante un acquisto. \\
Il proponente afferma che, grazie al meccanismo di \textit{Escrow\glo{}}, non sia più necessaria la possibilità, da parte di un'azienda, di rifiutare un saldo IVA, in quanto il controllo sugli eventuali errori derivanti dalle fatture effettuate da altre aziende sono già stati effettuati in un momento precedente.

\pagebreak

\section{Riepilogo tracciamenti}

	%\renewcommand{\arraystretch}{1.5}
	\rowcolors{2}{pari}{dispari}
	
\begin{longtable}{ >{\centering}p{0.20\textwidth} >{}p{0.70\textwidth}}
	\caption{Decisioni della riunione interna del 2018-01-04}\\	
	\rowcolorhead
	\textbf{\color{white}Codice} 
	& \centering\textbf{\color{white}Decisione} 
	\tabularnewline 
	\endfirsthead
		VE\_3.1 & Eliminazione del pagamento dilazionato delle fatture dai casi d'uso.
		
		\tabularnewline 
		VE\_3.2 & Concordata la definizione di \textit{"put on hold"}.
		
		\tabularnewline 
		VE\_3.3 & Chiarito che la dilazione deve essere permessa alle aziende nel versamento al governo dell'IVA riguardante un trimestre.
		\tabularnewline 
		VE\_3.4 & Concordato che il meccanismo di Escrow\glosp è una valida interpretazione ad alcuni requisiti, presenti nel capitolato\glo, di difficile comprensione da parte degli analisti.

		
	\end{longtable}
	


