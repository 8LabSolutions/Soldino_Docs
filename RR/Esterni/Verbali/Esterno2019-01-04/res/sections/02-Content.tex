\section{Verbale della riunione}

Si espone al proponente il meccanismo di compravendita di beni e servizi attraverso una “conferma d’ordine” attraverso il quale il cliente conferma o meno l’acquisto. Tutto ciò implementato attraverso smart contract. Il proponente conferma il meccanismo ESCROW adottato dal gruppo. \newline \newline
Viene richiesta una delucidazione sulla dicitura “put on hold”, il proponente conferma che si intende il pagamento dilazionato del saldo IVA, ad esempio nel caso in cui non ci siano abbastanza fondi nel wallet dell’azienda per effettuare il saldo. Il proponente specifica che non è necessario dilazionare il pagamento delle fatture tra privati. \newline \newline
Il proponente afferma che grazie al meccanismo ESCROW non è più necessaria la possibilità di rifiutare la proposta di pagamento IVA in quanto grazie alle conferme d’ordine si ha la certezza che l’IVA calcolata sarà corretta.

\pagebreak

\section{Riepilogo tracciamenti}
\begin{centering}
\begin{longtable}{ >{\centering}p{4cm} >{\centering}p{11cm} }

\hline
\\[0.5pt]
	\textbf{Codice} & \textbf{Decisione} 
	
	\tabularnewline 
	\hline
						
				\\[0.5pt]
				VE\_3.1 & Eliminazione del pagamento dilazionato delle fatture dai casi d’uso
				\\[0.5pt]
				\tabularnewline
				\hline
				
				\\[0.5pt]
				VE\_3.2 & Possibilità di dilazione del  versamento  IVA momento del pagamento del saldo IVA
				\\[0.5pt]
				\tabularnewline
				\hline
                
        %\end{tabularx}
        \\[0.7pt]
        \caption{Decisioni della riunione esterna del 2019-01-04}
\end{longtable}
\end{centering}

