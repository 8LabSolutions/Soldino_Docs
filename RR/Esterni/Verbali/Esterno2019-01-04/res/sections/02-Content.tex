\section{Verbale della riunione}
Si espone al proponente il meccanismo di compravendita di beni e servizi attraverso
una "conferma d'ordine" attraverso il quale il cliente conferma o meno l'acquisto.
Tutto ciò viene implementato attraverso smart contract\glo{}. Inoltre il proponente
conferma il meccanismo \textit{ESCROW\glo{}} adottato dal gruppo. \\
Viene richiesta una delucidazione sulla dicitura "put on hold" e il proponente conferma
che si intende il pagamento dilazionato del saldo IVA, ad esempio nel caso in cui non 
ci siano abbastanza fondi nel wallet\glo{} dell'azienda per effettuare il saldo. 
Il proponente specifica che non è necessario dilazionare il pagamento delle fatture 
tra privati. \\
Il proponente afferma che, grazie al meccanismo \textit{ESCROW\glo{}}, non è più necessaria
la possibilità di rifiutare la proposta di pagamento dell'IVA, in quanto grazie alle 
conferme dell'ordine si ha la certezza che l'IVA calcolata sarà corretta.

\pagebreak

\section{Riepilogo tracciamenti}
\begin{table}[H]
	%\renewcommand{\arraystretch}{1.5}
	\rowcolors{2}{pari}{dispari}
	
	\begin{longtable}{ >{\centering}p{0.20\textwidth} >{\centering}p{0.70\textwidth}}
			
		\rowcolorhead
		\centering \textbf{\color{white}Codice} 
		& \centering \textbf{\color{white}Decisione} 
		
		\tabularnewline 
		VE\_3.1 & Eliminazione del pagamento dilazionato delle fatture dai casi d'uso.
		
		\tabularnewline 
		VE\_3.2 & Concordata definizione di \textit{"put on hold"}.
		
		\tabularnewline 
		VE\_3.3 & Possibilità di dilazione del versamento dell'IVA al momento del 
						pagamento del saldo dell'IVA.

		
	\end{longtable}
	\caption{Decisione della riunione esterna del 2019-01-04}	

\end{table}

