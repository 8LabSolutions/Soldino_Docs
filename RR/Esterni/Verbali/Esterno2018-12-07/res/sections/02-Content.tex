\section{Verbale della riunione}
L'incontro è stato organizzato con lo scopo principale di presentare il team di
lavoro ai rappresentanti dell'azienda proponente. Nella prima fase di tale 
incontro è avvenuta una rapida presentazione delle parti e una breve discussione
su tempistiche e modalità da adottare nello sviluppo del progetto. Successivamente 
il team ha sottoposto ai proponenti alcuni interrogativi sul progetto.
Sono stati discussi i seguenti punti: 
\begin{itemize}
	\item \textbf{Visione Azienda}: ciascuna azienda operante nella piattaforma
			 deve essere considerata come persona giuridica, ovvero nel sistema 
			 l'azienda non deve essere collegata ad un secondo account utente con
			 il ruolo di rappresentante dell'azienda;
	\item \textbf{Transazioni}: il proponente vuole attenersi al modello attuale 
			per le transazioni di pagamento dell'IVA dalle aziende verso lo stato,
			ovvero ricadrà su ciascuna azienda l'onere di versare autonomamente 
			l'importo dovuto entro il termine stabilito. Non sarà quindi la piattaforma
			ad effettuare (almeno per ora) la transazione in automatico;
	\item \textbf{Tasso di Cambio}: non rientra nei requisiti del progetto la conversione
			da Euro a Ether\glo{};
	\item \textbf{Browser Supportati}: la piattaforma da realizzare deve essere compatibile
			con i browser Google Chrome e Mozilla Firefox, a partire dalla ultima versione 
			attualmente disponibile. Al momento della redazione del verbale le versioni di 
			riferimento, che dovranno essete supportate dal prodotto finale, sono le seguenti:
	\begin{itemize}
		\item \textbf{Google Chrome}: versione 71;
		\item \textbf{Mozilla Firefox}: versione 64;
	\end{itemize}
	\item \textbf{Lingua Documenti}: i documenti con funzionalità di manuale per il software, che siano essi volti all'utente o all'amministratore, devono essere redatti in lingua inglese.

\end{itemize}
Al termine della chiamata il gruppo ha riassunto gli argomenti discussi e organizzato l'incontro successivo.

\pagebreak

\section{Riepilogo tracciamenti}

	%\renewcommand{\arraystretch}{1.5}
	\rowcolors{2}{pari}{dispari}
	
\begin{longtable}{ >{\centering}p{0.20\textwidth} >{}p{0.70\textwidth}}
	\caption{Decisioni della riunione interna del 2018-12-07}\\	
	\rowcolorhead
	\textbf{\color{white}Codice} 
	& \centering\textbf{\color{white}Decisione} 
	\tabularnewline 
	\endfirsthead
		VE\_1.1 & Le aziende sono considerate persone giuridiche.
		
		\tabularnewline 
		VE\_1.2 & Le transazioni da azienda a stato non avvengono in automatico.
		
		\tabularnewline 
		VE\_1.3 & Il progetto non prevede il cambio tra moneta fisica a criptovaluta.
	
		\tabularnewline 
		VE\_1.4 & La piattaforma deve essere supportata da Google 
		Chrome versione 71.
		
		\tabularnewline 
		VE\_1.5 & La piattaforma deve essere supportata da Mozilla 
		Firefox versione 64.
		
		\tabularnewline 
		VE\_1.6 & I manuali relativi al software prodotto devono essere redatti 
		in lingua inglese.
	
	\end{longtable}

