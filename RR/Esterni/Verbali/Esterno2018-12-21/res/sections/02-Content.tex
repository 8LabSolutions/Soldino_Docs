\section{Verbale della riunione}

Da subito è stata lasciata la precedenza alle domande del secondo gruppo, il cui intervento è servito a prendere appunti in merito a questioni interessanti che potevano rispondere ad alcuni dubbi.\newline
Successivamente, arrivato il nostro turno, sono state esposte una per una le domande preparate durante l'incontro precedente riassunto nel verbale \textit{Interno2018-12-20}.\newline
Le domande hanno riguardato:
\begin{itemize}
	\item \textbf{Aspetti generali del capitolato}: ovvero se sviluppare una \textit{DApps} per ogni funzionalità e chiarimento su alcuni requisiti specificati nel capitolato;
	\item \textbf{Database}: precisazione sul concetto di database di un sito statico e chiarimento su come interfacciarsi da pagina hostata sul server;
	\item \textbf{Termini}: chiarimento su alcuni termini del capitolato dal significato poco chiaro;
	\item \textbf{Standard}: richiesta nello specifico relativa agli standard a cui sottostare e come rispettarli;
	\item \textbf{Plug-in}: capire meglio alcuni aspetti del plug-in Metamask;
	\item \textbf{Accesso} precisazione relativa alla login e alla registrazione di utente/azienda;
	\item \textbf{Token}: capire come si intende creare e distribuire token;
	\item \textbf{Avvisi}: capire dove vogliono che l'utente/azienda/governo venga notificato e dove non è strettamente necessario.
\end{itemize}
Alla fine della chiamata con RedBabel, il gruppo si è subito trovato per definire un resoconto della call che si era appena conclusa. Sono state elencate e rivisitate le risposte alle domande fatte e si è definita una nuova suddivisione dei compiti da svolgere in vista del prossimo incontro.
\pagebreak

\section{Riepilogo tracciamenti}
\begin{centering}
\begin{longtable}{ >{\centering}p{4cm} >{\centering}p{11cm} }

\hline
\\[0.5pt]
	\textbf{Codice} & \textbf{Decisione} 
	
	\tabularnewline 
	\hline
						
				\\[0.5pt]
				VE\_2.1 & Specificato l'utilizzo non necessario di un server per il database, con la possibilità di salvare dati sulla blockchain
				\\[0.5pt]
				\tabularnewline
				\hline
				
				\\[0.5pt]
				VE\_2.2 & \textit{"Put on hold"}: concordato di informarmarsi sul significato del termine
				\\[0.5pt]
				\tabularnewline
				\hline
				
				\\[0.5pt]				
				VE\_2.3 & \textit{Standard}: deciso di seguire standard EIP-712 e ESLint e informarsi individualmente su di essi
				\\[0.5pt]
				\tabularnewline
				\hline
				
				\\[0.5pt]
				VE\_2.4 & \textit{Plug-in} Metamask gestisce l'accessibilità. Non richiesta versione per screen reader. A livello di browser concordato che funzioni su Chrome e Firefox(ultime versioni)
				\\[0.5pt]
				\tabularnewline
				\hline
				
				\\[0.5pt]
				VE\_2.5 & \textit{Accesso} Scelto che gli utenti registrati sono salvati in uno smart contracts(array). Scelta opzionale sul fatto di creare due step di registrazione
				\\[0.5pt]
				\tabularnewline
				\hline  
				
				\\[0.5pt]
				VE\_2.6 & \textit{Token} Scelto che l'Admin può gestire i token, creandoli e distribuendoli a piacimento. Lo scopo è avere nel sito una moneta che si possa utilizzare
				\\[0.5pt]
				\tabularnewline
				\hline  
				
				\\[0.5pt]
				VE\_2.7 & \textit{Avvisi} Stabilito che la lista delle aziende sarà pubblica e che il governo potrà verificare le aziende che hanno pagato e quelle che ancora devono farlo.
				\\[0.5pt]
				\tabularnewline
				\hline  
                
        %\end{tabularx}
        \\[0.7pt]
        \caption{Decisioni della riunione esterna del 2018-12-21}
\end{longtable}
\end{centering}

