\section{Verbale della riunione}
Da subito è stata lasciata la precedenza alle domande del secondo gruppo,
il cui intervento è servito a prendere appunti in merito a questioni interessanti
che potevano rispondere ad alcuni dubbi. Successivamente, arrivato il nostro turno,
sono state esposte una per una le domande preparate durante l'incontro interno 
precedente.\newline
Le domande hanno riguardato:

\begin{itemize}
	\item \textbf{Aspetti generali del capitolato}: chiarire se sviluppare una 
			\textit{DApps} per ogni funzionalità e definire alcuni requisiti specificati
			 nel capitolato;
	\item \textbf{Database}: precisazione sul concetto di database di un sito statico
			 e chiarimento su come interfacciarsi da una pagina che viene ospitata sul server;
	\item \textbf{Termini}: chiarimento su alcuni termini del capitolato il cui significato
			 era risultato poco chiaro;
	\item \textbf{Standard}: richiesta nello specifico relativa agli standard a cui sottostare
			 e come rispettarli;
	\item \textbf{MetaMask}: capire meglio alcuni aspetti di questo plug-in;
	\item \textbf{Accesso}: precisazione relativa alla login e alla registrazione di 
			 utente/azienda;
	\item \textbf{Token\glo{}}: capire come si intende creare e distribuire i token;
	\item \textbf{Avvisi}: capire dove vogliono che l'utente/azienda/governo venga 
			 notificato e dove non è strettamente necessario.
\end{itemize}
Alla fine della chiamata con \textit{Red Babel}, alla luce dei chiarimenti dati,
il gruppo ha deciso di fissare un incontro interno per suddividere il lavoro rimanente.
\pagebreak

\section{Riepilogo tracciamenti}
\begin{table}[H]
	%\renewcommand{\arraystretch}{1.5}
	\rowcolors{2}{pari}{dispari}
	
	\begin{longtable}{ >{\centering}p{0.20\textwidth} >{\centering}p{0.70\textwidth}}
			
		\rowcolorhead
		\centering \textbf{\color{white}Codice} 
		& \centering \textbf{\color{white}Decisione} 
		
		\tabularnewline 
		VE\_2.1 & Specificato l'utilizzo non necessario di un server per il database,
				con la possibilità di salvare dati sulla blockchain.
		
		\tabularnewline 
		VE\_2.2 & \textit{"Put on hold"}: concordato di informarsi sul significato del termine.
		
		\tabularnewline 
		VE\_2.3 & Deciso di seguire gli standard EIP-712 e ESLint e informarsi su di essi.
	
		\tabularnewline 
		VE\_2.4 & \textit{MetaMask} gestisce l'accessibilità e non è richiesta una versione
				per screen reader\glo{}.
		
		\tabularnewline 
		VE\_2.5 & A livello di browser concordato che funzioni sulle ultime versioni 
				di \textit{Google Chrome} e \textit{Mozilla Firefox}.
		
		\tabularnewline 
		VE\_2.6 & Deciso che gli utenti registrati sono salvati in uno \textit{Smart 
				Contracts} ed è una scelta opzionale il creare due step di registrazione.
				
		\tabularnewline
		VE\_2.7 & Scelto che l'\textit{Admin} può gestire i token\glo{}, creandoli 
				e distribuendoli a piacimento. Lo scopo è avere nel sito una moneta che si 
				possa utilizzare.
				
		\tabularnewline
		VE\_2.8 & Stabilito che la lista delle aziende sarà pubblica e che il governo potrà 
				verificare le aziende che hanno pagato e quelle che ancora devono farlo.
		
	\end{longtable}
	\caption{Decisione della riunione esterna del 2018-12-21}	

\end{table}
