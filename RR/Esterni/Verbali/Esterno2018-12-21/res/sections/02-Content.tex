\section{Verbale della riunione}
Da subito è stata lasciata la precedenza alle domande del secondo gruppo,
il cui intervento è servito a prendere appunti in merito a questioni interessanti
che potevano rispondere ad alcuni dubbi. Successivamente, arrivato il nostro turno,
sono state esposte una per una le domande preparate durante l'incontro interno 
precedente.\newline
Le domande hanno riguardato:

\begin{itemize}
	\item \textbf{Aspetti generali del capitolato}: chiarire se sviluppare una 
			\textit{DApp\glo{}} per ogni funzionalità e definire alcuni 
			requisiti specificati
			 nel capitolato;
	\item \textbf{Database}: precisazione sul concetto di database di un sito statico
			 e chiarimento su come interfacciarsi da una pagina che viene ospitata sul server;
	\item \textbf{Termini}: chiarimento su alcuni termini del capitolato il cui significato
			 era risultato poco chiaro;
	\item \textbf{Standard}: richiesta nello specifico relativa agli standard a cui sottostare
			 e come rispettarli;
	\item \textbf{MetaMask\glo{}}: capire meglio alcuni aspetti di questo 
	plug-in\glo{};
	\item \textbf{Accesso}: precisazione relativa alla login e alla registrazione di 
			 utente/azienda;
	\item \textbf{Token\glo{}}: capire come si intende creare e distribuire i 
	token\glo{};
	\item \textbf{Avvisi}: capire dove vogliono che l'utente/azienda/governo venga 
			 notificato e dove non è strettamente necessario.
\end{itemize}
Alla fine della chiamata con \textit{Red Babel}, alla luce dei chiarimenti dati,
il gruppo ha deciso di fissare un incontro interno per suddividere il lavoro rimanente.
\pagebreak

\section{Riepilogo tracciamenti}

	%\renewcommand{\arraystretch}{1.5}
	\rowcolors{2}{pari}{dispari}
	
	\begin{longtable}{ >{\centering}p{0.20\textwidth} >{}p{0.70\textwidth}}
		\caption{Decisioni della riunione interna del 2018-12-21}\\	
		\rowcolorhead
		\textbf{\color{white}Codice} 
		& \centering\textbf{\color{white}Decisione} 
		\tabularnewline 
		\endfirsthead
		VE\_2.1 & La piattaforma non utilizzerà un server per il database,
				eventuali dati saranno salvati sulla blockchain\glo{}.
		
		\tabularnewline 
		VE\_2.2 & Nello sviluppo verranno utilizzati gli standard EIP-712 e 
		ESLint.
	
		\tabularnewline 
		VE\_2.3 & Il progetto non prevede lo sviluppo di nessuna versione per 
		screen reader\glo{} in quanto l'accessibilità è gestita da 
		MetaMask\glo{}.

		
		\tabularnewline 
		VE\_2.4 & Gli utenti registrati sono salvati in uno Smart 
				Contracts\glo{} ed è una scelta opzionale il creare due step 
				di registrazione.
				
		\tabularnewline
		VE\_2.5 & Admin può gestire i token\glo{}, creandoli 
				e distribuendoli a piacimento. Lo scopo è avere nel sito una 
				moneta che si possa utilizzare.
				
		\tabularnewline
		VE\_2.6 & La lista delle aziende sarà pubblica, il governo potrà 
				verificare le aziende che hanno pagato e quelle che ancora devono farlo.
		
	\end{longtable}

