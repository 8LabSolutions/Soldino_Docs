\section{Informazioni generali}

\subsection{Informazioni incontro}
\begin{itemize}
\item \textbf{Luogo}: Videochiamata Hangouts;
\item \textbf{Data}: 2018-12-21;
\item \textbf{Ora di inizio}: 10.30;
\item \textbf{Ora di fine}: 12.30;
\item \textbf{Partecipanti}:
\begin{itemize}
	\item Federico Bicciato;
	\item Mattia Bolzonella;
	\item Francesco Donè;
	\item Giacomo Greggio;
	\item Samuele Giuliano Piazzetta;
	\item Paolo Pozzan;
	\item Alessandro Maccagnan(proponente);
	\item Milo Ertola(proponente);
	\item membri gruppo \textit{The Walking Bug}.
\end{itemize}
\end{itemize}

\subsection{Argomenti affrontati}
Nel secondo incontro con \textit{Red Babel}, abbiamo effettuato una call per 
chiarire alcuni aspetti del capitolato, principalmente per i requisiti presenti
nel documento \textit{Analisi dei Requisiti v1.0.0}.
Oltre a \textit{Red Babel} come partecipante esterno era presente anche un altro
gruppo di studenti che sviluppa il progetto \textit{Soldino}.
In particolare ci siamo soffermati su quali requisiti richiedessero come obbligatori,
su quali definire desiderabili e quali invece opzionali.\newline
I proponenti si sono dimostrati parecchio disponibile a chiarire eventuali dubbi,
anche futuri, e a seguire passo passo lo sviluppo del progetto.
