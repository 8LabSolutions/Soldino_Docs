% STA FACENDO SARA, NON TOCCARE
\section{Preventivo}
Per facilitare la lettura delle seguenti tabelle, vengono utilizzate delle sigle 
per identificare i ruoli:
\begin{itemize}
\item \textbf{Ad:} Amministratore;
\item \textbf{An:} Analista;
\item \textbf{Pj:} Progettista;
\item \textbf{Pr:} Programmatore;
\item \textbf{Re:} Responsabile;
\item \textbf{Ve:} Verificatore.
\end{itemize}

Inoltre, se le ore ricoperte in un determinato ruolo fossero nulle, la cella 
presenterà il simbolo \textbf{-} per indicarne l'assenza. 

\subsection{Fase di Analisi}
\subsubsection{Prospetto orario}
In questa fase\glo{}, ogni componente del gruppo rivestirà i seguenti ruoli:
\begin{table}[H]
				\centering\renewcommand{\arraystretch}{1.5}
				\arrayrulecolor{white}
                \begin{tabular}{c|c|c|c|c|c|c|c}
                               
                \rowcolorhead
                 {\colorhead \textbf{Nominativo}} &
                 {\colorhead \textbf{Re}} & 
                 {\colorhead \textbf{Am}} & 
                 {\colorhead\textbf{An}} & 
                 {\colorhead \textbf{Pt}} & 
                 {\colorhead\textbf{Pr}} & 
                 {\colorhead \textbf{Ve}} & 
                 {\colorhead \textbf{Ore totali} }\\
				
                \rowcolorlight
                 {\colorbody Federico Bicciato} & {\colorbody } & 
                 {\colorbody } & {\colorbody } & {\colorbody } & 
                 {\colorbody } & {\colorbody } & { \colorbody } 
				\\
				
				\rowcolordark
                 {\colorbody Mattia Bolzonella} & {\colorbody } & 
                 {\colorbody } & {\colorbody } & {\colorbody } & 
                 {\colorbody } & {\colorbody } & { \colorbody } 
				\\	
				
				\rowcolorlight
                 {\colorbody Francesco Donè} & {\colorbody } & 
                 {\colorbody } & {\colorbody } & {\colorbody } & 
                 {\colorbody } & {\colorbody } & { \colorbody } 
				\\
				
				\rowcolordark
                 {\colorbody Sara Feltrin} & {\colorbody } & 
                 {\colorbody } & {\colorbody } & {\colorbody } & 
                 {\colorbody } & {\colorbody } & { \colorbody } 
				\\
                
                \rowcolorlight
                 {\colorbody Giacomo Greggio} & {\colorbody } & 
                 {\colorbody } & {\colorbody } & {\colorbody } & 
                 {\colorbody } & {\colorbody } & { \colorbody } 
				\\
				
				\rowcolordark
                 {\colorbody Samuele Piazzetta} & {\colorbody } & 
                 {\colorbody } & {\colorbody } & {\colorbody } & 
                 {\colorbody } & {\colorbody } & { \colorbody } 
				\\	
				
				\rowcolorlight
                 {\colorbody Paolo Pozzan} & {\colorbody } & 
                 {\colorbody } & {\colorbody } & {\colorbody } & 
                 {\colorbody } & {\colorbody } & { \colorbody } 
				\\
				
				\rowcolordark
                 {\colorbody Matteo Santinon} & {\colorbody } & 
                 {\colorbody } & {\colorbody } & {\colorbody } & 
                 {\colorbody } & {\colorbody } & { \colorbody } 
				\\
				
				\rowcolorlight
                 {\colorbody \textbf{Ore totali ruolo}} & {\colorbody } & 
                 {\colorbody } & {\colorbody } & {\colorbody } & 
                 {\colorbody } & {\colorbody } & { \colorbody } 
				\\
                

                \end{tabular}
                \caption{Distribuzione delle ore nel periodo di Analisi}
\end{table}

I dati ottenuti si possono riassumere nel seguente diagramma:
%aggiungere immagine del diagramma

\subsubsection{Prospetto economico}
In questa fase il costo per ogni ruolo è il seguente:
\begin{table}[H]
				\centering\renewcommand{\arraystretch}{1.5}
				\arrayrulecolor{white}
                \begin{tabular}{c|c|c}
                               
                \rowcolorhead
                 {\colorhead \textbf{Ruolo}} &
                 {\colorhead \textbf{Ore}} & 
                 {\colorhead \textbf{Costo}} \\
				
                \rowcolorlight
                 {\colorbody Responsabile} & {\colorbody } & 
                 {\colorbody }  
				\\
				
				\rowcolordark
                 {\colorbody Amministratore} & {\colorbody } & 
                 {\colorbody }
				\\	
				
				\rowcolorlight
                 {\colorbody Analista} & {\colorbody } & 
                 {\colorbody } 
				\\
				
				\rowcolordark
                 {\colorbody Progettista} & {\colorbody } & 
                 {\colorbody } 
				\\
				
				\rowcolorlight
                 {\colorbody Programmatore} & {\colorbody } & 
                 {\colorbody } 
				\\
				
				\rowcolordark
                 {\colorbody Verificatore} & {\colorbody } & 
                 {\colorbody } 
				\\
				
				\rowcolorlight
                 {\colorbody \textbf{Totale}} & {\colorbody } & 
                 {\colorbody } 
				\\
				
                

                \end{tabular}
                \caption{Prospetto dei costi per ruoli}
\end{table}

I dati ottenuti si possono riassumere nel seguente diagramma:
%aggiungere immagine del diagramma

\subsection{Fase di consolidamento dei requisiti}
Il periodo di consolidamento dei requisiti vede la seguente distribuzione oraria:
\begin{table}[H]
				\centering\renewcommand{\arraystretch}{1.5}
				\arrayrulecolor{white}
                \begin{tabular}{c|c|c|c|c|c|c|c}
                               
                \rowcolorhead
                 {\colorhead \textbf{Nominativo}} &
                 {\colorhead \textbf{Re}} & 
                 {\colorhead \textbf{Am}} & 
                 {\colorhead\textbf{An}} & 
                 {\colorhead \textbf{Pt}} & 
                 {\colorhead\textbf{Pr}} & 
                 {\colorhead \textbf{Ve}} & 
                 {\colorhead \textbf{Ore totali} }\\
				
                \rowcolorlight
                 {\colorbody Federico Bicciato} & {\colorbody } & 
                 {\colorbody } & {\colorbody } & {\colorbody } & 
                 {\colorbody } & {\colorbody } & { \colorbody } 
				\\
				
				\rowcolordark
                 {\colorbody Mattia Bolzonella} & {\colorbody } & 
                 {\colorbody } & {\colorbody } & {\colorbody } & 
                 {\colorbody } & {\colorbody } & { \colorbody } 
				\\	
				
				\rowcolorlight
                 {\colorbody Francesco Donè} & {\colorbody } & 
                 {\colorbody } & {\colorbody } & {\colorbody } & 
                 {\colorbody } & {\colorbody } & { \colorbody } 
				\\
				
				\rowcolordark
                 {\colorbody Sara Feltrin} & {\colorbody } & 
                 {\colorbody } & {\colorbody } & {\colorbody } & 
                 {\colorbody } & {\colorbody } & { \colorbody } 
				\\
                
                \rowcolorlight
                 {\colorbody Giacomo Greggio} & {\colorbody } & 
                 {\colorbody } & {\colorbody } & {\colorbody } & 
                 {\colorbody } & {\colorbody } & { \colorbody } 
				\\
				
				\rowcolordark
                 {\colorbody Samuele Piazzetta} & {\colorbody } & 
                 {\colorbody } & {\colorbody } & {\colorbody } & 
                 {\colorbody } & {\colorbody } & { \colorbody } 
				\\	
				
				\rowcolorlight
                 {\colorbody Paolo Pozzan} & {\colorbody } & 
                 {\colorbody } & {\colorbody } & {\colorbody } & 
                 {\colorbody } & {\colorbody } & { \colorbody } 
				\\
				
				\rowcolordark
                 {\colorbody Matteo Santinon} & {\colorbody } & 
                 {\colorbody } & {\colorbody } & {\colorbody } & 
                 {\colorbody } & {\colorbody } & { \colorbody } 
				\\
				
				\rowcolorlight
                 {\colorbody \textbf{Ore totali ruolo}} & {\colorbody } & 
                 {\colorbody } & {\colorbody } & {\colorbody } & 
                 {\colorbody } & {\colorbody } & { \colorbody } 
				\\

                \end{tabular}
                \caption{Distribuzione delle ore nel periodo di Consolidamento dei requisiti}
\end{table}

I dati ottenuti si possono riassumere nel seguente diagramma:
%aggiungere immagine del diagramma

\subsubsection{Prospetto economico}
In questa fase il costo per ogni ruolo è il seguente:
\begin{table}[H]
				\centering\renewcommand{\arraystretch}{1.5}
				\arrayrulecolor{white}
                \begin{tabular}{c|c|c}
                               
                \rowcolorhead
                 {\colorhead \textbf{Ruolo}} &
                 {\colorhead \textbf{Ore}} & 
                 {\colorhead \textbf{Costo}} \\
				
                \rowcolorlight
                 {\colorbody Responsabile} & {\colorbody } & 
                 {\colorbody }  
				\\
				
				\rowcolordark
                 {\colorbody Amministratore} & {\colorbody } & 
                 {\colorbody }
				\\	
				
				\rowcolorlight
                 {\colorbody Analista} & {\colorbody } & 
                 {\colorbody } 
				\\
				
				\rowcolordark
                 {\colorbody Progettista} & {\colorbody } & 
                 {\colorbody } 
				\\
				
				\rowcolorlight
                 {\colorbody Programmatore} & {\colorbody } & 
                 {\colorbody } 
				\\
				
				\rowcolordark
                 {\colorbody Verificatore} & {\colorbody } & 
                 {\colorbody } 
				\\
				
				\rowcolorlight
                 {\colorbody \textbf{Totale}} & {\colorbody } & 
                 {\colorbody } 
				\\
                

                \end{tabular}
                \caption{Prospetto dei costi per ruoli}
\end{table}

I dati ottenuti si possono riassumere nel seguente diagramma:
%aggiungere immagine del diagramma
