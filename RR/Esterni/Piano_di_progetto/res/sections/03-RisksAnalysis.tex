\section{Analisi dei rischi}
Nel corso dello sviluppo di un progetto complesso è possibile incorrere in problemi che potevano essere evitati tramite un'attività di analisi. Allo scopo quindi, di prevenire rallentamenti evitabili è stata effettutata un'approfondita attività di analisi dei principali fattori di rischio; per ciascuno delle voci nella tabella sottostante è stata utilizzata la seguente procedura di identificazione e risoluzione:
\begin{itemize}
	\item \textbf{Individuazione}: identificare i vari fattori di rischio che possono rallentare o impedire il normale proseguimento del progetto;
	\item \textbf{Analisi}: studio dei fattori di rischio e conseguente assegnazione di una probabilità che essi si verifichino e di un indice di gravità, ovvero l'impatto che avrebbero sul progetto;
	\item \textbf{Ricerca Contromisure e Pianificazione dei Controlli}: pianificazione di apposite misure di controllo e tecniche per la risoluzione di tali problemi.
\end{itemize}
Sono stati inoltre definiti i seguenti codici per raggruppare le varie tipologie di fattori di rischio:
\begin{itemize}
	\item \textbf{RT}: Rischi Tecnologici;
	\item \textbf{RO}: Rischi Organizzativi;
	\item \textbf{RI}: Rischi Interpersonali.
\end{itemize}
\begin{center}
	\begin{longtable}{ >{\centering}p{1cm} >{\centering}p{2cm}
			>{}p{5cm} >{\centering}p{1cm} >{\centering}p{3cm}}
\hline
\textbf{Cod.} & \textbf{Nome} & \textbf{Descrizione} & \textbf{Probabilità} &
\textbf{Gravità}\tabularnewline \hline
\hline
RT1 & Inesperienza Tecnologica & Molte delle tecnologie da adottare nello sviluppo del progetto richiesto sono nuove per molti componenti del team, di conseguenza potranno insorgere problemi operativi. & Alta & Alta
\tabularnewline \hline
\multicolumn{5}{p{13cm}}{\textbf{Contromisure}: Ciascun componente del team si impegnerà nello studio autonomo al fine di apprendere al meglio tali tecnologie.}
\tabularnewline \hline
RO1 & Calcolo Tempi & La massiccia presenza di tecnologie nuove per molti dei componenti del team, può comportare imprecisioni/variazioni nelle tempestiche inziali. & Alta & Alta
\tabularnewline \hline
\multicolumn{5}{p{13cm}}{\textbf{Contromisure}: Il gruppo ha predisposto apposite tabelle condivise, per monitorare i tempi di sviluppo ed evidenziare eventuali ritardi, il responsabile valuterà una eventuale riallocazione di risorse.}
\tabularnewline \hline
RO2 & Calcolo Costi & Considerata l'inesperienza del team nelle valutazioni
economiche, essere potranno risultare imprecise. & Media & Alta
\tabularnewline \hline
\multicolumn{5}{p{13cm}}{\textbf{Contromisure}: Utilizzando le stesse tabelle del caso precedente, a seguito di rilevanti cambiamenti nei costi e nelle tempistiche, tali variazioni verranno segnalate al proponente.}
\tabularnewline \hline
RO3 & Impegni Accademici & Possono verificarsi momenti in cui uno o più compenenti del team siano meno dispnibili a causa di alcuni impegni di tipo accademico. & Alta & Bassa
\tabularnewline \hline
\multicolumn{5}{p{13cm}}{\textbf{Contromisure}: Al fine di prevenire rallentamenti alle operazioni è stato predisposto un calendario condiviso nel quale ciascun componente deve segnalare i propri impegni; in questo modo il responsabile potrà soddividere il lavoro in maniera ottimale.}
\tabularnewline \hline
RO4 & Impegni Personali & Possono verificarsi momenti in cui uno o più compenenti del team siano meno dispnibili a causa di alcuni impegni di tipo personale. & Media & Bassa
\tabularnewline \hline
\multicolumn{5}{p{13cm}}{\textbf{Contromisure}: Come nel caso precedente è stato predisposto un calendario condiviso al fine di migliorare la suddivsione dei compiti.}
\tabularnewline \hline
RO5 & Ritardi & Una o più delle problematiche sopracitate possono comportare ritadi. & Media & Bassa
\tabularnewline \hline
\multicolumn{5}{p{13cm}}{\textbf{Contromisure}: Il responsabile provvederà ad una eventuale riassegnazione risorse e, se necessario, ad una segnalazione al committente/proponente.}
\tabularnewline \hline
RI1 & Cumunicazione Interna & Nel corso dello potrebbero verificarsi dei momenti in cui uno più membri del gruppo siano inreperibili. & Bassa & Alta
\tabularnewline \hline
\multicolumn{5}{p{13cm}}{\textbf{Contromisure}: Ciascun membro del team ha fornito più opzioni per essere contattato e si impegna a rispondere ad eventuali richieste. Sarà inoltre responsabilità personale segnalare eventuali momenti di inreperibilità.}
\tabularnewline \hline
RI2 & Cumunicazione Esterna & Il proponente ha la propria sede all'estero, di conseguenza le comunicazioni saranno più difficili. & Bassa & Media
\tabularnewline \hline
\multicolumn{5}{p{13cm}}{\textbf{Contromisure}: Come per le comunicazioni interne, sono stati predisposti più canali di comunicazione; le video conferenze con il proponente saranno organizzare con il dovuto preavviso. In occasione di tali conferena ciascun membro del gruppo si imegnerà a raggruppare domande, dubbi e chiarimenti da sottoporre al referente dell'azienda proponente.}
\tabularnewline \hline
RI3 & Contrasti interni & Come in qualsiasi gruppo, nel corso delle varie attività potranno emergere contrasti e tensioni tra i vari componenti. & Bassa & Media
\tabularnewline \hline
\multicolumn{5}{p{13cm}}{\textbf{Contromisure}: Ciascun membro del team si impegnerà a limitare tali tensioni e fare in modo che esse non influiscano il normale svolgersi delle attività; in caso necessario il responsabile provvederà a limitare tali contrasti.}
\tabularnewline \hline

\caption{Tabella dei Rischi di Progetto}		
		
		
\end{longtable}
\end{center}
