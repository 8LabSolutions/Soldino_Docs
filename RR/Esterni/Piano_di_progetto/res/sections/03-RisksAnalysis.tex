\section{Analisi dei rischi}

Nel corso dello sviluppo di un progetto complesso è possibile incorrere in 
problemi che possono essere evitati tramite un'attività di analisi. Quindi 
allo scopo di prevenire queste situazioni è stata effettuata 
un'approfondita attività di analisi dei principali fattori di rischio. Per 
ciascuna delle voci nella tabella sottostante è stata utilizzata la seguente 
procedura di identificazione e risoluzione:

\begin{itemize}
	\item \textbf{Individuazione dei rischi}: attività di identificazione dei vari fattori problematici che possono rallentare o impedire il normale proseguimento del progetto;
	\item \textbf{Analisi dei rischi}: attività di studio dei fattori di rischio con conseguente assegnazione di una probabilità che essi si verifichino e di un indice di gravità, determinando così l'impatto che avrebbero sul progetto;
	\item \textbf{Pianificazione di controllo}: attività di pianificazione di una metodologia per evitare che si verifichino i rischi individuati e si stabilisca come procedere nel caso in cui il problema si riscontrasse;
	\item \textbf{Monitoraggio dei rischi}: attività continua svolta al fine di 
	evitare che abbiano luogo queste complicazioni o, nel peggiore dei casi, 
	permetta di agire tempestivamente per contenerle.

\end{itemize}

Sono stati inoltre definiti i seguenti codici per raggruppare le varie 
tipologie di fattori di rischio:

\begin{itemize}
	\item \textbf{RT}: Rischi Tecnologici;
	\item \textbf{RO}: Rischi Organizzativi;
	\item \textbf{RI}: Rischi Interpersonali.
\end{itemize}


\renewcommand{\arraystretch}{1.5}
\rowcolors{2}{dispari}{pari}
	\arrayrulecolor{white}
	\begin{longtable}{ 
			>{\centering}p{0.17\textwidth} 
			>{\raggedright}p{0.28\textwidth}
			>{\raggedright}p{0.29\textwidth} 
			>{\centering}p{0.15\textwidth}
		}

	
	\caption{Tabella dei Rischi di Progetto}\\
	\rowcolorhead
	\colorhead\textbf{Nome \\ Codice} & \centering\colorhead\textbf{Descrizione} & 
	\centering\colorhead\textbf{Rilevamento} & 
	\colorhead\textbf{Grado di rischio} 
	\tabularnewline
	\endfirsthead
	\rowcolor{white}\caption[]{(continua)}\\
	\rowcolorhead
	\colorhead\textbf{Nome \\ Codice} & \centering\colorhead\textbf{Descrizione} & 
	\centering\colorhead\textbf{Rilevamento} & 
	\colorhead\textbf{Grado di rischio} 
	\tabularnewline
	\endhead
	
	%RT1---------------------------------------------------------
	 Inesperienza Tecnologica \\ RT1 & 
	 Molte delle tecnologie da adottare nello 
	 sviluppo del progetto richiesto sono nuove per molti componenti del team, 
	 di conseguenza potrebbero insorgere problemi operativi. &
	Il responsabile avrà il compito di rilevare conoscenze ed eventuali lacune 
	dei vari componenti del team. Ciascun menbro del gruppo inoltre provverà a 
	comunicare in assoluta trasparenza eventuali difficoltà. &
	 Occorrenza: \textbf{Alta} \\
	 Pericolosità: \textbf{Alta} 
	 \tabularnewline
	 \multicolumn{1}{p{0.17\textwidth}}{\centering\textbf{Piano di contingenza}}& 
	 \multicolumn{3}{p{0.7775\textwidth}}{I compiti più onerosi, o che 
	 richiedono maggiori conoscenze tecnologiche verranno assegnati a più 
	 persone favorendo così l'assistenza reciproca. }
	 \tabularnewline 
	 	
	 %RO1---------------------------------------------------------
	Calcolo Tempistice \\ R01 &
	La massiccia presenza di tecnologie 
	nuove per molti 
	dei componenti del team, può comportare imprecisioni/variazioni nel calcolo 
	delle tempestiche.&
	Nel corso dello sviluppo verrà assegnata una scadenza a ciascun task; sarà 
	responsibilità dell'owner del task comunicare problematiche nel 
	rispettare tali scadenze.&	
	Occorrenza: \textbf{Alta} \\
	Pericolosità: \textbf{Alta}
	\tabularnewline
	\multicolumn{1}{p{0.17\textwidth}}{\centering\textbf{Piano di contingenza}}& 
	\multicolumn{3}{p{0.7775\textwidth}}{All'insorgere di tali problematiche, 
	il responsabile in accordo con l'owner del task, provverà all'assegnazione 
	di maggiori risorse o allo spostamento della scadenza.}
	\tabularnewline	
	
	%R02------------------------------------------------------------
	Calcolo costi \\ RO2 &
	Considerata l'inesperienza del team 
	nelle valutazioni
	economiche, esse potranno risultare imprecise. &
	Il team ha predisposto specifiche tabelle condivise al fine di permettere 
	al responsabile di monitorare le ore di lavoro di ciascun componente.&
	Occorrenza: \textbf{Media} \\
	Pericolosità: \textbf{Alta}
	\tabularnewline
	\multicolumn{1}{p{0.17\textwidth}}{\centering\textbf{Piano di contingenza}}& 
	\multicolumn{3}{p{0.7775\textwidth}}{All'insorgere di rilevanti variazioni 
	orarie rispetto al preventivo iniziale, verranno comunicati tempestivamente 
	al committente tali mutamenti.}
	\tabularnewline	
	
	%R03------------------------------------------------------------
	Impegni Accademici \\ RO3 & 
	Possono verificarsi momenti in cui uno o più componenti del team siano meno disponibili a causa di alcuni impegni di tipo accademico. &
	Al fine di prevenire rallentamenti, è stato predisposto un calendario 
	condiviso nel quale segnalare eventuali impegni accademici.&
	Occorrenza: \textbf{Alta} \\
	Pericolosità: \textbf{Bassa}
	\tabularnewline
	\multicolumn{1}{p{0.17\textwidth}}{\centering\textbf{Piano di contingenza}}& 
	\multicolumn{3}{p{0.7775\textwidth}}{ L'assegnazione di incarchi e scadenze 
	avverrà nel rispetto degli impegni segnalati nel calendario.}
	\tabularnewline	
	
	%R04------------------------------------------------------------
    Impegni Personali \\ RO4 &
	Possono verificarsi momenti in cui uno o più componenti del team siano meno disponibili a causa di alcuni impegni di tipo personale.&
	Ciascun componente del team utilizzerà 
	il calendario già descritto nel caso precedente per segnalare i propri 
	impegni personali. Eventuali impegni imprevisti verranno tempestivamente 
	segnalati al responsabile.&
	Occorrenza: \textbf{Media} \\
	Pericolosità: \textbf{Bassa}
	\tabularnewline
	\multicolumn{1}{p{0.17\textwidth}}{\centering\textbf{Piano di contingenza}}& 
	\multicolumn{3}{p{0.7775\textwidth}}{L'assegnazione di incarchi e scadenze 
		avverrà nel rispetto degli impegni segnalati nel calendario. 
		All'insorgere di imprevisti, il reponsabile valurerà una riallocazione 
		di risorse oppure una riassegnazione del task.}
	\tabularnewline	
	
	%R05------------------------------------------------------------
	 Ritardi \\ RO5 &
	Una o più delle problematiche sopracitate 
	possono 
	comportare ritardi (RO1, RO3, RO4).&
	L'owner di ciascun task segnalerà in modo tempestivo l'impossibiltà di 
	rispettare le proprie scadenze.&
	Occorrenza: \textbf{Media} \\
	Pericolosità: \textbf{Bassa}
	\tabularnewline
	\multicolumn{1}{p{0.17\textwidth}}{\centering\textbf{Piano di contingenza}}& 
	\multicolumn{3}{p{0.7775\textwidth}}{ Il responsabile, se necessario, 
	riassegnerà le risorse al fine evitare rallentamenti.}
	\tabularnewline	
	
	%RI1------------------------------------------------------------
	Comunicazione Interna \\ RI1 & 
	Nel corso dello sviluppo del progetto potrebbero verificarsi dei 
	momenti in cui uno più membri del gruppo siano irreperibili. &
	Tutti i componenti del team sono tenuti a segnalare eventuali momenti di 
	inreperibilità e organizzare i propri impegni al fine di poter presenziare 
	alle riunioni del gruppo. &
	Occorrenza: \textbf{Bassa} \\
	Pericolosità: \textbf{Alta}
	\tabularnewline
	\multicolumn{1}{p{0.17\textwidth}}{\centering\textbf{Piano di contingenza}}& 
	\multicolumn{3}{p{0.7775\textwidth}}{ Il gruppo ha predisposto molteplici 
	vie di comunicazione interna. Inoltre verranno organizzati incontri a 
	scadenze fissa per discutere dell'avanzamento del progetto.}
	\tabularnewline	
	
	%RI2------------------------------------------------------------
	 Comunicazione Esterna \\ RI2 &
	Il proponente ha la propria 
	sede all'estero, di conseguenza le comunicazioni saranno più difficili. &
	Come per le comunicazioni interne, sono stati predisposti più canali di 
	comunicazione; le video conferenze con il proponente saranno organizzare 
	con il dovuto preavviso.&
	Occorrenza: \textbf{Bassa} \\
	Pericolosità: \textbf{Media}
	\tabularnewline
	\multicolumn{1}{p{0.17\textwidth}}{\centering\textbf{Piano di contingenza}}& 
	\multicolumn{3}{p{0.7775\textwidth}}{Il gruppo provvederà a raggruppare 
	quesiti e segnalazioni per il proponente.}
	\tabularnewline	
	
	%RI3------------------------------------------------------------
	 Contrasti interni \\ RI3 &
	Come in qualsiasi gruppo, nel 
	corso delle varie attività potranno emergere contrasti e tensioni tra i vari componenti. &
	Ciascun membro del team si impegnerà a limitare tali tensioni e fare in 
	modo che esse non influiscano sul normale svolgersi delle attività. &
	Occorrenza: \textbf{Bassa} \\
	Pericolosità: \textbf{Media}
	\tabularnewline
	\multicolumn{1}{p{0.17\textwidth}}{\centering\textbf{Piano di contingenza}}& 
	\multicolumn{3}{p{0.7775\textwidth}}{Il responsabile avrà la funzione di 
	mediatore in tali controversie.}
	\tabularnewline	
		
	\end{longtable}
		
\renewcommand{\arraystretch}{1}
