\section{Pianificazione}
Alla luce delle scadenze presentate nella \hyperlink{scadenze}{sottosezione 2.5} la pianificazione di progetto viene suddivisa nelle seguenti fasi:
\begin{enumerate}
	\item \textbf{Analisi dei requisiti};
	\item \textbf{Consolidamento dei requisiti};
	\item \textbf{text}
	\item \textbf{text}
\end{enumerate}
Ogni fase viene suddivisa in attività\glosp che verranno realizzate durante il 
periodo stabilito per la fase stessa. 
\subsection{Analisi dei Requisiti}
\textit{Periodo: dal 2018-11-16 al 2019-01-14}\\
L'inizio del periodo di questa fase coincide con la data di formazione dei 
gruppo e la fine coincide con la data di consegna dei documenti relativi alla 
revisione dei requisiti. Questa fase è stata scomposta nelle seguenti attività:
\begin{itemize}
	\item \textbf{Individuazione degli strumenti}: questa attività consiste nel 
	determinare quali strumenti il gruppo deve utilizzare per la comunicazione 
	e per la stesura dei documenti; 
	\item \textbf{Norme di Progetto}: viene redatto il documento \textit{Norme 
	di Progetto} dal responsabile di progetto e dall'amministratore. In questo 
	documento sono definite tutte le regole relative alla stesura dei documenti 
	e allo sviluppo del prodotto finale. Questo documento è identificato come 
	milestone\glo; %Inoltre sono 
	%sancite le norme che regolano%
	\item \textbf{Studio di fattibilità}: viene redatto il documento 
	\textit{Studio di fattibilità} nel quale vengono analizzano in breve tutti 
	i capitolati proposti e viene indicato il capitolato scelto dal gruppo. 
	Questa attività e da considerarsi bloccante per l'attività di Analisi dei 
	Requisiti;
	\item \textbf{Analisi dei Requisiti}: durante questa attività vengono 
	identificati ed analizzati i requisiti del capitolato scelto nell'attività 
	precendente e viene redatto di conseguenza il documento \textit{Analisi dei 
	Requisiti};
	\item \textbf{Piano di Progetto}: questa attività coincide con la stesura 
	del documento \textit{Piano di Progetto} nel quale viene pianificato il 
	lavoro del gruppo 8Lab Solutions, inteso come suddivisione di compiti, 
	risorse e attività. Inoltre si prospetta il preventivo per la realizzazione 
	del 
	progetto;
	\item \textbf{Piano di Qualifica}: 
	\item \textbf{Glossario}: 
	
	\item \textbf{lettera di presentazione}:
\end{itemize}

\subsection{Consolidamento dei Requisiti}