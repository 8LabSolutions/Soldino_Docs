\section{Introduzione}
\subsection{Scopo del documento}
Questo documento ha come obiettivo la stesura delle modalità attraverso le quali il gruppo 8Lab Solutions svilupperà il progetto Soldino. In particolare il documento tratta i seguenti argomenti:
\begin{itemize}
	\item analisi dei rischi;
	\item adozione di un modello di sviluppo;
	\item ripartizione dei compiti tra i membri del gruppo;
	\item scadenze;
	\item stima dei costi e delle risorse necessiarie allo sviluppo del progetto.
\end{itemize}
\subsection{Scopo del prodotto}
L'obiettivo del progetto è quello di creare una piattaforma di compravendita di beni e servizi tra aziende e persone fisiche per mezzo di cripto-valuta Cubit\glo, un token Ether\glosp custom. Sulla piattaforma è presente anche un altro attore: lo stato, il quale può coniare Cubit\glosp e distribuirli. Inoltre lo Stato, a cadenza trimestrale, riceve il versamento IVA delle aziende che utilizzano la piattaforma, mentre quest'ultime in caso di credito IVA riceveranno un rimborso.
\subsection{Glossario}
Per evitare possibili ambiguità relative al linguaggio utilizzato nei documenti, viene fornito il \textit{Glossario v1.0.0} in cui sono inserite le definizioni dei termini in corsivo marcati con G pedice.
\subsection{Riferimenti}
\subsubsection{Normativi}
\begin{itemize}
	\item \textbf{Norme di Progetto}: \textit{Norme di Progetto v1.0.0}.
\end{itemize}

\subsubsection{Informativi}
\begin{itemize}
	\item \textbf{Capitolato d'appalto C6}: Soldino: piattaforma Ethereum per pagamenti IVA \\
	\textsf{\url{https://www.math.unipd.it/~tullio/IS-1/2018/Progetto/C6.pdf}};
\end{itemize}
