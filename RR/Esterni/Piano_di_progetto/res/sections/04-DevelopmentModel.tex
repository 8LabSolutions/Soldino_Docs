\section{Modello di sviluppo}
La scelta di un modello di sviluppo è fondamentale per la pianificazione di 
progetto e a tal fine è stato adottato il \textbf{modello incrementale}.

\subsection{Modello incrementale}
Il modello di sviluppo incrementale permette di suddividere il sistema in 
attività ognuna delle quali soddisfa un requisito
individuato nell'\textit{Analisi dei Requisiti v1.0.0}. In questo modo è possibile 
identificare ogni incremento con il completamento di un'attività permettendo 
quindi lo sviluppo del sistema attraverso il versionamento dei suoi componenti. 
\\
L'aggiunta di requisiti è consentita previa discussione con il proponente e 
approvazione dello stesso, tuttavia la modifica di essi non è permessa 
durante la fase di sviluppo dell'incremento corrente.\\
L'adozione di questo modello di sviluppo comporta i seguenti vantaggi:
\begin{itemize}
	\item lo sviluppo prioritario delle attività primarie permette di 
	assicurare il soddisfacimento di tutti i requisiti obbligatori;
	\item la suddivisione in incrementi permette di isolare lo sviluppo in 
	moduli e in questo modo quest'ultimi sono sottoposti a fasi di verifica e 
	validazione mirate, portando ad una migliore individuazione di errori e 
	quindi correzione di essi.
\end{itemize}

