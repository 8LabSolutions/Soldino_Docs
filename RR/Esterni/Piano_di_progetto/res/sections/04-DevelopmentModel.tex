\section{Modello di sviluppo}
La scelta di un modello di sviluppo è fondamentale per la pianificazione di 
progetto e a tal fine è stato adottato il modello incrementale.
\subsection{Modello incrementale}
Il modello di sviluppo incrementale permette di suddividere i sistema in 
attività\glosp ognuna delle quali soddisfa un requisito\glosp 
individuato dall' \textit{Analisi dei Requisiti}. In questo modo è possibile 
identificare ogni incremento con il completamento di un'attività permettendo 
quindi lo sviluppo del sistema attraverso il versionamento di esso. \\
Si prevede la suddivisione delle attività in due categorie: 
\begin{itemize}
	\item \textbf{Attività primarie}: sono le attività che identificano i 
	requisiti obbligatori;
	\item \textbf{Attività secondarie}: sono le attività che identificano i 
	requisiti opzionali e desiderabili.
\end{itemize}
