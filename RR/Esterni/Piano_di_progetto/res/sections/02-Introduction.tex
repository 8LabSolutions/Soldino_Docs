\section{Introduzione}
\subsection{Scopo del documento}
Questo documento ha come obiettivo la stesura delle modalità attraverso le quali il gruppo 8Lab Solutions svilupperà il progetto Soldino. In particolare il documento tratta i seguenti argomenti:
\begin{itemize}
	\item Analisi dei rischi;
	\item Descrizione del modello di sviluppo adottato;
	\item Ripartizione dei compiti tra i membri del gruppo;
	\item Stima dei costi e delle risorse necessarie allo sviluppo del progetto.
\end{itemize}
\subsection{Scopo del prodotto}
L'obiettivo del progetto è quello di creare una piattaforma di compravendita di beni e servizi tra aziende e persone fisiche per mezzo di cripto-valuta Cubit\glo, un token Ether\glosp custom. Sulla piattaforma è presente anche un altro attore: lo stato, il quale può coniare Cubit\glosp e distribuirli. Inoltre lo stato, a cadenza trimestrale, riceve il versamento IVA delle aziende che utilizzano la piattaforma, mentre quest'ultime in caso di credito IVA riceveranno un rimborso.
\subsection{Glossario}
Per evitare possibili ambiguità relative al linguaggio utilizzato nei 
documenti, viene fornito il \textit{Glossario v1.0.0} in cui sono inserite le 
definizioni dei termini, i quali, all'interno del documento hanno la lettera "G" 
posta a pedice.
\subsection{Riferimenti}
\subsubsection{Normativi}
\begin{itemize}
	\item \textbf{Norme di Progetto}: \textit{Norme di Progetto v1.0.0};
	\item \textbf{Regolamento organigramma e specifica tecnico-economica} \\
	\url{https://www.math.unipd.it/~tullio/IS-1/2018/Progetto/RO.html}.
\end{itemize}

\subsubsection{Informativi}
\begin{itemize}
	\item \textbf{Capitolato d'appalto C6}: Soldino: piattaforma Ethereum per pagamenti IVA \\
	\url{https://www.math.unipd.it/~tullio/IS-1/2018/Progetto/C6.pdf};
	\item \textbf{Slide L05 del corso Ingegneria del Software}: Ciclo di vita 
	del software \\
	\url{https://www.math.unipd.it/~tullio/IS-1/2018/Dispense/L05.pdf};
	\item \textbf{Slide L06 del corso Ingegneria del Software}: Gestione di 
	Progetto \\
	\url{https://www.math.unipd.it/~tullio/IS-1/2018/Dispense/L06.pdf};
	\item \textbf{Software Engineering - Ian Sommerville - 9$^{th}$ Edition, 
	2010.}
\end{itemize}

\hypertarget{scadenze}{\subsection{Scadenze}}
Il gruppo 8Lab Solutions si impegna a rispettare le seguenti scandenze per lo 
sviluppo del progetto Soldino:

\begin{itemize}
	\item \textbf{Revisione dei Requisiti}: 2018-01-21;
	\item \textbf{Revisione di Progettazione}: 2018-03-15;
	\item \textbf{Revisione di Qualifica}: 2018-04-19;
	\item \textbf{Revisione di Accettazione}: 2018-05-17.
\end{itemize}