\section{Qualità di processo}
Per ricercare qualità nello svolgimento del progetto si adoperano dei processi. Inizialmente, tali processi sono stati scelti tra quelli proposti nello standard ISO/IEC/IEEE 12207:2017; successivamente sono stati semplificati.\newline 
Il risultato sono i processi esposti a seguito.

\subsection{Processi Primari}
	\subsubsection{Processo di Analisi dei Requisiti}
	Mediante il processo le informazioni ottenute dalle varie fonti sono trasformate in forma di casi d'uso e requisiti.
	Questa forma fornisce una descrizione dettagliata del sistema e definisce il funzionamento e le caratteristiche di ogni sua parte.
		\paragraph{Obiettivi}
		\begin{itemize}
			\item definire casi d'uso e requisiti;
			\item ottenere la loro approvazione;
			\item tracciare il loro cambiamento nel tempo.
		\end{itemize}		
		\paragraph{Strategia}
		Tracciare i requisiti del sistema.
		\paragraph{Metriche}
			\subparagraph{Numero di requisiti obbligatori soddisfatti} Indica la percentuale di requisiti obbligatori soddisfatti.
			\begin{itemize}
				\item \textbf{Misurazione}
				\item \textbf{Valore preferibile}
				\item \textbf{Valore accettabile}
			\end{itemize}
			
	\subsection{Processo di Progettazione Architetturale}
	Tradurre i requisiti in un modello architetturale del sistema.
	Il modello è ad alto livello di dettaglio: in esso il sistema è costituito da macro-componenti, utili a capire il funzionamento delle parti, ma non ancora realizzabili nella pratica.\newline
	Nel processo vanno considerati in particolare il tipo di software da produrre, le caratteristiche desiderate e i suoi requisiti non funzionali per scegliere l'architettura più adatta.
		\paragraph{Obiettivi}
		\begin{itemize}
			\item valutare il tipo di software da produrre, le caratteristiche desiderate, i casi d'uso e i requisiti da soddisfare;
			\item valutare i modelli architetturali secondo il punto precedente e scegliere il più adeguato;
			\item individuare nel modello i macro-componenti del sistema;
			\item comprendere e definire le relazioni tra i componenti.
		\end{itemize}
		\paragraph{Strategia}
		Tracciare le componenti del sistema.
		\paragraph{Metriche}
			\subparagraph{Structural Fan-in \textbf{SFIN}}
			Indica quante componenti utilizzano un dato modulo. Un alto valore indica un alto riuso della componente.
			\begin{itemize}
				\item \textbf{Misurazione}: numero intero;
				\item \textbf{Valore preferibile}: $\geq 1$;
				\item \textbf{Valore accettabile}: $\geq 0$.
			\end{itemize}
			\subparagraph{Structural Fan-out \textbf{SFOUT}}
			Indica quante componenti vengono utilizzate dalla componente in esame. Un alto valore indica un alto
accoppiamento della componente.
			\begin{itemize}
				\item \textbf{Misurazione}: numero intero;
				\item \textbf{Valore preferibile}: $=0$;
				\item \textbf{Valore accettabile}: $<=6$.
			\end{itemize}
			
	\subsection{Processo di Progettazione di Dettaglio}
	La progettazione di dettaglio segue la progettazione architetturale, e prevede la scomposizione delle macro-componenti in componenti più piccole che sono:
	\begin{itemize}
		\item immediatamente comprensibili;
		\item strettamente collegate ai requisiti funzionali;
		\item sviluppabili da un singolo programmatore.
	\end{itemize}
		\paragraph{Obiettivi}
		Arrivare alla definizione di dettaglio del sistema, cioè scomporlo in unità software dette componenti.
		\paragraph{Strategia}
		\paragraph{Metriche}
			\subparagraph{Numero di metodi per classe}
			\subparagraph{Numero di parametri per metodo}
		
\subsection{Processi di Supporto}
	\subsection{Processo di Verifica}
	Il processo consiste nella ricerca e correzione di anomalie nei processi e nei prodotti del progetto, mediante tecniche definite.
		\paragraph{Obiettivi}
		\begin{itemize}
			\item individuare e correggere le anomalie;
			\item provare che il sistema soddisfi i requisiti.
		\end{itemize}	
		\paragraph{Strategia}
		\begin{itemize}
			\item individuare tecniche e strumenti di verifica;
			\item affinare tecniche e strumenti con l'esperienza;
		\end{itemize}	
		\paragraph{Metriche}
			\subparagraph{Code Coverage}
				Indica il numero di righe di codice percorse dai test durante la loro esecuzione. 
				\begin{itemize}
					\item misurazione: $Le/Lt$;
					\item valore preferibile: $100\%$;
					\item valore accettabile: $75\%$.
				\end{itemize}
			% a grana più fine ci sarebbero: function coverage, loop coverage, branch coverage
	\subsubsection{Processo di Gestione della Documentazione}
		\paragraph{Obiettivi}
		\paragraph{Stragegia}
		\paragraph{Metriche}
			\subparagraph{} 
			\begin{itemize}
				\item \textbf{}
				\item \textbf{}
				\item \textbf{}
			\end{itemize}
		
\subsection{Processi Organizzativi}
	\subsection{Processo di Pianificazione di Progetto}	
		\paragraph{Obiettivi}
		\paragraph{Strategia}
		\paragraph{Metriche}
			\subparagraph{} 
			\begin{itemize}
				\item \textbf{}
				\item \textbf{}
				\item \textbf{}
			\end{itemize}
	\subsection{Processo di Gestione della Qualità}
		\paragraph{Obiettivi}
		\paragraph{Strategia}
		\paragraph{Metriche}
			\subparagraph{} 
			\begin{itemize}
				\item \textbf{}
				\item \textbf{}
				\item \textbf{}
			\end{itemize}
	\subsection{Processo di Misurazione} % cambia con qta
		\paragraph{Obiettivi}
		Scegliere metriche adeguate, effettuare misurazioni basate su tali metriche per migliorare, controllare e garandire la qualità di prodotto e processo.
		\paragraph{Strategia}
		\begin{itemize}
			\item definire gli oggetti di interesse da misurare;
			\item scegliere metriche appropriate;
			\item effettuare le misurazioni nel modo più possibile automatizzato;
		\end{itemize}
		\paragraph{Metriche}
			\subparagraph{} 
			\begin{itemize}
				\item \textbf{}
				\item \textbf{}
				\item \textbf{}
			\end{itemize}
		
%\subsection{Project Assessment and Control Process}
%\subsection{Processo di Costruzione del Software}
