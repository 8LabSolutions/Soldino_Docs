\section{Qualità di processo}
Per ricercare qualità nello svolgimento del progetto si adoperano dei processi. Inizialmente, tali processi sono stati scelti tra quelli proposti nello standard ISO/IEC/IEEE 12207:2017. Successivamente, essi sono stati adattati, semplificati e suddivisi in categorie (primari, di supporto, organizzativi).\newline 
Il risultato sono i processi esposti a seguito. 

\subsection{Processi Primari}
	\subsubsection{Processo di Analisi dei Requisiti}
		\paragraph{Scopo}
		\paragraph{Attività}
		\paragraph{Metriche}
	\subsection{Processo di Progettazione Architetturale}
		\paragraph{Scopo}
		\paragraph{Attività}
		\paragraph{Metriche}
	\subsection{Processo di Progettazione di Dettaglio}
		\paragraph{Scopo}
		\paragraph{Attività}
		\paragraph{Metriche}
\subsection{Processi di Supporto}
	\subsection{Processo di Verifica del Software}
		\paragraph{Scopo}
		\paragraph{Attività}
		\paragraph{Metriche}
	\subsubsection{Processo di Gestione della Documentazione}
		\paragraph{Scopo}
		\paragraph{Attività}
		\paragraph{Metriche}
\subsection{Processi Organizzativi}
	\subsection{Processo di Pianificazione di Progetto}	
		\paragraph{Scopo}
		\paragraph{Attività}
		\paragraph{Metriche}
	\subsection{Processo di Gestione della Qualità}
		\paragraph{Scopo}
		\paragraph{Attività}
		\paragraph{Metriche}
	\subsection{Processo di Misurazione}
		\paragraph{Scopo}
		Scegliere metriche adeguate, effettuare misurazioni basate su tali metriche per migliorare, controllare e garandire la qualità di prodotto e processo.
		\paragraph{Attività}
		\begin{itemize}
			\item definire gli oggetti di interesse da misurare;
			\item scegliere metriche appropriate;
			\item effettuare le misurazioni nel modo più possibile automatizzato;
		%\paragraph{Metriche}
		
%\subsection{Project Assessment and Control Process}
%\subsection{Processo di Costruzione del Software}
