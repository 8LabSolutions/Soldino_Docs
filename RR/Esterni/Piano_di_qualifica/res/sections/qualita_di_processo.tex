\section{Qualità di processo}
Per ricercare qualità nello svolgimento del progetto si adoperano dei processi. Inizialmente, tali processi sono stati scelti tra quelli proposti nello standard ISO/IEC/IEEE 12207:2017; successivamente sono stati semplificati.\newline 
Il risultato sono i processi esposti a seguito.

%\subsection{Processo di Costruzione del Software}. Opzione per il futuro.


\subsection{Processi Primari}
	\subsubsection{Processo di Analisi dei Requisiti}
	Mediante il processo le informazioni ottenute dalle varie fonti sono trasformate in forma di casi d'uso e requisiti.
	Questa forma fornisce una descrizione dettagliata del sistema e definisce il funzionamento e le caratteristiche di ogni sua parte.
		\paragraph{Obiettivi}
		\begin{itemize}
			\item formulare la definizione di casi d'uso e requisiti;
			\item ottenere la loro approvazione;
			\item tracciare il loro cambiamento nel tempo.
		\end{itemize}		
		\paragraph{Strategia}
		\begin{itemize}
			\item considerare lo scopo del progetto e le richieste degli stakeholder;
			\item esprimere ciò in forma di requisiti, classificati in obbligatori, desiderabili e opzionali;
			\item valutare il corpo dei requisiti e negoziare cambiamenti se necessario;
			\item ottenere la loro approvazione da parte del proponente;
		\end{itemize}
		Disporre del tracciamento dei requisiti del sistema.
		\paragraph{Metriche}
			\subparagraph{Percentuale di requisiti obbligatori soddisfatti - PROS} Indica appunto la percentuale di requisiti obbligatori soddisfatti.
			\begin{itemize}
				\item misurazione: valore percentuale: $ PROS = \frac{requisiti obbligatori soddisfatti}{requisiti obbligatori totali}$;
				\item valore preferibile: $100\%$;
				\item valore accettabile: $100\%$.
			\end{itemize}
		
	\subsubsection{Processo di Definizione dell'Architettura}
	Tradurre i requisiti in un modello architetturale del sistema.
	Il modello è ad alto livello di dettaglio: in esso il sistema è costituito da macro-componenti, utili a capire il funzionamento delle parti, ma non ancora realizzabili nella pratica.\newline
	Nel processo vanno considerati in particolare il tipo di software da produrre, le caratteristiche desiderate e i suoi requisiti non funzionali per scegliere l'architettura più adatta.
		\paragraph{Obiettivi}
		\begin{itemize}
			\item valutare il tipo di software da produrre, le caratteristiche desiderate, i casi d'uso e i requisiti da soddisfare;
			\item valutare i modelli architetturali secondo il punto precedente e scegliere il più adeguato;
			\item individuare nel modello i macro-componenti del sistema;
			\item comprendere e definire le relazioni tra i componenti.
		\end{itemize}
		\paragraph{Strategia}
		Tracciare le componenti del sistema.
		\paragraph{Metriche}
			\subparagraph{Structural Fan-in \textbf{SFIN}}
			Indica quante componenti utilizzano un dato modulo. Un alto valore indica un alto riuso della componente.
			\begin{itemize}
				\item \textbf{Misurazione}: valore intero: conteggio delle componenti;
				\item \textbf{Valore preferibile}: $SFIN \geq 1$;
				\item \textbf{Valore accettabile}: $SFIN \geq 0$.
			\end{itemize}
			\subparagraph{Structural Fan-out \textbf{SFOUT}}
			Indica quante componenti vengono utilizzate dalla componente in esame. Un alto valore indica un alto
accoppiamento della componente.
			\begin{itemize}
				\item \textbf{Misurazione}: valore intero: conteggio delle componenti;
				\item \textbf{Valore preferibile}: $SFOUT = 0$;
				\item \textbf{Valore accettabile}: $SFOUT \leq 6$.
			\end{itemize}
			
	\subsubsection{Processo di Progettazione di Design}
	La Progettazione di Design segue la Definizione dell'Architettura e prevede la scomposizione delle macro-componenti in componenti più piccole che sono:
	\begin{itemize}
		\item immediatamente comprensibili;
		\item strettamente collegate ai requisiti funzionali;
		\item sviluppabili da un singolo programmatore.
	\end{itemize}
		\paragraph{Obiettivi}
			Ottenere unità software atomiche dette componenti, facili da tradurre in codice e da testare.
		\paragraph{Strategia}
		
		\paragraph{Metriche}
			\subparagraph{Numero di metodi per classe}
			\subparagraph{Numero di parametri per metodo}
		
\subsection{Processi di Supporto}
	\subsubsection{Processo di Verifica}
	Il processo consiste nella ricerca e correzione di anomalie e difetti nei processi e prodotti del progetto, mediante tecniche predefinite e se possibile automatiche.
		\paragraph{Obiettivi}
		\begin{itemize}
			\item individuare e correggere le anomalie;
			\item provare che il sistema soddisfi i requisiti.
		\end{itemize}	
		\paragraph{Strategia}
		\begin{itemize}
			\item individuare tecniche e strumenti di verifica;
			\item applicarli;
			\item affinarli con l'esperienza;
		\end{itemize}	
		\paragraph{Metriche}
			\subparagraph{Code Coverage - CC}
				Indica il numero di righe di codice percorse dai test durante la loro esecuzione. Per linee di codice totali si intende tutte quelle appartenenti all'unità in fase di test.
				\begin{itemize}
					\item misurazione: valore percentuale: $CC = \frac{linee\ di\ codice\ percorse}{linee\ di\ codice\ totali}$;
					\item valore preferibile: $100\%$;
					\item valore accettabile: $75\%$.
				\end{itemize}
			% a grana più fine ci sarebbero: function coverage, loop coverage, branch coverage

	\subsubsection{Processo di Gestione dell'Informazione}
	Il processo consiste nella produzione di informazioni, la cui forma tipica è  documentale, e nella loro gestione. I documenti sono prodotti a supporto di tutte le attività di progetto.
		\paragraph{Obiettivi}
			Si costruisce la documentazione affinché sia un body of knowledge % PLACEHOLDER: glossario
			che raccoglie la conoscenza in modo:
			\begin{itemize}
				\item completo;
				\item non ambiguo;
				\item modulare e fatto di parti coese;
				\item trasparente, adatto alla trasmissione delle informazioni;
				\item disponibile esternamente.
			\end{itemize}
		\paragraph{Strategia}
		I documenti sono:
		\begin{itemize}
			\item prodotti in concomitanza con tutte le attività di sviluppo;
			\item prodotti in modo collaborativo;
			\item supportati da un glossario;
			\item supportati da Norme di Progetto;
			\item prodotti con strumenti software adatti alla collaborazione e alla modularità in \LaTeX{};
			\item ospitati in una repository pubblica su Github;
		\end{itemize}
		\paragraph{Metriche}
			% indice di fog omesso: contiene il parametro "numero parole complesse" che non è automatizzabile
			\subparagraph{Indice di Gulpease}
			Indice della leggibilità del testo. Valuta la lunghezza delle parole e delle frasi rispetto al numero totale di lettere. 
			\begin{itemize}
				\item misurazione: valore intero da 0 a 100:\newline 	
				$I_G = 89+ \frac{(300 * numero\ di\ frasi - 10 * numero\ di\ lettere)}{/numero\ di\ parole}$	
				\item valore preferibile: $80 < I_G < 100$;
				\item valore accettabile: $40 < I_G < 100$	
			\end{itemize}
		
\subsection{Processi Organizzativi}
	\subsubsection{Processo di Pianificazione di Progetto}
	Il processo consiste nella produzione di un Piano di Progetto e nella sua manutenzione. Il piano governa le risorse a disposizione, ovvero tempi, costi e ruoli.
		\paragraph{Obiettivi}
		\begin{itemize}
			\item produrre la pianificazione delle attività;
			\item mantenerla aggiornata mentre esse vengono svolte;
			\item usarla come riferimento e supporto.
		\end{itemize}
		\paragraph{Strategia}
		\paragraph{Metriche}
			\subparagraph{BAC: Budget at Completion}
			Budget totale allocato per il progetto
			\begin{itemize}
				\item misurazione: numero intero;
				% PLACEHOLDER: Aggiornare i 2 item qui sotto
				\item valore preferibile: $pari al preventivo$;
				\item valore accettabile: $di poco maggiore o pari al preventivo$.
			\end{itemize}
			\subparagraph{AC: Actual Cost}
			Il denaro speso fino al momento del calcolo.
			\begin{itemize}
				\item  misurazione: numero intero;
				\item  valore preferibile: $0 \leq AC < PV$;
				\item  valore accettabile: $0 \leq AC \leq budget totale$;
			\end{itemize}
			\subparagraph{EV - Earned Value}
			Metrica di utilità. Si tratta del valore del lavoro fatto fino al momento del calcolo; corrisponde al denaro guadagnato fino a quel momento.
			\begin{itemize}
				\item  misurazione: $BAC * \% di lavoro completato$;
				\item  valore preferibile: $EV \geq 0$;
				\item  valore accettabile: $EV \geq 0$.
			\end{itemize}
			\subparagraph{PV: Planned Value}
			Si tratta del valore del lavoro pianificato al momento del calcolo: corrisponde al denaro che si dovrebbe aver guadagnato in quel momento.
			\begin{itemize}
				\item  misurazione: $BAC * \% di lavoro pianificato$;
				\item  valore preferibile: $PV \geq 0$;
				\item  valore accettabile: $PV \geq 0$;
			\end{itemize}			
			\subparagraph{SV: Schedule Variance}
			Dice se si è avanti o indietro nello svolgimento del progetto rispetto alla pianificazione.
			\begin{itemize}
				\item misurazione: $SV = EV - PV$
				\item valore preferibile: $SV > 0$;
				\item valore accettabile: $SV = 0$;
			\end{itemize}
			\subparagraph{Cost Variance}
			Differenza tra il costo del lavoro effettivamente completato e quello pianificato. Una CV positiva indica che si sta rispettando il budget.
			\begin{itemize}
				\item misurazione: $CV = EV - AC$;
				\item valore preferibile: $CV > 0$
				\item valore accettabile: $CV \geq 0$
			\end{itemize}
		
	\subsubsection{Processo di Gestione della Qualità}
	Tale processo ha per scopo il raggiungimento di un grado soddisfacente di qualità nel progetto. Fornisce:
	\begin{itemize}
		\item obiettivi da perseguire;
		\item strumenti tecnici, come procedure e metriche.
	\end{itemize}
		\paragraph{Obiettivi}
		Garantire che i prodotti e processi rispettino gli standard di qualità richiesti.
		\paragraph{Strategia}
		\begin{itemize}
			\item pianificare le proprie azioni perseguendo gli obiettivi;
			\item utilizzare gli strumenti forniti per misurare e monitorare i risultati;
			\item reagire ai risultati aggiornando obiettivi, strategia e strumenti.
		\end{itemize}
		\paragraph{Metriche}
			\subparagraph{Percentuale di metriche soddisfatte - PMS}
			La percentuale di metriche soddisfatte valuta quante metriche raggiungono soglie accettabili sul numero totale delle metriche calcolate. Una bassa percentuale di soddisfazione può indicare poca qualità, metriche inadeguate o mancata correttezza nel calcolo.
			\begin{itemize}
				\item misurazione: $\frac{numero\ di\ metriche\ soddisfatte}{numero\ di\ metriche\ totali} $;
				\item valore preferibile: $PMS \geq 80\%$;
				\item valore accettabile: $PMS \geq 60\%$;
			\end{itemize}
	
