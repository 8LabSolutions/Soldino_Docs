\section{Introduzione}
\subsection{Scopo del documento}
Questo documento ha lo scopo di mostrare le strategie di verifica e validazione adottate al fine di garantire la qualità di prodotto e di processo. Per raggiungere questo obiettivo viene applicato un sistema di verifica continua sui processi in corso e sulle attività svolte. In questo modo è quindi possibile rilevare e correggere all'istante eventuali anomalie, riducendo al minimo lo spreco delle risorse.
\subsection{Scopo del prodotto}
Lo scopo del prodotto è quello di realizzare un software, in particolare un sito internet, che consenta il monitoraggio automatico dell'IVA, ovvero assiste il governo e gli utenti nell'esecuzione di operazioni come liquidazione, versamento e rimborso, e permetta l'acquisto di oggetti tramite una valuta denominata "Cubit"\glosp{}.
% PLACEHOLDER espandere
\subsection{Ambiguità}
All’interno dei documenti alcuni termini presentano significati ambigui  a seconda del contesto, fraintendibili, o che necessitano di una descrizione più approfondita. Per evitare questa ambiguità è stato creato il documento "Glossario" volto a fare chiarezza, ponendo a fianco di ogni termine il suo preciso significato. Questi termini sono pertanto marchiati con una "G" a pedice per ogni loro occorrenza all'interno di tutti i documenti.
\subsection{Riferimenti}
\subsubsection{Riferimenti normativi}
\begin{itemize}
\item \textbf{ISO/IEC 12207} \\* %PLACEHOLDER aggiungere dei riferimenti
%\href{https://en.wikipedia.org/wiki/ISO/IEC_12207}{https://en.wikipedia.org/wiki/ISO/IEC_12207};
\item \textbf{Testo del capitolato} \\*  \href{https://www.math.unipd.it/~tullio/IS-1/2018/Progetto/C6.pdf}{https://www.math.unipd.it/~tullio/IS-1/2018/Progetto/C6.pdf};
%verbali normativi
\end{itemize}
\subsubsection{Riferimenti informativi}
\begin{itemize}
\item Guide?(vedi Pro-tech)
\item \textbf{ISO/IEC 9126} \\*\href{https://en.wikipedia.org/wiki/ISO/IEC_9126}{https://en.wikipedia.org/wiki/ISO/IEC\_9126};
\end{itemize}