\section{Introduzione}
\subsection{Premessa}
Il Piano di Qualifica è un documento su cui si prevede di lavorare l'intera durata del progetto.\newline 
Molti dei contenuti del documento sono di natura instabile. Ad esempio, molte delle metriche scelte non sono applicabili nella fase iniziale, e solo con il loro utilizzo pratico si può valutarne l'effettiva utilità. Anche i processi selezionati possono essere soggetti a cambiamenti, rivelandosi insufficienti o inadeguati agli scopi del progetto e al modo di lavorare del team.\newline 
Alcune parti del documento sono prodotte in fasi temporali successive, come l'appendice sul resoconto delle verifiche.\newline 
Per tutte queste ragioni, il documento è prodotto in maniera incrementale, e i suoi contenuti iniziali sono da considerarsi incompleti: subiranno significative aggiunte e modifiche nel tempo.
\subsection{Scopo del documento}
Questo documento ha lo scopo di mostrare le strategie di verifica e validazione adottate al fine di garantire la qualità di prodotto e di processo. Per raggiungere questo obiettivo viene applicato un sistema di verifica continua sui processi in corso e sulle attività svolte. In questo modo è quindi possibile rilevare e correggere all'istante eventuali anomalie, riducendo al minimo lo spreco delle risorse.
\subsection{Scopo del prodotto}
Lo scopo del prodotto è quello di realizzare un software, in particolare un sito internet, che consenta il monitoraggio automatico dell'IVA, ovvero assiste il governo e gli utenti nell'esecuzione di operazioni come liquidazione, versamento e rimborso, e permetta l'acquisto di prodotti tramite una valuta denominata Cubit\glo.
% PLACEHOLDER espandere
\subsection{Glossario}
Al fine di evitare possibili ambiguità relative al linguaggio utilizzato nei documenti formali, viene fornito il \textit{Glossario v1.0.0}. In questo documento vengono definiti e descritti tutti i termini con un significato particolare. Per facilitare la lettura, i termini saranno contrassegnati da una 'G' a pedice.
\subsection{Riferimenti}
\subsubsection{Riferimenti normativi}
\begin{itemize}

\item \textbf{Capitolato d'appalto C6 - Soldino, piattaforma Ethereum per pagamenti IVA}: \\ \url{https://www.math.unipd.it/~tullio/IS-1/2018/Progetto/C6.pdf}
%verbali normativi
\end{itemize}
\subsubsection{Riferimenti informativi}
\begin{itemize}
% Guide?(vedi Pro-tech)
\item \textbf{ISO/IEC 9126}: \\* \url{https://en.wikipedia.org/wiki/ISO/IEC_9126}
\item \textbf{ISO/IEC 12207}: \\* 
\url{https://www.math.unipd.it/~tullio/IS-1/2009/Approfondimenti/ISO\_12207-1995.pdf}
\item \textbf{Indice di Gulpease}: \\* \url{https://it.wikipedia.org/wiki/Indice_Gulpease}
\item \textbf{Schedule Variance e metriche correlate}:\\* \url{https://www.smartsheet.com/hacking-pmp-how-calculate-schedule-variance}
\end{itemize}