\section{Resoconto attività di verifica}
In questa sezione vengono descritti e analizzati gli esiti delle attività
di verifica svolte su tutti i documenti che vengono consegnati nelle varie 
revisioni di avanzamento del progetto.

\subsection{Revisione dei Requisiti}
\subsubsection{Tracciamento dei casi d'uso e dei requisiti}
Per facilitare il tracciamento delle relazioni fra casi d'uso e requisiti che
fra requisiti e fonti, il gruppo ha deciso di utilizzare il software ???.

\subsubsection{Analisi statica dei documenti}
L'analisi dei documenti mediante \textit{Walkthrough}\glo{} ha portato 
all'individuazione di alcuni errori frequenti a partire dai quali è stata 
stilata una lista di controllo interna. In questo modo sarà possibile applicare
l'\textit{Inspection}\glo{} per le future attività di verifica.

\subsubsection{Esiti verifiche automatizzate}
Nella tabella seguente vengono riportati gli indici Gulpease\glo{} di tutti
i documenti prodotti finora.

\begin{table}[H]
	%\renewcommand{\arraystretch}{1.5}
	\rowcolors{2}{pari}{dispari}
	
	\begin{longtable}{ >{\centering}p{0.40\textwidth} >{\centering}p{0.25\textwidth}
			 >{\centering}p{0.14\textwidth}}
			
		\hline
		\rowcolorhead
		\centering\textbf{\color{white}Documento} 
		& \centering\textbf{\color{white}Indice Gulpease} 
		& \centering\textbf{\color{white}Esito}
		\tabularnewline \hline 	
		
		
		\textit{Analisi dei Requisiti v1.0.0} & & Superato
		
		\tabularnewline 
		\textit{Glossario v1.0.0} & & Superato
				
		\tabularnewline 
		\textit{Norme di Progetto v1.0.0} & & Superato
		
		\tabularnewline 
		\textit{Piano di Progetto v1.0.0} & & Superato
		
		\tabularnewline 
		\textit{Piano di Qualifica v1.0.0} & & Superato	
		
		\tabularnewline 
		\textit{Studio di Fattibilità v1.0.0} & & Superato
	
	\end{longtable}
	\caption{Esiti verifiche automatizzate - Revisione dei Requisiti}	

\end{table}