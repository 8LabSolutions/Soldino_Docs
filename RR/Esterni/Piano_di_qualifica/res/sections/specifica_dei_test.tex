\section{Specifica dei test}
Per assicurare la qualità del software prodotto, il gruppo \textit{8Lab
Solutions} adotterà come modello si sviluppo del software il
\textbf{Modello a V}, il quale prevede lo sviluppo dei test in parallelo alle
attività di analisi e progettazione. In questo modo i test permetteranno di
verificare sia la correttezza delle parti di programma sviluppati, sia che
tutti gli aspetti del progetto siano implementati e corretti. 
Seguirà quindi il tracciamento dei test e il loro esito per mezzo di tabelle che ne
semplificheranno la consultazione e che potranno fornire una precisa indicazione 
degli output prodotti, specificando se il risultato ottenuto sia quello atteso, errato
oppure non coerente a quanto fissato in precedenza.
Per definire lo stato dei test, vengono utilizzate le seguenti sigle:
\begin{itemize}
	\item \textbf{I}: per indicare che il test è stato implementato;
	\item \textbf{NI}: per indicare che il test non è stato implementato.
\end{itemize}
Inoltre per lo stato dei test si usano le seguenti abbreviazioni:
\begin{itemize}
	\item \textbf{S}: per indicare che il test ha soddisfatto la richiesta;
	\item \textbf{NS}: per indicare che il test non ha soddisfatto la richiesta.
\end{itemize}

\subsection{Tipi di test}
Vengono individuate quattro tipologie di test:
\begin{itemize}
	\item \textbf{Test di Accettazione [TA]}: essi vengono fatti alla fine per 
		verificare che il prodotto finale soddisfi quanto richiesto dal proponente;
	\item \textbf{Test di Sistema [TS]}: questi test vengono utilizzati quando il 
		sistema viene installato su una piattaforma e verificano che esso raggiunga gli 
		obiettivi fissati e soddisfi le richieste formulate in partenza;
	\item \textbf{Test di Integrazione [TI]}: lo scopo di questi test è quello di 
		testare come un gruppo i singoli moduli\glo{} del software. Essi vengono svolti 
		dopo i Test di Unità e prima dei Test di Sistema; 
	\item \textbf{Test di Unità [TU]}: questi test hanno il compito di verificare le 
		singole unità del software, ovvero le minime componenti del programma che hanno 
		un funzionamento autonomo. Il successo da parte di questi test non implica il 
		corretto funzionamento da parte del software.		
\end{itemize}

\subsubsection{Test di Accettazione}
I test di accettazione hanno lo scopo di dimostrare che il software sviluppato 
soddisfi le richieste del proponente ed essi vengono eseguiti durante il
collaudo finale.

\subsubsection{Test di Sistema}
I test di sistema sono impiegati per garantire il corretto funzionamento delle 
componenti dell'intero sistema. Tali test verranno indicati nel seguente modo:
	\centerline{\textbf{TS[id]}}
dove \textit{id} rappresenta il codice identificativo crescente del componente da
verificare.


\subsubsection{Test di Integrazione}
I test di integrazione sono usati per verificare il corretto funzionamento tra le
varie unità dell'architettura. Tali test verranno indicati nel seguente modo:
	\centerline{\textbf{TI[id]}}
dove \textit{id} rappresenta il codice identificativo crescente del componente da
verificare.

\subsubsection{Test di Unità}
I test di unità hanno l'obiettivo di verificare il corretto funzionamento della 
parte più piccola autonoma del lavoro realizzato. Tali test verranno indicati nel
seguente modo:
	\centerline{\textbf{TU[id]}}
dove \textit{id} rappresenta il codice identificativo crescente dell'unità da verificare.





