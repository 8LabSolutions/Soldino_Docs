\section{Valutazioni per il miglioramento}
In questa sezione viene riportata la valutazione fatta dal gruppo riguardo il 
lavoro svolto finora. Lo scopo di questa scelta è trattare i problemi sorti e
procedere alla loro più efficiente risoluzione in modo tale che non si verifichino
in futuro. \\
Verano dunque tracciati problemi riguardanti i seguenti ambiti:

\begin{itemize}
	\item \textbf{Organizzazione}: in cui vengono analizzati i problemi riguardanti 
		l'organizzazione e la comunicazione all'interno del gruppo;
	\item \textbf{Ruoli}: in cui vengono analizzati i problemi riguardanti il 
		corretto svolgimento di un ruolo;
	\item \textbf{Strumenti}: in cui vengono analizzati i problemi riguardanti 
		l'uso degli strumenti scelti.
\end{itemize}

\noindent Ogni problema viene sollevato sulla base dell'autovalutazione dei membri del 
gruppo, poichè non vi è una persona esterna al gruppo che possa dare una valutazione
oggettiva. Nonostante possa sembrare un sistema poco efficace, il gruppo ha 
beneficiato di questa scelta dal punto di vista della comunicazione e dal punto di
vista della produzione, migliorando progressivamente la qualità del lavoro.
Questa sezione verrà aggiornata con l'avanzamento del lavoro riportando nuove 
problematiche, nel caso in cui queste dovessero verificarsi.

\subsection{Valutazioni sull'organizzazione}
\begin{itemize}
\item Si è rivelato difficoltoso organizzare incontri nei quali fossero presenti tutti i membri, si è scelto quindi di organizzare incontri più frequentemente anche se non tutti i membri sono presenti;
\end{itemize}
\subsection{Valutazione sui ruoli}
\begin{itemize}
\item Il responsabile, a causa dell'inesperienza, ha trovato difficoltoso suddividere in modo bilanciato il lavoro tra i membri, si è scelto quindi di ridurre il carico assegnato ad ogni singolo membro;
\end{itemize}
\subsection{Valutazioni sugli strumenti}
\begin{itemize}
\item Si è verificato difficoltoso per alcuni membri, causa inesperienza, la stesura dei documenti \LaTeX  soprattutto per quanto riguarda la costruzione di tabelle e l'inserimento di figure, si è scelto quindi di far creare ai membri più esperti template e comandi personalizzati.
\item Si sono verificati alcuni conflitti durante il push di modifiche su Github, si è scelto quindi di togliere la possibilità di pushare direttamente sul master senza che almeno un altro membro approvati le modifiche. 
\end{itemize}