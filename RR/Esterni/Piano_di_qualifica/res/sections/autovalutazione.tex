\section{Valutazioni per il miglioramento}
In questa sezione viene riportata la valutazione fatta dal gruppo riguardo il 
lavoro svolto finora. Lo scopo di questa scelta è trattare i problemi sorti e
procedere alla loro più efficiente risoluzione in modo tale che non si verifichino
in futuro. \\
Verano dunque tracciati problemi riguardanti i seguenti ambiti:

\begin{itemize}
	\item \textbf{Organizzazione}: in cui vengono analizzati i problemi riguardanti 
		l'organizzazione e la comunicazione all'interno del gruppo;
	\item \textbf{Ruoli}: in cui vengono analizzati i problemi riguardanti il 
		corretto svolgimento di un ruolo;
	\item \textbf{Strumenti di lavoro}: in cui vengono analizzati i problemi riguardanti 
		l'uso degli strumenti scelti.
\end{itemize}

\noindent Ogni problema viene sollevato sulla base dell'autovalutazione dei membri del 
gruppo, poiché non vi è una persona esterna che possa dare una valutazione
oggettiva. Nonostante possa sembrare un sistema poco efficace, il gruppo ha 
beneficiato di questa scelta dal punto di vista della comunicazione e  della produzione, migliorando progressivamente la qualità del lavoro.
Questa sezione verrà aggiornata con l'avanzamento del lavoro riportando nuove 
problematiche, nel caso in cui queste dovessero verificarsi.
Per rendere le valutazioni più leggibili e consultabili, si è
deciso di organizzarle in forma tabellare, la cui struttura è 
consultabile nel documento \textit{Norme di Progetto v1.0.0}.

\subsection{Valutazioni sull'organizzazione}
\begin{table}[H]
	%\renewcommand{\arraystretch}{1.5}
	\rowcolors{2}{pari}{dispari}
	
	\begin{longtable}{ >{\centering}p{0.20\textwidth} >{\centering}p{0.30\textwidth}
			>{\centering}p{0.10\textwidth} >{\centering}p{0.30\textwidth}}
			
		\hline
		\rowcolorhead
		\textbf{\color{white}Problema} 
		& \textbf{\color{white}Descrizione} 
		& \centering\textbf{\color{white}Gravità}
		& \textbf{\color{white}Soluzione} 
		\tabularnewline \hline 	
		
		Incontro di gruppo &
		Si è riscontrata una certa difficoltà nell'organizzare gli incontri in modo
		tale che fossero presenti tutti i componenti. &
		2 &
		Si è deciso di utilizzare un calendario condiviso per scegliere il giorno
		in cui tutto il team potesse essere presente. 
				
		\tabularnewline 
		Incontro con il proponente &
		Poiché l'azienda proponente ha sede all'estero, gli incontri sono stati fatti 
		via skype e c'è stata un'iniziale difficoltà nel trovare una sede in cui 
		trovarsi con il gruppo per effettuare la videochiamata senza essere disturbati. &
		1 &
		Il Dipartimento di Matematica concede la possibilità di prenotare delle aule
		per necessità come la nostra, quindi, venuti a conoscenza di ciò, il problema 
		è stato risolto in breve tempo.
			
	\end{longtable}
	\caption{Tabella problematiche relative all'organizzazione}	
\end{table}

\subsection{Valutazione sui ruoli}
\begin{table}[H]
	%\renewcommand{\arraystretch}{1.5}
	\rowcolors{2}{pari}{dispari}
	
	\begin{longtable}{ >{\centering}p{0.20\textwidth} >{\centering}p{0.30\textwidth}
			>{\centering}p{0.10\textwidth} >{\centering}p{0.30\textwidth}}
			
		\hline
		\rowcolorhead
		\textbf{\color{white}Problema} 
		& \textbf{\color{white}Descrizione} 
		& \centering\textbf{\color{white}Gravità}
		& \textbf{\color{white}Soluzione} 
		\tabularnewline \hline 	
		
		Rivestire il ruolo di \textit{Responsabile} &
		A causa dell'inesperienza, chi ha lavorato come responsabile ha avuto discrete
		difficoltà nella suddivisione bilanciata delle ore tra i membri provocando 
		diverse ridistribuzioni delle ore. &
		3 &
		Per evitare eventuali ritardi nelle consegne, il gruppo ha deciso di dedicare 
		del tempo per analizzare meglio la mole di lavoro e compiere così una più
		accurata distribuzione delle ore.
						
		\tabularnewline 
		Rivestire il ruolo di \textit{Analista} &
		La scelta e la configurazione del software per il tracciamento dei requisiti
		ha richiesto più ore del previsto, come si può vedere anche dal documento
		\textit{Piano di Progetto v1.0.0}. &
		2 &
		Per evitare ritardi sul lavoro, chi ha svolto il ruolo di \textit{Verificatore}
		e ha avanzato ore, è stato affiancato agli \textit{Analisti} per completare 
		i loro compiti.
			
	\end{longtable}
	\caption{Tabella problematiche relative ai ruoli}	
\end{table}

\subsection{Valutazioni sugli strumenti di lavoro}
\begin{table}[H]
	%\renewcommand{\arraystretch}{1.5}
	\rowcolors{2}{pari}{dispari}
	
	\begin{longtable}{ >{\centering}p{0.20\textwidth} >{\centering}p{0.30\textwidth}
			>{\centering}p{0.10\textwidth} >{\centering}p{0.30\textwidth}}
			
		\hline
		\rowcolorhead
		\textbf{\color{white}Problema} 
		& \textbf{\color{white}Descrizione} 
		& \centering\textbf{\color{white}Gravità}
		& \textbf{\color{white}Soluzione} 
		\tabularnewline \hline 	
		
		GitHub & 
		Si sono riscontrati in più occasioni conflitti sui file in cui si
		stava lavorando e il tempo utilizzato per risolverli è stato sottratto dal tempo
		di lavoro. &
		2 &
		Il gruppo è stato istruito sull'uso di specifici rami di lavoro in modo tale che
		la modifiche di tutti i componenti si potessero integrare con il proprio lavoro 
		senza che quest'ultimo potesse soffrire di conflitti. Da questa scelta, il team
		ha potuto beneficiare nell'avere nel ramo \textit{master} la versione stabile 
		del proprio lavoro e \textit{branch} dedicati su cui lavorare.
						
		\tabularnewline 
		\LaTeX{} &
		A causa dell'inesperienza di alcuni membri del gruppo all'utilizzo di questo
		strumento, si sono riscontrare diverse difficoltà sopratutto nell'inserimento 
		di figure e nella costruzione di tabelle. &
		1 &
		Per risolvere in breve tempo questa problematica, si è deciso di affiancare
		ai membri meno esperti chi sapeva già utilizzare i comandi di \LaTeX{} dando
		così la possibilità ai primi di imparare e permettendo ai secondi di non 
		subire grossi rallentamenti nel lavoro.
		
			
	\end{longtable}
	\caption{Tabella problematiche relative agli strumenti di lavoro}	
\end{table}

