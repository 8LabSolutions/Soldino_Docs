\section*{S}
\subsection*{SCSS}
\index{SCSS}
Estensione del linguaggio CSS, che ne aumenta le funzionalità e l'espressività.

\subsection*{Server-side}
\index{Server-side}
Fa riferimento a operazioni compiute dal server in un ambito client-server contrapponendosi a tutto ciò che viene eseguito sul client.

\subsection*{Shortcut}
\index{Shortcut}
Una maniera alternativa e più veloce per svolgere la stessa attività.

\subsection*{Skill}
\index{Skill}
Funzioni aggiuntive per Amazon Alexa non previste nel pacchetto base.

\subsection*{Slack}
\index{Slack}
Sistema di messaggistica cloud-based orientato al mondo lavorativo. Permette la creazioni di diversi team di lavoro, ulteriormente suddivi in "canali" che trattano un particolare topic.

\subsection*{Smart Contract}
\index{Smart Contract}
Protocolli per facilitare, attuare e verificare la negoziazione di un contratto in versione digitale. Permettono di ottenere lo stesso valore di un contratto reale senza l'ausilio di un garante esterno. Le transazioni che avvengono con questo protocollo sono tracciabili e irreversibili. Uno smart contract rappresenta del codice che può essere eseguito.

\subsection*{SonarQube}
\index{SonarQube}
Piattaforma open source che permette disvolgere una continua revisione del codice caricato in una repository, svolgendo analisi statica per individuare possibili bug e vulnerabilità in sicurezza.

\subsection*{Staging}
\index{Staging}
Fase successiva alla fase di sviluppo del software che consiste nell'assemblare tutte le componenti e testarle in un server svolgere verifiche e test. Se il software possiede il comportamento desiderato allora può passare alla fase di produzione.

\subsection*{Stand-up}
\index{Stand-up}
Riunioni giornaliere di breve durata in cui le persone che partecipano ad un progetto espongono i progressi raggiunti e le attività pianificate per la giornata. Sono uno degli elementi principali dei metodi di sviluppo Agile\glo. 

