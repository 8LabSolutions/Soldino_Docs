\subsection*{\quad$A\quad$}
\subsubsection*{Add-on}
\index{Add-on}
Parte di software aggiuntiva che estende le funzionalità e/o caratteristiche del software di base. Di solito è eseguito insieme al software in cui è aggiunto.

\subsubsection*{Agile}
\index{Agile}
Modello di sviluppo software caratterizzato da: team di piccole dimensioni che comunicano informalmente, importanza del software rispetto ai documenti, pianificazione adattiva, coinvolgimento del cliente nel processo di sviluppo.

\subsubsection*{Aliquota}
\index{Aliquota}
Percentuale del reddito, patrimonio o valore imponibile, fissata dalla legge, in base alla quale si determina il debito individuale di imposta.

\subsubsection*{Apache Kafka}
\index{Apache Kafka}
Piattaforma open source di streaming distribuito, usata per stream processing, monitoraggio di dati e traffico web, messaggistica e varie operazioni su file di log.

\subsubsection*{API}
\index{API}
Con Application Programming Interface si intende un insieme di procedure e funzioni offerte ai programmatori per facilitare lo sviluppo. Le API espongono blocchi di codice delle librerie di cui fanno parte, per permetterne il riuso.

\subsubsection*{Area di staging}
\index{Area di staging}
L'area di staging è un'area di immagazzinamento intermedia usata per i dati in lavorazione durante l'estrazione, la trasformazione e il caricamento di un processo. Su Git è usata come step precedente al processo di commit.

\subsubsection*{Artefatti}
\index{Artefatti}
Un artefatto è un sottoprodotto che viene realizzato durante lo sviluppo software. Sono artefatti i casi d'uso, i diagrammi delle classi, i modelli UML\glo, il codice sorgente e la documentazione varia.

\subsubsection*{Attore}
\index{Attore}
Entità che interagisce con il sistema per svolgere delle attività. Può essere umana o un'altro sistema.

\subsubsection*{AWS}
\index{AWS}
Amazon Web Services è un insieme di servizi di cloud computing che compongono la piattaforma on demand offerta dall'azienda Amazon

