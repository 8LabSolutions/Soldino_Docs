\section*{Tabelle delle modifiche}
\renewcommand{\arraystretch}{1.5}
	\begin{longtable}{ >{\centering}p{1.5cm} >{\centering}p{1.8cm}
			>{\centering}p{2.9cm} >{\centering}p{2cm} >{}p{5cm} }
		
		\caption{Tabelle delle modifiche di questo documento} \\
		\hline
		\textbf{Versione} & \textbf{Data} & \textbf{Nominativo} & \textbf{Ruolo} &
		\textbf{Descrizione} \tabularnewline \hline
		\endfirsthead
			\caption{Tabella delle modifiche (continua)} \\
		\hline
		\textbf{Versione} & \textbf{Data} & \textbf{Nominativo} & 
		\textbf{Ruolo} &
		\textbf{Descrizione} \tabularnewline \hline
		\endhead
		
		
		0.0.11 & 2018-12-30 & Francesco Donè & 
		\textit{Analista} & Stesura UC5 e UC6.
		\tabularnewline
		\hline
		
		0.0.10 & 2018-12-28 & Francesco Donè & 
		\textit{Analista} & Stesura UC9 e UC10.
		\tabularnewline
		\hline
		
		0.0.9 & 2018-12-28 & Samuele Giuliano Piazzetta & 
		\textit{Analista e verificatore} & Stesura UC11 e revisione UC12.
		\tabularnewline
		\hline
		
		0.0.8 & 2018-12-27 & Francesco Donè & 
		\textit{Analista} & Stesura UC12, sottocasi di UC12 e sottocasi di UC4.
		\tabularnewline
		\hline
		
		0.0.7 & 2018-12-26 & Francesco Donè & 
		\textit{Analista} & Stesura UC11 e sottocasi di UC11.
		\tabularnewline
		\hline
		
		0.0.6 & 2018-12-26 & Samuele Giuliano Piazzetta & 
		\textit{Redattore} & Stesura UC1, UC2 e sottocasi di UC2.
		\tabularnewline
		\hline
		
		0.0.5 & 2018-12-24 & Francesco Donè & 
		\textit{Analista} & Stesura UC3, UC4 e sottocasi di UC3.
		\tabularnewline
		\hline
		
		0.0.4 & 2018-12-24 & Francesco Donè & 
		\textit{Analista} & Stesura generale dei casi d'uso.
		\tabularnewline
		\hline
		
		0.0.3 & 2018-12-03 & Samuele Giuliano Piazzetta & 
		\textit{Analista} & Stesura descrizione generale.
		\tabularnewline
		\hline
		
		0.0.2 & 2018-12-03 & Samuele Giuliano Piazzetta & 
		\textit{Analista} & Stesura introduzione.
		\tabularnewline
		\hline
		
		0.0.1 & 2018-12-03 & Samuele Giuliano Piazzetta & 
		\textit{Redattore} &
		Creata struttura del documento in \LaTeX{}.
		\tabularnewline
		\hline
		
		%\end{tabularx}
		
	\end{longtable}
\renewcommand{\arraystretch}{1} 
