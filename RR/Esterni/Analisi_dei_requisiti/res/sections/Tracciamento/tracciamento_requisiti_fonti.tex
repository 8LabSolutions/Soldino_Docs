\subsection{Requisito-Fonti}
\begin{center}
	\rowcolors{2}{pari}{dispari}
	
	\begin{longtable}{ >{\centering}p{0.5\textwidth}
			>{\centering}p{0.5\textwidth}}
		\caption{Tabella tracciamento requisito-fonti}\\
		\rowcolorhead 
		\textbf{\color{white}Requisito}
		& \textbf{\color{white}Fonti} 
		\tabularnewline 
		\endfirsthead
		\caption{(continua)}\\	
		\rowcolorhead 
		\textbf{\color{white}Requisito}
		& \textbf{\color{white}Fonti} 
		\tabularnewline 
		\endhead
		
		R1F1 & Obbligatorio 
		\tabularnewline
		R1F1.1 & Obbligatorio 
		\tabularnewline
		R1F1.2 & Obbligatorio 
		\tabularnewline
		R1F1.3 & Obbligatorio 
		\tabularnewline
		
		R1F2 & Obbligatorio 
		\tabularnewline
		R1F2.1 & Obbligatorio 
		\tabularnewline
		%R1F2.2 & Obbligatorio & Il governo  &
		R1F3 & Obbligatorio 
		\tabularnewline 
		R1F3.1 & Obbligatorio 
		\tabularnewline
		
		
		
		
	\end{longtable}
\end{center}


