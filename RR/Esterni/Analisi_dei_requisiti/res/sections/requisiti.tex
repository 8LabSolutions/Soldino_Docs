
\section{Requisiti} 
Ogni requisito è composto dalla seguente struttura:
\begin{itemize}
	\item \textbf{codice identificativo}: ogni codice identificativo è univoco e conforme alla seguente codifica: \\
	\centerline{\textbf{R[Importanza][Tipologia][Codice]}} \\ \\
	Il significato delle cui voci è:
	\begin{itemize}
		\item \textbf{Importanza}: ogni requisito può assumere uno dei seguenti valori:
		\begin{itemize}
			\item \textit{1}: requisito obbligatorio: irrinunciabili per qualcuno degli stakeholder;
			\item \textit{2}: requisito desiderabile: non strettamente necessari ma  a valore aggiunto riconoscibile;
			\item \textit{3}: requisito opzionale: relativamente utili oppure contrattabili più avanti nel progetto;	
		\end{itemize}
		\item \textbf{Tipologia}: ogni requisito può assumere uno dei seguenti valori:
		\begin{itemize}
			\item \textit{F}: funzionale;
			\item \textit{Q}: prestazionale;
			\item \textit{P}: qualitativo;
			\item \textit{V}: vincolo.
		\end{itemize}
		\item \textbf{Codice}: è un identificatore univoco del requisito in forma gerarchica padre/figlio.
	\end{itemize}
	\item \textbf{classificazione}: viene riportata l'importanza del requisito. Sebbene questa sia un'informazione ridondante ne facilita la lettura;
	\item \textbf{descrizione}: descrizione breve ma completa del requisito, meno ambigua possibile;
	\item \textbf{fonti}: ogni requisito può derivare da una o più tra le seguenti opzioni:
	\begin{itemize}
		\item \textit{capitolato\glo}: si tratta di un requisito individuato dalla lettura del capitolato\glo;
		\item \textit{interno}: si tratta di un requisito che gli analisti hanno ritenuto opportuno aggiungere;
		\item \textit{caso d'uso}: il requisito è estrapolato da uno o più casi d'uso. In questo caso è riportato il codice univoco del caso d'uso;
		\item \textit{verbale}: si tratta di un requisito individuato in seguito ad una richiesta di chiarimento con il proponente. Tali informazioni sono riportate nei verbali in cui ogni requisito individuato è segnato da un codice presente nella tabella dei tracciamenti.
	\end{itemize}
\end{itemize}
\renewcommand{\arraystretch}{1.5}

\subsection{Requisiti funzionali}

\rowcolors{2}{pari}{dispari}

\begin{longtable}{ >{\centering}p{0.15\textwidth} >{\centering}p{0.20\textwidth}
		>{\raggedright}p{0.35\textwidth} >{\centering}p{0.14\textwidth}}
	\caption{Tabella dei requisiti funzionali}\\
	\rowcolorhead 
	\textbf{\color{white}Requisito} 
	& \textbf{\color{white}Classificazione} 
	& \centering\textbf{\color{white}Descrizione}
	& \textbf{\color{white}Fonti} 
	\endfirsthead
	\rowcolor{white}\caption[]{(continua)}\\
	\rowcolorhead 
	\textbf{\color{white}Requisito} 
	& \textbf{\color{white}Classificazione} 
	& \centering\textbf{\color{white}Descrizione}
	& \textbf{\color{white}Fonti} 
	\endhead	
	
	R2F1	&	Desiderabile	&	L’utente può leggere una breve guida sull’uso di MetaMask\glosp e sul pagamento delle operazioni	&	Interno\\ UC1	\tabularnewline
	R1F2	&	Obbligatorio	&	Un utente non registrato può effettuare la registrazione	&	Capitolato	\tabularnewline
	R1F2.1	&	Obbligatorio	&	Il sistema permette la registrazione di un nuovo cittadino	&	Capitolato	\tabularnewline
	R1F2.1.1	&	Obbligatorio	&	La registrazione da parte di un cittadino necessita di email	&	Interno \\ UC2.2.1	\tabularnewline
	R1F2.1.2	&	Obbligatorio	&	La registrazione da parte di un cittadino necessita di indirizzo	&	Interno \\ UC2.2.2	\tabularnewline
	R1F2.1.3	&	Obbligatorio	&	La registrazione da parte di un cittadino necessita di nome	&	Interno \\ UC2.3.1	\tabularnewline
	R1F2.1.4	&	Obbligatorio	&	La registrazione da parte di un cittadino necessita di cognome	&	Interno \\ UC2.3.2	\tabularnewline
	R1F2.2	&	Obbligatorio	&	Il sistema permette la registrazione di una nuova azienda	&	Capitolato	\tabularnewline
	R1F2.2.1	&	Obbligatorio	&	La registrazione da parte di un'azienda necessita di email	&	Interno \\ UC2.2.1	\tabularnewline
	R1F2.2.2	&	Obbligatorio	&	La registrazione da parte di un'azienda necessita della sede	&	Interno \\ UC2.2.2	\tabularnewline
	R1F2.2.3	&	Obbligatorio	&	La registrazione da parte di un'azienda necessita di partita IVA	&	Interno \\ UC2.4.1	\tabularnewline
	R1F2.2.4	&	Obbligatorio	&	La registrazione da parte di un'azienda necessita del nome dell'azienda	&	Interno \\ UC2.4.2	\tabularnewline
	R1F2.3	&	Obbligatorio	&	La fase di registrazione di un nuovo utente non va a buon fine se la chiave\glosp reperita da MetaMask\glosp è già presente nel sistema	&	Interno \\ UC2.7	\tabularnewline
	R1F3	&	Obbligatorio	&	Un utente può effettuare il login	&	Interno \\ UC3	\tabularnewline
	R1F3.1	&	Obbligatorio	&	Il login sulla piattaforma deve avvenire in modo automatico attraverso MetaMask\glo	&	Capitolato\\ UC3.1	\tabularnewline
	R1F3.2	&	Obbligatorio	&	Il processo di login e/o registrazione dell'utente non va a buon fine se MetaMask\glosp non è presente	&	Interno \\ UC2.5	\tabularnewline
	R1F3.3	&	Obbligatorio	&	Il processo di login e/o registrazione dell'utente non va a buon fine se non vi è alcuna chiave disponibile nel plug-in	&	Interno \\ UC2.6	\tabularnewline
	R1F3.4	&	Obbligatorio	&	Il processo di login si interrompe se la chiave inserita non è registrata nel sistema	&	Interno \\ UC3.2	\tabularnewline
	R1F3.5	&	Obbligatorio	&	Il processo di login e/o registrazione dell'utente non va a buon fine se l'utente risulta "disabilitato"	&	Interno \\ UC3.3	\tabularnewline
	R1F4	&	Obbligatorio	&	L'utente può effettuare il logout	&	Interno\\ UC4	\tabularnewline
	R1F5	&	Obbligatorio	&	Il governo può coniare Cubit\glo	&	Capitolato \\ UC9	\tabularnewline
	R1F5.1	&	Obbligatorio	&	Il governo può inserire la quantità di Cubit\glosp da coniare	&	Interno \\ UC9	\tabularnewline
	R1F6	&	Obbligatorio	&	Il governo può distribuire Cubit\glosp a cittadini ed aziende	&	Capitolato \\ UC10	\tabularnewline
	R1F6.1	&	Obbligatorio	&	Il governo può determinare l’ammontare di Cubit\glosp pro capite da trasferire	&	Interno\\ UC10.1	\tabularnewline
	R1F6.2	&	Obbligatorio	&	Il governo può selezionare la lista dei destinatari del trasferimento dalla lista degli utenti	&	Interno\\ UC10.2	\tabularnewline
	R1F6.3	&	Obbligatorio	&	La distribuzione dei Cubit\glosp da parte del governo fallisce mostrando un errore se è stata raggiunta la soglia massima distribuibile	&	Interno\\ UC10.4	\tabularnewline
	R1F7	&	Obbligatorio	&	Il governo può visualizzare e gestire la lista degli utenti registrati	&	Interno \\ UC11	\tabularnewline
	R1F7.1	&	Obbligatorio	&	Il governo può visualizzare e gestire lo stato di "abilitato" o "disabilitato" per un utente registrato	&	Interno \\ UC11.1	\tabularnewline
	R1F7.1.1	&	Obbligatorio	&	Il governo può disabilitare l'account di un utente registrato	&	Interno \\ UC12.2	\tabularnewline
	R3F7.1.1.1	&	Opzionale	&	Il governo può motivare con un messaggio la scelta di disabilitazione di un account	&	Interno \\ UC12.3	\tabularnewline
	R1F7.1.2	&	Obbligatorio	&	Il governo può abilitare l'account, precedentemente disabilitato, di un utente registrato	&	Interno \\ UC12.1	\tabularnewline
	R1F7.2	&	Obbligatorio	&	Il governo può visualizzare e gestire la lista delle aziende registrate	&	Capitolato \\ UC11.1	\tabularnewline
	R1F7.2.1	&	Obbligatorio	&	Il governo può visualizzare la chiave Ethereum\glosp di ogni azienda	&	Interno \\ UC11.1	\tabularnewline
	R1F7.2.2	&	Obbligatorio	&	Il governo può visualizzare il nome di ogni azienda	&	Interno \\ UC11.1	\tabularnewline
	R1F7.2.3	&	Obbligatorio	&	Il governo può visualizzare la partita IVA di ogni azienda	&	Interno \\ UC11.1	\tabularnewline
	R1F7.2.4	&	Obbligatorio	&	Il governo può visualizzare la sede di ogni azienda	&	Interno \\ UC11.1	\tabularnewline
	R1F7.2.5	&	Obbligatorio	&	Il governo può visualizzare lo stato del pagamento del saldo IVA e relativo importo di ogni azienda	&	Interno \\ UC11.1	\tabularnewline
	R3F7.2.6	&	Opzionale	&	Il governo può visualizzare la lista delle aziende filtrando i risultati per il valore \texttt{stato di liquidazione IVA}, che può assumere i valori: insolvente,  in fase di pagamento, in dilazione,  regolare, in attesa di rimborso	&	Interno \\ UC11.1.1	\tabularnewline
	R1F7.2.7	&	Obbligatorio	&	Il governo può effettuare il rimborso alle aziende il cui stato di liquidazione IVA risulta in attesa di rimborso	&	Interno  \\ UC13	\tabularnewline
	R1F7.3	&	Obbligatorio	&	Il governo può visualizzare e gestire la lista dei cittadini registrati	&	Interno \\ UC11.2	\tabularnewline
	R1F7.3.1	&	Obbligatorio	&	Il governo può visualizzare la chiave\glosp di ogni cittadino	&	Interno \\ UC11.2	\tabularnewline
	R1F7.3.2	&	Obbligatorio	&	Il governo può visualizzare il nome di ogni cittadino	&	Interno \\ UC11.2	\tabularnewline
	R1F7.3.3	&	Obbligatorio	&	Il governo può visualizzare il cognome di ogni cittadino	&	Interno\\ UC11.2	\tabularnewline
	R1F7.3.4	&	Obbligatorio	&	Il governo può visualizzare l'indirizzo di ogni cittadino	&	Interno\\ UC11.2	\tabularnewline
	R1F7.3.5	&	Obbligatorio	&	Il governo può visualizzare l'email di ogni cittadino	&	Interno\\ UC11.2	\tabularnewline
	R1F8	&	Obbligatorio	&	Aziende e cittadini possono visualizzare i prodotti in vendita nel sito	&	Interno\\ UC5	\tabularnewline
	R1F8.1	&	Obbligatorio	&	Aziende e cittadini possono visualizzare il nome di ciascun prodotto in vendita nel sito	&	Interno\\ UC5	\tabularnewline
	R1F8.2	&	Obbligatorio	&	Aziende e cittadini possono visualizzare il prezzo lordo\glosp{} di ciascun prodotto in vendita nel sito	&	Interno\\ UC5	\tabularnewline
	R1F8.3	&	Obbligatorio	&	Aziende e cittadini possono visualizzare la descrizione di ciascun prodotto in vendita nel sito	&	Interno\\ UC5	\tabularnewline
	R1F8.4	&	Obbligatorio	&	Aziende e cittadini possono visualizzare la quantità di prodotto selezionato relativa a ciascun prodotto in vendita nel sito	&	Interno\\ UC5	\tabularnewline
	R1F8.4.1	&	Obbligatorio	&	Aziende e cittadini possono modificare la quantità di prodotto selezionata relativa a ciascun prodotto in vendita nel sito	&	Interno\\ UC5.1	\tabularnewline
	R1F9	&	Obbligatorio	&	Aziende e cittadini possono aggiungere prodotti al carrello	&	Interno\\ UC6.1	\tabularnewline
	R1F10	&	Obbligatorio	&	Aziende e cittadini possono visualizzare i prodotti nel carrello	&	Interno\\ UC6.2	\tabularnewline
	R1F10.1	&	Obbligatorio	&	Aziende e cittadini possono visualizzare il nome di ciascun prodotto nel carrello	&	Interno\\ UC6.2	\tabularnewline
	R1F10.2	&	Obbligatorio	&	Aziende e cittadini possono visualizzare la quantità di ciascun prodotto nel carrello	&	Interno\\ UC6.2	\tabularnewline
	R1F10.3	&	Obbligatorio	&	Aziende e cittadini possono visualizzare il prezzo unitario di ciascun prodotto nel carrello	&	Interno\\ UC6.2	\tabularnewline
	R1F10.4	&	Obbligatorio	&	Aziende e cittadini possono visualizzare il prezzo totale dei prodotti nel carrello	&	Interno\\ UC6.2	\tabularnewline
	R1F11	&	Obbligatorio	&	Aziende e cittadini possono modificare la quantità di un bene da acquistare presente nel carrello	&	Interno \\UC6.3	\tabularnewline
	R1F12	&	Obbligatorio	&	Aziende e cittadini possono rimuovere prodotti dal carrello	&	Interno\\ UC6.4	\tabularnewline
	R2F13	&	Desiderabile	&	Aziende e cittadini possono resettare il contenuto del carrello	&	Interno	\tabularnewline
	R1F14	&	Obbligatorio	&	Aziende e cittadini posso acquistare prodotti venduti nella piattaforma	&	Capitolato\\ UC7	\tabularnewline
	R1F14.1	&	Obbligatorio	&	Aziende e cittadini possono effettuare il checkout dei prodotti contenuti nel carrello	&	Interno\\ UC7.1	\tabularnewline
	R1F14.1.1	&	Obbligatorio	&	In fase di checkout viene visualizzato un messaggio di errore se il carrello è vuoto	&	Interno\\ UC7.2	\tabularnewline
	R1F14.2	&	Obbligatorio	&	La fase di acquisto necessita di un indirizzo di spedizione	&	Interno\\ UC7.3	\tabularnewline
	R1F14.2.1	&	Obbligatorio	&	L'acquirente può selezionare come indirizzo di spedizione il proprio indirizzo di residenza	&	Interno\\ UC7.3	\tabularnewline
	R1F14.2.2	&	Obbligatorio	&	L'acquirente può inserire un nuovo indirizzo per la spedizione	&	Interno\\ UC7.3	\tabularnewline
	R1F14.3	&	Obbligatorio	&	Viene visualizzato un messaggio di errore se il pagamento non va a buon fine	&	Interno\\ UC7.5	\tabularnewline
	R1F14.4	&	Obbligatorio	&	Il sistema genera la conferma d'ordine\glo{}, che viene visualizzata nella pagina dedicata dell'acquirente (meccanismo di escrow\glo)	&	Interno	\tabularnewline
	R1F14.5	&	Obbligatorio	&	Il sistema trattiene l'ammontare del pagamento fino all'approvazione/rifiuto della conferma d'ordine\glosp (meccanismo di escrow\glo)	&	Interno	\tabularnewline
	R3F15	&	Opzionale	&	Aziende e cittadini  possono visualizzare la lista degli acquisti effettuati	&	Interno\\ UC14.1	\tabularnewline
	R3F15.1	&	Opzionale	&	La visualizzazione di un acquisto effettuato include la data dell'acquisto	&	Interno\\ UC14.1	\tabularnewline
	R3F15.2	&	Opzionale	&	La visualizzazione di un acquisto effettuato include il numero dell'acquisto	&	Interno\\ UC14.1	\tabularnewline
	R3F15.3	&	Opzionale	&	La visualizzazione di un acquisto effettuato include i prodotti inclusi nell'acquisto	&	Interno\\ UC14.1	\tabularnewline
	R3F15.4	&	Opzionale	&	La visualizzazione di un acquisto effettuato include il totale IVA dell'acquisto	&	Interno\\ UC14.1	\tabularnewline
	R3F15.5	&	Opzionale	&	La visualizzazione di un acquisto effettuato include il  prezzo lordo\glosp dell'acquisto	&	Interno\\ UC14.1	\tabularnewline
	R3F15.6	&	Opzionale	&	La visualizzazione di un acquisto effettuato include l'indirizzo di spedizione dell'acquisto	&	Interno\\ UC14.1	\tabularnewline
	R3F15.7	&	Opzionale	&	Aziende e cittadini possono visualizzare e gestire le conferme d'ordine da approvare	&	Interno \\ UC14.2	\tabularnewline
	R3F15.7.1	&	Opzionale	&	Aziende e cittadini possono rifiutare le conferme d'ordine in attesa di approvazione	&	Interno \\ UC14.5	\tabularnewline
	R3F15.7.1.1	&	Opzionale	&	La somma totale dell'ordine rifutato viene rimborsata all'acquirente	&	Interno \\ UC14.4	\tabularnewline
	R3F15.7.1.2	&	Opzionale	&	Aziende e cittadini possono annullare un'ordine d'acquisto entro 2 giorni dal ricevimento dello stesso	&	Interno	\tabularnewline
	R3F15.7.2	&	Opzionale	&	Aziende e cittadini possono approvare le conferme d'ordine in attesa di approvazione	&	Interno \\ UC14.6	\tabularnewline
	R3F15.8	&	Opzionale	&	Aziende e cittadini ricevono notifiche via email con aggiornamenti sull'ordine	&	Interno	\tabularnewline
	R1F16	&	Obbligatorio	&	Un'azienda può gestire i prodotti in vendita	&	Capitolato\\ UC8	\tabularnewline
	R1F16.1	&	Obbligatorio	&	Un'azienda può mettere in vendita beni e servizi	&	Capitolato\\ UC8.1	\tabularnewline
	R1F16.1.1	&	Obbligatorio	&	L'azienda deve inserire il prezzo netto\glosp di vendita per ogni nuovo prodotto inserito	&	Interno\\ UC8.1.3	\tabularnewline
	R1F16.1.2	&	Obbligatorio	&	L'azienda deve inserire la descrizione per ogni nuovo prodotto inserito	&	Interno\\ UC8.1.2	\tabularnewline
	R1F16.1.3	&	Obbligatorio	&	L'azienda deve inserire l'aliquota IVA per ogni nuovo prodotto inserito	&	Interno\\ UC8.1.4	\tabularnewline
	R1F16.1.4	&	Obbligatorio	&	L'azienda deve inserire il nome per ogni nuovo prodotto inserito	&	Interno\\ UC8.1.1	\tabularnewline
	R1F16.2	&	Obbligatorio	&	Un'azienda può modificare i dati relativi ad un proprio bene o servizio in vendita	&	Capitolato\\ UC8.3	\tabularnewline
	R1F16.2.1	&	Obbligatorio	&	Durante la conferma della modifica di un prodotto, viene visualizzato un errore se non è stato modificato alcun campo	&	Interno\\ UC8.3.6	\tabularnewline
	R1F16.3	&	Obbligatorio	&	Un'azienda può eliminare un proprio bene o servizio in vendita	&	Capitolato\\ UC8.4	\tabularnewline
	R3F17	&	Opzionale	&	Un'azienda può visualizzare la lista delle vendite	&	Interno\\ UC14.3	\tabularnewline
	R3F17.1	&	Opzionale	&	La visualizzazione di una vendita effettuata include la data della vendita	&	Interno\\ UC14.1	\tabularnewline
	R3F17.2	&	Opzionale	&	La visualizzazione di una vendita effettuata include il numero della vendita	&	Interno\\ UC14.1	\tabularnewline
	R3F17.3	&	Opzionale	&	La visualizzazione di una vendita effettuata include i prodotti includi della vendita	&	Interno\\ UC14.1	\tabularnewline
	R3F17.4	&	Opzionale	&	La visualizzazione di una vendita effettuata include il totale IVA della vendita	&	Interno\\ UC14.1	\tabularnewline
	R3F17.5	&	Opzionale	&	La visualizzazione di una vendita effettuata include il prezzo lordo\glosp della vendita	&	Interno\\ UC14.1	\tabularnewline
	R3F17.6	&	Opzionale	&	La visualizzazione di una vendita effettuata include l'indirizzo di spedizione della vendita	&	Interno\\ UC14.1	\tabularnewline
	R3F17.7	&	Opzionale	&	La visualizzazione di una vendita effettuata include la data di approvazione dell'acquisto	&	Interno\\ UC14.3	\tabularnewline
	R3F17.8	&	Opzionale	&	La visualizzazione di una vendita effettuata include il nome dell'acquirente se esso è un cittadino	&	Interno\\ UC14.3	\tabularnewline
	R3F17.9	&	Opzionale	&	La visualizzazione di una vendita effettuata include il cognome dell'acquirente se esso è un cittadino	&	Interno\\ UC14.3	\tabularnewline
	R3F17.10	&	Opzionale	&	La visualizzazione di una vendita effettuata include il nome dell'azienda-cliente se essa è l'acquirente	&	Interno\\ UC14.3	\tabularnewline
	R3F17.11	&	Opzionale	&	La visualizzazione di una vendita effettuata include la partita IVA dell'azienda-cliente se essa è l'acquirente	&	Interno\\ UC14.3	\tabularnewline
	R1F18	&	Obbligatorio	&	Un'azienda riceve il pagamento a seguito della conferma d'ordine\glosp{} da parte di un acquirente (meccanismo di escrow\glo)	&	Interno \\ UC14.5	\tabularnewline
	R1F19	&	Obbligatorio	&	Un'azienda può gestire le transazioni riguardanti l'IVA	&	Capitolato \\ UC15	\tabularnewline
	R1F19.1	&	Obbligatorio	&	Un'azienda può visualizzare il saldo IVA relativo al trimestre corrente	&	Interno \\ UC15.3	\tabularnewline
	R1F19.2	&	Obbligatorio	&	Un'azienda può visualizzare il saldo IVA relativo ad un trimestre concluso selezionato	&	Interno \\ UC15.4	\tabularnewline
	R1F19.3	&	Obbligatorio	&	Un'azienda può visualizzare, per ogni saldo IVA, la lista delle fatture relative al trimestre selezionato (vista generale delle fatture) ed il valore del saldo stesso	&	Interno \\ UC15.2	\tabularnewline
	R1F19.3.1	&	Obbligatorio	&	La vista generale della fattura include il numero identificativo della fattura	&	Interno \\ UC15.2	\tabularnewline
	R1F19.3.2	&	Obbligatorio	&	La vista generale della fattura include la data della fattura	&	Interno \\ UC15.2	\tabularnewline
	R1F19.3.3	&	Obbligatorio	&	La vista generale della fattura include il nome dell'azienda emittente (inteso come ragione sociale)	&	Interno \\ UC15.2	\tabularnewline
	R1F19.3.4	&	Obbligatorio	&	La vista generale della fattura include informazioni relative all'acquirente	&	Interno \\ UC15.2	\tabularnewline
	R1F19.3.4.1	&	Obbligatorio	&	La vista generale della fattura include nome e cognome dell'acquirente nel caso questo sia un cittadino	&	Interno \\ UC15.2	\tabularnewline
	R1F19.3.4.2	&	Obbligatorio	&	La vista generale della fattura include nome dell'azienda acquirente nel caso sia appunto un'azienda	&	Interno \\ UC15.2	\tabularnewline
	R1F19.3.5	&	Obbligatorio	&	La vista generale della fattura include l'importo totale dell'ordine	&	Interno \\ UC15.2	\tabularnewline
	R1F19.3.6	&	Obbligatorio	&	La vista generale della fattura include l'IVA a credito/debito derivato dalla transazione	&	Interno \\ UC15.2	\tabularnewline
	R1F19.4	&	Obbligatorio	&	Un'azienda può effettuare il versamento IVA di un trimestre nel quale il saldo IVA risulti a debito	&	Capitolato \\ UC15.6	\tabularnewline
	R1F19.5	&	Obbligatorio	&	Un'azienda può effettuare la dilazione del versamento IVA di un trimestre nel quale il saldo IVA risulti a debito	&	Verbale 2019-01-04 VE\_3.2 \\ UC15.7	\tabularnewline
	R1F19.6	&	Obbligatorio	&	Un'azienda può scaricare in formato PDF la lista dei movimenti IVA dell'attuale trimestre	&	Capitolato\\ UC15.5	\tabularnewline
	R1F19.7	&	Obbligatorio	&	Un'azienda può visualizzare in dettaglio una particolare fattura	&	Interno\\ UC15.8	\tabularnewline
	R1F19.7.1	&	Obbligatorio	&	La visializzazione di una fattura include la data della fattura	&	Interno\\ UC15.8	\tabularnewline
	R1F19.7.2	&	Obbligatorio	&	La visializzazione di una fattura include il numero identificativo della fattura	&	Interno\\ UC15.8	\tabularnewline
	R1F19.7.3	&	Obbligatorio	&	La visializzazione di una fattura include la data dell'ordine relativo alla fattura	&	Interno\\ UC15.8	\tabularnewline
	R1F19.7.4	&	Obbligatorio	&	La visializzazione di una fattura include il numero identificativo dell'ordine relativo alla fattura	&	Interno\\ UC15.8	\tabularnewline
	R1F19.7.5	&	Obbligatorio	&	La visializzazione di una fattura include la visualizzazione dei prodotti	&	Interno\\ UC15.8	\tabularnewline
	R1F19.7.5.1	&	Obbligatorio	&	Per ogni prodotto elencato in una fattura è visualizzato il nome	&	Interno\\ UC15.8	\tabularnewline
	R1F19.7.5.2	&	Obbligatorio	&	Per ogni prodotto elencato in una fattura è visualizzata la descizione	&	Interno\\ UC15.8	\tabularnewline
	R1F19.7.5.3	&	Obbligatorio	&	Per ogni prodotto elencato in una fattura è visualizzata la quantità	&	Interno\\ UC15.8	\tabularnewline
	R1F19.7.5.4	&	Obbligatorio	&	Per ogni prodotto elencato in una fattura è visualizzato il prezzo netto\glo	&	Interno\\ UC15.8	\tabularnewline
	R1F19.7.5.5	&	Obbligatorio	&	Per ogni prodotto elencato in una fattura è visualizzata l'aliquota IVA	&	Interno\\ UC15.8	\tabularnewline
	R1F19.7.5.6	&	Obbligatorio	&	Per ogni prodotto elencato in una fattura è visualizzato il prezzo lordo\glo	&	Interno\\ UC15.8	\tabularnewline
	R1F19.7.6	&	Obbligatorio	&	La visializzazione di una fattura include l'importo totale IVA della fattura	&	Interno\\ UC15.8	\tabularnewline
	R1F19.7.7	&	Obbligatorio	&	La visializzazione di una fattura include l'importo totale della fattura	&	Interno\\ UC15.8	\tabularnewline
	R1F19.7.8	&	Obbligatorio	&	La visializzazione di una fattura include il nome dell'azienda emittente	&	Interno\\ UC15.8	\tabularnewline
	R1F19.7.9	&	Obbligatorio	&	La visializzazione di una fattura include la partita IVA dell'azienda emittente	&	Interno\\ UC15.8	\tabularnewline
	R1F19.7.10	&	Obbligatorio	&	La visializzazione di una fattura include il nome dell'acquirente se esso è un cittadino	&	Interno\\ UC15.8	\tabularnewline
	R1F19.7.11	&	Obbligatorio	&	La visializzazione di una fattura include il cognome dell'acquirente se esso è un cittadino	&	Interno\\ UC15.8	\tabularnewline
	R1F19.7.12	&	Obbligatorio	&	La visualizzazione di una fattura include la ragione sociale (nome) se l'acquirente è un'azienda 	&	Interno\\ UC15.8	\tabularnewline
	R1F19.7.13	&	Obbligatorio	&	La visializzazione di una fattura include la partita IVA dell'azienda-cliente se essa è l'acquirente	&	Interno\\ UC15.8	\tabularnewline
	R1F19.7.14	&	Obbligatorio	&	La visializzazione di una fattura include l'indirizzo di spedizione dell'ordine	&	Interno\\ UC15.8	\tabularnewline
	R1F19.8	&	Obbligatorio	&	Un'azienda può scaricare in formato PDF la lista dei movimenti IVA per un trimestre concluso	&	Capitolato\\ Interno	\tabularnewline
	R3F19.9	&	Opzionale	&	L'azienda può scaricare in formato PDF la fattura relativa ad un acquisto	&	Interno	\tabularnewline
	
	
	
\end{longtable}


\pagebreak
\subsection{Requisiti di qualità}

\rowcolors{2}{pari}{dispari}
\LTcapwidth=\linewidth
\begin{longtable}{ >{\centering}p{0.10\textwidth} >{\centering}p{0.25\textwidth}
		>{\raggedright}p{0.35\textwidth} >{\centering}p{0.14\textwidth}}
	\caption{Tabella dei requisiti di qualità}\\
	\rowcolorhead 
	\textbf{\color{white}Requisito} 
	& \textbf{\color{white}Classificazione} 
	& \centering\textbf{\color{white}Descrizione}
	& \textbf{\color{white}Fonti} 
	\endfirsthead
	\rowcolor{white}\caption[]{(continua)}\\
	\rowcolorhead 
	\textbf{\color{white}Requisito} 
	& \textbf{\color{white}Classificazione} 
	& \centering\textbf{\color{white}Descrizione}
	& \textbf{\color{white}Fonti} 
	\endhead
	R1Q1	&	Obbligatorio	&	La progettazione e la codifica devono rispettare le norme e le metriche definite nel documento \textit{Piano di qualifica v1.0.0}	&	Interno	\tabularnewline
	R1Q2	&	Obbligatorio	&	L’approccio di scrittura per il codice JavaScript deve rispettare il promise centric approach	&	Capitolato	\tabularnewline
	R1Q2.1	&	Obbligatorio	&	L'utilizzo delle callback\glosp deve essere possibilmente evitato, utilizzato solo se strettamente necessario	&	Capitolato	\tabularnewline
	R1Q3	&	Obbligatorio	&	La guida sullo stile di programmazione Airbnb JavaScript style guide deve essere rispettata nella stesura del codice JavaScript	&	Capitolato	\tabularnewline
	R1Q4	&	Obbligatorio	&	Lo sviluppo del codice JavaScript deve essere supportato dal software di analisi del codice ESLint\glo	&	Capitolato	\tabularnewline
	
\end{longtable}
	


\subsection{Requisiti di vincolo}

	\rowcolors{2}{pari}{dispari}
	
	\begin{longtable}{ >{\centering}p{0.10\textwidth} >{\centering}p{0.25\textwidth}
			>{\raggedright}p{0.35\textwidth} >{\centering}p{0.14\textwidth}}
		\caption{Tabella dei requisiti di vincolo}\\
		\rowcolorhead 
		\textbf{\color{white}Requisito} 
		& \textbf{\color{white}Classificazione} 
		& \centering\textbf{\color{white}Descrizione}
		& \textbf{\color{white}Fonti} 
			\endfirsthead
		\rowcolor{white}\caption[]{(continua)}\\
		\rowcolorhead 
		\textbf{\color{white}Requisito} 
		& \textbf{\color{white}Classificazione} 
		& \centering\textbf{\color{white}Descrizione}
		& \textbf{\color{white}Fonti} 
		\endhead	
		
		
		R1V1	&	Obbligatorio	&	La Dapp\glosp sviluppata dovrà avere il front end\glosp sviluppato attraverso l'uso di tecnologie web	&	Capitolato	\tabularnewline
		R1V1.1	&	Obbligatorio	&	L'applicativo dovrà essere sviluppato utilizzando JavaScript 8 (ES8)	&	Capitolato	\tabularnewline
		R1V1.2	&	Obbligatorio	&	L'applicativo dovrà essere sviluppato utilizzando React\glo.	&	Capitolato	\tabularnewline
		R1V1.3	&	Obbligatorio	&	L'applicativo dovrà essere sviluppato utilizzando Redux\glo.	&	Capitolato	\tabularnewline
		R2V1.4	&	Opzionale	&	L'applicativo dovrà essere implementato affidandosi al boilerplate\glosp Gigacore\glo.	&	Capitolato	\tabularnewline
		R3V1.5	&	Desiderabile	&	\`E desiderabile che l'applicativo venga sviluppato utilizzando SCSS\glo/Sass\glo	&	Capitolato	\tabularnewline
		R1V2	&	Obbligatorio	&	Soldino deve rispettare lo standard EIP-712\glo.	&	Capitolato	\tabularnewline
		R1V3.1	&	Obbligatorio	&	Gli smart contracts\glosp devono essere upgradable	&	Capitolato	\tabularnewline
		R1V3.2	&	Obbligatorio	&	Gli smart contracs\glosp devono essere scritti in linguaggio Solidity\glo	&	Capitolato	\tabularnewline
		R1V4.1	&	Obbligatorio	&	Per connettersi alla rete Ethereum deve essere utilizzato il plugin MetaMask\glo	&	Capitolato	\tabularnewline
		R1V4.2	&	Obbligatorio	&	Per poter validare ed effettuare una transazione che richiede un trasferimento di Cubit\glo, deve essere utilizzato MetaMask\glo	&	Capitolato	\tabularnewline
		R1V5	&	Obbligatorio	&	Il sistema, ed in particolare lo sviluppo degli smart contracts\glo, deve avvenire per mezzo del framework\glosp Truffle\glo	&	Capitolato	\tabularnewline
		R1V6	&	Obbligatorio	&	Lo sviluppo di Soldino deve comprendere l'implementazione di test di unità e di intergrazione	&	Capitolato	\tabularnewline
		R1V7	&	Obbligatorio	&	Il progetto deve utilizzare come minimo 4 ambienti di sviluppo: ambiente di sviluppo locale, ambiente per il testing, ambiente per lo staging\glosp e ambiente di produzione. Il deploy in produzione è apprezzabile ma non obbligatorio	&	Capitolato	\tabularnewline
		R3V7.1	&	Desiderabile	&	Gli ambienti per le fasi di sviluppo locale e testing possono essere fornite da Truffle\glo, utilizzando la rete \textit{testrpc}\glosp ed il web server locale	&	Capitolato	\tabularnewline
		R1V7.2	&	Obbligatorio	&	Durante la fase di staging l'applicazione deve pubblicamente accessibile	&	Capitolato	\tabularnewline
		R3V7.3	&	Desiderabile	&	Come ambiente per la fase di staging è desiderabile l'utilizzo della rete Ropsten\glosp e del web server Surge.sh	&	Capitolato	\tabularnewline
		R1V7.4	&	Opzionale	&	L'ambiente di produzione deve essere composto dall'Ethereum\glosp main network assieme al web server Surge.sh	&	Capitolato	\tabularnewline
		R1V8	&	Obbligatorio	&	L'utente non può effettuare alcuna azione prima di essersi autenticato al sistema con MetaMask\glo	&	Capitolato	\tabularnewline
		R1V9.1	&	Obbligatorio	&	La Dapp\glosp dovrà essere accessibile ed utilizzabile utilizzando i browser Google Chrome, dalla versione 71	&	Verbale 2018-12-07  VE\_1.4	\tabularnewline
		R1V9.2	&	Obbligatorio	&	La Dapp\glosp dovrà essere accessibile ed utilizzabile utilizzando il browser Mozilla FireFox, dalla versione 64	&	Verbale 2018-12-07 VE\_1.4	\tabularnewline
		R1V10	&	Obbligatorio	&	Il token utilizzato nella piattaforma, il \textit{Cubit}\glo, dovrà rispettare lo standard ECR20\glo	&	Capitolato	\tabularnewline
		R1V11	&	Obbligatorio	&	Il sistema deve essere pubblicato con licenza MIT	&	Capitolato	\tabularnewline
		
		
		
		
	\end{longtable}


\subsection{Requisiti prestazionali}
Non sono stati individuati requisiti prestazionali in quanto il progetto sarà costituito da una ÐApp\glo, e come database di supporto verrà utilizzato un database distribuito. L'utilizzo di queste tecnologie prevede l'interazione con una rete Ethereum\glosp nel primo caso, e con una rete peer-to-peer\glosp nel secondo. Le operazioni che avvengono in una rete Ethereum\glosp reale hanno un tempo di soddisfacimento che dipende dal carico della rete nel momento della richiesta. Nel caso queste operazioni vadano a modificare un contratto, sarà inoltre il quantitativo di Ether per unità di Gas\glosp a determinarne la priorità di esecuzione all'interno della rete. Situazione simile per quanto riguarda il database distribuito, dove la velocità di reperimento delle informazioni dipende dal numero di peer attualmente disponibili.
\pagebreak
\subsection{Tracciamento}  
\subsubsection{Fonte - Requisiti}

	\rowcolors{2}{pari}{dispari}
	
	\begin{longtable}{ >{\centering}p{0.5\textwidth}
			>{\centering}p{0.5\textwidth}}
		\caption{Tabella di tracciamento fonte-requisiti}\\
		\rowcolorhead 
		\textbf{\color{white}Fonte}
		& \textbf{\color{white}Requisiti} 
		\tabularnewline 	
		\endfirsthead
		\rowcolor{white}\caption[]{(continua)} \\
		\rowcolorhead 
		\textbf{\color{white}Fonte}
		& \textbf{\color{white}Requisiti} 
		\tabularnewline 
		\endhead
		
		
	
		
		
Capitolato	&	R1F2\\
R1F2.1\\
R1F2.2\\
R1F3.1\\
R1F5\\
R1F6\\
R1F7.2\\
R1F14\\
R1F16\\
R1F16.1\\
R1F16.2\\
R1F16.3\\
R1F19\\
R1F19.4\\
R1F19.5\\
R1F19.6\\
R1F19.8\\
R1Q2\\
R1Q2.1\\
R1Q3\\
R1Q4\\
R1Q5\\
R1Q6.1\\
R1Q6.2\\
R1V1\\
R1V1.1\\
R1V1.2\\
R1V1.3\\
R3V1.4\\
R2V1.5\\
R1V2\\
R1V3.1\\
R1V3.2\\
R1V4.2\\
R1V5\\
R1V6\\
R1V7\\
R2V7.1\\
R1V7.2
\tabularnewline  \rowcolordark & \tabularnewline
Capitolato	& R2V7.3\\

R3V7.4\\
R1V8\\
R1Q6.2\\

R1V10\\
R1V11\\
R1V12\\
R2V14\\
R1V15.1\\
R1V15.2\\
R1V16\\
R1V17\\
	\tabularnewline
Interno	&	 R2F1\\
R1F2.1.1\\
R1F2.1.2\\
R1F2.1.3\\
R1F2.1.4\\
R1F2.1.5\\
R1F2.2.1\\
R1F2.2.2\\
R1F2.2.3\\
R1F2.2.4\\
R1F2.2.5\\
R1F2.3\\
R1F2.4\\
R1F3\\
R1F3.2\\
R1F3.3\\
R1F3.4\\
R1F3.5\\
R1F4\\
R1F5.1\\
R1F6.2\\
R1F6.3\\
R1F6.4\\
R1F7\\
R1F7.1\\
R1F7.1.1\\
R3F7.1.1.1 \\

R1F7.1.2\\
R1F7.2.1\\
R1F7.2.2\\
R1F7.2.3
\tabularnewline  \rowcolorlight & \tabularnewline
Interno	&
R1F7.2.4\\
R1F7.2.5\\
R3F7.2.6\\
R1F7.2.7\\
R1F7.3\\
R1F7.3.1\\
R1F7.3.2\\
R1F7.3.3\\
R1F7.3.4\\
R1F7.3.5\\
R1F8\\
R1F8.1\\
R1F8.2\\
R1F8.3\\
R1F8.4\\
R1F8.4.1\\
R1F9\\
R1F10\\
R1F10.1\\
R1F10.2\\
R1F10.3\\
R1F10.4\\
R1F11\\
R1F12\\
R2F13\\
R1F14.1\\
R1F14.1.1\\
R1F14.2\\
R1F14.2.1\\
R1F14.2.2\\
R1F14.3\\
R1F14.4\\
R1F14.5\\
R3F15\\
R3F15.1\\
R3F15.2\\
R3F15.3\\
R3F15.4\\

R3F15.5\\
R3F15.6\\

R3F15.7\\
R3F15.7.1
\tabularnewline  \rowcolorlight & \tabularnewline
Interno	&
R3F15.7.1.1\\
R3F15.7.1.2\\
R3F15.7.2\\
R2F15.8\\
R1F16.1.1\\
R1F16.1.2\\
R1F16.1.3\\
R1F16.1.4\\
R1F16.1.5\\
R1F16.2.1\\
R3F17\\
R3F17.1\\
R3F17.2\\
R3F17.3\\
R3F17.4\\
R3F17.5\\
R3F17.6\\
R3F17.7\\
R3F17.8\\
R3F17.9\\
R3F17.10\\
R3F17.11\\
R1F18\\
R1F19.1\\
R1F19.2\\
R1F19.3\\
R1F19.3.1\\
R1F19.3.2\\
R1F19.3.3\\
R1F19.3.4\\
R1F19.3.4.1\\
R1F19.3.4.2\\
R1F19.3.5\\
R1F19.3.6\\
R1F19.7\\
R1F19.7.1\\
R1F19.7.2\\
R1F19.7.3\\
R1F19.7.4\\
R1F19.7.5\\
R1F19.7.5.1\\
R1F19.7.5.2
\tabularnewline  \rowcolorlight & \tabularnewline
Interno	& R1F19.7.5.3\\
R1F19.7.5.4\\

 R1F19.7.5.5\\
R1F19.7.5.6\\

R1F19.7.6\\
R1F19.7.7\\
R1F19.7.8\\
R1F19.7.9\\
R1F19.7.10\\
R1F19.7.11\\
R1F19.7.12\\
R1F19.7.13\\
R1F19.7.14\\
R3F19.9\\
R1Q1\\
R1V12

 	\tabularnewline
UC1	&	R2F1	\tabularnewline
UC2.2.1	&	 R1F2.1.1\\
R1F2.2.1 	\tabularnewline
UC2.2.2	&	 R1F2.1.2\\
R1F2.2.2 	\tabularnewline
UC2.2.3	&	 R1F2.1.3\\
R1F2.2.3 	\tabularnewline
UC2.3.1	&	R1F2.1.4	\tabularnewline
UC2.3.2	&	R1F2.1.5	\tabularnewline
UC2.4.1	&	R1F2.2.4	\tabularnewline
UC2.4.2	&	R1F2.2.5	\tabularnewline
UC2.5	&	R1F3.2	\tabularnewline
UC2.6	&	R1F3.3	\tabularnewline
UC2.7	&	R1F2.3	\tabularnewline
UC2.9	&	R1F2.4	\tabularnewline
UC3	&	R1F3	\tabularnewline
UC3.1	&	R1F3.1	\tabularnewline
UC3.2	&	R1F3.4	\tabularnewline
UC3.3	&	R1F3.5	\tabularnewline
UC4	&	R1F4	\tabularnewline
UC5	&	 R1F8\\
R1F8.1\\
R1F8.2\\
R1F8.3\\
R1F8.4 	\tabularnewline
UC5.1	&	R1F8.4.1	\tabularnewline
UC6.1	&	R1F9	\tabularnewline
UC6.2	&	 R1F10\\
R1F10.1\\
R1F10.2\\
R1F10.3\\
R1F10.4 	\tabularnewline
UC6.3	&	R1F11	\tabularnewline
UC6.4	&	R1F12	\tabularnewline
UC7	&	R1F14	\tabularnewline
UC7.1	&	R1F14.1	\tabularnewline
UC7.2	&	R1F14.1.1	\tabularnewline
UC7.3	&	 R1F14.2\\
R1F14.2.1\\
R1F14.2.2 	\tabularnewline
UC7.5	&	R1F14.3	\tabularnewline
UC8	&	R1F16	\tabularnewline
UC8.1	&	R1F16.1	\tabularnewline
UC8.1.1	&	R1F16.1.4	\tabularnewline
UC8.1.2	&	R1F16.1.2	\tabularnewline
UC8.1.3	&	R1F16.1.1	\tabularnewline
UC8.1.4	&	R1F16.1.3	\tabularnewline
UC8.3	&	R1F16.2	\tabularnewline
UC8.3.6	&	R1F16.2.1	\tabularnewline
UC8.4	&	R1F16.3	\tabularnewline
UC9	&	 R1F5\\
R1F5.1 	\tabularnewline
UC10	&	R1F6	\tabularnewline
UC10.1	&	R1F6.1	\tabularnewline
UC10.2	&	R1F6.2	\tabularnewline
UC10.4	&	R1F6.3	\tabularnewline
UC11	&	R1F7	\tabularnewline
UC11.1	&	 R1F7.1\\
R1F7.2\\
R1F7.2.1\\
R1F7.2.2\\
R1F7.2.3\\
R1F7.2.4\\
R1F7.2.5 	\tabularnewline
UC11.1.1	&	R3F7.2.6	\tabularnewline
UC11.2	&	 R1F7.3\\
R1F7.3.1\\
R1F7.3.2\\
R1F7.3.3\\
R1F7.3.4\\
R1F7.3.5 	\tabularnewline
UC12.1	&	R1F7.1.2	\tabularnewline
UC12.2	&	R1F7.1.1	\tabularnewline
UC12.3	&	R3F7.1.1.1	\tabularnewline
UC13	&	R1F7.2.7	\tabularnewline
UC14.1	&	 R3F15\\
R3F15.1\\
R3F15.2\\
R3F15.3\\
R3F15.4\\
R3F15.5\\
R3F15.6\\
R3F17.1\\
R3F17.2\\
R3F17.3\\
R3F17.4\\
R3F17.5\\
R3F17.6 	\tabularnewline
UC14.2	&	R3F15.7	\tabularnewline
UC14.3	&	 R3F17\\
R3F17.7\\
R3F17.8\\
R3F17.9\\
R3F17.10\\
R3F17.11 	\tabularnewline
UC14.4	&	R3F15.7.1.1	\tabularnewline
UC14.5	&	 R3F15.7.1\\
R3F18 	\tabularnewline
UC14.6	&	R3F15.7.2	\tabularnewline
UC15	&	R1F19	\tabularnewline
UC15.2	&	 R1F19.3\\
R1F19.3.1\\
R1F19.3.2\\
R1F19.3.3\\
R1F19.3.4\\
R1F19.3.4.1\\
R1F19.3.4.2\\
R1F19.3.5\\
R1F19.3.6 	\tabularnewline
UC15.3	&	R1F19.1	\tabularnewline
UC15.4	&	R1F19.2	\tabularnewline
UC15.5	&	R1F19.6	\tabularnewline
UC15.6	&	R1F19.4	\tabularnewline
UC15.7	&	R1F19.5	\tabularnewline
UC15.8	&	 R1F19.7\\
R1F19.7.1\\
R1F19.7.2\\
R1F19.7.3\\
R1F19.7.4\\
R1F19.7.5\\
R1F19.7.5.1\\
R1F19.7.5.2\\
R1F19.7.5.3\\
R1F19.7.5.4\\
R1F19.7.5.5\\
R1F19.7.5.6
\tabularnewline  \rowcolorlight & \tabularnewline
UC15.8	&
R1F19.7.6\\
R1F19.7.7\\
R1F19.7.8\\
R1F19.7.9\\
R1F19.7.10\\
R1F19.7.11\\
R1F19.7.12\\
R1F19.7.13\\
R1F19.7.14 	\tabularnewline
Verbale 2018-12-07 VE\_1.4	&	 R1V9.1\\
R1V9.2 	\tabularnewline
Verbale 2018-12-07 VE\_1.5	&	R1Q6	\tabularnewline
Verbale 2018-12-07 VE\_1.2	&	R1V13	\tabularnewline
Verbale 2018-12-07, VE\_1.6	&	R1V15	\tabularnewline
Verbale 2019-01-04  VE\_3.4	&	R1V12	\tabularnewline
	
	
	\end{longtable}

\pagebreak
\subsubsection{Requisito - Fonti}

	\rowcolors{2}{pari}{dispari}
	
\begin{longtable}{ >{\centering}p{0.5\textwidth}
		>{\centering}p{0.5\textwidth}}
	
	\caption{Tabella tracciamento requisito-fonti}\\
	\rowcolorhead 
	\textbf{\color{white}Requisito}
	& \textbf{\color{white}Fonti} 
	\tabularnewline 
	\endfirsthead
	\rowcolor{white}\caption[]{(continua)}\\	
	\rowcolorhead 
	\textbf{\color{white}Requisito}
	& \textbf{\color{white}Fonti} 
	\tabularnewline 
	\endhead
	
R2F1	&	Interno\\ UC1	\tabularnewline
R1F2	&	Capitolato	\tabularnewline
R1F2.1	&	Capitolato	\tabularnewline
R1F2.1.1	&	Interno \\ UC2.2.1	\tabularnewline
R1F2.1.2	&	Interno \\ UC2.2.2	\tabularnewline
R1F2.1.3	&	Interno \\ UC2.2.3	\tabularnewline
R1F2.1.4	&	Interno \\ UC2.3.1	\tabularnewline
R1F2.1.5	&	Interno \\ UC2.3.2	\tabularnewline
R1F2.2	&	Capitolato	\tabularnewline
R1F2.2.1	&	Interno \\ UC2.2.1	\tabularnewline
R1F2.2.2	&	Interno \\ UC2.2.2	\tabularnewline
R1F2.2.3	&	Interno \\ UC2.2.3	\tabularnewline
R1F2.2.4	&	Interno \\ UC2.4.1	\tabularnewline
R1F2.2.5	&	Interno \\ UC2.4.2	\tabularnewline
R1F2.3	&	Interno \\ UC2.7	\tabularnewline
R1F2.4	&	Interno \\ UC2.9	\tabularnewline
R1F3	&	Interno \\ UC3	\tabularnewline
R1F3.1	&	Capitolato\\ UC3.1	\tabularnewline
R1F3.2	&	Interno \\ UC2.5	\tabularnewline
R1F3.3	&	Interno \\ UC2.6	\tabularnewline
R1F3.4	&	Interno \\ UC3.2	\tabularnewline
R1F3.5	&	Interno \\ UC3.3	\tabularnewline
R1F4	&	Interno\\ UC4	\tabularnewline
R1F5	&	Capitolato \\ UC9	\tabularnewline
R1F5.1	&	Interno \\ UC9	\tabularnewline
R1F6	&	Capitolato \\ UC10	\tabularnewline
R1F6.1	&	Interno\\ UC10.1	\tabularnewline
R1F6.2	&	Interno\\ UC10.2	\tabularnewline
R1F6.3	&	Interno\\ UC10.4	\tabularnewline
R1F7	&	Interno \\ UC11	\tabularnewline
R1F7.1	&	Interno \\ UC11.1	\tabularnewline
R1F7.1.1	&	Interno \\ UC12.2	\tabularnewline
R3F7.1.1.1	&	Interno \\ UC12.3	\tabularnewline
R1F7.1.2	&	Interno \\ UC12.1	\tabularnewline
R1F7.2	&	Capitolato \\ UC11.1	\tabularnewline
R1F7.2.1	&	Interno \\ UC11.1	\tabularnewline
R1F7.2.2	&	Interno \\ UC11.1	\tabularnewline
R1F7.2.3	&	Interno \\ UC11.1	\tabularnewline
R1F7.2.4	&	Interno \\ UC11.1	\tabularnewline
R1F7.2.5	&	Interno \\ UC11.1	\tabularnewline
R3F7.2.6	&	Interno \\ UC11.1.1	\tabularnewline
R1F7.2.7	&	Interno  \\ UC13	\tabularnewline
R1F7.3	&	Interno \\ UC11.2	\tabularnewline
R1F7.3.1	&	Interno \\ UC11.2	\tabularnewline
R1F7.3.2	&	Interno \\ UC11.2	\tabularnewline
R1F7.3.3	&	Interno\\ UC11.2	\tabularnewline
R1F7.3.4	&	Interno\\ UC11.2	\tabularnewline
R1F7.3.5	&	Interno\\ UC11.2	\tabularnewline
R1F8	&	Interno\\ UC5	\tabularnewline
R1F8.1	&	Interno\\ UC5	\tabularnewline
R1F8.2	&	Interno\\ UC5	\tabularnewline
R1F8.3	&	Interno\\ UC5	\tabularnewline
R1F8.4	&	Interno\\ UC5	\tabularnewline
R1F8.4.1	&	Interno\\ UC5.1	\tabularnewline
R1F9	&	Interno\\ UC6.1	\tabularnewline
R1F10	&	Interno\\ UC6.2	\tabularnewline
R1F10.1	&	Interno\\ UC6.2	\tabularnewline
R1F10.2	&	Interno\\ UC6.2	\tabularnewline
R1F10.3	&	Interno\\ UC6.2	\tabularnewline
R1F10.4	&	Interno\\ UC6.2	\tabularnewline
R1F11	&	Interno \\UC6.3	\tabularnewline
R1F12	&	Interno\\ UC6.4	\tabularnewline
R2F13	&	Interno	\tabularnewline
R1F14	&	Capitolato\\ UC7	\tabularnewline
R1F14.1	&	Interno\\ UC7.1	\tabularnewline
R1F14.1.1	&	Interno\\ UC7.2	\tabularnewline
R1F14.2	&	Interno\\ UC7.3	\tabularnewline
R1F14.2.1	&	Interno\\ UC7.3	\tabularnewline
R1F14.2.2	&	Interno\\ UC7.3	\tabularnewline
R1F14.3	&	Interno\\ UC7.5	\tabularnewline
R1F14.4	&	Interno	\tabularnewline
R1F14.5	&	Interno	\tabularnewline
R3F15	&	Interno\\ UC14.1	\tabularnewline
R3F15.1	&	Interno\\ UC14.1	\tabularnewline
R3F15.2	&	Interno\\ UC14.1	\tabularnewline
R3F15.3	&	Interno\\ UC14.1	\tabularnewline
R3F15.4	&	Interno\\ UC14.1	\tabularnewline
R3F15.5	&	Interno\\ UC14.1	\tabularnewline
R3F15.6	&	Interno\\ UC14.1	\tabularnewline
R3F15.7	&	Interno \\ UC14.2	\tabularnewline
R3F15.7.1	&	Interno \\ UC14.5	\tabularnewline
R3F15.7.1.1	&	Interno \\ UC14.4	\tabularnewline
R2F15.7.1.2	&	Interno	\tabularnewline
R3F15.7.2	&	Interno \\ UC14.6	\tabularnewline
R2F15.8	&	Interno	\tabularnewline
R1F16	&	Capitolato\\ UC8	\tabularnewline
R1F16.1	&	Capitolato\\ UC8.1	\tabularnewline
R1F16.1.1	&	Interno\\ UC8.1.3	\tabularnewline
R1F16.1.2	&	Interno\\ UC8.1.2	\tabularnewline
R1F16.1.3	&	Interno\\ UC8.1.4	\tabularnewline
R1F16.1.4	&	Interno\\ UC8.1.1	\tabularnewline
R1F16.1.5	&	Interno\\ UC2.9	\tabularnewline
R1F16.2	&	Capitolato\\ UC8.3	\tabularnewline
R1F16.2.1	&	Interno\\ UC8.3.6	\tabularnewline
R1F16.3	&	Capitolato\\ UC8.4	\tabularnewline
R3F17	&	Interno\\ UC14.3	\tabularnewline
R3F17.1	&	Interno\\ UC14.1	\tabularnewline
R3F17.2	&	Interno\\ UC14.1	\tabularnewline
R3F17.3	&	Interno\\ UC14.1	\tabularnewline
R3F17.4	&	Interno\\ UC14.1	\tabularnewline
R3F17.5	&	Interno\\ UC14.1	\tabularnewline
R3F17.6	&	Interno\\ UC14.1	\tabularnewline
R3F17.7	&	Interno\\ UC14.3	\tabularnewline
R3F17.8	&	Interno\\ UC14.3	\tabularnewline
R3F17.9	&	Interno\\ UC14.3	\tabularnewline
R3F17.10	&	Interno\\ UC14.3	\tabularnewline
R3F17.11	&	Interno\\ UC14.3	\tabularnewline
R1F18	&	Interno \\ UC14.5	\tabularnewline
R1F19	&	Capitolato \\ UC15	\tabularnewline
R1F19.1	&	Interno \\ UC15.3	\tabularnewline
R1F19.2	&	Interno \\ UC15.4	\tabularnewline
R1F19.3	&	Interno \\ UC15.2	\tabularnewline
R1F19.3.1	&	Interno \\ UC15.2	\tabularnewline
R1F19.3.2	&	Interno \\ UC15.2	\tabularnewline
R1F19.3.3	&	Interno \\ UC15.2	\tabularnewline
R1F19.3.4	&	Interno \\ UC15.2	\tabularnewline
R1F19.3.4.1	&	Interno \\ UC15.2	\tabularnewline
R1F19.3.4.2	&	Interno \\ UC15.2	\tabularnewline
R1F19.3.5	&	Interno \\ UC15.2	\tabularnewline
R1F19.3.6	&	Interno \\ UC15.2	\tabularnewline
R1F19.4	&	Capitolato \\ UC15.6	\tabularnewline
R1F19.5	&	Capitolato \\ UC15.7	\tabularnewline
R1F19.6	&	Capitolato\\ UC15.5	\tabularnewline
R1F19.7	&	Interno\\ UC15.8	\tabularnewline
R1F19.7.1	&	Interno\\ UC15.8	\tabularnewline
R1F19.7.2	&	Interno\\ UC15.8	\tabularnewline
R1F19.7.3	&	Interno\\ UC15.8	\tabularnewline
R1F19.7.4	&	Interno\\ UC15.8	\tabularnewline
R1F19.7.5	&	Interno\\ UC15.8	\tabularnewline
R1F19.7.5.1	&	Interno\\ UC15.8	\tabularnewline
R1F19.7.5.2	&	Interno\\ UC15.8	\tabularnewline
R1F19.7.5.3	&	Interno\\ UC15.8	\tabularnewline
R1F19.7.5.4	&	Interno\\ UC15.8	\tabularnewline
R1F19.7.5.5	&	Interno\\ UC15.8	\tabularnewline
R1F19.7.5.6	&	Interno\\ UC15.8	\tabularnewline
R1F19.7.6	&	Interno\\ UC15.8	\tabularnewline
R1F19.7.7	&	Interno\\ UC15.8	\tabularnewline
R1F19.7.8	&	Interno\\ UC15.8	\tabularnewline
R1F19.7.9	&	Interno\\ UC15.8	\tabularnewline
R1F19.7.10	&	Interno\\ UC15.8	\tabularnewline
R1F19.7.11	&	Interno\\ UC15.8	\tabularnewline
R1F19.7.12	&	Interno\\ UC15.8	\tabularnewline
R1F19.7.13	&	Interno\\ UC15.8	\tabularnewline
R1F19.7.14	&	Interno\\ UC15.8	\tabularnewline
R1F19.8	&	Capitolato\\ Interno	\tabularnewline
R3F19.9	&	Interno	\tabularnewline
R1Q1	&	Interno	\tabularnewline
R1Q2	&	Capitolato	\tabularnewline
R1Q2.1	&	Capitolato	\tabularnewline
R1Q3	&	Capitolato	\tabularnewline
R1Q4	&	Capitolato	\tabularnewline
R1Q5	&	Capitolato	\tabularnewline
R1Q6	&	Verbale 2018-12-07 VE\_1.5	\tabularnewline
R1Q6.1	&	Capitoltato	\tabularnewline
R1Q6.2	&	Capitoltato	\tabularnewline
R1V1	&	Capitolato	\tabularnewline
R1V1.1	&	Capitolato	\tabularnewline
R1V1.2	&	Capitolato	\tabularnewline
R1V1.3	&	Capitolato	\tabularnewline
R3V1.4	&	Capitolato	\tabularnewline
R2V1.5	&	Capitolato	\tabularnewline
R1V2	&	Capitolato	\tabularnewline
R1V3.1	&	Capitolato	\tabularnewline
R1V3.2	&	Capitolato	\tabularnewline
R1V4.2	&	Capitolato	\tabularnewline
R1V5	&	Capitolato	\tabularnewline
R1V6	&	Capitolato	\tabularnewline
R1V7	&	Capitolato	\tabularnewline
R2V7.1	&	Capitolato	\tabularnewline
R1V7.2	&	Capitolato	\tabularnewline
R2V7.3	&	Capitolato	\tabularnewline
R3V7.4	&	Capitolato	\tabularnewline
R1V8	&	Capitolato	\tabularnewline
R1V9.1	&	Verbale 2018-12-07 VE\_1.4	\tabularnewline
R1Q6.2	&	Capitolato	\tabularnewline
R1V9.2	&	Verbale 2018-12-07 VE\_1.4	\tabularnewline
R1V10	&	Capitolato	\tabularnewline
R1V11	&	Capitolato	\tabularnewline
R1V12	&	Verbale 2019-01-04  VE\_3.4	\tabularnewline
R1V13	&	Verbale 2018-12-07 VE\_1.2	\tabularnewline
R2V14	&	Capitolato	\tabularnewline
R1V15	&	Verbale 2018-12-07, VE\_1.6	\tabularnewline
R1V15.1	&	Capitolato	\tabularnewline
R1V15.2	&	Capitolato	\tabularnewline
R1V16	&	Capitolato	\tabularnewline
R1V17	&	Capitolato	\tabularnewline
	
		
\end{longtable}




\subsection{Considerazioni}
I requisiti potranno subire delle variazioni in futuro, per apportare degli aggiornamenti alle voci presenti o delle migliorie. Nel caso in cui le attività pianificate terminassero prima del previsto, e dovessero avanzare delle ore di lavoro, potranno essere presi in carico nuovi requisiti per aggiungere del valore al prodotto. Dunque eventuali espansioni sono lasciate a momenti futuri. 