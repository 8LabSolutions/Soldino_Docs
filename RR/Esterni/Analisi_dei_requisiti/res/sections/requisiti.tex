
\section{Requisiti} 
I requisiti vengono classificati ed assegnati ad un identificativo  univoco come definito nel documento \textit{Norme di progetto v 1.0.0}.
%\LTXtable{\textwidth}{res/sections/tabella_requisiti.tex}
\renewcommand{\arraystretch}{1.5}
\begin{center}
	\rowcolors{2}{pari}{dispari}
	
	\begin{longtable}{ >{\centering}p{0.20\textwidth} >{\centering}p{0.25\textwidth}
			>{\raggedright}p{0.30\textwidth} >{\centering}p{0.14\textwidth}}
		
		\rowcolorhead 
		\textbf{\color{white}Requisito} 
		& \textbf{\color{white}Classificazione} 
		& \centering\textbf{\color{white}Descrizione}
		& \textbf{\color{white}Fonti} 
		\tabularnewline 	
		
		Prova 
		& Prova
		& Prova 
		& Prova 
		\tabularnewline 
		R1 & Cl & Desc & In
		
		
	\end{longtable}
\end{center}

