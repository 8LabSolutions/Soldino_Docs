\section{Introduzione} 
\subsection{Scopo del documento}
Il presente documento ha lo scopo di descrivere in maniera dettagliata i requisiti individuati per il prodotto. Tali requisiti sono stati identificati dall'analisi del capitolato\glosp C6 ed i successivi incontri con il proponente, \textit{Red Babel}.
\subsection{Scopo del prodotto}
Lo scopo del prodotto è lo sviluppo di una ÐApp\glosp accessibile attraverso interfaccia web con l'ausilio del plug-in\glosp MetaMask\glo, che offra le funzionalità di piattaforma per la compravendita di beni e/o servizi, gestita dal governo\glo. L'interfaccia web verrà sviluppata con React\glosp e Redux\glo, mentre il back end\glosp sarà gestito da un set di smart contracts\glosp scritti utilizzando il linguaggio Solidity\glo, che verranno eseguiti su una rete Ethereum\glo.  Il mezzo di pagamento sarà un token\glosp basato sullo standard ECR20\glo, denominato Cubit\glosp e gestito dal governo\glo.

\subsection{Glossario}
Al fine di evitare possibili ambiguità relative al linguaggio utilizzato nei documenti formali, viene fornito il \textit{Glossario v1.0.0}. In questo documento vengono definiti e descritti tutti i termini con un significato particolare. Per facilitare la lettura, i termini saranno contrassegnati da una 'G' a pedice.
\subsection{Riferimenti}
\subsubsection{Normativi}
\begin{itemize}
	\item \textbf{Norme di Progetto}: \textit{Norme di Progetto v1.0.0};

	\item \textbf{Capitolato d'appalto C6 - Soldino: piattaforma Ethereum per pagamenti IVA}: \\ \url{ https://www.math.unipd.it/~tullio/IS-1/2018/Progetto/C6.pdf};
	\item \textbf{Verbale esterno}: \textit{Verbale esterno 2018-12-07};
	\item \textbf{Verbale esterno}: \textit{Verbale esterno 2018-12-21};
	\item \textbf{Verbale esterno}: \textit{Verbale esterno 2019-01-04};
\end{itemize}
\subsubsection{Informativi}
\begin{itemize}
	\item \textbf{Studio di Fattibilità}: \textit{Studio di Fattibilità v1.0.0};
	\item \textbf{Capitolato d'appalto C6 - Soldino: piattaforma Ethereum per pagamenti IVA}: \\ \url{ https://www.math.unipd.it/~tullio/IS-1/2018/Progetto/C6.pdf};
	\item \textbf{Software Engineering - Ian Sommerville - 10\textsuperscript{th} Edition 2014}
	\subitem - Chapter 4: Requirements engineering;
	\item \textbf{Sito informativo riguardante il funzionamento del plug-in MetaMask}:\\ \textsf{\url{ https://metamask.io/}};
	\item \textbf{Sito informativo riguardante la struttura di una fattura}:\\ \textsf{\url{ https://it.wikipedia.org/wiki/Fattura}};
	\item \textbf{Sito ufficiale della blockchain Ethereum}: \textsf{\url{https://www.ethereum.org/}}. Il sito presenta i concetti alla base del back end\glosp dell'applicativo da sviluppare, come la rete Ethereum\glo, gli smart contracts\glosp ed il meccanismo di escrow\glo, la coniazione di nuovi token\glo. 

\end{itemize}