\section{Introduzione} 
\subsection{Scopo del documento}
Il presente documento ha lo scopo di descrivere in maniera dettagliata i requisiti individuati per il prodotto. Tali requisiti sono stati identificati dall'analisi del capitolato\glosp C6 ed i successivi incontri con il proponente, RedBabel.
\subsection{Scopo del prodotto}
Lo scopo del prodotto è lo sviluppo di una ÐApp\glosp accessibile attraverso interfaccia web con l'ausilio del plug-in MetaMask\glo, che offra le funzionalità di piattaforma per la compravendita di prodotti e/o servizi, gestita dal governo\glo. L'interfaccia web verrà sviluppata con React\glosp e Redux\glo, mentre il back end\glosp sarà gestito da un set di smart contracts\ scritti in linguaggio Solidity\glo, che verranno eseguiti su una rete Ethereum\glo.  Il mezzo di pagamento sarà un token\glosp basato sullo standard ECR20\glo, denominato \textit{Cubit}\glosp e gestito dal governo\glo.

\subsection{Glossario}
Al fine di evitare possibili ambiguità relative al linguaggio utilizzato nei documenti formali, viene fornito il \textit{"Glossario v1.0.0"}. In questo documento vengono definiti e descritti tutti i termini con un significato particolare. Per facilitare la lettura, i termini saranno contrassegnati da una 'G' a pedice.
\subsection{Riferimenti}
\subsubsection{Normativi}
\begin{itemize}
	\item \textbf{Norme di progetto}: \textit{"Norme di progetto v1.0.0"};
	\item \textbf{Capitolato d'appalto C6}: Soldino: piattaforma Ethereum per pagamenti IVA. Il documento può essere consultato all'indirizzo: \textsf{\url{ https://www.math.unipd.it/~tullio/IS-1/2018/Progetto/C6.pdf}}.
\end{itemize}
\subsubsection{Informativi}
\begin{itemize}
	\item \textbf{Studio di Fattibilità}: \textit{"Studio di Fattibilità v1.0.0"};
	\item \textbf{Capitolato d'appalto C6}: Soldino: piattaforma Ethereum per pagamenti IVA. Il documento può essere consultato all'indirizzo: \textsf{\url{ https://www.math.unipd.it/~tullio/IS-1/2018/Progetto/C6.pdf}};
	\item \textbf{Software Engineering - Ian Sommerville - 10\textsuperscript{th} Edition 2014}
	\subitem - Chapter 4: Requirements engineering;
	\item \textbf{830-1998 - IEEE Recommended Practice for Software Requirements Specifications}: \textsf{\url{https://ieeexplore.ieee.org/document/720574}}
	\item ---INSERIRE I VERBALI CONTENENTI LA DISCUSSIONE DEI REQUISITI ---
	
\end{itemize}