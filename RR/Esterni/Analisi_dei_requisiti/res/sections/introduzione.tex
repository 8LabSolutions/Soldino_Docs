\section{Introduzione} 
\subsection{Scopo del documento}
Il presente documento ha lo scopo di descrivere in maniera dettagliata i requisiti individuati per il prodotto. Tali requisiti sono stati identificati dall'analisi del capitolato\glosp C6 ed i successivi incontri con il proponente.
\subsection{Scopo del prodotto}
Il progetto \textit{Soldino} prevede lo sviluppo di una ÐApp\glosp che offra le funzionalità di piattaforma per la compravendita di prodotti e/o servizi gestita dal governo\glo. Lo scopo finale è l'automatizzazione del tracciamento dell'IVA riguardante le transazioni sostenute da cittadini ed aziende. \\ L'interfaccia web deve interagire con l'add-on\glosp "Metamask"\glo, responsabile dell'autenticazione e della conferma delle transazioni. Il back end\glosp sfrutta invece il meccanismo degli Smart Contracts\glo, per validare le transazioni senza doversi affidare a terze parti. Il mezzo di pagamento sarà un token\glosp basato sullo standard ECR20\glo, denominato \textit{Cubit}\glosp e gestito dal governo\glo.
\subsection{Glossario}
Al fine di evitare possibili ambiguità relative al linguaggio utilizzato nei documenti formali, viene fornito il \textit{"Glossario v1.0.0"}. In questo documento vengono definiti e descritti tutti i termini con un significato particolare. Per facilitare la lettura, i termini saranno contrassegnati da una 'G' a pedice.
\subsection{Riferimenti}
\subsubsection{Normativi}
\begin{itemize}
	\item \textbf{Norme di progetto}: \textit{"Norme di progetto v1.0.0"};
	\item \textbf{Capitolato d'appalto C6}: Soldino: piattaforma Ethereum per pagamenti IVA. Il documento può essere consultato all'indirizzo: \textsf{\url{ https://www.math.unipd.it/~tullio/IS-1/2018/Progetto/C6.pdf}}.
\end{itemize}
\subsubsection{Informativi}
\begin{itemize}
	\item \textbf{Studio di Fattibilità}: \textit{"Studio di Fattibilità v1.0.0"};
	\item \textbf{Capitolato d'appalto C6}: Soldino: piattaforma Ethereum per pagamenti IVA. Il documento può essere consultato all'indirizzo: \textsf{\url{ https://www.math.unipd.it/~tullio/IS-1/2018/Progetto/C6.pdf}};
	\item \textbf{Software Engineering - Ian Sommerville - 10\textsuperscript{th} Edition 2016}
	\subitem - Chapter 4: Requirements engineering;
	\item \textbf{830-1998 - IEEE Recommended Practice for Software Requirements Specifications}: \textsf{\url{https://ieeexplore.ieee.org/document/720574}}
	\item ---INSERIRE I VERBALI CONTENENTI LA DISCUSSIONE DEI REQUISITI ---
	
\end{itemize}