\subsection{Requisiti di qualità}

\rowcolors{2}{pari}{dispari}
\LTcapwidth=\linewidth
\begin{longtable}{ >{\centering}p{0.10\textwidth} >{\centering}p{0.25\textwidth}
		>{\raggedright}p{0.35\textwidth} >{\centering}p{0.14\textwidth}}
	\caption{Tabella dei requisiti di qualità}\\
	\rowcolorhead 
	\textbf{\color{white}Requisito} 
	& \textbf{\color{white}Classificazione} 
	& \centering\textbf{\color{white}Descrizione}
	& \textbf{\color{white}Fonti} 
	\endfirsthead
	\rowcolor{white}\caption[]{(continua)}\\
	\rowcolorhead 
	\textbf{\color{white}Requisito} 
	& \textbf{\color{white}Classificazione} 
	& \centering\textbf{\color{white}Descrizione}
	& \textbf{\color{white}Fonti} 
	\endhead
	R1Q1	&	Obbligatorio	&	La progettazione e la codifica devono rispettare le norme e le metriche definite nel documento \textit{Piano di qualifica v1.0.0}	&	Interno	\tabularnewline
	R1Q2	&	Obbligatorio	&	L’approccio di scrittura per il codice JavaScript deve rispettare il promise centric approach	&	Capitolato	\tabularnewline
	R1Q2.1	&	Obbligatorio	&	L'utilizzo delle callback\glosp deve essere possibilmente evitato, utilizzato solo se strettamente necessario	&	Capitolato	\tabularnewline
	R1Q3	&	Obbligatorio	&	La guida sullo stile di programmazione Airbnb JavaScript style guide deve essere rispettata nella stesura del codice JavaScript	&	Capitolato	\tabularnewline
	R1Q4	&	Obbligatorio	&	Lo sviluppo del codice JavaScript deve essere supportato dal software di analisi del codice ESLint\glo	&	Capitolato	\tabularnewline
	
\end{longtable}
	

