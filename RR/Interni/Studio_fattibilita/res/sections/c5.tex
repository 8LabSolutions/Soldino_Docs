\subsection{Capitolato C5 - P2PCS}
\begin{itemize}
\item \textbf{Nome:} P2PCS: piattaforma di peer-to-peer car sharing;
\item \textbf{Proponente:} GaiaGo S.r.l;
\item \textbf{Committente:} Prof. Tullio Vardanega e Prof. Riccardo Cardin.
\end{itemize}
\subsubsection{Descrizione}
Lo scopo di questo progetto è creare un'applicazione per il car sharing 
condominiale che permetta agli utenti che posseggono un'automobile di prestarla 
ai vicini che ne fanno richiesta, consentendo così di evitare che il veicolo 
diventi un peso economico per chi lo possiede ma non lo utilizza spesso.

\subsubsection{Studio del dominio}
\paragraph{Dominio applicativo} \mbox{}\\
Un utente registrato può cercare un'auto libera nella zona interessata, 
prenotarla per quando ne avrà bisogno e andare a ritirare le chiavi. I 
proprietari, invece, possono offrire la propria macchina nei giorni in cui 
segnalano che non è utilizzata.

\paragraph{Dominio tecnologico} \mbox{}\\
\begin{itemize}
	\item \textbf{Node.js}: framework impiegato per la scrittura di applicazioni 
JavaScript dal lato server con un modello asincrono di I/O basato su eventi, 
permettendo un'ottimizzazione di tempi e risorse;
	\item \textbf{Google Cloud}: per la gestione del database;
	\item \textbf{Octalysis}: framework per integrare una strategia di gamification 
all'interno dell'applicazione da sviluppare;
	\item \textbf{Movens}: piattaforma open source progettata per la gestione dei 
servizi nelle smart cities. Essa fornisce gli strumenti per permettere ad ogni 
	utente, nello scenario tecnologico di questa applicazione, di interagire con il 
proprio dispositivo;
	\item \textbf{Android Studio}: framework per lo sviluppo dell'app.
\end{itemize}

\subsubsection{Aspetti Positivi}
I fattori positivi rilevati sono:
\begin{itemize}
	\item Possibilità di imparare e approfondire tecnologie e temi nuovi per il 
gruppo, come Node.js e l'architettura peer-to-peer;
	\item Possibilità di capire come funziona uno stand-up di una metodologia Agile 
all'interno di un'azienda;
	\item Possibilità di comprendere la teoria della gamification e capire come 
utilizzarla all'interno di un'applicazione;
\end{itemize}

\subsubsection{Criticità}
Ad ogni modo, sono stati riscontrati anche degli aspetti negativi:
\begin{itemize}
	\item Recentemente progetti per il car sharing simili, proposti in Italia, si 
sono rivelati fallimentari in quanto il numero di utenti che vi hanno preso 
parte è stato esiguo;

	\item L'argomento di car sharing non ha suscitato un forte interesse 
all'interno del team.
\end{itemize}
\subsubsection{Conclusioni}
Il gruppo ha espresso un giudizio principalmente negativo su questo capitolato 
soprattutto considerando i fallimenti dei altre compagnie in questo campo, 
andando così a prediligere un altro progetto.
