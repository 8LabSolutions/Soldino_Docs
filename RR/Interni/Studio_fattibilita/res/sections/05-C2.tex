\subsection{Capitolato C2 - Colletta}
\subsubsection{Informazioni sul capitolato}
\begin{itemize}
	\item \textbf{Nome}: Colletta;
	\item \textbf{Proponente}: Mivoq s.r.l.;
	\item \textbf{Committente}: Prof. Tullio Vardanega e Prof. Riccardo Cardin.
\end{itemize}

\subsubsection{Descrizione}
Lo scopo del progetto è la raccolta di dati relativi alla classificazione grammaticale di parole nel
contesto in cui vengono utilizzate e la possibilità di rendere facilmente disponibili ed esportabili
tali informazioni. La raccolta dati non deve avvenire in modo esplicito, ma gli utenti devono
trovare un’utilità intrinseca nell’utilizzo della piattaforma. A tal fine, il proponente suggerisce
l’implementazione di un sistema predisposto alla gestione di esercizi di grammatica, come l’ana-
lisi grammaticale. 

 
\subsubsection{Finalità del progetto}
Il prodotto finale sarà una piattaforma multilingua, che offrirà funzionalità diverse ai tre attori:
\begin{itemize}
	\item \textbf{Insegnante}: dovrà poter creare esercizi in modo agevole. Dopo l'inserimento di nuove frasi
	 nel sistema, un tool integrato provvederà automaticamente allo svolgimento
	 dell'esercizio, proponendo una soluzione. L'insegnante dovrà 
	 successivamente correggere e/o validare il risultato proposto, al fine di
	 garantire che i propri allievi ricevano il materiale controllato e corretto;
	\item \textbf{Allievo}:
	dovrà poter svolgere gli esercizi proposti
	dall'insegnante e ricevere una valutazione immediata. La scelta
	dell'esercizio da svolgere avverrà tramite un elenco di frasi proposte o
	inserendo autonomamente una frase nel sistema (la soluzione in quest'ultimo caso verrà fornita da un tool automatico). \`E previsto
    anche uno storico dei progressi nel tempo e un sistema di ricompensa;
	\item \textbf{Sviluppatori}:
	dovranno poter accedere ai dati
	raccolti al fine di utilizzarli nella fase di addestramento di sistemi di apprendimento automatico. Allo sviluppatore dovrà essere fornita più di una versione dell'annotazione di ogni frase, con relativo storico delle modifiche, dalle quali estrarre solo i dati d'interesse.
\end{itemize}
\subsubsection{Tecnologie interessate}
\begin{itemize}
	\item \textbf{Hunpos\glo/Freeling\glo}: sono due software specializzati nel "Part
	of Speech (PoS) tagging"\glo;
	
	\item \textbf{Firebase\glosp Storage}: piattaforma offerta da Google per il salvataggio dei dati relativi agli utenti di un'applicazione. Verrà utilizzata come database per la raccolta dei dati;
	 
	\item \textbf{Web/Mobile programming}: il proponente richiede che la
	piattaforma sia sviluppata sotto forma di pagina web oppure come
	applicazione mobile. L'azienda non ha imposto l'adozione di nessuna
	tecnologia specifica per quanto riguarda questa parte del progetto, quindi
	la scelta spetta agli sviluppatori.		
\end{itemize}
\subsubsection{Aspetti positivi}
\begin{itemize}
	\item Il proponente non ha specificato nessuna tecnologia con la quale
	 sviluppare la piattaforma, viene quindi lasciata agli sviluppatori totale
	 libertà di scelta;
	\item Nel capitolato i requisiti obbligatori, sia espliciti che impliciti,
	 sono in numero inferiore rispetto agli opzionali rendendo maggiormente
	 flessibile la quantità di requisiti da gestire;
	\item La piattaforma Google FireBase potrebbe risultare una conoscenza
	 utile da applicare successivamente nel mondo del lavoro.
	
\end{itemize}
\subsubsection{Criticità e fattori di rischio}
\begin{itemize}
	\item Uno dei requisiti di maggior interesse da parte del
	 proponente consiste nel multilinguismo della piattaforma e il tempo
	 necessario da dedicare allo studio dell'analisi grammaticale di lingue straniere è
	 complesso da quantificare;
	
	\item Nel progetto sono presenti temi già ampiamente studiati nel corso di
	 studi universitario, per cui non si amplierebbe il bagaglio di
	 tecnologie conosciute.
 	
\end{itemize}
\subsubsection{Conclusioni}
Sebbene il gruppo abbia trovato interessante questa proposta, ha deciso
di orientarsi verso progetti rivolti a nuove tecnologie, considerate più
stimolanti e che potranno arricchire maggiormente le abilità di ogni 
componente. 


