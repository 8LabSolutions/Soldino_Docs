% NOTE: Descrizione e dominio applicativo, che differenza?
% La descrizione va bene?
\subsection{Capitolato C3 - G\&B}
\subsubsection{Informazioni generali}
\begin{itemize}
	\item \textbf{Nome}: G\&B: monitoraggio intelligente di processi DevOps\glo{};
	\item \textbf{Proponente}: Zucchetti;
	\item \textbf{Committente}: Prof. Tullio Vardanega e Prof. Riccardo Cardin.
\end{itemize}
\subsubsection{Descrizione}
Il capitolato prevede la costruzione di un software per monitorare un sistema 
DevOps\glo{}, cioè un sistema in cui, a livello aziendale, chi produce il software 
e chi lo usa collaborano strettamente. Per migliorare ulteriormente il servizio 
erogato si richiede un secondo software che permetta di visualizzare, 
analizzare, misurare e controllare i dati forniti dal primo.
\subsubsection{Finalità del progetto}
La struttura del software da realizzare è conforme ai seguenti punti: 
\begin{itemize}
	\item un flusso di dati in input viene associato a una rete Bayesiana\glo{}, 
	composta di nodi contenenti informazioni di probabilità;
	\item la rete riceve il flusso e lo usa per fare dei calcoli, aggiornando
	 quindi le probabilità dei propri nodi;
	\item sia il flusso di dati che la rete sono monitorati in una dashboard\glo{};
	\item l'andamento dei dati determina l'eventuale generazione di allarmi
	 e notifiche.
\end{itemize}
\subsubsection{Tecnologie interessate}
\begin{itemize}
	\item \textbf{Grafana}: software open-source\glo{} che, ricevuti i dati in 
input,
	 consente di raccoglierli in un cruscotto, visualizzarli, analizzarli, 
	 misurarli e controllarli;
	\item \textbf{InfluxDB}: database per l'archiviazione di Time Series\glo{}, generate con 
continuità
	 temporale e atte a essere lette e monitorate costantemente per misurarne 
	 le variazioni;
	\item \textbf{JavaScript}: linguaggio di programmazione richiesto per costruire i plug-in\glo{} di Grafana e per definire la rete di Bayes\glo{} in formato JSON\glo;
	\item \textbf{Rete di Bayes}: rete di nodi che contengono informazioni di 
probabilità; quando un evento significativo si verifica, le probabilità dei nodi si aggiornano di conseguenza.
\end{itemize}
\subsubsection{Aspetti positivi}
\begin{itemize}
	\item L'azienda si presenta come la prima software house italiana e il gruppo 
mostra 
	notevole interesse a collaborare con essa;
	\item La presentazione del problema è chiara e i requisiti sono ben definiti;
\end{itemize}
\subsubsection{Criticità e fattori di rischio}
\begin{itemize}
	\item Scarso numero di nuove tecnologie da apprendere;
	\item L'apprendimento di \textit{Grafana} non ha suscito molto 
	entusiasmo all'interno del team di lavoro.
\end{itemize}
\subsubsection{Conclusioni}
Dopo un'attenta valutazione, il capitolato è stato escluso dalle preferenze. Ci 
sono stati diversi elementi positivi che hanno portato il gruppo a considerarlo,
ma non sono stati abbastanza determinanti da poterlo preferire ad altri.

