\section{Capitolato scelto C6 - Soldino}
\subsection{Informazioni generali}
% ------ Inserire qui testo ------
% Eliminare parte dei controlli e burocrazia (Guardia di Finanza -> controlli a 
% campione)
% 
\begin{itemize}
\item
\textbf{Nome:} Soldino;
\item
\textbf{Proponente:} Red Babel;
\item
\textbf{Committente:} Prof. Tullio Vardanega e Prof. Riccardo Cardin.
\end{itemize}

\subsection{Descrizione}
Il progetto proposto ha lo scopo di sviluppare una piattaforma che permetta a 
tutti gli utenti di tenere sotto controllo i propri investimenti, inserire le 
proprie spese/vendite e registrare le tasse. Nello specifico, \textit{Soldino} 
consente il monitoraggio automatico dell'IVA, ovvero assiste il governo e gli 
utenti nell'esecuzione di operazioni come: liquidazione, versamento, rimborso, 
ecc. Lo scambio di denaro tra utenti e governo avviene per mezzo di una 
criptovaluta basata sullo standard ECR20\glo{}, denominata "Cubit"\glo{}. 


\subsection{Studio del dominio}
\subsubsection{Dominio applicativo}
Sono stati individuati tre attori:
\begin{itemize}	
	\item \textbf{Ente governativo: }\`E in grado di coniare e distribuire la 
	criptovaluta utilizzata in \textit{Soldino}. Inoltre, ha la possibilità di 
accedere e gestire la lista delle aziende iscritte alla piattaforma e, a cadenza 
trimestrale, di controllare se le aziende che sono in uno stato di debito 
abbiano pagato;
	\item \textbf{Proprietario di un'azienda:} Deve poter registrare la propria 
impresa nella piattaforma \textit{Soldino}, controllare i prodotti/servizi 
offerti ed acquistare prodotti da altre aziende. In aggiunta, deve poter gestire
tutto ciò che riguarda l'IVA, ovvero creare un documento di resoconto periodico,
gestirne le ricevute e eventualmente pagare il saldo a cadenza trimestrale;
	\item \textbf{Persona fisica\glo: }Può convertire Euro in Cubit e 
successivamente acquistare i beni e servizi offerti sulla piattaforma. 
\end{itemize}

\subsubsection{Dominio tecnologico}
Per lo sviluppo del lato backend si individuano le seguenti tecnologie:
\begin{itemize}
    \item \textbf{Blockchain:} tecnologia che permette la gestione di un grande 
database distribuito per la gestione di transazioni condivisibili tra più nodi 
di una rete;
	\item \textbf{Ethereum:} è una piattaforma decentralizzata per la creazione
	e la pubblicazione peer-to-peer di smart contracts;
	\item \textbf{Ethereum Virtual Machine (EVM):} è la macchina virtuale che 
esegue gli smart contracts;
	\item \textbf{Smart Contracts:} sono programmi che facilitano, verificano e 
fanno rispettare la negoziazione e la validità dei contratti stipulati tra due 
enti o tra due semplici persone;
	\item \textbf{Solidity:} linguaggio orientato ai contratti, in particolare 
utilizzato per implementare smart contracts.
	\item \textbf{DApps:} è un acronimo per indicare un'applicazione 
decentralizzata, ovvero un software creato attraverso i contratti intelligenti 
nella blockchain di Ethereum;
	\item \textbf{Metamask:} plugin browser che permette di autenticarsi nella rete 
Ethereum e supporta tutti i token tramite il sistema tecnologico ERC20.
\end{itemize}

Per lo sviluppo del lato front-end:
\begin{itemize}
	\item \textbf{Javascript:} linguaggio di scripting client-side. Il proponente 
impone di sottostare alle linee guida esposte nella "Airbnb Javascript style 
guide", e di utilizzare un tool di analisi statica del codice e syntax checking, 
"ESlint"\glo;
	\item \textbf{React:}  libreria open source per JavaScript per la creazione di 
interfacce grafiche e la gestione delle interazioni in ambito web;
	\item \textbf {Redux:} libreria open source Javascript per la gestione degli 
stati di React;
	\item \textbf{SCSS:} estensione del linguaggio CSS, che ne aumenta le 
funzionalità e l'espressività.
\end{itemize}


\subsection{Aspetti positivi}
Gli aspetti positivi emersi dall'analisi di questo capitolato sono:
\begin{itemize}
	\item Con l'avvento dell'obbligo di fatturazione elettronica è sicuramente
	utile a livello curriculare averci lavorato;
	\item Le criptovalute sono state un argomento molto discusso negli ultimi anni 
suscitando nel gruppo un forte interesse nell'approfondire il loro utilizzo;
	\item Il capitolato prevede l'uso di tecnologie largamente utilizzate e quindi 
documentate in modo esaustivo;
	\item Il fatto di utilizzare librerie come React e SCSS permetterà di
	sperimentare come realmente si lavora nelle aziende, distaccandoci così dalle 
tecniche puramente accademiche.
\end{itemize}

\subsection{Criticità}
Invece gli elementi negativi sono:
\begin{itemize}
	\item Il capitolato prevede l'utilizzo di tecnologie nuove, che quindi 
porteranno ad una mole di studio autonomo non indifferente;
	\item L'azienda proponente ha sede all'estero, quindi la comunicazione con i 
referenti sarà meno agevole rispetto ai rapporti con un'azienda che ha sede nel 
territorio nazionale.
\end{itemize}

\subsection{Conclusioni} Il gruppo ha accolto con entusiasmo il capitolato, in 
quanto tra tutte le proposte è quella che copre più tecnologie innovative, ma 
che sono già ampiamente sfruttate e che probabilmente avranno un impatto sempre 
maggiore nel mercato. Con questo progetto tutti i componenti del team avranno 
l'opportunità di studiare un campo dell'informatica che all'università è poco 
trattato, potendo aggiungere al proprio bagaglio curricolare delle voci 
particolarmente interessanti. Inoltre, il gruppo ha apprezzato che RedBabel abbia
proposto un approccio moderno allo sviluppo delle interfacce web, scegliendo 
tecnologie quali React e Redux, sempre più diffuse e richieste anche in ambito lavorativo.\\
L'interesse verso queste nuove tecnologie ha spinto il gruppo 8Lab Solutions ad 
optare per questa scelta nonostante la consapevolezza che, essendo tecnologie 
poco conosciute, il loro studio richiederà un impegno notevole.


