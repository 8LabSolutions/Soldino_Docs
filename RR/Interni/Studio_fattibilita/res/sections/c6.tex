\section{Capitolato scelto C6 - Soldino}
\subsection{Informazioni generali}
% ------ Inserire qui testo ------
% Eliminare parte dei controlli e burocrazia (Guardia di Finanza -> controlli a 
campione)
% 
\begin{itemize}
\item
\textbf{Nome:} Soldino
\item
\textbf{Proponente:} \textit{Red Babel} 
\item
\textbf{Committente:} Prof.Tullio Vardanega e Prof. Riccardo Cardin
\end{itemize}
\subsection{Descrizione}
Attualmente il sistema di pagamento dell'IVA prevede che siano le aziende a 
registrare i loro acquisti/vendite, e che con cadenza trimestrale venga 
calcolato il saldo. Se l'azienda risulta debitrice allora deve versare allo 
stato il rispettivo ammontare, altrimenti ottiene rimborso.
 \\Soldino nasce per automatizzare la gestione di 
questo processo, proponendo una piattaforma di e-commerce controllata dal 
governo, nella quale le aziende e le persone fisiche\glosp possano comprare e/o vendere beni 
e servizi per mezzo di una criptovaluta.

\subsection{Obiettivo finale}
L'obiettivo di Soldino è la creazione di un insieme di ÐApps\glosp per la 
gestione di un e-commerce. Governo ed aziende verranno assistite da Soldino 
nella gestione dell'IVA sottostante le operazioni di compravendita di beni e 
servizi tra aziende e clienti. La business logic deve affidarsi al meccanismo 
degli Smart Contracts\glo, che arricchisce le transazioni con regole 
user-defined, garantendone maggiore controllo e sicurezza. I diversi attori 
potranno accedere alla piattaforma attraverso un'interfaccia web dedicata, 
previa autenticazione con Metamask\glosp (disponibile come add-on per i browser 
come Chrome\glo, Opera\glo, Firefox\glo e dal browser Brave\glo).  Il 
mezzo di pagamento sarà un criptovaluta basata sullo standard ECR20\glo , 
denominata "Cubit"\glo.

\subsection{Studio del dominio}
\subsubsection{Dominio applicativo}
Si individuano tre attori:
\begin{itemize}	
	\item \textbf{Ente governativo: }\`E in grado di coniare e distribuire la 
	criptovaluta utilizzata in Soldino. Potrà inoltre accedere e gestire la lista 
delle aziende iscritte alla piattaforma, e controllare che a cadenza trimestrale 
esse abbiano pagato in caso di stato di debito.
	\item \textbf{Proprietario di un'azienda:} Deve poter registrare la propria 
impresa nell'e-commerce, gestire i prodotti/servizi offerti ed acquistare 
prodotti da altre aziende. Inoltre deve poter gestire tutto ciò che riguarda 
l'IVA, ovvero pagare l'eventuale saldo a cadenza trimestrale (in caso di 
debito), creare un documento di resoconto di periodo e gestire le ricevute IVA.
	\item \textbf{Persona fisica\glo: }Può convertire Euro in Cubit e 
successivamente acquistare i beni e servizi offerti sulla piattaforma. 
\end{itemize}
\subsubsection{Dominio tecnologico}
Per lo sviluppo del lato backend si individuano le seguenti tecnologie:
\begin{itemize}
    \item \textbf{Blockchain: }è un database distribuito e 
    decentralizzato che utilizza una rete peer to peer.
	\item \textbf{Ethereum: }è un'infrastruttura decentralizzata open source di 
computing che
	esegue Smart Contracts. Utilizza la blockchain per sincronizzare e salvare i 
cambiamenti di stato
	del sistema.
	\item \textbf{Ethereum Virtual Machine (EVM): }è la macchina virtuale che 
esegue smart contracts
	\item \textbf{Smart Contracts: } sono programmi che fungono da intermediari tra 
due parti, come ad esempio
	compratore e venditore. [da finire]
	\item \textbf{Solidity: }linguaggio orientato ai contratti. \`E utilizzato per 
implementare smart contracts.
	\item \textbf{ÐApps: }è una web application che si interfaccia con uno o più 
smart contract
	\item \textbf{Metamask:} plugin browser che permette di autenticarsi alla rete 
Ethereum

\end{itemize}
Per lo sviluppo del lato front-end:
\begin{itemize}
	\item \textbf{Javascript:} linguaggio di scripting client-side. Il proponente 
impone di sottostare allo standard esposto nella "Airbnb Javascript style 
guide", e di utilizzare un toll di analisi statica del codice e syntax checking, 
"ESlint";
	\item \textbf{React:}  libreria open source JavaScript per la creazione di 
interfacce grafiche e la gestione delle interazioni in ambito web
	\item \textbf {Redux:} libreria open source Javascript per la gestione degli 
stati di React;
	
	\item \textbf{SCSS:} estensione del linguaggio CSS, che ne aumenta le 
funzionalità ed espressività.
\end{itemize}


\subsection{Aspetti positivi}
\begin{itemize}
	\item Con l'avvento dell'obbligo di fatturazione elettronica è sicuramente
	utile a livello curriculare averci lavorato.
	\item le cripto valute hanno avuto un notevole slancio nell'ultimo periodo.
	\item Il capitolato prevede l'utilizzo di tecnologie, se pur nuove per noi,
	largamente utilizzate e quindi documentate in odo esaustivo.
	\item Il fatto di utilizzare sistemi come React e SCSS ci permetterà di
	sperimetare come realmente si lavora nelle aziende, distaccandoci così dalle
	tecniche puramente accademiche.
\end{itemize}

\subsection{Criticità e fattori di rischio}
\begin{itemize}
	\item Il capitolato prevede l'utilizzo di tecnologie a noi nuove, che quindi 
portaranno ad una mole di studio autonomo non indifferente.
	\item L'azienda proponende ha sede all'estero, quindi la cominicazione con i 
referenti sarà meno agevole rispetto ai rapporti con un'azienda del territorio.
\end{itemize}

\subsection{Conclusioni} Il gruppo ha accolto con entusiasmo il capitolato, in 
quanto tra tutte le proposte è quella che copre più tecnologie innovative, ma 
che sono già ampiamente sfruttate e che probabilmente avranno un impatto sempre 
maggiore nel mercato. Con questo progetto avremo l'opportunità di studiare un 
campo dell'informatica che all'università è poco trattato, potendo aggiungere al 
nostro bagaglio curricolare delle voci particolarmente interessanti. Inoltre il 
gruppo ha apprezzato che RedBabel abbia proposto un approccio moderno allo 
sviluppo delle interfacce web, scegliendo tecnologie quali React e Redux, sempre 
più diffuse e richieste anche in ambito lavorativo.\\
 L'interesse verso queste nuove tecnologie ha spinto il gruppo 8Lab Solutions ad 
optare per questa scelta nonostante la consapevolezza che, essendo tecnologie a 
noi poco conosciute, il loro studio richiederà un impegno notevole. Siamo 
inoltre consci del fatto che la documentazione/risorse disponibili potranno 
essere inferiori a quelle tecnologie già consolidate.
