\subsection{Capitolato C1 - Butterfly}
\subsubsection{Informazioni generali}
% ------ Inserire qui testo ------
\begin{itemize}
\item
\textbf{Nome:} Butterfly;
\item
\textbf{Proponente:} Imola Informatica;
\item
\textbf{Committente:} Prof. Tullio Vardanega e Prof. Riccardo Cardin.
\end{itemize}
\subsubsection{Descrizione}
\textit{Butterfly} mira allo sviluppo di una piattaforma che permetta di
 accentrare, standardizzare, automatizzare e personalizzare le segnalazioni di
 diversi strumenti di versionamento, di continuous integration\glosp e continuous delivery\glo , così da permettere all'utente di interfacciarsi ad un'unica dashboard per la loro gestione.
\subsubsection{Finalità del progetto}
Il prodotto finale, integrando al suo interno le segnalazioni delle diverse 
applicazioni, semplifica e organizza il lavoro in un progetto. L'azienda propone,
per la realizzazione di questa soluzione, l'utilizzo di quattro componenti:
\begin{itemize}
	\item \textbf{Producers}, che hanno la funzionalità di recuperare le segnalazioni dalle applicazioni interessate e di pubblicarle, associandole ad un Topic;
	\item \textbf{Broker}, uno strumento per istanziare e gestire i Topic;
	\item \textbf{Consumers}, che dovranno abbonarsi a dei Topic, recuperarne i messaggi e procedere al loro invio ai destinatari finali;
    \item \textbf{Componente custom specifico}, applicativo che permetta, previa lettura di metadati relativi agli utenti, di decidere quali tra questi sono i più appropriati a ricevere la notifica.
\end{itemize}
\subsubsection{Tecnologie interessate}
Per lo sviluppo dei componenti applicativi, l'azienda proponente consiglia:
\begin{itemize}
	\item uno dei linguaggi di programmazione tra \textbf{Java},
	\textbf{Python}\glosp o \textbf{Node.js}\glo;
	\item il sistema open source \textbf{Apache Kafka}\glosp per la gestione delle
	operazioni tra i vari client, da utilizzare come Broker.
	\item \textbf{Docker}\glosp per creare i container relativi alle diverse componenti;
	\item \textbf{API Redmine}\glo,  \textbf{GitLab}\glo,  \textbf{SonarQube}\glo,  \textbf{Telegram}\glo, \textbf{Slack}\glo, per potersi interfacciare con queste applicazioni.
\end{itemize}
Inoltre il proponente richiede di:
\begin{itemize}
	\item fornire delle \textbf{API Rest}\glosp per ognuna delle componenti utilizzate nell'applicazione; 
	\item utilizzare test unitari e d'integrazione per ogni componente; 
	\item rispettare i 12 fattori presenti in "The Twelve-Factor App" nelle applicazioni sviluppate.
\end{itemize}
\subsubsection{Aspetti positivi}

\begin{itemize}
	\item Le tecnologie proposte hanno larga diffusione nel mondo lavorativo ed
	 approfondire la conoscenza su di esse è un aspetto apprezzato dal gruppo;
	\item Java è materia di studio nel nostro corso di laurea, per cui il
	 capitolato offre la possibilità di migliorare la padronanza di questo
	 linguaggio.
\end{itemize}
\subsubsection{Criticità e fattori di rischio}

\begin{itemize}
	\item Lo sviluppo del componente Producer permetterebbe solamente l'apprendimento di aspetti marginali delle tecnologie coinvolte.
	\item Il lavoro per la raccolta dati appare ripetitivo 
	 e le API da utilizzare sembrano altamente specifiche per il progetto. Probabilmente queste conoscenze acquisite saranno poco spendibili nel futuro, specie se comparate alle offerte di altri capitolati;
	\item L'interesse da parte del gruppo di lavoro per questo capitolato si è dimostrato scarso.
\end{itemize}

\subsubsection{Conclusioni}
Lo scopo del capitolato non è risultato molto stimolante, in quanto lo
sviluppo di alcune componenti sembra caratterizzato da attività ripetitive.
Inoltre, il dover apprendere tecnologie per le quali è richiesta solamente 
l'integrazione di un sottoinsieme di funzionalità, ha demotivato il gruppo nella scelta di questo progetto.
