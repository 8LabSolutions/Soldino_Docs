\subsection{Capitolato C1 - Butterfly}
\subsubsection{Informazioni generali}
\begin{itemize}
\item
\textbf{Nome:} Butterfly;
\item
\textbf{Proponente:} Imola Informatica;
\item
\textbf{Committente:} Prof. Tullio Vardanega e Prof. Riccardo Cardin.
\end{itemize}
\subsubsection{Descrizione}
\textit{Butterfly} mira allo sviluppo di una piattaforma che permette di
 accentrare, standardizzare, automatizzare e personalizzare le segnalazioni di
 diversi strumenti di versionamento, di continuous integration\glosp e continuous delivery\glo , così da permettere all'utente di interfacciarsi ad un'unica dashboard per la gestione di queste.
\subsubsection{Finalità del progetto}
Il prodotto finale, integrando al suo interno le segnalazioni delle diverse 
applicazioni, semplifica e organizza il lavoro dell'utente. L'azienda propone,
per la realizzazione di questa soluzione, l'utilizzo di quattro componenti:
\begin{itemize}
	\item \textbf{Producers}, che hanno la funzionalità di recuperare le
	segnalazioni e mostrarle come messaggi nei rispettivi Topic;
	\item \textbf{Broker}, ovvero uno strumento per istanziare e gestire i
	 Topic;
	\item \textbf{Consumers}, che hanno il compito di iscriversi a diversi
	Topic specifici così da reindirizzare i messaggi verso gli utenti finali;
    \item \textbf{Componente custom specifico}, inteso come un componente da
    implementare per l'azienda che permetta di indirizzare la notifica alla
    persona più idonea.
\end{itemize}
\subsubsection{Tecnologie interessate}
Per lo sviluppo dei componenti applicativi, l'azienda proponente consiglia:
\begin{itemize}
	\item uno dei linguaggi di programmazione tra \textbf{Java}\glo,
	\textbf{Python}\glosp o \textbf{Node.js}\glo;
	\item il sistema open source \textbf{Apache Kafka}\glosp per la gestione delle
	operazioni tra i vari client, da utilizzare come Broker.
\end{itemize}
I requisiti obbligatori, invece, sono:
\begin{itemize}
	\item rispettare i 12 fattori presenti in "The Twelve-Factor App" nelle 
	applicazioni sviluppate;
	\item utilizzare \textbf{Docker}\glosp come container per l'istanziazione dei componenti;
	\item esporre le \textbf{API Rest}\glosp dei componenti per l'utilizzo dell'applicazione; 
	\item utilizzare test unitari e d'integrazione per ogni componente 
	realizzato.
\end{itemize}
\subsubsection{Aspetti positivi}
\begin{itemize}
	\item Le tecnologie proposte hanno larga diffusione nel mondo lavorativo ed
	 approfondire la conoscenza su di esse è un aspetto apprezzato dal gruppo.
	\item Java è materia di studio nel nostro corso di laurea, per cui il
	 capitolato offre la possibilità di migliorare la padronanza di questo
	 linguaggio.
\end{itemize}
\subsubsection{Fattori di rischio}
\begin{itemize}
	\item Lo sviluppo del componente Producer permetterebbe solamente l'apprendimento di aspetti marginali delle tecnologie coinvolte.
	\item Il lavoro per la raccolta dati appare ripetitivo 
	 e le API da utilizzare sembrano altamente specifiche per il progetto, senza l'opportunità di acquisire esperienza spendibile in futuro.
	\item L'interesse da parte del gruppo di lavoro per questo capitolato si è 
dimostrato scarso.
	
\end{itemize}

\subsubsection{Conclusioni}
Lo scopo del capitolato non è risultato molto stimolante, in quanto lo
sviluppo di alcune componenti sembra caratterizzato da attività ripetitive.
Inoltre, il dover apprendere tecnologie per le quali è richiesta solamente 
l'integrazione di un sottoinsieme di funzionalità, ha demotivato il gruppo nella scelta di questo progetto.
