\section{Capitolato scelto C6 - Soldino}
\subsection{Informazioni generali}
% ------ Inserire qui testo ------
% Eliminare parte dei controlli e burocrazia (Guardia di Finanza -> controlli a 
% campione)
% 
\begin{itemize}
\item \textbf{Nome:} Soldino;
\item \textbf{Proponente:} Red Babel;
\item \textbf{Committente:} Prof. Tullio Vardanega e Prof. Riccardo Cardin.
\end{itemize}

\subsection{Descrizione}
Soldino prevede lo sviluppo di una piattaforma di e-commerce con lo scopo di automatizzare la gestione dell'IVA tra governo, aziende e cittadini. Cittadini ed aziende devono poter acquistare e vendere prodotti e/o servizi. Trimestralmente avviene il saldo dell'IVA: se l'azienda è in stato di debito allora deve poter versare la relativa somma, e viceversa lo stato deve risarcire le aziende in stato di credito. Inoltre il governo deve poter accedere a tutte le informazioni riguardanti le aziende iscritte alla piattaforma.

\subsection{Finalità del progetto}
L’obiettivo finale è la creazione di un e-commerce con il back end che si appoggia alla rete Ethereum\glo. La business logic deve affidarsi al meccanismo degli Smart Contracts\glo, gestiti attraverso delle ÐApps\glo. I diversi
attori potranno accedere alla piattaforma attraverso un’interfaccia web dedicata, sviluppata per interagire con l'add-on per browser "Metamask"\glosp (su Chrome e Firefox). Il mezzo di pagamento sarà un token basato sullo standard ECR20\glo, denominato "Cubit"\glo.



\subsection{Tecnologie interessate}
\begin{itemize}
  
	\item \textbf{Ethereum\glo:} blockchain\glosp che verrà utilizzata per approvare e archiviare le transazioni che avvengono sulla piattaforma;
	\item \textbf{Ethereum Virtual Machine\glosp (EVM):} macchina virtuale distribuita sulla rete Ethereum\glo, verrà utilizzata per eseguire le ÐApps\glo; 
	\item \textbf{Smart Contracts\glo:} utilizzati per amministrare i contratti e/o transazioni trai vari attori;
	\item \textbf{Solidity:} linguaggio utilizzato per programmare gli smart-contracts;
	\item \textbf{ÐApps\glo:} è un acronimo per indicare un'applicazione decentralizzata, ovvero un software creato attraverso i contratti intelligenti nella blockchain di Ethereum\glo;
	\item \textbf{Metamask\glo:} add-on che permette agli utenti di gestire i propri account/wallet ed interagire con la rete Ethereum\glo. Verrà utilizzato per verificare l'identità degli utenti ed approvare le transazioni; 
	\item \textbf{Web3}: API JavaScript\glosp per effettuare chiamate remote a un nodo Ethereum\glo;
	\item \textbf{Truffle}: framework per lo sviluppo di smart contracts\glosp su rete Ethereum\glo, verrà utilizzato per lo sviluppo iniziale e per il testing;
	\item \textbf {Ropsten}: rete che esegue gli stessi protocolli di Ethereum, utilizzata per la fase di staging;
	\item \textbf{Javascript, React\glo, Redux\glo, SCSS\glo, HTML}: per l'implementazione della parte di front end. Il proponente impone di sottostare alle linee guida esposte nella "Airbnb Javascript style guide".
	\item \textbf{ESlint\glo}: tool di analisi statica del codice e syntax checking.

\end{itemize}

\subsection{Aspetti positivi}
Gli aspetti positivi emersi dall'analisi di questo capitolato sono:
\begin{itemize}
	\item Con l'avvento dell'obbligo di fatturazione elettronica è sicuramente
	utile a livello curriculare averci lavorato;
	\item Le criptovalute sono state un argomento molto discusso negli ultimi anni 
suscitando nel gruppo un forte interesse nell'approfondire il loro utilizzo;
	\item Il capitolato prevede l'uso di tecnologie largamente utilizzate e quindi 
documentate in modo esaustivo;
	\item Il fatto di utilizzare librerie come React e SCSS permetterà di
	sperimentare come realmente si lavora nelle aziende, distaccandoci così dalle 
tecniche puramente accademiche.
\end{itemize}

\subsection{Criticità  e fattori di rischio}
Invece gli elementi negativi sono:
\begin{itemize}
	\item Il capitolato prevede l'utilizzo di tecnologie nuove, che quindi 
porteranno ad una mole di studio autonomo non indifferente;
	\item L'azienda proponente ha sede all'estero, quindi la comunicazione con i 
referenti sarà meno agevole rispetto ai rapporti con un'azienda che ha sede nel 
territorio nazionale.
\end{itemize}

\subsection{Conclusioni} Il gruppo ha accolto con entusiasmo il capitolato, in 
quanto tra tutte le proposte è quella che copre più tecnologie innovative, ma 
che sono già ampiamente sfruttate e che probabilmente avranno un impatto sempre 
maggiore nel mercato. Con questo progetto tutti i componenti del team avranno 
l'opportunità di studiare un campo dell'informatica che all'università è poco 
trattato, potendo aggiungere al proprio bagaglio curricolare delle voci 
particolarmente interessanti. Inoltre, il gruppo ha apprezzato che RedBabel abbia
proposto un approccio moderno allo sviluppo delle interfacce web, scegliendo 
tecnologie quali React e SCSS, sempre più diffuse e richieste anche in ambito lavorativo.\\
L'interesse verso queste nuove tecnologie ha spinto il gruppo 8Lab Solutions ad 
optare per questa scelta nonostante la consapevolezza che, essendo tecnologie 
poco conosciute, il loro studio richiederà un impegno notevole.


