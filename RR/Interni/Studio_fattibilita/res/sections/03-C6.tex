\section{Capitolato scelto C6 - Soldino}
\subsection{Informazioni generali}
% ------ Inserire qui testo ------
% Eliminare parte dei controlli e burocrazia (Guardia di Finanza -> controlli a 
% campione)
% 
\begin{itemize}
\item \textbf{Nome}: Soldino: piattaforma Ethereum per pagamenti IVA;
\item \textbf{Proponente}: Red Babel;
\item \textbf{Committente}: Prof. Tullio Vardanega e Prof. Riccardo Cardin.
\end{itemize}

\subsection{Descrizione}
\textit{Soldino} prevede lo sviluppo di una piattaforma gestita dal governo\glo. I proprietari delle aziende potranno registrare la propria attività commerciale e vendere/acquistare beni e/o servizi, oltre a poter ricevere e registrare le tasse. Il governo\glosp potrà coniare e distribuire la moneta utilizzata in queste transazioni. I cittadini potranno acquistare i beni utilizzando tale valuta.

\subsection{Finalità del progetto}
L'obiettivo finale è il tracciamento automatico dell'IVA riguardante le operazioni che avvengono nella suddetta piattaforma. Per raggiungere questo risultato è richiesta la creazione di un'interfaccia web sviluppata per interagire con l'add-on\glosp per browser "MetaMask"\glosp (su Chrome e Firefox). La business logic deve affidarsi al meccanismo degli Smart Contracts\glo, gestiti attraverso delle ÐApps\glo, che verranno eseguite sulla EVM\glo. Il mezzo di pagamento sarà un token\glosp basato sullo standard ECR20\glo, denominato \textit{Cubit}\glo. 


\subsection{Tecnologie interessate}
\begin{itemize}
  
	\item \textbf{Ethereum\glo}: nel progetto useremo questa blockchain\glosp per approvare ed archiviare le transazioni che avvengono sulla piattaforma;
	\item \textbf{Ethereum Virtual Machine\glosp (EVM)}: macchina virtuale distribuita sulla rete Ethereum\glo, verrà utilizzata per eseguire le ÐApps\glo; 
	\item \textbf{Smart Contracts\glo}: utilizzati per amministrare i contratti e/o transazioni trai vari attori;
	\item \textbf{Solidity}: linguaggio utilizzato per programmare gli smart-contracts\glo;
	\item \textbf{ÐApps\glo}: è un acronimo per indicare un'applicazione decentralizzata, ovvero un software creato attraverso i contratti intelligenti nella blockchain di Ethereum\glo;
	\item \textbf{MetaMask\glo}: add-on\glosp che permette agli utenti di gestire i propri account/wallet ed interagire con la rete Ethereum\glo. Verrà utilizzato per verificare l'identità degli utenti ed approvare le transazioni; 
	\item \textbf{Web3}: API JavaScript per effettuare chiamate remote a un nodo Ethereum\glo;
	\item \textbf{Truffle}: framework\glosp per lo sviluppo di smart contracts\glosp su rete Ethereum\glo, verrà utilizzato per lo sviluppo iniziale e per il testing;
	\item \textbf {Ropsten}: rete che esegue gli stessi protocolli di Ethereum\glo, utilizzata per la fase di staging\glo;
	\item \textbf{Javascript, React\glo, Redux\glo, SCSS\glo, HTML}: linguaggi e framework\glosp per l'implementazione del front end\glo. Il proponente impone di sottostare alle linee guida esposte nella "Airbnb Javascript style guide"\footnote{https://github.com/airbnb/javascript};
	\item \textbf{ESlint\glo}: tool di analisi statica del codice e syntax checking.

\end{itemize}

\subsection{Aspetti positivi}
\begin{itemize}
	\item Con l'avvento dell'obbligo di fatturazione elettronica è sicuramente
	utile a livello curriculare aver lavorato ad argomenti inerenti;
	\item Le criptovalute sono state un argomento molto discusso negli ultimi anni 
suscitando nel gruppo un forte interesse nell'approfondire il loro utilizzo;
	
	\item Il fatto di utilizzare librerie come React e SCSS permetterà di
	apprendere nuove conoscenze fortemente richieste nel mondo del lavoro.
\end{itemize}


\subsection{Criticità  e fattori di rischio}

\begin{itemize}
	\item Il capitolato\glosp prevede l'utilizzo di tecnologie nuove, che quindi 
porteranno ad una mole di studio autonomo non indifferente;
	\item L'azienda proponente ha sede all'estero, quindi la comunicazione con i 
referenti sarà meno agevole rispetto ai rapporti con un'azienda che ha sede nel 
territorio nazionale.
\end{itemize}

\subsection{Conclusioni} Il gruppo ha accolto con entusiasmo il capitolato\glo, in 
quanto tra tutte le proposte è quella che copre più tecnologie innovative, ma 
che sono già ampiamente sfruttate e che probabilmente avranno un impatto sempre 
maggiore nel mercato. Con questo progetto tutti i componenti del team avranno 
l'opportunità di studiare un campo dell'informatica che all'università è poco 
trattato, potendo aggiungere al proprio bagaglio curricolare delle voci 
particolarmente interessanti. Inoltre, il gruppo ha apprezzato che RedBabel abbia
proposto un approccio moderno allo sviluppo delle interfacce web, scegliendo 
tecnologie quali React e SCSS, sempre più diffuse e richieste anche in ambito lavorativo.\\
L'interesse verso queste nuove tecnologie ha spinto il gruppo 8Lab Solutions ad 
optare per questa scelta nonostante la consapevolezza che, essendo tecnologie 
poco conosciute, il loro studio richiederà un impegno notevole.


