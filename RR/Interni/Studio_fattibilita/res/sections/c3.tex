% NOTE: Descrizione e dominio applicativo, che differenza?
% La descrizione va bene?
\subsection{Capitolato C3 - G\&B}
\subsubsection{Informazioni sul capitolato}
\begin{itemize}
	\item \textbf{Nome}: G\&B: monitoraggio intelligente di processi DevOps;
	\item \textbf{Proponente}: Zucchetti;
	\item \textbf{Committente}: Prof. Tullio Vardanega e Prof. Riccardo Cardin.
\end{itemize}
\subsubsection{Descrizione}
Il capitolato prevede la costruzione di un software per monitorare un sistema 
DevOps, cioè un sistema in cui, a livello aziendale, chi produce il software 
e chi lo usa collaborano strettamente. Per migliorare ulteriormente il servizio 
erogato si richiede un secondo software che permetta di visualizzare, 
analizzare, misurare e controllare i dati forniti dal primo.
\subsubsection{Studio del dominio}
\paragraph{Dominio applicativo} \mbox{}\\
La struttura del software da realizzare è conforme ai seguenti punti: 
\begin{itemize}
	\item un flusso di dati in input viene associato a una rete Bayesiana, 
	composta di nodi contenenti informazioni di probabilità;
	\item la rete riceve il flusso e lo usa per fare dei calcoli, aggiornando
	 quindi le probabilità dei propri nodi;
	\item sia il flusso di dati che la rete sono monitorati in una dashboard;
	\item l'andamento dei dati determina l'eventuale generazione di allarmi
	 e notifiche.
\end{itemize}
\paragraph {Dominio tecnologico} \mbox{} \\
Le tecnologie proposte per lo sviluppo del progetto sono:
\begin{itemize}
	\item \textbf{Grafana}: software \textit{open-source} che, ricevuti i dati in 
input,
	 consente di raccoglierli in un cruscotto, visualizzarli, analizzarli, 
	 misurarli e controllarli;
	\item \textbf{InfluxDB}: database di tipo \textit{Time Series}, generati con 
continuità
	 temporale e atti a essere letti e monitorati costantemente per misurarne 
	 le variazioni;
	\item \textbf{JavaScript}: Linguaggio di programmazione richiesto per costruire 
i 
	\textit{plug-in} di Grafana e per definire la rete di Bayes in formato JSON;
	\item \textbf{Rete di Bayes}: rete di nodi che contengono informazioni di 
probabilità; quando un evento significativo si verifica, le probabilità dei nodi si 
aggiornano di conseguenza.
\end{itemize}
\subsubsection{Aspetti positivi}
L'analisi di questo capitolato ha portato a considerare diversi aspetti 
positivi:
\begin{itemize}
	\item L'azienda si presenta come la prima software house italiana e il gruppo 
mostra 
	notevole interesse a collaborare con essa;
	\item La presentazione del problema è chiara e i requisiti sono ben definiti;
	\item Le tecnologie riguardano vari ambiti (database, linguaggi, probabilità, 
	monitoraggio) e sono in numero ragionevole da apprendere.
\end{itemize}
\subsubsection{Criticità}
Oltre ai punti a favore citati in precedenza, sono emersi anche dei fattori a 
sfavore per
questo progetto:
\begin{itemize}
	\item Per alcuni componenti del gruppo determinate tecnologie da  utilizzare 
	sono sconosciute;
	\item L'apprendimento di \textbf{Grafana} non ha suscito molto 
	entusiasmo all'interno del team di lavoro.
\end{itemize}
\subsubsection{Conclusioni}
Dopo un'attenta valutazione, il capitolato è stato escluso dalle preferenze. Ci 
sono stati diversi elementi positivi che hanno portato il gruppo a considerarlo,
ma non sono stati abbastanza determinanti da poterlo preferire ad altri.

