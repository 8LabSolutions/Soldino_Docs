\documentclass[a4paper]{article}
\usepackage[margin=2cm]{geometry}
\usepackage[italian]{babel}
\usepackage[utf8]{inputenc}
\usepackage[T1]{fontenc}
%\usepackage{fancyhdr}
%\usepackage{amsmath, amssymb}
%\usepackage{caption}
\usepackage{graphicx}
%\usepackage{float}
%\pagestyle{fancy}

\begin{document}
\begin{figure}
\centering
\includegraphics[scale=0.07]{Verbali/res/images/logo8_crop.png}
\end{figure}

\section*{Verbale incontro 4 Dicembre 2018}
Ecco i principali oggetti di discussione del giorno:
\begin{itemize}
	\item Cambo dei ruoli: verrà effettuato ogni due settimane;
	
	\item Pianificazione: sono state definite attività da fare per la prossima settimana, relative ai documenti delle Norme, del Piano di Progetto, del Glossario e dello studio di Fattibilità, con maggiori dettagli su Trello;

	\item Convenzioni: si èdiscusso di alcune convenzioni, tra cui:
	\begin{itemize}
		\item la lingua inglese da adottare nel codice;
		\item la struttura con i comandi di sezione: si usano, a cascata, section, subsection, subsubsection, paragraph e subparagraph;
		\item il glossario e i suoi riferimenti: si utilizza una G a pedice dei termini presenti in glossario in tutte le loro occorrenze;
		\item gli elenchi numerati, in cui ogni voce comincia per lettera minuscola e termina con punto e virgola, a eccezione dell'ultima voce che termina per punto;
		\item nel caso di parole in grassetto o corsivo, seguite da punteggiatura, solo la parola subisce formattazione, mentre la punteggiatura non va in grassetto né corsivo;
	\end{itemize}
	\item Branching: si è discusso sul branching, e sull'utliità di usarne uno per ogni attività  di documentazione in corso.
\end{itemize}

\end{document}
