\section{Verbale della riunione}
Il gruppo ha voluto fin da subito affrontare il problema del nome da attribuirsi 
e tutti i componenti si sono mostrati d'accordo sullo stilare una lista di 
possibili scelte per adottare quella che si è rivelata la più apprezzata. 
All'unanimità si è deciso anche per il logo da usare. 
Successivamente si sono trattate le problematiche relative agli strumenti 
necessari da utilizzare durante lo svolgimento del progetto, sia per la parte 
organizzativa sia per quella implementativa, arrivando alla seguente 
conclusione:

\begin{itemize}
\item \textbf{Git}: come strumento di controllo del versionamento;
\item \textbf{GitHub}: come piattaforma di hosting web;
\item \textbf{Slack}: come mezzo per gestire la comunicazione, in quanto 
permette di dividere le discussioni in argomento e a tal proposito si è deciso 
di creare i canali:
	\begin{itemize}
		\item \textbf{StudioFattibilità}: per confrontarsi nella stesura dell'omonimo 
		documento;
		\item \textbf{NormeProgetto}: per definire le regole, gli strumenti e le 
		convenzioni che si andranno ad adottare per lo svolgimento del progetto;
		\item \textbf{Incontri}: per accordarsi sul giorno e orario in cui riunirsi;
		\item \textbf{General}: per comunicare informazioni relative ad altri 
		argomenti;
	\end{itemize}
\item \textbf{Trello}: come  tool di collaborazione strutturato in bacheche 
condivise e personali, con scadenze e assegnazione dei task, integrabile anche 
in Slack;
\item \textbf{Google Drive}: come spazio per la condivisione di materiale 
consultabile come guide e paper per approfondire i temi trattati nel capitolato 
scelto.
\end{itemize} 

\noindent L'ultima parte dell'incontro ha riguardato la scelta del capitolato da 
aggiudicarsi: ogni componente ha esposto quelli che, secondo il proprio 
giudizio, erano gli aspetti positivi e gli aspetti negativi di ogni progetto 
candidato. Questo confronto ha indirizzato il gruppo verso il capitolato C6,
ovvero \textit{Soldino}. 
\pagebreak
\section{Riepilogo tracciamenti}

	%\renewcommand{\arraystretch}{1.5}
	\rowcolors{2}{pari}{dispari}
	
	\begin{longtable}{ >{\centering}p{0.20\textwidth} >{}p{0.70\textwidth}}
		\caption{Decisioni della riunione interna del 2018-12-04}\\	
		\rowcolorhead
		\textbf{\color{white}Codice} 
		& \centering\textbf{\color{white}Decisione} 
		\tabularnewline 
		\endfirsthead
		VI\_1.1 & Scelto \textit{8Lab Solutions} come nome del gruppo.
		
		\tabularnewline 
		VI\_1.2 & Scelto il logo per rappresentare il gruppo.
		
		\tabularnewline 
		VI\_1.3 & Scelto Git come software di controllo di versione.
	
		\tabularnewline 
		VI\_1.4 & Scelto GitHub come servizio di hosting per il progetto.
		
		\tabularnewline 
		VI\_1.5 & Scelto Slack come canale comunicativo.
		
		\tabularnewline 
		VI\_1.6 & Scelto Trello come strumento di collaborazione per la 
		divisione 
				dei compiti.
	
		\tabularnewline 
		VI\_1.7 & Scelto Google Drive come spazio per il materiale ausiliario.
		
		\tabularnewline
		VI\_1.8 & Scelto il capitolato C6.	
	
	\end{longtable}
	




