\section{Verbale della riunione}
Il primo argomento trattato durante l'incontro è stato il cambio di ruoli.
Essendo a conoscenza che a rotazione tutti i membri avrebbero occupato per un 
certo periodo di tempo un determinato ruolo, si è deciso, per questo primo periodo, di 
eseguire la rotazione dei ruoli ogni due settimane. Il ruolo iniziale concordato 
per ogni membro del team rispecchiava le qualità dimostrate dal primo incontro 
fino ad oggi.
\newline \newline
L'argomento sul quale è stata spesa la maggior parte del tempo è stata la 
pianificazione del lavoro fino al successivo incontro. Son state prese decisioni 
sulla suddivisione della stesura del documento \textit{Norme di Progetto v1.0.0} con dovuta supervisione 
da parte di tutti i membri. \'E cominciata la stesura dei punti fondamentali del 
\textit{Piano di Progetto v1.0.0}. Successivamente si è discusso a proposito di 
un metodo per automatizzare l'inserimento di termini nel glossario e si è 
deciso di effettuare una revisione totale relativa al documento \textit{Studio di
Fattibilità v1.0.0} in modo tale che ogni componente esprimesse il suo parere in 
merito così, nel caso ci fosse stato il bisogno di apportare modifiche, il tutto 
fosse in accordo con il resto del team. Ogni attività è stata organizzata 
tramite l'assegnazione ai componenti del gruppo di schede di lavoro/compiti su 
\textit{Trello}. \\

\pagebreak

Si sono trattate alcune convenzioni da adottare durante lo sviluppo del 
progetto:
\begin{itemize}
\item \textbf{Lingua inglese}: la lingua da utilizzare per la scrittura di un 
futuro codice sorgente;
\item \textbf{Struttura documenti con comandi di sezione}: organizzazione della 
struttura dei documenti in \LaTeX{} secondo le seguenti sezioni a cascata:
\begin{itemize}
	\item section;
	\item subsection;
	\item subsubsection;
	\item paragraph;
	\item subparagraph.
\end{itemize}
\item \textbf{Riferimenti glossario}: si è deciso di utilizzare per ogni 
occorrenza di ogni termine che andrà inserito nel glossario una G a pedice;
\item \textbf{Elenchi numerati}: ogni parola iniziale di un elenco numerato 
dovrà avere la lettera iniziale minuscola. Ogni voce dell'elenco terminerà con 
un punto e virgola(;) ad eccezione dell'ultima voce che terminerà con un 
punto(.);
\item \textbf{Grassetto e corsivo}: nel caso di parole in grassetto o corsivo, 
seguite da punteggiatura, solo la parola subisce formattazione, mentre la 
punteggiatura non va né in grassetto né corsivo.
\end{itemize}
Si è discusso e concordato sul fatto di usare un branch per ogni documento 
perchè permette di velocizzare il tracciamento delle ultime modifiche, per 
mettersi d'accordo più velocemente sui campi da integrare o modificare e senza 
preoccuparsi di eventuali modifiche ad altri documenti in ogni caso 
rintracciabili sul master dopo il merge o sull'apposito branch. Si è deciso, 
infine, di bloccare il merge sul master per evitare modifiche che erroneamente 
potrebbero intaccare documenti revisionati e confermati dal team.


\pagebreak
\section{Riepilogo tracciamenti}
\begin{centering}
\begin{longtable}{ >{\centering}p{4cm} >{\centering}p{11cm} }

\hline
\\[0.5pt]
	\textbf{Codice} & \textbf{Decisione} 
	
	\tabularnewline 
	\hline
	
	
				\\[0.5pt]
				VI\_2.1 & Ruoli stabiliti. \textit{Responsabile}: Federico Bicciato; \textit{Revisore}: Sara Feltrin, Matteo Santinon; \textit{Analista}: Giacomo Greggio, Samuele Giuliano Piazzetta, Francesco Donè, Mattia Bolzonella, Paolo Pozzan
				\\[0.5pt]
				\tabularnewline
				\hline
						
				\\[0.5pt]
				VI\_2.2 & \textit{Norme di progetto}: Giacomo Greggio assegnati "processi primari", Federico Bicciato assegnati "processi di supporto", Paolo Pozzan assegnati "processi organizzativi". Supervisione assegnata ai restanti
				\\[0.5pt]
				\tabularnewline
				\hline
				
				\\[0.5pt]
				VI\_2.3 & \textit{Piano di progetto}: Sara Feltrin assegnato "budget", Matteo Santinon assegnato "analisi dei Rischi", Mattia Bolzonella assegnati introduzione, "modello di sviluppo" e "pianificazione".
				Supervisione assegnata ai restanti
				\\[0.5pt]
				\tabularnewline
				\hline
				
				\\[0.5pt]				
				VI\_2.4 & \textit{Glossario} creazione script per inserimento termini in modo automatico assegnato a Samuele Giuliano Piazzetta
				\\[0.5pt]
				\tabularnewline
				\hline
				
				\\[0.5pt]
				VI\_2.5 & \textit{Studio di fattibilità}: revisione assegnata a Francesco Donè e Samuele Giuliano Piazzetta
				\\[0.5pt]
				\tabularnewline
				\hline
				
				\\[0.5pt]
				VI\_2.6 & Scelta delle convenzioni da utilizzare(specificate sopra: vedi elenco puntato)
				\\[0.5pt]
				\tabularnewline
				\hline
		
				\\[0.5pt]
				VI\_2.7 & Concordato utilizzo di branching suddiviso per ogni documento
				\\[0.5pt]
				\tabularnewline
				\hline          	
                
        %\end{tabularx}
        \\[0.7pt]
        \caption{Decisioni della riunione interna del 2018-12-10}
\end{longtable}
\end{centering}

