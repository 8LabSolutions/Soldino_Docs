\section{Verbale della riunione}
Il primo argomento trattato durante l'incontro è stata la scelta di uno strumento 
condiviso da poter usare per l'organizzazione dei futuri incontri tra i componenti
del team e la decisione di utilizzare \textit{Google Calendar} è stata unanime.
Successivamente si è discusso su quale strumento per il tracciamento dei requisiti
usare, optando per \textit{PragmaDB}\glo{}. Poiché nessun membro del gruppo ha 
avuto modo di usarlo in precedenza, si è deciso di assegnare agli \textit{Analisti}
delle ore di investimento dedicate per lo studio di questo strumento. 
L'argomento sul quale è stata spesa la maggior parte del tempo è stata la 
pianificazione del lavoro da svolgere. Son state prese decisioni sulle norme a
cui tutti i componenti devono aderire per dare coerenza e coesione al progetto
che si è scelto di svolgere. Queste saranno presenti nel documento 
\textit{Norme di Progetto v1.0.0}. Alla fine si è discusso sulla divisione dei 
compiti con relative scadenze per ogni membro al fine di consegnare tutto il 
materiale richiesto per la Revisione dei Requisiti. 
\pagebreak
\section{Riepilogo tracciamenti}

	%\renewcommand{\arraystretch}{1.5}
	\rowcolors{2}{pari}{dispari}
	
\begin{longtable}{ >{\centering}p{0.20\textwidth} >{}p{0.70\textwidth}}
	\caption{Decisioni della riunione interna del 2018-12-10}\\	
	\rowcolorhead
	\textbf{\color{white}Codice} 
	& \centering\textbf{\color{white}Decisione} 
	\tabularnewline 
	\endfirsthead
		VI\_2.1 & Utilizzo di \textit{PragmaDB} come strumento per il tracciamento dei
				requisiti.
		
		\tabularnewline 
		VI\_2.2 & Scelto Google Calendar per l'organizzazione degli 
				incontri del gruppo.
		
		\tabularnewline 
		VI\_2.3 & Decise regole da adottare per uniformare il lavoro da svolgere.
	
		\tabularnewline 
		VI\_2.4 & Assegnazione dei task concordati ad ogni membro del gruppo.
	
	\end{longtable}
	



