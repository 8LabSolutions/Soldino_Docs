\section{Verbale della riunione}
All'inizio si è ribadito lo stato in cui devono trovarsi i requisiti a seconda della revisione di avanzamento a cui sono riferiti e in base a quello che sancisce il \textit{SEMAT}.\newline
Si è iniziato a pensare ad una soluzione nel caso si dovesse usare un database distribuito per il salvataggio di dati(in particolare immagini) e a come si potesse mettere in comunicazione con il browser e con \textit{Metamask}(tecnologia che gestisce le transazioni). La soluzione concordata consiste nel non usare un server per il database, ma salvare tutto nella blockchain.\newline
E' stato creata una sezione su Slack apposita che comprende tutti i  link utili, suddivisi per argomento, per avere un accesso più rapido alla comprensione e ad esempi su diversi topic. \newline
E' stato introdotto e approvato un meccanismo in grado di segnalare velocemente sui documenti l'utilizzo di un termine da \textit{Glossario} tramite la creazione di un nuovo comando per \LaTeX. \newline
Si è concordato uno standard per la stesura dei casi d'uso, insieme agli strumenti che verranno utlizzati per tracciare schemi e UML. \newline
E' stata chiarita la situazione relativa al fatto della gestione del rimborso nel caso una transazione non andasse a buon fine. Si è scesi poi nel dettaglio per quanto riguarda lo storico dell'IVA che ogni trimestre deve essere reso disponibile e consultabile. \newline
In un secondo momento si è parlato del \textit{Piano di qualifica} in merito alla struttura da utilizzare per questo documento. C'è stata una suddivisione delle diverse parti su cui informarsi per poi svilupparle nel documento. Il tutto prevedeva una revisione continua da parte di tutti i membri del team a rotazione. \newline
Infine sono state prese in considerazione alcune modifiche da apportare sia alle \textit{Norme di progetto} sia al \textit{Piano di progetto} che favoriscano uniformità nei documenti. Una volta pianificato il lavoro da svolgere fino al successivo incontro si è stabilita una data che fosse in accordo con tutti.
\pagebreak
\section{Riepilogo tracciamenti}
\begin{centering}
\begin{longtable}{ >{\centering}p{4cm} >{\centering}p{11cm} }

\hline
\\[0.5pt]
	\textbf{Codice} & \textbf{Decisione} 
	
	\tabularnewline 
	\hline
						
				\\[0.5pt]
				VI\_3.1 & Scelto di non usare un server non necessario ma di salvare il tutto sulla blockchain
				\\[0.5pt]
				\tabularnewline
				\hline
				
				\\[0.5pt]
				VI\_3.2 & Creata sezione link utili su \textit{Slack}
				\\[0.5pt]
				\tabularnewline
				\hline
				
				\\[0.5pt]				
				VI\_3.3 & Creato comando per la visualizzazione più rapida di termini \textit{Glossario} per i documenti
				\\[0.5pt]
				\tabularnewline
				\hline
				
				\\[0.5pt]
				VI\_3.4 & Scelto di seguire il modello reale di un qualsiasi sistema di compravendita lasciando che lo smart contracts faccia da intermediario durante la transazione e lasciandogli dunque l'incarico di gestire il rimborso. Infine il resoconto dell'IVA viene fornito all'azienda trimestralmente
				\\[0.5pt]
				\tabularnewline
				\hline
				
				\\[0.5pt]
				VI\_3.5 & \textit{Piano di qualifica}: assegnato a Federico Bicciato "qualità dei processi", assegnato a Paolo Pozzan "qualità di prodotto"(secondo standard ISO/IEC/IEE 12207:2017 e standard ISO/IEC 9126), assegnato a Sara Feltrin la parte riservata ai test. "Appendice A" assegnata a Paolo Pozzan e Federico Bicciato, "appendice B" assegnata a Paolo Pozzan. Supervisione del documento assegnata ai rimanenti del team
				\\[0.5pt]
				\tabularnewline
				\hline
		
				\\[0.5pt]
				VI\_3.6 & \textit{Norme di progetto}: deciso di apportare modifiche per la sezione \textit{Analisi dei requisiti}; \textit{Piano di progetto}: apportare modifiche a "budget"
				\\[0.5pt]
				\tabularnewline
				\hline          	
                
        %\end{tabularx}
        \\[0.7pt]
        \caption{Decisioni della riunione interna del 2018-12-24}
\end{longtable}
\end{centering}

