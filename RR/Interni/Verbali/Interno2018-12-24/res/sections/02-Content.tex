\section{Verbale della riunione}

\subsection{Descrizione}
All'inizio si è ribadito lo stato in cui devono trovarsi i requisiti a seconda della revisione di avanzamento a cui sono riferiti e in base a quello che sancisce il \textit{SEMAT}.\newline
Si è iniziato a pensare ad una soluzione nel caso si dovesse usare un database distribuito per il salvataggio di dati(in particolare immagini) e a come si potesse mettere in comunicazione con il browser e con \textit{Metamask}(tecnologia che gestisce le transazioni).\newline
E' stato creata una sezione su Slack apposita che comprende tutti i  link utili, suddivisi per argomento, per avere un accesso più rapido alla comprensione e ad esempi su diversi topic. \newline
E' stato introdotto e approvato un meccanismo in grado di segnalare velocemente sui documenti l'utilizzo di un termine da \textit{Glossario} tramite la creazione di un nuovo comando per \LaTeX. \newline
Si è concordato uno standard per la stesura dei casi d'uso, insieme agli strumenti che verranno utlizzati per tracciare schemi e UML. \newline
E' stata chiarita la situazione relativa al fatto della gestione del rimborso nel caso una transazione non andasse a buon fine, causa cliente troppo esigente o venditore poco onesto. Si è scesi poi nel dettaglio per quanto riguarda lo storico dell'IVA che ogni trimestre deve essere reso disponibile e consultabile. \newline
In un secondo momento si è parlato del \textit{Piano di qualifica} in merito alla struttura da utilizzare per questo documento. C'è stata una suddivisione delle diverse parti su cui informarsi per poi svilupparle nel documento. Il tutto prevedeva una revisione continua da parte di tutti i membri del team a rotazione. \newline
Infine sono state prese in considerazione alcune modifiche da apportare sia alle \textit{Norme di progetto} sia al \textit{Piano di progetto} che favoriscano uniformità nei documenti. Una volta pianificato il lavoro da svolgere fino al successivo incontro si è stabilita una data che fosse in accordo con tutti.
\pagebreak
\subsection{Riepilogo tracciamenti}
\begin{centering}
\begin{longtable}{ >{\centering}p{4cm} >{\centering}p{11cm} }

\hline
\\[0.5pt]
	\textbf{Codice} & \textbf{Decisione} 
	
	\tabularnewline 
	\hline
	
	
				\\[0.5pt]
				VI\_3.1 & Definizione stato requisiti secondo \textit{SEMAT}
				\\[0.5pt]
				\tabularnewline
				\hline
						
				\\[0.5pt]
				VI\_3.2 & Discussione su salvataggio dati tramite database distribuito e comunicazione fra le diverse parti
				\\[0.5pt]
				\tabularnewline
				\hline
				
				\\[0.5pt]
				VI\_3.3 & Creata sezione link utili su \textit{Slack}
				\\[0.5pt]
				\tabularnewline
				\hline
				
				\\[0.5pt]				
				VI\_3.4 & Creato comando per la visualizzazione più rapida di termini \textit{Glossario} per i documenti
				\\[0.5pt]
				\tabularnewline
				\hline
				
				\\[0.5pt]
				VI\_3.5 & Chiarimento su gestione rimborso e dettagli storico IVA trimestrale
				\\[0.5pt]
				\tabularnewline
				\hline
				
				\\[0.5pt]
				VI\_3.6 & Trattazione sui punti fondamentali e sulla struttura del \textit{Piano di qualifica}
				\\[0.5pt]
				\tabularnewline
				\hline
		
				\\[0.5pt]
				VI\_3.7 & Trattazione sulle modifiche alle \textit{Norme di progetto} e al \textit{Piano di progetto}
				\\[0.5pt]
				\tabularnewline
				\hline          	
                
                \\[0.5pt]
                VI\_3.8 & Pianificazione lavoro fino al prossimo incontro
                \\[0.5pt]
                \tabularnewline
                \hline  
                
        %\end{tabularx}
        \\[0.7pt]
        \caption{Decisioni della riunione interna del 2018-12-24}
\end{longtable}
\end{centering}

