\section{Verbale della riunione}
All'inizio della riunione, sono stati modificati alcuni dei requisiti già analizzati e ne sono 
stati individuati dei nuovi emersi dalla riunione precedente con il proponente.
Successivamente si è trattato il problema riguardante l'utilizzo di un database per il salvataggio dei dati, nel particolare di immagini. L'utilizzo di un database distribuito è risultato la soluzione più appropriata al nostro progetto, in quanto permette di mantenere i vantaggi derivanti dalla decentralizzazione dei dati della rete Ethereum\glo. \newline
Si è poi deciso di creare un canale su Slack per condividere i link utili degli
approfondimenti riguardanti il progetto scelto e di creare uno script per segnalare
velocemente l'utilizzo di termini presenti nel documento \textit{Glossario v1.0.0}.
Infine si è concordato uno standard per la stesura dei casi d'uso, insieme agli 
strumenti che verranno utilizzati per tracciare schemi e disegnare i diagrammi UML.

\pagebreak

\section{Riepilogo tracciamenti}

	%\renewcommand{\arraystretch}{1.5}
	
	\rowcolors{2}{pari}{dispari}
		
	\begin{longtable}{ >{\centering}p{0.20\textwidth} >{}p{0.70\textwidth}}
		\caption{Decisioni della riunione interna del 2018-12-24}\\	
		\rowcolorhead
		\textbf{\color{white}Codice} 
		& \centering\textbf{\color{white}Decisione} 
		\tabularnewline 
		\endfirsthead
		
		VI\_4.1 & Revisionati requisiti già presenti e inseriti quelli nuovi.
		
		\tabularnewline 
		VI\_4.2 & Creazione del canale Link Utili su Slack.
	
		\tabularnewline 
		VI\_4.3 & Creazione di un script per il \textit{Glossario v1.0.0}.
		
		\tabularnewline
		VI\_4.4 & Apportate modifiche in \textit{Norme di Progetto v1.0.0} per 
		la 
				stesura di casi d'uso e requisiti.
		
		\tabularnewline
		VI\_4.5 & Utilizzo di Draw.io per la creazione di diagrammi 
		UML.			
	
	\end{longtable}



