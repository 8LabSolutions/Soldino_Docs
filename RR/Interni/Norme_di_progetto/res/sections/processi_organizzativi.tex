\section{Processi Organizzativi}
	\subsection{Gestione Organizzativa}
		\subsubsection{Scopo}
		Il processo di gestione contiene le attività e i compiti  che possono essere usati da qualunque parte che gestisce i relativi processi. Il responsabile di  	progetto ha la responsabilità della gestione del prodotto, di progetto e dei compiti dei processi applicabili, come i processi di acquisizione, fornitura, funzionamento, manutenzione e supporto.
		\subsubsection{Descrizione}
		Questo processo consiste nelle seguenti attività:
		\begin{itemize}
			\item iniziazione e definizione dello scopo;
			\item pianificazione;
			\item esecuzione e controllo;
			\item verifica e valutazione;
			\item chiusura.
		\end{itemize}
		\subsubsection{Ruoli di progetto}
		Ciascun membro del gruppo, a rotazione, si impegnerà a ricoprire i  ruoli che corrispondono alle omonime figure aziendali. Nel Piano di Progetto 	vengono organizzate e pianificate le attività assegnate ai specifici ruoli previsti nell'attività di progetto. I ruoli che ogni componente del gruppo sarà tenuto a rispettare sono descritti di seguito.
			\paragraph{Responsabile di progetto}
			Il responsabile di progetto è una figura importante in quanto ricadono su di lui le responsabilità di pianificazione, gestione, controllo e coordinamento. Un altro ruolo del responsabile è quello di rappresentare il gruppo all'esterno di questo, saranno quindi a suo carico le comunicazioni con committente e proponente. \\
			Il responsabile deve quindi:
			\begin{itemize}
				\item gestire, controllare e coordinare risorse e attività del gruppo;
				\item  gestire, controllare e coordinare gli altri componenti del gruppo;
				\item analizzare e gestire le criticità;
				\item approvare i documenti.
			\end{itemize}
			\paragraph{Amministratore}
			L'amministratore ha il compito di supporto e controllo dell'ambiente di lavoro. \\
			L'amministratore deve quindi:
			\begin{itemize}
				\item amministrazione delle infrastrutture di supporto;
				\item risoluzione dei problemi legati alla gestione dei processi;
				\item gestione della documentazione;
				\item controllo di versioni e configurazioni.
			\end{itemize}
			\paragraph{Analista}
			L'analista si occupa di analisi dei problemi e del dominio applicativo. Questa figura non sarò sempre presente durante il progetto. \\
			L'analista deve quindi:
			\begin{itemize}
				\item studio del dominio del problema e del problema stesso, definendone complessità e requisiti;
				\item redazione di Analisi dei Requisiti e Studio di Fattibilità.
			\end{itemize}
			\paragraph{Progettista}
			Il progettista gestisce gli aspetti tecnologici e tecnici del progetto.\\
			Il progettista deve:
			\begin{itemize}
				\item effettuare scelte efficienti ed ottimizzate su aspetti tecnici del progetto;
				\item sviluppare un'architettura che sfrutti tecnologie note ed ottimizzate su cui basare un prodotto stabile e mantenibile.
			\end{itemize}
			\paragraph{Programmatore}
			Il programmatore è il responsabile della codifica del progetto e delle componenti di supporto, che serviranno per effettuare le prove di verifica e validazione sul prodotto.\\
			Il programmatore deve:
			\begin{itemize}
				\item implementare le decisioni del progettista;
				\item creare o gestire componenti di supporto per la verifica e la validazione del codice.
			\end{itemize}
		\subsubsection{Attività}
		Seguono le modalità che il gruppo adotterà durante la realizzazione del progetto. Le comunicazioni saranno interne, ossia coinvolgono i partecipanti del gruppo, o esterne, ossia coinvolgono anche proponente e committente.
			\paragraph{Comunicazioni interne}
			Le comunicazioni interne al gruppo avvengono utilizzando Slack, strumento di collaborazione aziendale che offre un ambiente adatto a gestire al meglio il lavoro di gruppo. Questo è stato preferito rispetto ad altri software per la comunicazione perchè permette di integrare alcuni dei servizi che il team utilizza.
			\paragraph{Comunicazioni esterne}
				Le comunicazioni con soggetti esterni al gruppo sono di competenza del responsabile che utilizzerà la mail 8labsolutions@gmail.com creata in fase di creazione del team e, su suggerimento del proponente, su un canale Slack creato proprio da Red Babel .  Il responsabile poi avrà il compito di tenere informati gli altri componenti del gruppo.
			\paragraph{Incontri interni del team}
			Le riunioni interne del team verranno organizzate dal responsabile in accordo con i membri del team. Per stabilire data e ora viene utilizzato Google Calendar, dove ogni membro segnala gli orari nel quale ha impegni.
			\paragraph{Verbali di riunioni interne}
			Ad ogni riunione interna fatta corrisponderà un verbale. Questo sarà redatto da segretario, persona nominata dal responsabile, che dovrà tenere nota delle discussioni fatte e delle decisioni prese. 
			\paragraph{Incontri esterni del team}
			Il responsabile ha il compito di comunicare e organizzare gli incontri con  proponente e committente. Se un membro del gruppo o uno tra proponente e committente ritiene necessario organizzare un incontro il responsabile deve decidere una data, in accordo tra le due parti, e comunicarla tramite i canali sopra citati.
			\paragraph{Verbali esterni di riunione}
			Come per le riunioni interne verrà redatto un verbale che seguirà le regole con struttura uguale a quelli interni.
			\paragraph{Gestione di progetto}
			Per dividere le attività da svolgere e per assicurarsi che tutto venga svolto entro una deadline si è scelto di utilizzare Trello. Questo permette di assegnare una specifica attività ad un membro e impostare anche una scadenza entro la quale l'attività dovrà essere terminata. Sarà il responsabile a gestire la suddivisione degli incarichi tra i membri e a determinare le scadenze per queste.
		\subsubsection{Strumenti utilizzati}
		Il gruppo, nel corso del progetto, ha utilizzato o utilizzerà i seguenti strumenti:
		\begin{itemize}
			\item \textbf{Telegram}: strumento di messaggistica utilizzato per una prima gestione del gruppo;
			\item \textbf{Slack}: per la comunicazione interna del team ed eventuali comunicazioni col proponente;
			\item \textbf{Trello}: per assegnare determinate attività e imporre una scadenza;
			\item \textbf{GitHub}: per il versionamento e il salvataggio di tutti i file riguardanti il progetto.
			\item \textbf{Google Drive e Documenti}: utilizzato per la stesura di file che sono soggetti a molti cambiamenti e devono essere visibili a tutti nella sua versione più aggiornata come ad esempio il glossario;
			\item \textbf{Google Calendar}: per facilitare il lavoro al responsabile ogni settimana ciascun membro indica quando non è disponibile per un incontro;
			\item \textbf{Skype}: servizio che offre possibilità di fare video conferenze e chiamate VoIP, utilizzato per il primo contatto con il proponente;
			\item \textbf{\LaTeX}: per la stesura dei documenti;
			\item \textbf{Sistemi operativi}: i requisiti non indicano la necessità di usare un sistema operativo specifico, verranno quindi utilizzati Windows e Linux; %se qualcuno usa MacOS aggiungerlo non so neanche se ha senso aggiungerlo nella lista
			\item \textbf{} %PLACEHOLDER qualche software per fare grafici tipo R
			\item \textbf{} %PLACEHOLDER qualcosa per fare i grafici UML cardin consigliava qualcosa ma non mi ricordo il nome
		\end{itemize}
	\subsection{Formazione del team}
	Per quando riguarda l'apprendimento basilare delle tecnologie richieste per il progetto, il gruppo ha optato per l'auto-formazione. Nel caso di necessità membri con più conoscenze o esperienze in materia offriranno aiuto a chi ne ha necessità.			