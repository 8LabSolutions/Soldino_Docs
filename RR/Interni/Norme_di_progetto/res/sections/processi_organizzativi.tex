\section{Processi Organizzativi}
	\subsection{Gestione Organizzativa}
		\subsubsection{Scopo}

		Lo scopo di questo processo è:
		\begin{itemize}
			\item creare un modello organizzativo volto a specificare i rischi che si possono verificare;
			\item definire un modello di sviluppo\glosp da seguire;
			\item pianificare il lavoro in base alle scadenze;
			\item calcolare un prospetto economico suddiviso per ruoli;
			\item effettuare un bilancio finale sulle spese.
		\end{itemize}
		Tutte queste attività sono a carico del RdP e devono essere raccolte nel documento \textit{Piano di Progetto}.

		\subsubsection{Aspettative}
		Gli obiettivi di questo processo sono:
		\begin{itemize}
			\item ottenere una pianificazione ragionevole delle attività da seguire;
			\item coordinare i membri del team assegnando loro ruoli, compiti e facilitando la comunicazione tra loro;
			\item adoperare processi per regolare le attività e renderle economiche;
			\item mantenere il controllo sul progetto in modo efficace ma non invasivo, monitorando il team, i processi e i prodotti.
		\end{itemize}
		\subsubsection{Descrizione}
		Le attività di gestione sono:
		\begin{itemize}
			\item inizio e definizione dello scopo;
			\item istanziazione dei processi;
			\item pianificazione e stima di tempi, ricorse, costi;
			\item assegnazione ruoli e compiti;
			\item esecuzione e controllo;
			\item revisione e valutazione periodica delle attività.
		\end{itemize}
		\subsubsection{Ruoli di progetto}
		Ciascun membro del gruppo, a rotazione, deve ricoprire il ruolo che gli viene assegnato e che corrisponde all'omonima figura aziendale. Nel \textit{Piano di Progetto v1.0.0} vengono organizzate e pianificate le attività assegnate agli specifici ruoli. I ruoli che ogni componente del gruppo è tenuto a rappresentare sono descritti di seguito.
			\paragraph{Responsabile di progetto} \mbox{}\\ \mbox{}\\
			Il responsabile di progetto è una figura importante in quanto ricadono su di lui le responsabilità di pianificazione, gestione, controllo e coordinamento. Un altro compito del responsabile è quello interfacciare il gruppo con le persone esterne, facendo da intermediario: sono quindi di sua competenza le comunicazioni con committente e proponente. \\
			Riassumendo, si occupa di:
			\begin{itemize}
				\item gestire, controllare, coordinare risorse e attività del gruppo;
				\item  gestire, controllare, coordinare gli altri componenti del gruppo;
				\item analizzare e gestire le criticità;
				\item approvare i documenti.
			\end{itemize}
			\paragraph{Amministratore di progetto} \mbox{}\\ \mbox{}\\
			L'amministratore ha il compito di supporto e controllo dell'ambiente di lavoro. \\
			Egli deve quindi:
			\begin{itemize}
				\item dirigere le infrastrutture di supporto;
				\item risolvere problemi legati alla gestione dei processi;
				\item gestire la documentazione;
				\item controllare versioni e configurazioni.%povero can
			\end{itemize}
			\paragraph{Analista} \mbox{}\\ \mbox{}\\
			L'analista si occupa di analisi dei problemi e del dominio applicativo. Questa figura non sarà sempre presente durante il progetto. \\
			Le sue responsabilità sono:
			\begin{itemize}
				\item studio del dominio del problema; definizione della complessità  e dei requisiti dello stesso;
				\item redazione del documenti: \textit{Analisi dei Requisiti} e \textit{Studio di Fattibilità}.
			\end{itemize}
			\paragraph{Progettista} \mbox{}\\ \mbox{}\\
			Il progettista gestisce gli aspetti tecnologici e tecnici del progetto.\\
			Il progettista deve:
			\begin{itemize}
				\item effettuare scelte efficienti ed ottimizzate su aspetti tecnici del progetto;
				\item sviluppare un'architettura che sfrutti tecnologie note ed ottimizzate, su cui basare un prodotto stabile e mantenibile.
			\end{itemize}
			\paragraph{Programmatore} \mbox{}\\ \mbox{}\\
			Il programmatore è responsabile della codifica del progetto e delle componenti di supporto che serviranno per effettuare le prove di verifica e validazione sul prodotto.\\
			Il programmatore si occupa di:
			\begin{itemize}
				\item implementare le decisioni del progettista;
				\item creare o gestire componenti di supporto per la verifica e validazione del codice.
			\end{itemize}
			\paragraph{Verificatore} \mbox{}\\ \mbox{}\\
			Il verificatore si occupa di controllare il prodotto del lavoro svolto dagli altri membri del team, sia esso codice o documentazione. Per le correzioni si affida agli standard definiti nelle \textit{Norme di Progetto v1.0.0}, nonché alla propria esperienza e capacità di giudizio.\\
			Il verificatore deve:
			\begin{itemize}
				\item ispezionare i prodotti in fase di revisione, avvalendosi delle tecniche e degli strumenti definiti nelle NdP;
				\item correggere difetti ed errori del prodotto in esame;
				\item decidere se applicare le correzioni in autonomia o in collaborazione con l'autore del prodotto in questione, a seconda della natura dell'errore.
			\end{itemize}
		\subsubsection{Procedure}

		Seguono le procedure che il gruppo adotterà durante la realizzazione del progetto. Le comunicazioni saranno interne, ossia coinvolgeranno i partecipanti del gruppo, o esterne, ossia includeranno anche proponente e committente.
			\paragraph{Gestione delle comunicazioni} \mbox{}\\ \mbox{}\\
			\textbf{Comunicazioni interne} \newline \newline
			Le comunicazioni interne al gruppo avvengono utilizzando Slack\glo. Slack\glosp è un software di collaborazione aziendale adatto al lavoro di gruppo; permette di separare i vari argomenti di discussione in canali e i team hanno la possibilità di crearsi il proprio workspace; permette anche di integrare servizi di terze parti utili al team, tra cui bot e videochiamate. Le sue funzionalità e la sua conformità agli scopi del progetto ne hanno indotto la scelta. \newline \newline
			\textbf{Comunicazioni esterne} \mbox{}\\ \mbox{}\\
			Le comunicazioni con soggetti esterni al gruppo sono di competenza del responsabile. Gli strumenti predefiniti sono la posta elettronica, che utilizza l'indirizzo \href{mailto:8labsolutions@gmail.com}{8labsolutions@gmail.com}.
			Per comunicare con Red Babel si utilizzano un canale Slack\glosp per la chat testuale, e il servizio Google Hangouts per le chiamate. Il responsabile ha il compito di tenere informati gli altri componenti del gruppo in caso di assenza.
			\newline
			\paragraph{Gestione degli incontri} \mbox{}\\ \mbox{}\\
			\textbf{Incontri interni del team} \newline \newline
			Le riunioni interne del team sono organizzate dal responsabile in accordo con i membri del team. Per stabilire data e ora viene utilizzato Google Calendar, dove ogni membro segnala gli orari nei quali non è disponibile. \newline \newline
			\textbf{Verbali di riunioni interne} \newline \newline
			Ad ogni riunione interna corrisponde un \textit{Verbale}. Questo sarà redatto da un segretario, persona nominata dal responsabile, che dovrà tenere nota delle discussioni fatte e delle decisioni prese. \newline \newline
			\textbf{Incontri esterni del team} \newline \newline
			Il responsabile ha il compito di comunicare e organizzare gli incontri con proponente e committente. Se un membro del gruppo, il proponente o il committente ritiene necessario organizzare un incontro il responsabile decide una data, in accordo tra le due parti, e la comunica tramite i canali sopra citati.
			\newline \newline
			\textbf{Verbali esterni di riunione} \newline \newline
			Come per le riunioni interne, anche per le esterne viene redatto un \textit{Verbale}. La struttura delle due tipologie è analoga, ma essendo le riunioni esterne di maggiore criticità, il \textit{Verbale} va redatto con maggiore attenzione, evitando di tralasciare informazioni importanti.
			\paragraph{Gestione degli strumenti di coordinamento} \mbox{}\\ \mbox{}\\
			\textbf{Tickecting} \newline \newline
			Il ticketing consente ai membri di avere chiaro in ogni momento quali attività sono in corso; permette al RdP di assegnare compiti ai membri del team e di controllare l'andamento delle attività; permette ai membri del team di conoscere e gestire il proprio carico di lavoro.\newline Lo strumento di ticketing scelto è Trello: consiste in una lavagna virtuale online dove sono esposti i task. A ogni task sono associati una data di scadenza e un insieme di membri assegnatari. Ogni compito passa attraverso i seguenti stati:
			\begin{itemize}
				\item da fare(to do);
				\item in lavorazione(doing);
				\item in revisione;

				\item completato(done).
			\end{itemize}

			\noindent{La scelta di Trello è stata determinata dalla sua facilità di apprendimento, di controllo, dall'usabilità e dalla trasparenza che fornisce. La gestione dei ticket dev'essere scrupolosa, perché la lavagna di Trello tende ad affollarsi velocemente.}

			\paragraph{Gestione dei rischi} \mbox{}\\ \mbox{}\\
			Il responsabile di progetto ha il compito di rilevare i rischi e di renderli noti, documentando quest'attività nel \textit{Piano di Progetto v1.0.0}. La procedura da seguire per la gestione dei rischi è la seguente:
			\begin{itemize}
				\item individuare nuovi problemi e monitorare i rischi già previsti;
				% PLACEHOLDER: sostituire la frase sottostante, in particolare 'riscontro'
				\item registrare ogni riscontro previsto dei rischi nel Piano di Progetto;
				\item aggiungere i nuovi rischi individuati nel Piano di Progetto;
				\item ridefinire, se necessario, le strategie di gestione dei rischi.

			\end{itemize}
				\subparagraph{Codifica dei rischi}
				Le tipologie di rischi sono così codificate:
				\begin{itemize}
					\item \textbf{RT}: Rischi Tecnologici;
					\item \textbf{RO}: Rischi Organizzativi;
					\item \textbf{RI}: Rischi Interpersonali.
				\end{itemize}

		\subsubsection{Strumenti}
		Il gruppo, nel corso del progetto, ha utilizzato o utilizzerà i seguenti strumenti:
		\begin{itemize}
			\item \textbf{Telegram\glo}: strumento di messaggistica utilizzato inizialmente per la gestione del gruppo;
			\item \textbf{Slack\glo}: per la comunicazione interna del team ed eventuali comunicazioni col proponente;
			\item \textbf{Trello}: per assegnare determinate attività e imporre una scadenza;
			\item \textbf{Git}: sistema di controllo di versionamento;
			\item \textbf{Gitflow, GitKraken}: interfacce per utilizzare Git più comodamente sul proprio desktop;
			\item \textbf{GitHub}: per il versionamento e il salvataggio in remoto di tutti i file riguardanti il progetto.
			\item \textbf{Google Drive}: utilizzato per la stesura di file che sono soggetti a molti cambiamenti e devono essere visibili a tutti nella loro versione più aggiornata, come ad esempio il \textit{Glossario};
			\item \textbf{Google Calendar}: per facilitare il lavoro al responsabile, ogni settimana ciascun membro indica quando non è disponibile, in modo sa semplificare l'organizzazione degli incontri;
			\item \textbf{Skype}: servizio che offre possibilità di fare videoconferenze e chiamate VoIP, utilizzato per il primo contatto con il proponente;
			\item \textbf{Hangouts}: servizio che offre la possibilità di fare videoconferenze e chiamate VoIP, utilizzato per parlare con il proponente e per alcuni incontri interni;
			\item \textbf{Sistemi operativi}: i requisiti non indicano la necessità di usare un sistema operativo specifico, verranno quindi utilizzati sia Windows, sia Linux, sia Mac OS dai diversi membri del team. %niente mac perché viva la povertà!
		\end{itemize}
	