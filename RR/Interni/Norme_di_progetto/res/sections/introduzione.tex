\section{Introduzione}
\subsection{Scopo del documento}
Questo documento ha lo scopo di definire delle regole di base che tutti i membri di 8LabSolutions devono rispettare nello svolgimento del progetto, così da garantire uniformità in tutto il materiale. Verrà utilizzato un approccio incrementale, volto a normare passo passo ogni decisione discussa e concordata tra tutti i membri del gruppo. Ogni eventuale modifica sarà in accordo fra tutti i componenti allo scopo di perseguire la coesione all'interno del team.
\subsection{Scopo del prodotto}
Il capitolato C6 si prefigge di creare un set di ÐApps per la gestione di un e-commerce, basata sull'infrastruttura Ethereum e funzionante tramite il meccanismo degli smart contracts. Soldino ha lo scopo di assistere il governo e le aziende nella gestione dell'IVA riguardante tutte le operazioni di compravendita di beni e servizi tra aziende e clienti. La piattaforma intende divenire un punto di incontro tra governo, aziende e cittadini.
% PLACEHOLDER espandere
\subsection{Ambiguità}
All’interno dei documenti alcuni termini presentano significati ambigui  a  seconda del contesto, fraintendibili, o che necessitano di una descrizione più approfondita. Per evitare questa ambiguità è stato creato il documento "Glossario" volto a fare chiarezza, ponendo a fianco di ogni termine il suo preciso significato. Questi termini sono pertanto marchiati con una "G" a pedice per ogni loro occorrenza all'interno di tutti i documenti.
\subsection{Riferimenti}
\subsubsection{Riferimenti normativi}
\begin{itemize}
\item \textbf{ISO/IEC 12207} \\* %PLACEHOLDER aggiungere dei riferimenti
%\href{https://en.wikipedia.org/wiki/ISO/IEC_12207}{https://en.wikipedia.org/wiki/ISO/IEC_12207};
\item \textbf{Testo del capitolato} \\*  \href{https://www.math.unipd.it/~tullio/IS-1/2018/Progetto/C6.pdf}{https://www.math.unipd.it/~tullio/IS-1/2018/Progetto/C6.pdf};
%verbali normativi
\end{itemize}
\subsubsection{Riferimenti informativi}
\begin{itemize}
\item Guide?(vedi Pro-tech)
\end{itemize}