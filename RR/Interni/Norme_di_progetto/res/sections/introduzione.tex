\section{Introduzione}
\subsection{Scopo del documento}
Questo documento ha lo scopo di definire delle regole di base che tutti i membri 
di \textit{8Lab Solutions} devono rispettare nello svolgimento del progetto, 
così da garantire uniformità in tutto il materiale. Verrà utilizzato un 
approccio incrementale, volto a normare passo passo ogni decisione discussa e 
concordata tra tutti i membri del gruppo. Ciascun componente è obbligato a 
prendere visione di tale documento e a rispettare le norme in esso descritte 
allo scopo di perseguire la coesione\glosp all'interno del team.

\subsection{Scopo del prodotto}
Il capitolato C6 ha per obiettivo lo sviluppo di una piattaforma chiamata \textit{Soldino} che è basata sull'infrastruttura Ethereum\glo{} e funzionante tramite il meccanismo degli smart contract\glo{}. \textit{Soldino} è gestita dal governo e ha lo scopo di assistere le aziende nella gestione dell'IVA riguardo tutte le operazioni di compravendita di beni e servizi, e della gestione delle tasse. Il governo può coniare e 
distribuire la moneta utilizzata in queste transazioni, mentre i cittadini 
potranno acquistare i beni utilizzando tale valuta. La piattaforma, dunque, 
intende divenire un punto di incontro tra governo, aziende e cittadini.

\subsection{Ambiguità}
All'interno dei documenti alcuni termini presentano significati ambigui  a  
seconda del contesto, fraintendibili, o che necessitano di una descrizione più approfondita. Per evitare questa ambiguità è stato creato il documento 
\textit{Glossario v1.0.0} volto a fare chiarezza, ponendo a fianco di ogni 
termine il suo preciso significato. Questi termini sono pertanto marchiati con una \textbf{G} a pedice all'interno di tutti i documenti dove necessario, almeno nella prima occorrenza di ogni documento.

\subsection{Riferimenti}
\subsubsection{Riferimenti normativi}
\begin{itemize}
	\item \textbf{Standard ISO/IEC 12207:1995} \\* 
		\url{https://www.math.unipd.it/~tullio/IS-1/2009/Approfondimenti/ISO_12207-1995.pdf};
	\item \textbf{Testo del capitolato} \\*  
		\url{https://www.math.unipd.it/~tullio/IS-1/2018/Progetto/C6.pdf};
%verbali normativi PLACEHOLDER
\end{itemize}

\subsubsection{Riferimenti informativi}
\begin{itemize}
	\item \textit{Piano di Progetto v1.0.0};
	\item \textit{Piano di Qualifica v1.0.0};
	\item Amministrazione di progetto - Slide del corso “Ingegneria del
		Software”: \\*
		\url{https://www.math.unipd.it/~tullio/IS-1/2016/Dispense/L05.pdf};
	\item Guide to the Software Engineering Body of Knowledge(SWEBOK), 2004: \\*
		\url{http://www.math.unipd.it/~	tullio/IS-1/2007/Approfondimenti/SWEBOK.pdf};
	\item Software Engineering - Ian Sommerville - 9 th Edition (2010): \\*
		(formato cartaceo);
	\item Documentazione git: \\*
		\url{https://www.atlassian.com/git};
	\item Standard ISO 8601: \\*
		\url{https://it.wikipedia.org/wiki/ISO_8601}
	\item Snake Case: \\*
	\url{https://it.wikipedia.org/wiki/Snake_case}	
	
\end{itemize}
