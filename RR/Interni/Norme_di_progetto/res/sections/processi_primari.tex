\section{Processi primari}
\subsection{Fornitura}
\subsubsection{Scopo del processo}
Il processo di fornitura si compone delle attività e dei compiti del fornitore. Una volta comprese le richieste del proponente e aver stilato uno studio di fattibilità, può essere avviato questo processo con fine di soddifare ognuna di queste richieste. D'altra parte si deve stipulare e concordare con l'acquirente un contratto per la consegna del prodotto.
Il processo precedentemente avviato continua con la determinazione delle procedure e delle risorse necessarie per il completamento del progetto, incluso lo sviluppo di un piano di progetto e l'esecuzione di questo fino alla consegna del materiale prodotto.
Il processo di fornitura è composto dalle seguenti attività:
\begin{itemize}
	\item avvio;
	\item approntamento di risposte alle richieste;
	\item contrattazione;
	\item pianificazione;
	\item esecuzione e controllo;
	\item revisione e valutazione;
	\item consegna e completamento;
\end{itemize}
\subsubsection{Aspettative}
Il gruppo intende mantenere un costante dialogo con il proponente per avere un riscontro efficace sul lavoro svolto e instaurare un rapporto di collaborazione in termini di:
\begin{itemize}
	\item determinare aspetti chiave per far fronte ai bisogni del proponente;
	\item stilare requisiti e vincoli sui processi;
	\item stimare le tempistiche di lavoro;
	\item promuovere una verifica continua;
	\item chiarire eventuali dubbi emersi;
	\item accordarsi sulla qualifica del prodotto.
\end{itemize}
\subsubsection{Descrizione}
Questa sezione tratta le norme che i membri del gruppo 8Lab Solutions devono rispettare in tutte le fasi di progettazione, sviluppo e consegna del prodotto Soldino, al fine di diventare fornitori nei confronti del proponente Red Babel e dei committenti Prof. Tullio Vardanega e Prof. Riccardo Cardin.
\subsubsection{Attività}
\paragraph{Studio di fattibilità} \mbox{}\\ 
E' compito del responsabile di progetto organizzare riunioni tra i membri del gruppo al fine di permettere lo scambio di opinioni sui capitolati proposti.
Il documento ``Studio di Fattibilità'', redatto dagli analisti, indica per ogni capitolato:
\begin{itemize}
	\item \textbf{Descrizione e obiettivo finale:} viene fatta una presentazione del capitolato in generale, una descrizione delle caratteristiche principali richieste per il prodotto e viene definito dell'obiettivo che si vuole raggiungere;
	\item \textbf{Studio del dominio:} vengono individuati i principali attori che saranno poi coinvolti, facenti parte del dominio applicativo e viene fatto un elenco delle tecnologie richieste per lo svolgimento, che rientrano nel dominio tecnologico;
	\item \textbf{Aspetti positivi e di rischio:} vengono esposte le considerazione fatte dal gruppo sugli aspetti positivi e sui fattori di rischio del capitolato;
	\item \textbf{Conclusioni:} vengono esposte le ragioni per la quale il gruppo ha deciso di accettare o scartare il capitolato.
\end{itemize}
\paragraph{Piano di progetto} \mbox{}\\
Il responsabile, con l'aiuto degli amministratori, redigerà un piano da seguire durante lo svolgimento del progetto, questo documento conterrà:
\begin{itemize}
	\item \textbf{Analisi dei rischi:} vengono analizzati nel dettaglio i rischi che potranno presentarsi e viene fornita una possibile soluzione. Viene anche fornita la probabilità con la quale questi possono presentarsi e il livello di gravità per ciascuno;
	%\item \textbf{Modello di sviluppo:} viene descritto il modello si sviluppo che è stato scelto;
	\item \textbf{Pianificazione:} vengono pianificate le attività da eseguire nelle diverse fasi del progetto e vengono stabilite le loro scadenze temporali;
	\item \textbf{Preventivo e consuntivo:} viene data una stima di lavoro necessaria per ciascuna fase proponendo così un preventivo per il costo totale del progetto. Verrà anche tracciato, alla fine di ogni attività, un consuntivo di periodo relativo all'andamento rispetto a ciò che è stato preventivato.
\end{itemize}
\paragraph{Piano di qualifica} \mbox{}\\
I verificatori dovranno redigere un documento contenente le strategie da adottare per verifica e validazione del materiale che sarà prodotto dal gruppo. Quest'ultimo conterrà:
\begin{itemize}
	\item \textbf{Strategia di gestione della qualità:} vengono definiti gli obiettivi di qualità di processo e di prodotto e individuate le metriche e le relative misure;
	\item \textbf{Gestione amministrativa della revisione:} vengono stabiliti gli obiettivi da raggiungere per la qualità di processo e di prodotto, assegnando range alle metriche;
	\item \textbf{Standard di qualità:} vengono descritti gli standard di qualità scelti dal gruppo;
	\item \textbf{Valutazioni per il miglioramento:} vengono riportati i problemi e le relative soluzioni nel ricoprire un determinato ruolo e nell'uso degli strumenti scelti;
	\item \textbf{Resoconto delle attività di verifica:} per ogni attività si devono riportare le metriche calcolate e un resoconto sulla verifica di tale attività.
\end{itemize}
\textbf{(queste 2 ultime sezioni non sono presenti nella struttura valutare se tenerle)}
\subsubsection{Documentazione fornita}
Al fine di garantire la massima trasparenza con il proponente e i committenti verranno forniti i seguenti documenti:
\begin{itemize}
	\item \textbf{Piano di Progetto:} descrive pianificazione, consegna e completamento del progetto;
	\item \textbf{Analisi dei requisiti:} specifica l'analisi dei requisiti e i casi d'uso;
	\item \textbf{Piano di Qualifica:} indica le procedure che garantiscono la qualità dei processi e di prodotto
	\item \textbf{Glossario:} raccoglie i termini presenti nei documenti che potrebbero generare ambiguità o che potrebbero necessitare di una descrizione.
\end{itemize}
\subsubsection{Collaudo e consegna del prodotto}
Al fine di consegnare il prodotto terminato il gruppo deve effettuare un collaudo in presenza del proponente e dei committenti. Precedentemente a questo test il gruppo deve assicurare correttezza, completezza e affidabilità per ogni parte del materiale consegnato, permettendo così che tutti i requisiti obbligatori siano soddisfatti e l'esecuzione dei test abbiano un esito positivo. In seguito al collaudo finale il responsabile di progetto consegna il prodotto su un supporto fisico.                  
\subsection{Sviluppo}
\subsubsection{Scopo del processo}
Attività e compiti svolti dal gruppo, al fine di arrivare al prodotto finale richiesto dal proponente, sono contenuti in questo processo.
\subsubsection{Aspettative}
Per una corretta implementazione di tale processo le aspettative sono le seguenti:
\begin{itemize}
	\item realizzare un prodotto finale conforme alle richieste del proponente;
	\item realizzare un prodotto finale che soddisfa i test di verifica;
	\item realizzare un prodotto finale che soddisfa i test di validazione;
	%\item fissare gli obiettivi di sviluppo;
	\item fissare i vincoli tecnologici;
	\item fissare i vincoli di design.
\end{itemize}
\subsubsection{Descrizione}
Il processo di sviluppo si articola in:
\begin{itemize}
	\item Analisi dei requisiti
	\item Progettazione
	\item Codifica	
\end{itemize}
\subsubsection{Attività}
\paragraph{Analisi dei Requisiti} \mbox{}\\
%fin qui
\subparagraph{Scopo} \mbox{}\\
\subparagraph{Aspettative} \mbox{}\\
\subparagraph{Descrizione} \mbox{}\\
\subparagraph{Casi d'uso} \mbox{}\\
\subparagraph{Codice identificativo dei casi d'uso} \mbox{}\\
\subparagraph{Requisiti} \mbox{}\\
\subparagraph{Codice identificativo dei requisiti} \mbox{}\\
\subparagraph{UML} \mbox{}\\
\textbf{(vedere di inserire le sezioni sottostanti nei subparagrph creati)}
I requisiti si raccolgono secondo modalità predefinite:
\begin{itemize}
	\item Lettura del capitolato, analisi e approfondimento dello stesso;
	\item Confronto con il proponente;
	\item Confronto tra membri del team di progetto.
\end{itemize}
I requisiti sono descritti nel documento chiamato "Analisi dei
requisiti. Essi sono conformi alla seguente codifica:\newline
\centerline{R[Importanza][Tipo][Codice]}\newline
Il significato delle cui voci è:
\begin{itemize}
	\item \textbf{Importanza} distingue i requisiti in:
	\begin{itemize}
		\item[1] Obbligatorio;
		\item[2] Desiderabile;
		\item[3] Opzionale.
	\end{itemize}
	\item \textbf{Tipo} distingue i requisiti in:
	\begin{itemize}
		\item[F] Funzionale;
		\item[P] Prestazionale;
		\item[Q] Qualitativo;
		\item[V] Vincolo.
	\end{itemize}
\end{itemize}
Ogni requisito contiene inoltre una descrizione testuale.
\paragraph{Progettazione} \mbox{}\\
\subparagraph{Scopo} \mbox{}\\
\subparagraph{Aspettative} \mbox{}\\
\subparagraph{Descrizione} \mbox{}\\
\subparagraph{Specifica tecnica} \mbox{}\\
\subparagraph{Definizione di prodotto} \mbox{}\\
% PLACEHOLDER da completare
\paragraph{Codifica} \mbox{}\\
\subparagraph{Scopo} \mbox{}\\
\subparagraph{Aspettative} \mbox{}\\
\subparagraph{Descrizione} \mbox{}\\
\subparagraph{Stile di codifica} \mbox{}\\
\subparagraph{Intestazione} \mbox{}\\
\subparagraph{Versionamento} \mbox{}\\
\subparagraph{Ricorsione} \mbox{}\\
Nel codice sorgente si rispettano le seguenti convenzioni:
\begin{itemize}
	\item Parentesi graffe: si segue lo stile di indentazione Stroustroup;
	\item Tabulazioni: l'indentazione del codice si effettua mediante tabulazioni;
	\item Nomi: 
	\begin{itemize}
		\item i nomi di classe cominciano per maiuscola;
		\item i nomi di funzioni cominciano per lettera minuscola;
		\item i nomi di variabili cominciano per lettera minuscola, a eccezione delle costanti, che sono scritte interamente in maiuscolo;
		\item ogni nome composto da più parole è scritto tutto attaccato, con le iniziali dei nomi successivi al primo in maiuscolo; fanno 							eccezione i nomi composti di costanti, da separare con underscore.
	\end{itemize}
\end{itemize}
\subsubsection{Strumenti}
\paragraph{IntelliJ IDEA} \mbox{}\\
\paragraph{IntelliJ IDEA} \mbox{}\\
\paragraph{IntelliJ IDEA} \mbox{}\\