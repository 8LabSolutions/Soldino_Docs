\section{Processi primari}
\subsection{Fornitura}
\subsubsection{Scopo del processo}
Il processo di fornitura contiene le attività e i compiti del fornitore. Questo può essere avviato appena viene presa la decisione di preparare una risposta alle richieste del proponente, sia con la stipulazione di un contratto con l'acquirente per la consegna del prodotto.
Il processo precedentemente avviato continua con la determinazione delle procedure e delle risorse necessarie per il completamento del progetto, incluso lo sviluppo di un piano di progetto e l'esecuzione di questo fino alla consegna del materiale prodotto all'acquirente.
Il processo di fornitura è composto dalle attività che seguono:
\begin{itemize}
	\item avvio;
	\item preparazione delle risposte;
	\item contrattazione;
	\item pianificazione;
	\item esecuzione e controllo;
	\item revisione e valutazione;
	\item consegna e completamento;
\end{itemize}
\subsubsection{Descrizione}
Questa sezione tratta le norme che i membri del gruppo 8Lab Solutions devono rispettare in tutte le fasi di progettazione, sviluppo e consegna del prodotto Soldino, al fine di diventare fornitori nei confronti del proponente Red Babel e dei committenti Prof. Tullio Vardanega e Prof. Riccardo Cardin.
\subsubsection{Attività}
\paragraph{Studio di fattibilità}
Sarà compito del responsabile di progetto organizzare riunioni tra i membri del gruppo al fine di permettere lo scambio di opinioni sui capitolati proposti.
Il documento ``Studio di Fattibilità'', redatto dall'analista, indica per ogni capitolato:
\begin{itemize}
	\item \textbf{Descrizione generale:} viene fatta piccola presentazione del progetto, descrizione delle caratteristiche principali richieste per il prodotto;
	\item \textbf{Studio del dominio:} viene esposto l'obbiettivo finale del progetto e viene fatto un elenco delle tecnologie richieste per il suo svolgimento;
	\item \textbf{Aspetti positivi e di rischio:} vengono esposte le considerazione fatte dal gruppo sugli aspetti positivi e i fattori di rischio del capitolato;
	\item \textbf{Conclusioni:} viene spiegato perchè il gruppo ha deciso di accettare o scartare il capitolato.
\end{itemize}
\paragraph{Piano di progetto}
Il responsabile, con l'aiuto degli amministratori, redigerà un piano da seguire durante lo svolgimento del progetto, questo documento conterrà:
\begin{itemize}
	\item \textbf{Analisi dei rischi:} vengono analizzate nel dettaglio i rischi che potranno presentarsi e viene fornita una possibile soluzione. Viene anche fornita la probabilità con la quale questi possono presentarsi e il livello di gravità che presentano;
	\item \textbf{Modello di sviluppo:} viene descritto il modello si sviluppo che è stato scelto;
	\item \textbf{Pianificazione:} vengono pianificate le attività da eseguire nelle fasi del progetto e vengono fornite le loro scadenze temporali;
	\item \textbf{Preventivo:} viene data una stima di lavoro necessaria per ciascuna fase e quindi un preventivo per il costo che il progetto avrà;
\end{itemize}
\paragraph{Piano di qualifica}
I verificatori dovranno redigere un documento contenente le strategie da adottare per verifica e validazione del materiale che sarà prodotto dal gruppo, in particolare conterrà:
\begin{itemize}
	\item \textbf{Strategia di gestione delle qualità:} vengono definiti gli obiettivi di qualità di processo e di prodotto e individuate le metriche e le relative misure;
	\item \textbf{Gestione amministrativa della revisione:} vengono stabiliti gli obiettivi da raggiungere per la qualità di processo e di prodotto, assegnando range alle metriche;
	\item \textbf{Standard di qualità:} vengono descritti gli standard di qualità scelti dal gruppo;
	\item \textbf{Valutazioni per il miglioramento:} vengono riportati i problemi e le relative soluzioni nel ricoprire un determinato ruolo e nell'uso degli strumenti scelti;
	\item \textbf{Resoconto delle attività di verifica:} per ogni attività si devono riportare le metriche calcolate e un resoconto sulla verifica di tale attività.
\end{itemize}
\subsubsection{Rapporti con il proponente}
Durante lo svolgimento del progetto il gruppo desidera instaurare con l'azienda proponente Red Babel un rapporto di collaborazione costante al fine di:
\begin{itemize}
	\item determinare i bisogni del proponente;
	\item specificare come saranno definiti ed eseguiti i processi;
	\item fare una stima dei costi;
	\item accordarsi sulla qualifica del prodotto.
\end{itemize}
\subsubsection{Documentazione fornita}
Al fine di garantire la massima trasparenza con il proponente e i committenti verranno forniti i seguenti documenti:
\begin{itemize}
	\item \textbf{Piano di Progetto:} descrive pianificazione, consegna e completamento del progetto;
	\item \textbf{Analisi dei requisiti:} specifica l'analisi dei requisiti e i casi d'uso;
	\item \textbf{Piano di Qualifica:} indica le procedure che garantiscono la qualità dei processi e di prodotto
	\item \textbf{Glossario:} raccoglie i termini presenti nei documenti che potrebbero generare ambiguità o che potrebbero necessitare di una descrizione.
\end{itemize}
\subsubsection{Collaudo e consegna del prodotto}
Al fine di consegnare il prodotto terminato il gruppo deve effettuare un collaudo in presenza del proponente e dei committenti. Precedentemente a questo test il gruppo deve assicurare correttezza, completezza e affidabilità per ogni parte del materiale consegnato, permettendo così che tutti i requisiti obbligatori siano soddisfatti e l'esecuzione dei test abbiano un esito positivo. In seguito al collaudo finale il responsabile di progetto consegna il prodotto su un supporto fisico.                  
\subsection{Sviluppo}
\subsubsection{Analisi dei Requisiti}
\paragraph{Modalità di raccolta}
I requisiti si raccolgono secondo modalità predefinite:
\begin{itemize}
	\item Lettura del capitolato, analisi e approfondimento dello stesso;
	\item Confronto con il proponente;
	\item Confronto tra membri del team di progetto.
\end{itemize}
\paragraph{Descrizione dei requisiti}
I requisiti sono descritti nel documento chiamato "Analisi dei
requisiti. Essi sono conformi alla seguente codifica:\newline
\centerline{R[Importanza][Tipo][Codice]}\newline
Il significato delle cui voci è:
\begin{itemize}
	\item \textbf{Importanza} distingue i requisiti in:
	\begin{itemize}
		\item[1] Obbligatorio;
		\item[2] Desiderabile;
		\item[3] Opzionale.
	\end{itemize}
	\item \textbf{Tipo} distingue i requisiti in:
	\begin{itemize}
		\item[F] Funzionale;
		\item[P] Prestazionale;
		\item[Q] Qualitativo;
		\item[V] Vincolo.
	\end{itemize}
\end{itemize}
Ogni requisito contiene inoltre una descrizione testuale.
\subsubsection{Progettazione}
% PLACEHOLDER da completare
\subsubsection{Codifica}
Nel codice sorgente si rispettano le seguenti convenzioni:
\begin{itemize}
	\item Parentesi graffe: si segue lo stile di indentazione \textit{Stroustroup};
	\item Tabulazioni: l'indentazione del codice si effettua mediante tabulazioni;
	\item Nomi: 
	\begin{itemize}
		\item i nomi di classe cominciano per maiuscola;
		\item i nomi di funzioni cominciano per lettera minuscola;
		\item i nomi di variabili cominciano per lettera minuscola, a eccezione delle costanti, che sono scritte interamente in maiuscolo;
		\item ogni nome composto da più parole è scritto tutto attaccato, con le iniziali dei nomi successivi al primo in maiuscolo; fanno 							eccezione i nomi composti di costanti, da separare con \textit{underscore}.
	\end{itemize}
\end{itemize}
