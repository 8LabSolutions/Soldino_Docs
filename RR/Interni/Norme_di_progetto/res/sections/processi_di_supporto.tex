\section{Processi di Supporto}
	\subsection{Documentazione}
		Ogni processo e attività significativi svolti per lo sviluppo del progetto sono documentati. Lo scopo della documentazione è rendere questi processi e attività chiari, trasparenti, registrarli e facilitare il processo di manutenzione. I documenti sono consultabili nell'apposita \href{https://github.com/8LabSolutions/Soldino}{\underline{sezione della repository}}. 
		\subsubsection{Strumenti di scrittura}
		Lo strumento scelto per la scrittura di documenti è \LaTeX. La sua utilità sta nel permettere di scrivere documenti in modo ordinato, modulare, collaborativo e scalabile.
		\subsubsection{Tipologie di documenti}
		Il progetto prevede la redazione di un insieme di documenti, suddivisi in documenti interni e documenti esterni. Essi sono elencati di seguito con le rispettive sigle.\newline
		I documenti esterni sono:		
		\begin{itemize}
			\item \textbf{AdR}: Analisi dei Requisiti: stabilisce le caratteristiche che il software deve rispettare;
			\item \textbf{MU}: Manuale Utente: ad uso degli utilizzatori del software;
			\item \textbf{MS}: Manuale Sviluppatore: per gli sviluppatori e manutentori;
			\item \textbf{PdP}: Piano di Progetto: concerne la gestione del progetto, evidenziandone la fattibilità e le criticità; tratta di tempi, costi, obiettivi, rischi, vincoli;
			\item \textbf{PdQ}: Piano di Qualifica: descrive la qualità del software e dei processi, e come la si intende raggiungere mediante l'uso di strumenti, metriche e processi;
			\item \textbf{RA}: Revisione di Accettazione: determina l'entrata;
			\item \textbf{RP}: Revisione di Progettazione;
			\item \textbf{RQ}: Revisione di Qualifica;
			\item \textbf{RR}: Revisione dei Requisiti.
		\end{itemize}	
		I documenti interni sono:
		\begin{itemize}
			\item \textbf{G}: Glossario: raccoglie i termini di interesse per il team di sviluppo e sui quali il team vuole concordare;
			\item \textbf{NdP}: Norme di Progetto: sono un riferimento per lo svolgimento delle attività di progetto;
			\item \textbf{SdF}: Studio di Fattibilità: descrive sommariamente i capitolati e spiega la loro scelta o esclusione;
			\item \textbf{V}: Verbale: descrizione le interazioni e i loro prodotti avvenuti durante un incontro con il proponente o un incontro interno di interesse per il progetto; sono orientati a dare informazioni semplici e immediatamente fruibili. 
		\end{itemize}
		\subsubsection{Ciclo di vita del documento}
		Ogni documento segue le fasi del seguente ciclo di vita:
		\begin{enumerate}
			\item Creazione del documento: dopo attenta valutazione, nel momento in cui diventa necessario, il documento viene creato, applicando le norme per i documenti e adeguandosi a documenti precedenti dello stesso tipo; si usa un template se disponibile;
			\item Creazione della struttura: se il documento è sufficientemente grande, viene creato il suo indice, che traccia gli argomenti trattati nel documento;
			\item In lavorazione: il documento viene scritto in modo incrementale e per moduli;
			\item In revisione: ogni voce del documento è stata prodotta interamente, ma è soggetta a revisioni per correggere, ampliare, semplificare quanto presente; se possibile, la revisione di ciascun frammento è svolta da almeno una persona diversa da chi ha scritto quel frammento;
			\item Approvato: il responsabile di progetto sancisce che il documento è stato completato; esso è pronto per il rilascio.
		\end{enumerate}
		Le fasi sono ordinate, ma non sono strettamente sequenziali. Sono consentiti ritorni a fasi precedenti.
		\subsubsection{Struttura del documento}
		Ogni documento è composto in maniera modulare. Un main.tex (il cui nome rispecchia il nome del documento) raccoglie in input le sezioni di cui è composto il documento.\newline Tutte le pagine dei documenti contengono un frontespizio, che riporta logo, nome del team e nome del documento. Le pagine contengono, in ordine:
		\begin{enumerate}
			\item Facciata, contenente:
			\begin{itemize}
				\item logo del team, nome del team e nome del progetto;
				\item nome del documento;
				\item tabella contenente informazioni sul documento, la sua versione, chi ci ha lavorato e a chi è destinato;
				\item descrizione sommaria;
				\item email del team.
			\end{itemize}
			\item Changelog;
			\item Indice;
			\item Corpo.
		\end{enumerate}
		\subsubsection{Norme tipografiche}
			\paragraph{Formati}
			\subparagraph{Data}
			La data è scritta secondo il formato:\newline
			\centerline{YYYY-MM-DD}\newline
			che rappresenta la sequenza anno-mese-giorno.
			\subparagraph{Versione}
			Ogni documento ha una storia, ricostruibile attraverso le sue versioni. Ogni documento ha un numero di versione composto da tre cifre:
			\begin{center}
				X.Y.Z
			\end{center}
			dove:
			\begin{itemize}
				\item \textbf{X}: rappresenta una versione stabile del documento, resa tale dopo approvazione del responsabile di progetto;
				\item \textbf{Y}: rappresenta una versione parzialmente stabile del documento che è stata soggetta a verifica da parte di un verificatore;
				\item \textbf{Z}: rappresenta una versione instabile del documento in fase di lavorazione da parte dei redattori.
			\end{itemize}
			\paragraph{Elenchi e sottoelenchi}
			Ogni voce di un elenco comincia per lettera minuscola e termina per "\textbf{;}" eccetto l'ultima che termina per "\textbf{.}". I sottoelenchi sono innestati dentro una voce di elenco e rispettano le medesime regole, poiché la loro funzione è analoga.
	\subsection{Verifica}
	\subsection{Validazione}
	\subsection{Strumenti utilizzati}
	