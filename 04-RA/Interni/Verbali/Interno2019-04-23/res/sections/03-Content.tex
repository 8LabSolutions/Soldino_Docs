\section{Diario della riunione}
La riunione consiste in una chiamata su Hangouts, durante la quale ci aggiorniamo sull'esito della RQ e sul da farsi per la RA.
Sono stati trattati i seguenti punti:
\begin{itemize}
	\item discussione dell'esito della RQ;
	\item individuazione dei presunti punti di debolezza di tale revisione e discussione su come correggerli;
	\item spartizione del lavoro tra i membri del team per la RA;
	
\end{itemize}

\hspace{3cm}

\section{Riepilogo delle decisioni}

	%\renewcommand{\arraystretch}{1.5}
	\rowcolors{2}{pari}{dispari}
	
	\begin{longtable}{ >{\centering}p{0.20\textwidth} >{}p{0.70\textwidth}}
		\caption{Decisioni della riunione interna del 2019-04-23}\\	
		\rowcolorhead
		\textbf{\color{white}Codice} 
		& \centering\textbf{\color{white}Decisione} 
		\tabularnewline 
		\endfirsthead
		VI\_11.1 & Scelto di ridurre a due il numero di rami dei documenti: il \textbf{master} per le versioni stabili, il \textbf{develop} per quelle in lavorazione.
		\tabularnewline 
		VI\_11.2 & Suddivisione temporanea in due team con due oggetti di lavoro distinti: documenti e codice.
		\tabularnewline 
		VI\_11.3 & Prima pianificazione, consistente in assegnazione dei ruoli nei team e dei compiti associati.
		
	\end{longtable}
	




