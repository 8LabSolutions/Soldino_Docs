\section{Introduzione}
\subsection{Scopo del documento}
Questo documento ha lo scopo di definire delle regole di base che tutti i membri 
di \textit{8Lab Solutions} devono rispettare nello svolgimento del progetto, 
così da garantire uniformità in tutto il materiale. Verrà utilizzato un 
approccio incrementale, volto a normare passo passo ogni decisione discussa e 
concordata tra tutti i membri del gruppo. Ciascun componente è obbligato a 
prendere visione di tale documento e a rispettare le norme in esso descritte 
allo scopo di perseguire la coesione\glosp all'interno del team.

\subsection{Scopo del prodotto}
Il capitolato\glosp C6 ha per obiettivo lo sviluppo di una piattaforma chiamata \textit{Soldino} che è basata sull'infrastruttura Ethereum\glo{} e funzionante tramite il meccanismo degli smart contracts\glo{}. \textit{Soldino} è gestita dal governo\glosp e ha lo scopo di assistere le aziende nella gestione dell'IVA riguardo tutte le operazioni di compravendita di beni e servizi. Il governo può coniare e 
distribuire la moneta utilizzata in queste transazioni, mentre i cittadini 
potranno acquistare i beni utilizzando tale valuta. La piattaforma, dunque, 
intende divenire un punto di incontro tra governo, aziende e cittadini.

\subsection{Glossario}
Al fine di evitare possibili ambiguità relative al linguaggio utilizzato nei 
documenti formali, viene fornito il \textit{Glossario v4.0.0}. In questo 
documento vengono definiti e descritti tutti i termini con un significato 
particolare. Per facilitare la lettura, i termini saranno contrassegnati da una 
'G' a pedice.

\subsection{Riferimenti}
\subsubsection{Riferimenti normativi}
\begin{itemize}

	\item \textbf{Capitolato d'appalto C6 - Soldino, piattaforma Ethereum per pagamenti IVA}: \\
		\url{https://www.math.unipd.it/~tullio/IS-1/2018/Progetto/C6.pdf};
%verbali normativi PLACEHOLDER
\end{itemize}

\subsubsection{Riferimenti informativi}
\begin{itemize}
	\item \textbf{Terry Rout. Standard ISO/IEC 12207:1995: New Zealand, Australia. 63 pages (Process' Type)}:\\* 
	\url{https://www.math.unipd.it/~tullio/IS-1/2009/Approfondimenti/ISO_12207-1995.pdf};
	\item \textbf{Ercole F. Colonese. La qualità secondo ISO/IEC 9126. Versione 2.0. 2006. 114 pages (Quality Model, Internal Metrics)}: \\*
	\url{http://www.colonese.it/00-Manuali_Pubblicatii/07-ISO-IEC9126_v2.pdf};
	\item \textbf{Slide L05 del corso Ingegneria del Software - Ciclo di vita 
	del software}:\\ 
		\url{https://www.math.unipd.it/~tullio/IS-1/2016/Dispense/L05.pdf};
	\item \textbf{Pierre Bourque, Robert Dupuis. Guide to the Software Engineering Body of Knowledge(SWEBOK),Versione 2004, 2004: California. 204 pages (Software quality)}: \\*
		\url{http://www.math.unipd.it/~tullio/IS-1/2007/Approfondimenti/SWEBOK.pdf};
	\item \textbf{Ian Sommerville. Software Engineering. 9th Edition. 2010. 615 pages};
	\item \textbf{Documentazione git}: \\*
		\url{https://www.atlassian.com/git};
	\item \textbf{Standard ISO 8601}: \\*
		\url{https://it.wikipedia.org/wiki/ISO_8601};
	\item \textbf{Snake Case}\glo: \\*
		\url{https://it.wikipedia.org/wiki/Snake_case};
	\item \textbf{Airbnb JavaScript style guide}: \\*
		\url{https://github.com/airbnb/javascript/blob/master/README.md};
	\item \textbf{Solidity style guide}: \\*
		\url{https://solidity.readthedocs.io/en/v0.5.7/style-guide.html}.
	
	
	
	
\end{itemize}
