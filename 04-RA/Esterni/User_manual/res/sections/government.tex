\section{Government}
	\subsection{Registration}
	The government cannot register a new account like citizens and 
	businesses because only one account is allowed. This account is provided 
	by the developer of \textit{Soldino}.
	\subsection{Login}
	If you want to log on the Government account press the "login" button on the 
	top right of the homepage, you will automatically log in your account 
	(there is no need for a username or password, all is done via MetaMask). 
	\\To be able to log in make sure you are logged in the correct MetaMask\glosp 
	account.
	\subsection{Logout}
	To log out of \textit{Soldino} you just have press the "Logout" button on 
	the top right of the page but you should also to log out of MetaMask\glosp{} 
	for better security. To do this you have to press MetaMask's icon on the top 
	right of the browser, press your account's icon and then press \texttt{Log out}
	on the top right.
	\begin{figure}[H]
		\includegraphics[width=7cm]{res/images/logout_metamask.png}
		\centering
		\caption{Logging out}
	\end{figure}
\pagebreak
	\subsection{Cubit}
	The Government can mint and distribute Cubits\glosp by pressing the "Cubit 
	Manager" button in the navigation bar at the top of the page.
	\begin{figure}[H]
		\includegraphics[width=15cm]{res/images/cubit_manager.png}
		\centering
		\caption{Goverment Cubit page}
	\end{figure}
		\subsubsection{Minting}
		To mint Cubits\glosp you have to insert the quantity you want in the 
		"Amount" field at the bottom of the page then press "Mint". You will 
		now see that the "Government supply" is increased by the chosen amount.
		\begin{figure}[H]
			\includegraphics[width=15cm]{res/images/minting_cubits.png}
			\centering
			\caption{Minting Cubits}
		\end{figure}
		\subsubsection{Distributing}
		To distribute Cubits\glosp you have to insert the amount in the "Amount" 
		field and select the address of the accounts you wish to send the Cubits to 
		from the drop down menu. Every account you selected will have a checked 
		box next to their name. You can also search an account by name using the 
		specific search bar. When you are done press "Distribute".\\
		\begin{figure}[H]
			\includegraphics[width=15cm]{res/images/distributing.png}
			\centering
			\caption{Distributing Cubits}
		\end{figure} \mbox{}\\
		\noindent Note that the amount of cubit must be lower than the "Government  
		supply" otherwise you will not be able to send them.
	\subsection{Managing users}
	The Government can activate disabled accounts or deactivate active accounts.
	This can be done by clicking on "Citizens List" or "Businesses List" in the 
	navigation bar at the top of the page. Here you will find all users 
	registered in \textit{Soldino} with their informations.
	\begin{figure}[H]
		\includegraphics[width=13cm]{res/images/users_list.png}
		\centering
		\caption{Example of users list}
	\end{figure}
		\subsubsection{Deactivating users}
		After you have found the account you want to disable press the 
		"Disable" button. Disabling an account means that it will not be able 
		to make purchases on \textit{Soldino} 
		until it is enabled.
		\begin{figure}[H]
			\includegraphics[width=15cm]{res/images/user_disable.png}
			\centering
			\caption{Example of disabling an activated account}
		\end{figure}
		\subsubsection{Activating users}
		After you have found the account you want to enable press the button 
		"Enable". After being enabled an account will again be able to make 
		purchases on \textit{Soldino}.
		\begin{figure}[H]
			\includegraphics[width=15cm]{res/images/user_enable.png}
			\centering
			\caption{Example of enabling a deactivated account}
		\end{figure}
	\subsection{Refund VAT}
	To refund businesses of their VAT output you have to press the "VAT 
	Refund" button in the navigation bar at the top of the page, in the 
	page that loads you will find all businesses with VAT output for the 
	current quarter.\\
	After selecting the quarter you can search a Business by name and select 
	them by status.
	\begin{figure}[H]
		\includegraphics[width=15cm]{res/images/business_list.png}
		\centering
		\caption{List of Business}
	\end{figure}
	\noindent Press "Refund" to refund the selected business.\\
%	\begin{figure}[H]
%		\includegraphics[width=7cm]{res/images/business_refund.png}
%		\centering
%		\caption{Example of reimbursing a business}
%	\end{figure}