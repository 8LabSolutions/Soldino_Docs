\section{Glossary}

\subsection*{A}
\addcontentsline{toc}{subsection}{A}

\subsubsection*{Add-on}
\index{Add-on}
Piece of software that attaches to another software extending its functionalities. Usually the application and its addons are executed together as a whole.

\subsubsection*{API}
\index{API}
Abbreviation for Application Programming Interface, it's a set of procedures and functions available to the developers to ease the development process. APIs expose chuncks of code that comes from libraries fostering the reuse of library code.

\subsubsection*{Authentication}
\index{Autenticazione}
It's an essential feature of security. It means to demonstrate one's own identity. In other words, the one who declares to be somebody must be able to proof it, e.g. showing some ID, or exchanging cryptographic keys.

\subsection*{B}
\addcontentsline{toc}{subsection}{B}

\subsubsection*{Backend}
\index{Back end}
It's the part of a computer program that provides core functionalities, but does not expose them in a fancy way to the users.
The user can access these functionalities from an interface provided by the frontend part of the application.


\subsubsection*{Blockchain}
\index{Blockchain}
Immutable and shared data structure shaped as a registister: the register is made of chronologically-ordered chained blocks 
Once a block has been added to the structure it can't be changed. 
This technology, similar to a distributed database, is managed by a network of nodes, each possessing a copy of all data. Each transaction is controlled by a protocol that verifies if the transaction is valid, hence approves it or not. Every approved transactions is added to the blockchain and can't be modified, just like any other block. This way there's no need of external authorities to approve transactions and grant for their validity.

\subsection*{C}
\addcontentsline{toc}{subsection}{C}

\subsubsection*{Purchase confirmation}
\index{Conferma d'acquisto}
It's a document in use in the \textit{Soldino} platform.
Its purpose is supplying an estimate of the VAT invoice, so that a client-business can verify that all products and applications of the IVA invoices applied upon itself are correct. Confirming a purchase is equal to carrying out the effective payment to the seller-business and receiving the actual invoice.


\subsubsection*{Contract}
\index{Contract}
See Smart Contract. 
% add label

\subsubsection*{Cubit}
\index{Cubit}
Custom token of \textit{Soldino}\glosp forked from the Ether\glosp currency, imposed by the \glosp C6 specification.

\subsection*{D}
\addcontentsline{toc}{subsection}{D}

\subsubsection*{DApp}
\index{DApp}

Known also as ÐApp, it stands for Decentralized App, running on a blockchain and managing smart contracts. Transactions on DApps are enrichened by user-defined rules that guarantee more security and control over them. Usually DApps are open-source and operates with cripted data.

\subsubsection*{Design pattern}
\index{Design pattern}
General solution to a common problem. It comes in the form of a model to apply to solve this problem. Each problem can be solved with different patterns, but every pattern is only good for to certain type of problems.

\subsection*{E}
\addcontentsline{toc}{subsection}{E}

\subsubsection*{ECR20}
\index{ECR20}
Abbreviation for Ethereum Request for Comment. It's the standard to adopt to implement tokens on the Ethereum platform.
It defines rules concerning token transfer and data access. Every token that follows the standard has well defined and predictable operations.

%\subsubsection*{EIP-712}
%\index{EIP-712}
%Standard per le transazioni Ethereum. Implementa la firma digitale delle transazioni semplificando la procedura all'utente. In questa maniera l'utente utilizzerà tool come MetaMask\glosp per gestire le transazioni, senza la necessità di inserire le chiavi esadecimali a mano.

\subsubsection*{ESlint}
\index{ESlint}
Tool for code static analysis and pattern identification in JavaScript. Used to achieve correctness, readability and conformity to given rules on your codebase.

\subsubsection*{Ether}
\index{Ether}
Main currency of the Ethereum platform\glo.

\subsubsection*{Ethereum}
\index{Ethereum}
Platform\glosp for the creation and spread of smart contracts\glo. Contracts must pay Ethereum in Ether\glosp to be run on it for the computational power provided by the platform. The data are saved on Ethereum on a blockchain data structure. Every person can access to Ethereum from a node, so the platform can also be seen as a network. 
Like other blockchains, Ethereum allows money exchange.

%\subsubsection*{Ethereum Virtual Machine}
%\index{Ethereum Virtual Machine}
%Macchina virtuale decentralizzata sulla rete Ethereum, che permette di eseguire ÐApps\glo.

\subsection*{F}
\addcontentsline{toc}{subsection}{F}

\subsubsection*{Framework}
\index{Framework}
Un framework è un'astrazione software, in cui è scritto del codice che fornisce funzionalità generiche e atte a essere estese, scrivendo così del software specifico. Può includere vari componenti, come librerie, compilatori ed API\glo, tutte atte a migliorare il processo di sviluppo software.

\subsubsection*{Front end}
\index{Front end}
Il front end è la parte di un'applicazione con la quale l'utente interagisce direttamente, responsabile dell'acquisizione dei dati di ingresso e per la loro elaborazione. Tali dati sono poi utilizzabili dal back end\glo. 

\subsection*{G}
\addcontentsline{toc}{subsection}{G}

\subsubsection*{Gamification}
\index{Gamification}
Riutilizzo di concetti ed elementi tipici dei giochi in applicazioni di diverso contesto. Utilizzando questi principi si ottengono sia maggiore coinvolgimento degli utenti dell'applicazione, sia maggiore produttività ed organizzazione nel lavoro.

\subsubsection*{Ganache}
\index{Ganache}

\subsubsection*{Ganache CLI}
\index{Ganache CLI}
Versione da riga di comando di Ganache che simula un client di Ethereum per sviluppare in modo veloce e sicuro. Fa parte della suite di strumenti di Truffle\glo.


\subsubsection*{Gigacore}
\index{Gigacore}
Boilerplate\glosp per lo sviluppo di applicazioni che fanno uso di React\glo, Redux\glosp e Sass\glo.

\subsubsection*{GitHub}
\index{GitHub}

\subsubsection*{GitHub}
\index{GitHub}
Sistema di gestione repository online, che integra servizi quali issue tracking e l'integrazione di pipeline volte a continuos delivery\glosp e continuos integration\glo.

%\subsubsection*{Gross price}
%\index{Gross price}
%Prezzo del prodotto dopo l'applicazione dell'imposta IVA.


\subsubsection*{Government}
\index{Government}
Termine usato per riferirsi a tutti gli enti governativi interessati nella gestione dell'IVA.


\subsection*{I}
\addcontentsline{toc}{subsection}{I}

\subsubsection*{IPFS}
\index{IPFS}
Protocollo progettato per sostenere la diffusione e l'utilizzo di una rete che possa facilitare la la condivisione di file e documenti attraverso un file system distribuito e con un approccio peer-to-peer\glo.

\subsection*{J}
\addcontentsline{toc}{subsection}{J}

\subsubsection*{JSON}
\index{JSON}
JavaScript Object Notation. \'E una sintassi per il salvataggio e scambio di dati.


\subsection*{K}
\addcontentsline{toc}{subsection}{K}

\subsubsection*{Key}
\index{Key}
Il termine viene utilizzato per indicare la chiave pubblica di un wallet\glosp Ethereum\glosp. Una chiave pubblica identifica un wallet\glosp ed è nota a tutti. Può essere utilizzata per identificare un account destinatario di un versamento. Una chiave privata invece è conosciuta solamente dall'utente proprietario del wallet ed è utilizzata per verificare una transazione effettuata dall'account stesso. 

%\subsection*{L}
%\addcontentsline{toc}{subsection}{L}

\subsection*{M}
\addcontentsline{toc}{subsection}{M}

\subsubsection*{MetaMask}
\index{MetaMask}
Add-on\glosp disponibile nei browser Chrome, Firefox, Opera e Brave che permette agli utenti di interfacciarsi con la rete Ethereum\glosp senza ospitare un nodo della rete. Permette di autenticarsi in maniera sicura e di eseguire ÐApps\glosp sul proprio browser. Fornisce un'interfaccia utente per gestire più wallet/account e salvare tutti i dati direttamente nel browser. 

\subsection*{N}
\addcontentsline{toc}{subsection}{N}

\subsubsection*{Net price}
\index{Net price}
Prezzo del prodotto esente da imposta IVA.


\subsubsection*{Node.js}
\index{Node.js}
Ambiente open-source per l'esecuzione di codice Javascript a run-time, al di fuori dei browser. Ciò permette l'esecuzione di codice Javascript server-side.

\subsubsection*{NPM}
\index{NPM}
NPM è quindi un package manager (Node.js\glosp Package Manager) lo strumento che permette di includere, rimuovere e aggiornare le librerie all'interno di un proprio progetto.

%\subsection*{O}
%\addcontentsline{toc}{subsection}{O}

\subsection*{P}
\addcontentsline{toc}{subsection}{P}


\subsubsection*{Peer-to-peer}
\index{Peer-to-peer}
Nelle reti telematiche, architettura in cui tutti i computer connessi svolgono la funzione sia di client che di server.

\subsubsection*{Plugin}
\index{Plug-in}
Componente aggiuntivo che può essere aggiunto a un'altro software per ampliarne le funzionalità. Di solito può essere eseguito in modo indipendente.



\subsection*{Q}
\addcontentsline{toc}{subsection}{Q}

\subsection*{R}
\addcontentsline{toc}{subsection}{R}

\subsubsection*{React}
\index{React}
Libreria open source per JavaScript per la creazione di interfacce grafiche e la gestione delle interazioni in ambito web.

\subsubsection*{Redux}
\index{Redux}
Libreria open source JavaScript per la gestione degli stati di React\glo.

\subsubsection*{Repository}
\index{Repository}
In generale, locazione di salvataggio dei dati. Nei sistemi di versionamento è una struttura dati più complessa, contenente metadati e operazioni per maneggiarla.


\subsection*{S}
\addcontentsline{toc}{subsection}{S}



\subsubsection*{Smart contract}
\index{Smart contract}
Protocolli per facilitare, attuare e verificare la negoziazione di un contratto in versione digitale. Permettono di ottenere lo stesso valore di un contratto reale senza l'ausilio di un garante esterno. Le transazioni che avvengono con questo protocollo sono tracciabili e irreversibili. Uno smart contract rappresenta del codice che può essere eseguito.

\subsubsection*{Solidity}
\index{Solidity}
Linguaggio di programmazione utilizzato per lo sviluppo di smart contracts\glosp eseguibili in diverse blockchain\glo.

\subsubsection*{Surge}
\index{Surge}
Short for Surge.sh
% add label here

\subsubsection*{Surge.sh}
\index{Surge.sh}
Web server che offre il servizio di hosting per siti web statici.


\subsection*{T}
\addcontentsline{toc}{subsection}{T}

\subsubsection*{Token}
\index{Token}
Rappresentazione di una particolare risorsa o utilità che opera su una blockchain\glo. Un token può rappresentare un qualsiasi bene commerciabile, da materie prime fino alle criptovalute. Si differenzia dagli alctoin per il fatto che usa una blockchain già esistente.

\subsubsection*{Truffle}
\index{Truffle}
Framework per lo sviluppo ed il testing di codice in una blockchain\glo. Gestisce l'environment necessario per eseguire e testare gli smart contracts\glo.


\subsection*{U}
\addcontentsline{toc}{subsection}{U}

%\subsubsection*{UML}
%\index{UML}
%Unified Modeling Language è un linguaggio standard per specififcare, visualizzare, costruire e documentare gli artefatti\glosp di un sistema software.

\subsection*{V}
\addcontentsline{toc}{subsection}{V}

\subsection*{Z}
\addcontentsline{toc}{subsection}{Z}
