\section{Glossary}

\subsection*{A}
\addcontentsline{toc}{subsection}{A}

\subsubsection*{Add-on}
\index{Add-on}
Parte di software aggiuntiva che estende le funzionalità e/o caratteristiche del software di base. Di solito è eseguito insieme al software in cui è aggiunto.

\subsubsection*{API}
\index{API}
Con Application Programming Interface si intende un insieme di procedure e funzioni offerte ai programmatori per facilitare lo sviluppo. Le API espongono blocchi di codice delle librerie di cui fanno parte, per permetterne il riuso.

\subsubsection*{Autenticazione}
\index{Autenticazione}
Proprietà fondamentale della sicurezza: significa fornire la prova della propria identità. In altre parole, chi afferma di essere qualcuno a un'altro ente deve anche provarlo, ad esempio con un identificativo o uno scambio di chiavi crittografiche.

\subsection*{B}
\addcontentsline{toc}{subsection}{B}

\subsubsection*{Backend}
\index{Back end}
Back end è un termine largamente utilizzato per caratterizzare le interfacce che hanno come destinatario un programma. Una applicazione back end è un programma con il quale l'utente interagisce indirettamente, in generale attraverso l'utilizzo di una applicazione front end.

\subsubsection*{Blockchain}
\index{Blockchain}
Struttura dati condivisa ed immutabile, composta da un registro di blocchi concatenati in ordine cronologico. I blocchi una volta inseriti non sono più modificabili. Questa tecnologia, assimilabile ad un database distribuito, è gestito da una rete di nodi, ognuno dei quali possiede una copia dei dati.  Ogni transazione è regolata da un protocollo che ne verifica l'autenticità e la approva o meno. Una transazione approvata non è più modificabile e viene aggiunta alla blockchain. Con questo sistema non è necessaria la presenza di un'autorità esterna alla transazione che faccia da garante.


\subsection*{C}
\addcontentsline{toc}{subsection}{C}


\subsubsection*{Conferma d'acquisto}
\index{Conferma d'acquisto}
Documento utilizzato nella piattaforma Soldino. Lo scopo del documento è fornire un preventivo della fattura IVA in maniera tale che l'azienda-cliente possa verificare che tutti i prodotti e le applicazioni delle imposte IVA su di loro applicate siano corretti. Confermare tale proposta corrisponde ad effettuare il pagamento effettivo all'azienda-venditrice ed a ricevere la vera fattura.


\subsubsection*{Contract}
\index{Contract}
See Smart Contract. 
% add label


\subsubsection*{Cubit}
\index{Cubit}
Custom token\glosp di Ethereum\glosp previsto nel capitolato\glosp C6.

\subsection*{D}
\addcontentsline{toc}{subsection}{D}

\subsubsection*{DApp}
\index{DApp}
Anche scritta ÐApp, è un'applicazione decentralizzata che gestisce gli smart contracta\glo, arricchendo le transazioni con regole user-defined, garantendo maggiore sicurezza e controllo. Solitamente una ÐApp è open-source e  gestisce operazioni con dati criptati.

\subsubsection*{Design pattern}
\index{Design pattern}
Soluzione progettuale generale ad un problema ricorrente. Una descrizione o un modello da applicare per risolvere un problema che può presentarsi in diverse situazioni durante la progettazione e lo sviluppo del software.


\subsection*{E}
\addcontentsline{toc}{subsection}{E}

\subsubsection*{ECR20}
\index{ECR20}
Sigla per "Ethereum Request for Comment". È lo standard per implementare token\glosp su Ethereum. Definisce regole sul trasferimento di token e su come accedere ai dati contenuti in essi, allo scopo di rendere queste operazioni ben definite e prevedibili.

%\subsubsection*{EIP-712}
%\index{EIP-712}
%Standard per le transazioni Ethereum. Implementa la firma digitale delle transazioni semplificando la procedura all'utente. In questa maniera l'utente utilizzerà tool come MetaMask\glosp per gestire le transazioni, senza la necessità di inserire le chiavi esadecimali a mano.

\subsubsection*{ESlint}
\index{ESlint}
Linter\glo per l'analisi statica del codice e per l'identificazione di pattern in JavaScript. Si usa per ottenere codice corretto, comprensibile, conforme a delle regole date.

\subsubsection*{Ether}
\index{Ether}
Valuta in uso nella piattaforma Ethereum\glo.

\subsubsection*{Ethereum}
\index{Ethereum}
Blockchain\glosp per la creazione e pubblicazione di smart contracts\glo. Uno smart contract\glosp che vuole girare su questa rete paga la sua potenza computazionale tramite la valuta Ether. Come le altre blockchain\glosp permette anche lo scambio di denaro.

%\subsubsection*{Ethereum Virtual Machine}
%\index{Ethereum Virtual Machine}
%Macchina virtuale decentralizzata sulla rete Ethereum, che permette di eseguire ÐApps\glo.

\subsection*{F}
\addcontentsline{toc}{subsection}{F}

\subsubsection*{Framework}
\index{Framework}
Un framework è un'astrazione software, in cui è scritto del codice che fornisce funzionalità generiche e atte a essere estese, scrivendo così del software specifico. Può includere vari componenti, come librerie, compilatori ed API\glo, tutte atte a migliorare il processo di sviluppo software.

\subsubsection*{Front end}
\index{Front end}
Il front end è la parte di un'applicazione con la quale l'utente interagisce direttamente, responsabile dell'acquisizione dei dati di ingresso e per la loro elaborazione. Tali dati sono poi utilizzabili dal back end\glo. 

\subsection*{G}
\addcontentsline{toc}{subsection}{G}

\subsubsection*{Gamification}
\index{Gamification}
Riutilizzo di concetti ed elementi tipici dei giochi in applicazioni di diverso contesto. Utilizzando questi principi si ottengono sia maggiore coinvolgimento degli utenti dell'applicazione, sia maggiore produttività ed organizzazione nel lavoro.

\subsubsection*{Ganache}
\index{Ganache}

\subsubsection*{Ganache CLI}
\index{Ganache CLI}
Versione da riga di comando di Ganache che simula un client di Ethereum per sviluppare in modo veloce e sicuro. Fa parte della suite di strumenti di Truffle\glo.


\subsubsection*{Gigacore}
\index{Gigacore}
Boilerplate\glosp per lo sviluppo di applicazioni che fanno uso di React\glo, Redux\glosp e Sass\glo.

\subsubsection*{GitHub}
\index{GitHub}

\subsubsection*{GitHub}
\index{GitHub}
Sistema di gestione repository online, che integra servizi quali issue tracking e l'integrazione di pipeline volte a continuos delivery\glosp e continuos integration\glo.

%\subsubsection*{Gross price}
%\index{Gross price}
%Prezzo del prodotto dopo l'applicazione dell'imposta IVA.


\subsubsection*{Government}
\index{Government}
Termine usato per riferirsi a tutti gli enti governativi interessati nella gestione dell'IVA.


\subsection*{I}
\addcontentsline{toc}{subsection}{I}

\subsubsection*{IPFS}
\index{IPFS}
Protocollo progettato per sostenere la diffusione e l'utilizzo di una rete che possa facilitare la la condivisione di file e documenti attraverso un file system distribuito e con un approccio peer-to-peer\glo.

\subsection*{J}
\addcontentsline{toc}{subsection}{J}

\subsubsection*{JSON}
\index{JSON}
JavaScript Object Notation. \'E una sintassi per il salvataggio e scambio di dati.


\subsection*{K}
\addcontentsline{toc}{subsection}{K}

\subsubsection*{Key}
\index{Key}
Il termine viene utilizzato per indicare la chiave pubblica di un wallet\glosp Ethereum\glosp. Una chiave pubblica identifica un wallet\glosp ed è nota a tutti. Può essere utilizzata per identificare un account destinatario di un versamento. Una chiave privata invece è conosciuta solamente dall'utente proprietario del wallet ed è utilizzata per verificare una transazione effettuata dall'account stesso. 

%\subsection*{L}
%\addcontentsline{toc}{subsection}{L}

\subsection*{M}
\addcontentsline{toc}{subsection}{M}

\subsubsection*{MetaMask}
\index{MetaMask}
Add-on\glosp disponibile nei browser Chrome, Firefox, Opera e Brave che permette agli utenti di interfacciarsi con la rete Ethereum\glosp senza ospitare un nodo della rete. Permette di autenticarsi in maniera sicura e di eseguire ÐApps\glosp sul proprio browser. Fornisce un'interfaccia utente per gestire più wallet/account e salvare tutti i dati direttamente nel browser. 

\subsection*{N}
\addcontentsline{toc}{subsection}{N}

\subsubsection*{Net price}
\index{Net price}
Prezzo del prodotto esente da imposta IVA.


\subsubsection*{Node.js}
\index{Node.js}
Ambiente open-source per l'esecuzione di codice Javascript a run-time, al di fuori dei browser. Ciò permette l'esecuzione di codice Javascript server-side.

\subsubsection*{NPM}
\index{NPM}
NPM è quindi un package manager (Node.js\glosp Package Manager) lo strumento che permette di includere, rimuovere e aggiornare le librerie all'interno di un proprio progetto.

%\subsection*{O}
%\addcontentsline{toc}{subsection}{O}

\subsection*{P}
\addcontentsline{toc}{subsection}{P}


\subsubsection*{Peer-to-peer}
\index{Peer-to-peer}
Nelle reti telematiche, architettura in cui tutti i computer connessi svolgono la funzione sia di client che di server.

\subsubsection*{Plugin}
\index{Plug-in}
Componente aggiuntivo che può essere aggiunto a un'altro software per ampliarne le funzionalità. Di solito può essere eseguito in modo indipendente.



\subsection*{Q}
\addcontentsline{toc}{subsection}{Q}

\subsection*{R}
\addcontentsline{toc}{subsection}{R}

\subsubsection*{React}
\index{React}
Libreria open source per JavaScript per la creazione di interfacce grafiche e la gestione delle interazioni in ambito web.

\subsubsection*{Redux}
\index{Redux}
Libreria open source JavaScript per la gestione degli stati di React\glo.

\subsubsection*{Repository}
\index{Repository}
In generale, locazione di salvataggio dei dati. Nei sistemi di versionamento è una struttura dati più complessa, contenente metadati e operazioni per maneggiarla.


\subsection*{S}
\addcontentsline{toc}{subsection}{S}



\subsubsection*{Smart contract}
\index{Smart contract}
Protocolli per facilitare, attuare e verificare la negoziazione di un contratto in versione digitale. Permettono di ottenere lo stesso valore di un contratto reale senza l'ausilio di un garante esterno. Le transazioni che avvengono con questo protocollo sono tracciabili e irreversibili. Uno smart contract rappresenta del codice che può essere eseguito.

\subsubsection*{Solidity}
\index{Solidity}
Linguaggio di programmazione utilizzato per lo sviluppo di smart contracts\glosp eseguibili in diverse blockchain\glo.

\subsubsection*{Surge}
\index{Surge}
Short for Surge.sh
% add label here

\subsubsection*{Surge.sh}
\index{Surge.sh}
Web server che offre il servizio di hosting per siti web statici.


\subsection*{T}
\addcontentsline{toc}{subsection}{T}

\subsubsection*{Token}
\index{Token}
Rappresentazione di una particolare risorsa o utilità che opera su una blockchain\glo. Un token può rappresentare un qualsiasi bene commerciabile, da materie prime fino alle criptovalute. Si differenzia dagli alctoin per il fatto che usa una blockchain già esistente.

\subsubsection*{Truffle}
\index{Truffle}
Framework per lo sviluppo ed il testing di codice in una blockchain\glo. Gestisce l'environment necessario per eseguire e testare gli smart contracts\glo.


\subsection*{U}
\addcontentsline{toc}{subsection}{U}

\subsection*{V}
\addcontentsline{toc}{subsection}{V}

\subsection*{Z}
\addcontentsline{toc}{subsection}{Z}
