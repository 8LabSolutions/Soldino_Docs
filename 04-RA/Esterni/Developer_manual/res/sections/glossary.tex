\section{Glossary}

\subsection*{A}
\addcontentsline{toc}{subsection}{A}

\subsubsection*{Add-on}
\index{Add-on}
Piece of software that attaches to another software extending its functionalities. Usually the application and its addons are executed together as a whole.

\subsubsection*{API}
\index{API}
Abbreviation for Application Programming Interface, it's a set of procedures and functions available to the developers to ease the development process. APIs expose chuncks of code that come from libraries, fostering the reuse of library code.

\subsubsection*{Authentication}
\index{Autenticazione}
It's an essential feature of security. It means to demonstrate one's own identity. In other words, the one who declares to be somebody must be able to proof it, e.g. showing some ID, or exchanging cryptographic keys.


\subsection*{B}
\addcontentsline{toc}{subsection}{B}

\subsubsection*{Backend}
\index{Back end}
It's the layer of a computer program that provides core functionalities, but does not expose them in a fancy way to the users.
The user can access these functionalities from an interface provided by the frontend\glosp.

\subsubsection*{Blockchain}
\index{Blockchain}
A blockchain is an immutable and distributed data structure. You can think of it as a registister made of chronologically-ordered chained blocks: once a block has been added to the structure it can't be changed. \\
This technology actually consists in a network of nodes, each possessing a copy of all data put on the network.
The blockchain permits transactions to store or exchange data between nodes. Each transaction is controlled by a protocol that verifies if the transaction is valid, hence approves it or not. Every approved transactions is then added to the blockchain, where it can be read but it can't be modified, just like any other block.
This way there's no need of external authorities to approve transactions and grant for their validity.


\subsection*{C}
\addcontentsline{toc}{subsection}{C}

\subsubsection*{Cubit}
\index{Cubit}
Custom token of \textit{Soldino}\glosp forked from the Ether\glosp currency, imposed by the \textit{Soldino} specification.


\subsection*{D}
\addcontentsline{toc}{subsection}{D}

\subsubsection*{DApp}
\index{DApp}

Known also as ÐApp, it stands for Decentralized App, which runs on a blockchain and has it's backend\glosp made of smart contracts\glo. Transactions on DApps are enrichened by user-defined rules that guarantee more security and control over them. Usually DApps are open-source and operate with cripted data.

\subsubsection*{Design pattern}
\index{Design pattern}
General solution to a common problem. It comes in the form of a model to apply to solve this problem. Each problem can be solved with different patterns, but every pattern is only good for certain type of problems.


\subsection*{E}
\addcontentsline{toc}{subsection}{E}

\subsubsection*{ECR20}
\index{ECR20}
Abbreviation for Ethereum Request for Comment. It's the standard to follow for implementing tokens on Ethereum\glo.
It defines rules concerning token transfer and data access. Every token that follows the standard has well defined and predictable operations.

%\subsubsection*{EIP-712}
%\index{EIP-712}
%Standard per le transazioni Ethereum. Implementa la firma digitale delle transazioni semplificando la procedura all'utente. In questa maniera l'utente utilizzerà tool come MetaMask\glosp per gestire le transazioni, senza la necessità di inserire le chiavi esadecimali a mano.

%\subsubsection*{ESlint}
%\index{ESlint}
%Tool for code static analysis and pattern identification in JavaScript. Use it to achieve correctness, readability and conformity to given rules in your codebase.

\subsubsection*{Ether}
\index{Ether}
Main currency of the Ethereum platform\glo. Every operation on Ethereum has a cost in Ether proportional to the computational power required to complete that operation: this cost is limited due to practical reasons. 

\subsubsection*{Ethereum}
\index{Ethereum}
Platform\glosp for the creation and spread of smart contracts\glo. Contracts must pay some Ether\glosp to be run on it for the computational power provided by the platform. A blockchain is used in Ethereum for data storage. Every person can access Ethereum from a node, so the platform can also be seen as a network. 
Like other blockchains, Ethereum allows money exchange.

%\subsubsection*{Ethereum Virtual Machine}
%\index{Ethereum Virtual Machine}
%Macchina virtuale decentralizzata sulla rete Ethereum, che permette di eseguire ÐApps\glo.


\subsection*{F}
\addcontentsline{toc}{subsection}{F}

\subsubsection*{Framework}
\index{Framework}
A framework is a software abstraction. A framework can be made of many tools, such as libraries, compilers, APIs\glo, all working together for the improvement of the software development process. Among these tools there is usually some software library, providing generic and reusable code that can be extended and adapted to particular cases.

\subsubsection*{Frontend}
\index{Frontend}
The part of the software application providing actual interaction with the user, therefore responsible for gathering user input data. These data flow to the backend\glosp for storage, computation, manipulation etc..


\subsection*{G}
\addcontentsline{toc}{subsection}{G}

\subsubsection*{Ganache}
\index{Ganache}
Ethereum client simulator that permits fast and safe development, emulating a local Ethereum network with limited accounts and currencies. This tool is part of the Truffle\glosp suite. There is also a command line version available called \texttt{ganache-cli}.

\subsubsection*{GitHub}
\index{GitHub}
Online repository management system built upon Git. It also provides other services, such as issue tracking system and some metrics.

%\subsubsection*{Gross price}
%\index{Gross price}
%Prezzo del prodotto dopo l'applicazione dell'imposta IVA.

\subsubsection*{Government}
\index{Government}
Term used to refer to all the entities of the State interested in the VAT management.

%\subsection*{H}
%\addcontentsline{toc}{subsection}{H}


\subsection*{I}
\addcontentsline{toc}{subsection}{I}

\subsubsection*{IPFS}
\index{IPFS}
Protocol designed to ease file sharing on a peer-to-peer network built on the top of a distributed file system.
\subsection*{J}
\addcontentsline{toc}{subsection}{J}

\subsubsection*{JSON}
\index{JSON}
JavaScript Object Notation. It's a common format to represent data that is easy to read, analyse and exchange.


%\subsection*{J}
%\addcontentsline{toc}{subsection}{J}

\subsection*{K}
\addcontentsline{toc}{subsection}{K}

\subsubsection*{Key}
\index{Key}
Term indicating the public key of an Ethereum\glosp wallet\glosp. A wallet has a couple of keys, a public one and a private one. The public key identifies the wallet and it's visible to everyone. The private key is only known to its owner and it's used to sign and verify transactions.


%\subsection*{L}
%\addcontentsline{toc}{subsection}{L}


\subsection*{M}
\addcontentsline{toc}{subsection}{M}

\subsubsection*{MetaMask}
\index{MetaMask}
Addon\glosp available on Chrome, Firefox, Opera and Brave browsers.
It allows the users simplified access to Ethereum without the need for them to host a node of the network.
It also allows to securely authenticate and execute DApps\glosp on browsers.
Furthermore, it provides an interface to manage multiple wallets and accounts, and to save all data directly on the browser.


\subsection*{N}
\addcontentsline{toc}{subsection}{N}

\subsubsection*{Net price}
\index{Net price}
Net price of the product, VAT excluded.

\subsubsection*{Node.js}
\index{Node.js}
Open source environment for runtime JavaScript code execution. A common choice if you run JS code server-side. npm\glosp is its default package manager.

\subsubsection*{npm}
\index{npm}
npm stands for Node Package Manager, and permits to manage JavaScript libraries within a filesystem or a project.


%\subsection*{O}
%\addcontentsline{toc}{subsection}{O}


\subsection*{P}
\addcontentsline{toc}{subsection}{P}

\subsubsection*{Peer-to-peer}
\index{Peer-to-peer}
Architecture where all the computers are not in a client-server hierarchy. There is no server that only provides data, no client that only uses it: every node is a peer and plays both roles. This network is suitable to bidirectional data transfer; file sharing is a common peer-to-peer use case.

\subsubsection*{Plugin}
\index{Plug-in}
Additional component that can be attached to a bigger software to extend its functionalities, but usually can be run independently. See also addon\glo.

%\subsubsection*{Purchase confirmation}
%\index{Conferma d'acquisto}
%It's a document in use in the \textit{Soldino} platform.
%Its purpose is supplying an estimate of the VAT invoice, so that a client-business can verify that all products and applications of the IVA invoices applied upon itself are correct. Confirming a purchase is equal to carrying out the effective payment to the seller-business and receiving the actual invoice.


%\subsection*{Q}
%\addcontentsline{toc}{subsection}{Q}


\subsection*{R}
\addcontentsline{toc}{subsection}{R}

\subsubsection*{React}
\index{React}
JavaScript open-source library for building user interfaces and managing interactions in web apps.

\subsubsection*{Redux}
\index{Redux}
JavaScript open source library to manage React\glosp states.

\subsubsection*{Repository}
\index{Repository}
In general, a place to store data. In version control systems it's a more complex structure including metadata and operations for data management.


\subsection*{S}
\addcontentsline{toc}{subsection}{S}

\subsubsection*{Smart contract}
\index{Smart contract}
These are protocols to ease, actualize and verify the negotiation of the digital version of a contract. Smart contracts have the same value of real world contracts but need no external guarantor. The transactions made by this protocol are traceable and irreversible. A smart contract consists in executable code.

%\subsubsection*{Solidity}
%\index{Solidity}
%Programming language used to write smart contracts\glosp for many different blockchains\glo. 

\subsubsection*{Surge}
\index{Surge}
Abbreviation for Surge.sh
% add label here

\subsubsection*{Surge.sh}
\index{Surge.sh}
Web server for static sites hosting. These sites can still run JavaScript code.


\subsection*{T}
\addcontentsline{toc}{subsection}{T}

\subsubsection*{Token}
\index{Token}
Representation of a particular resource on the blockchain\glo. A token can represent any commercial good, from raw materials to cryptovalues. 

\subsubsection*{Truffle}
\index{Truffle}
It's a framwork\glosp for developing and testing code in a blockchain\glo: it builds the environment needed for compiling contrants, migrating them on the blockchain, testing them and much more. Highly customizable.


%\subsection*{U}
%\addcontentsline{toc}{subsection}{U}

%\subsubsection*{UML}
%\index{UML}
%Unified Modeling Language è un linguaggio standard per specififcare, visualizzare, costruire e documentare gli artefatti\glosp di un sistema software.

%\subsection*{V}
%\addcontentsline{toc}{subsection}{V}

%\subsection*{Z}
%\addcontentsline{toc}{subsection}{Z}
