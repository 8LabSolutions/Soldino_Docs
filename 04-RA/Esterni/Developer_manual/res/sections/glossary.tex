\section{Glossary}

\subsection*{A}
\addcontentsline{toc}{subsection}{A}

\subsubsection*{Add-on}
\index{Add-on}
Parte di software aggiuntiva che estende le funzionalità e/o caratteristiche del software di base. Di solito è eseguito insieme al software in cui è aggiunto.

\subsubsection*{API}
\index{API}
Con Application Programming Interface si intende un insieme di procedure e funzioni offerte ai programmatori per facilitare lo sviluppo. Le API espongono blocchi di codice delle librerie di cui fanno parte, per permetterne il riuso.

\subsubsection*{Autenticazione}
\index{Autenticazione}
Proprietà fondamentale della sicurezza: significa fornire la prova della propria identità. In altre parole, chi afferma di essere qualcuno a un'altro ente deve anche provarlo, ad esempio con un identificativo o uno scambio di chiavi crittografiche.

\subsection*{B}
\addcontentsline{toc}{subsection}{B}

\subsubsection*{Backend}
\index{Back end}
Back end è un termine largamente utilizzato per caratterizzare le interfacce che hanno come destinatario un programma. Una applicazione back end è un programma con il quale l'utente interagisce indirettamente, in generale attraverso l'utilizzo di una applicazione front end.

\subsubsection*{Blockchain}
\index{Blockchain}
Struttura dati condivisa ed immutabile, composta da un registro di blocchi concatenati in ordine cronologico. I blocchi una volta inseriti non sono più modificabili. Questa tecnologia, assimilabile ad un database distribuito, è gestito da una rete di nodi, ognuno dei quali possiede una copia dei dati.  Ogni transazione è regolata da un protocollo che ne verifica l'autenticità e la approva o meno. Una transazione approvata non è più modificabile e viene aggiunta alla blockchain. Con questo sistema non è necessaria la presenza di un'autorità esterna alla transazione che faccia da garante.


\subsection*{C}
\addcontentsline{toc}{subsection}{C}


\subsubsection*{Conferma d'acquisto}
\index{Conferma d'acquisto}
Documento utilizzato nella piattaforma Soldino. Lo scopo del documento è fornire un preventivo della fattura IVA in maniera tale che l'azienda-cliente possa verificare che tutti i prodotti e le applicazioni delle imposte IVA su di loro applicate siano corretti. Confermare tale proposta corrisponde ad effettuare il pagamento effettivo all'azienda-venditrice ed a ricevere la vera fattura.


\subsubsection*{Contract}
\index{CamelCase}


\subsubsection*{Cubit}
\index{Cubit}
Custom token\glosp di Ethereum\glosp previsto nel capitolato\glosp C6.





\subsection*{K}
\addcontentsline{toc}{subsection}{K}

\subsubsection*{Key}
\index{Key}
Il termine viene utilizzato per indicare la chiave pubblica di un wallet\glosp Ethereum\glosp. Una chiave pubblica identifica un wallet\glosp ed è nota a tutti. Può essere utilizzata per identificare un account destinatario di un versamento. Una chiave privata invece è conosciuta solamente dall'utente proprietario del wallet ed è utilizzata per verificare una transazione effettuata dall'account stesso. 
