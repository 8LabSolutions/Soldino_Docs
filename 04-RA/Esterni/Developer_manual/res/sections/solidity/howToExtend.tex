\subsection{How to extend}
As described above, smart contracts have been developed with the aim of being upgradable. This means that all the logic contracts can be upgraded without losing the stored data. In order to upgrade a logic contract you have to deploy the new contract, and update the contract manager, changing the address associated with the upgraded contract:
\begin{lstlisting}[language=JavaScript]
//import the new version of the contract
var Contract_v2 = artifacts.require("Name_of_the_contract");
//deploy the new contract
deployer.deploy(Contract_v2, contractManagerInstance.address)
.then((newContractInstance)=>{
	//change the deployment address of the upgraded contract
	//in the contract manager
	return contractManagerInstance.setContractAddress(
		"Name_of_the_contract", 
		newContractInstance.address);
});
\end{lstlisting}
At line 4 the object \texttt{deployer}, provided by Truffle framework, is used to deploy the contract on the blockchain. \texttt{deployer.deploy} returns a JavaScript Promise\glosp which if resolved returns an instance of the contract deployed. In fact at line 5 \texttt{then} is used to construct an anonymous function with the contract instance as parameter. The parameter is used to get its address used as parameter for the \texttt{setContractAddress} function. \\
As mentioned in  \hyperlink{cm}{\underline{section 8.2.1.3}} the contract manager provides to all contracts the address of each contract at its latest version. In this way, when a contract \texttt{A} needs some function provided by contract \texttt{B}, \texttt{A} simply asks the contract manager the address of \texttt{B} and then use the address returned to call \texttt{B}'s functions.

