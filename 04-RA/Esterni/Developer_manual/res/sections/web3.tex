\section{Web3} 
web3.js is a collection of libraries that allows you to interact with a local or remote Ethereum node. Moreover, it is able to communicate with MetaMask, which is the add-on for Chrome and Mozilla Firefox that lets the user manage his Ethereum wallet and confirm transactions in an intuitive and secure way. Using MetaMask we offer secure payments and automatic login to \textit{Soldino}.
\subsection{Architecture overview}
To interact with web3 library calls we organized the \texttt{web3functions} package into five different modules:
\begin{figure}[h]
	\centering
	\includegraphics[scale=0.47]{res/images/web3.png}
	\caption{Class diagram of the \texttt{web3functions} package}
\end{figure}
\\Since the programming language used to code this package is JavaScript, the type of many of the methods in the diagram shown above are just JSON with specific fields. In particular:
\begin{itemize}
	\item if the returned type is \texttt{Promise}, the returnd type is a Promise that resolves if the function call went well, otherwise rejects with an error message;
	\item the other non-standard types have to be intended as JSONs with specific fields, and are better explained below, alongside with the description of the first method that use it.
\end{itemize}
\subsection{Methods and constants}

The five modules groups some methods that are responsible for setting data to Ethereum or retrieving them from the blockchain.

\subsubsection{web\_utils}
This module groups some utility functions that are used by the otherother modules to make the web3 call easier:
\begin{itemize}
	\item \textbf{init}: tries to get an instance of a web3 object, which will be used to make all the Web3 calls. Specifically, it tries to get the Web3 object (defined by the Web3 library) instance injected by MetaMask\glo, if this is possible, otherwise it returns an error;
	\item \textbf{splitIPFSHash}: a function that converts the base58 string representing the IPFS CID into three variables(object ID, size of the ID, hash function ID), since this information is saved in this way in the blockchain for cost and scalability reasons;
	\item \textbf{recomposeIPFSHash}: it's the the inverse of the function descripted above, it recomposes the IPFS CID starting from the three variables mentioned above;
	\item \textbf{getVATPeriod}: returns the VAT period id starting from the current date. The format is YYYY-Q, where YYYY is the current year, and Q is the current trimester (a value between 1 and 4);
	\item \textbf{getCurrentAccount}: returns the current default account reading it from the current Web3 object instance;
	\item \textbf{getContractInstance}: retreives an object of the last version instance of the deployed smart-contract related to the parameter passed. The parameter contractJSON is the JSON given as result from the compilation of the desired smart-contract produced with the command \texttt{truffle compile};
	\item \textbf{tokenTransferApprove}: gives the authorization, to the last version instance of the deployed smart-contract related to the parameter passed, to withdraw the specified amount of Cubit\glosp from the current user account;
		\item \textbf{TOKENMULTIPLIER}:	 in order to save the decimal numbers in Solidity, since this language does not support only integers, the value passed to and retrieved from Solidity must be multiplied for and divided by this constant. The constant is currently set to 100, in order to save 2 decimal digits.
\end{itemize} 

\subsubsection{authentication\glo}
This module manages the registration and login of all the types of users:
\begin{itemize}
	\item \textbf{listenForChanges}: this function is triggered after the user login. Using the Web3 object injected by MetaMask\glo, it is possible to listen to an event related to account or network changes. In this cases, this function resolves the promise returned that could be use, as we are currently using it, to log out the current user;
	\item \textbf{addCitizen}: registers a new citizen using the Ethereum address provided by MetaMask, and saves the IPFS CID, that represents the related information saved in the peer-to-peer\glosp network;
	\item \textbf{addBusiness}: registers a new business using the Ethereum address provided by MetaMask, and saves the IPFS CID, that represents the related information saved in the peer-to-peer network;
	\item \textbf{getUser}: retrieves the information related to the current account from Solidity. In particular, the UserInfo type returned corresponds to a JSON with the following fields:
	\begin{itemize}
		\item the IPFS hash;
		\item the state of the user (a boolean that indicated if the user is enable or disabled);
		\item the user type:
		\begin{itemize}
			\item 0: not registered;
			\item 1: citizen
			\item 2: business
			\item 3: government.
		\end{itemize}
	\end{itemize}
\end{itemize}
\subsubsection{user}
This module manages the common functionality offered to citizen and business:
\begin{itemize}
	\item \textbf{getBalance}:returns the Cubit\glosp balance of the current user account;
	\item \textbf{purchase}: manage a new order making the user transfer the due amount of Cubit\glosp to the target companies;
	\item \textbf{getPurchase}: gets the IPFS CID related to the purchases of the current user;
	\item \textbf{getPurchaseNumber}: gets the number of purchases carried out from the current user.
\end{itemize}
\subsubsection{business}
This module manages the functionality offered to business:
\begin{itemize}
	\item \textbf{addProduct}: inserts the product passed as JSON in the Ethereum blockchain; 
	\item \textbf{modifyProduct}: allows a business to modify one of its product that was previously been added;
	\item \textbf{deleteProduct}: allows a business to delete one of its products that were previously been added. The product will not be shown in the store anymore;
	\item \textbf{modifyProduct}: allows a business to modify one of its product that was previously been added;
	\item \textbf{getProductHash}: get the full IPFS CID of a product starting from the object ID;
	\item \textbf{getSalesInvoices}: returns an array containing all the IPFS CID related to the sales invoices;
	\item \textbf{getTotalProducts}: returns the amount of products currently available in the store. If the user sets the "sender" parameter to true, the function will return the number of products available in the store, inserted by the current account (a business);
	\item \textbf{getProducts}: returns an array of arrays with the information about the products in the store. Specifically each array contains the IPFS CID of the product, the product's seller and the product ID. If the flag "sender" is set to true, the function returns only the products owned by the logged-in business.
	\item \textbf{getInvoices}: returns an array containing all the IPFS CID related to the business invoices;
	\item \textbf{getVATPeriodInfo}: the function returns the datails about the passed VAT period, related to the logged-in business. The returned value (VATInfo) is an array with the following format [businessAddress, state, amount];
	\item \textbf{payVATPeriod}: allows a business to pay the VAT owed to the government related to a particular VAT quarter;
	\item \textbf{putOnHoldVATPariod}: allows a business to postpone of a month the VAT payment to the government related to a particular VAT quarter.

\end{itemize}
\subsubsection{government}
Module to manage the common functionality offered to the government:
\begin{itemize}
	\item \textbf{mint}: lets the government mint a specific amount of \textit{Cubit}. This amount is deposited in the government account;
	\item \textbf{distribute}: lets the government transfer a specific amount of \textit{Cubit} from his own account to a target account;
	\item \textbf{distributeToMultipleAddresses}: lets the government transfer a specific amount of Cubit from his own account to each of the account in the targets list passed to the function;
	\item \textbf{disableAccount}: lets the government set the state of the target account to "disabled". The target account will not be able to use the platform until it has been restored;
	\item \textbf{enableAccount}: lets the government set the state of the target account to "enabled". The target account will be able to use the platform;
	\item \textbf{refundVAT}: lets the government refund a business. The owed amount of Cubit, related to a specific VAT quarter, will be transferred to the target business;
	\item \textbf{getUserList}: returns an array containing \texttt{amount} JSONs with the users' data, skipping the first \texttt{amount*index} users. The user type is the one passed to the function. The JSONs (the UserInfo type in the diagrams above) contains all the informations inserted by the user in the registration process, plus the state of abled/disabled;
	\item \textbf{getTotalCubit}: returns the total amount of Cubit that were minted in Soldino;
	
	\item \textbf{getInvoicesGovernment}: lets the government refund a business. The owed amount of Cubit, related to a specific VAT quarter, will be transferred to the target business;
	
	\item \textbf{getVATQuarterInfo}: Return the following info about the VAT period:
	\begin{itemize}
		\item businessAddress;
		\item state of the period: [DUE, OVERDUE, PAID, TO\_BE\_REFUNDED, REFUNDED] enum 0->4;
		\item amount: VAT amount of the period.
	\end{itemize} 
\end{itemize}
\subsection{How to extend with new features}
 In order to extend this package, keep in mind the following thing:


