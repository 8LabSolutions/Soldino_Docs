\section{Analisi dei rischi}

Nel corso dello sviluppo di un progetto complesso è possibile incorrere in 
problemi che possono essere evitati tramite un'attività di analisi. Quindi 
allo scopo di prevenire queste situazioni è stata effettuata 
un'approfondita attività di analisi dei principali fattori di rischio. Per 
ciascuna delle voci nella tabella sottostante è stata utilizzata la seguente 
procedura di identificazione e risoluzione:

\begin{itemize}
	\item \textbf{Individuazione dei rischi}: attività di identificazione dei 
	vari fattori problematici che possono rallentare o impedire il normale 
	proseguimento del progetto;
	\item \textbf{Analisi dei rischi}: attività di studio dei fattori di 
	rischio con conseguente assegnazione di una probabilità che essi si 
	verifichino e di un indice di gravità, determinando così l'impatto che 
	avrebbero sul progetto;
	\item \textbf{Pianificazione di controllo}: attività di pianificazione di 
	una metodologia per evitare che si verifichino i rischi individuati e si 
	stabilisca come procedere nel caso in cui il problema si riscontrasse;
	\item \textbf{Monitoraggio dei rischi}: attività continua svolta al fine di 
	evitare che abbiano luogo queste complicazioni o, nel peggiore dei casi, 
	permetta di agire tempestivamente per contenerle.

\end{itemize}

Sono stati inoltre definiti i seguenti codici per raggruppare le varie 
tipologie di fattori di rischio:

\begin{itemize}
	\item \textbf{RT}: Rischi Tecnologici;
	\item \textbf{RO}: Rischi Organizzativi;
	\item \textbf{RI}: Rischi Interpersonali.
\end{itemize}

\counterwithin{table}{section}
\renewcommand{\arraystretch}{1.5}
\rowcolors{2}{dispari}{pari}
	\arrayrulecolor{white}
	\begin{longtable}{ 
			>{\centering}p{0.17\textwidth} 
			>{\raggedright}p{0.28\textwidth}
			>{\raggedright}p{0.29\textwidth} 
			>{\centering}p{0.15\textwidth}
		}

	
	\caption{Tabella dei Rischi di Progetto}\\
	\rowcolorhead
	\colorhead\textbf{Nome \\ Codice} & \centering\colorhead\textbf{Descrizione} & 
	\centering\colorhead\textbf{Rilevamento} & 
	\colorhead\textbf{Grado di rischio} 
	\tabularnewline
	\endfirsthead
	\rowcolor{white}\caption[]{(continua)}\\
	\rowcolorhead
	\colorhead\textbf{Nome \\ Codice} & \centering\colorhead\textbf{Descrizione} & 
	\centering\colorhead\textbf{Rilevamento} & 
	\colorhead\textbf{Grado di rischio} 
	\tabularnewline
	\endhead
	
	%RT1---------------------------------------------------------
	 \rowcolorlight Inesperienza tecnologica \\ RT1 & 
	 Molte delle tecnologie da adottare nello sviluppo del progetto richiesto sono nuove per molti componenti del team, di conseguenza potrebbero insorgere problemi operativi. &
	Il \textit{responsabile} avrà il compito di rilevare conoscenze ed 
	eventuali lacune 
	dei vari componenti del team. Ciascun membro del gruppo inoltre provvederà a 
	comunicare in assoluta trasparenza eventuali difficoltà. &
	 Occorrenza: \textbf{Alta} \\
	 Pericolosità: \textbf{Alta} 
	 \tabularnewline
	\rowcolorlight \multicolumn{1}{p{0.17\textwidth}}{\centering\textbf{Piano di contingenza}}& 
	 \multicolumn{3}{p{0.7775\textwidth}}{I compiti più onerosi, o che 
	 richiedono maggiori conoscenze tecnologiche, verranno assegnati a più 
	 persone favorendo così l'assistenza reciproca. }
	 \tabularnewline 
	 
	 %RT2---------------------------------------------------------
	 \rowcolordark Tecnologie instabili \\ RT2 & 
	 Le tecnologie adoperate nel progetto sono recenti. Alcune di esse non sono stabili, e riportano bug e scarsa documentazione. L'integrazione perfetta di tutte le tecnologie appare incerta. &
	 Gli addetti a ciascuna tecnologia segnalano le rispettive falle tecnologiche. &
	 Occorrenza: \textbf{Alta} \\
	 Pericolosità: \textbf{Alta} 
	 \tabularnewline
	 \rowcolordark \multicolumn{1}{p{0.17\textwidth}}{\centering\textbf{Piano di contingenza}}& 
	 \multicolumn{3}{p{0.7775\textwidth}}{Se possibile, si tenta di mantenere le ultime versioni di ciascuna tecnologia, per favorire la durata del progetto nel tempo. Se no, si opta per una versione più datata ma più stabile della tecnologia in questione.}
	 \tabularnewline 
	
	%RT3---------------------------------------------------------
	\rowcolorlight
	Guasto tecnico \\ RT3 &
	I computer del team rischiano possono guastarsi, anche a causa del frequente utilizzo e l'installazione di nuovi software. &
	Ogni membro del team deve monitorare il corretto funzionamento del proprio pc.&
	Occorrenza: \textbf{Bassa} \\
	Pericolosità: \textbf{Media}
	\tabularnewline
	\rowcolorlight\multicolumn{1}{p{0.17\textwidth}}{\centering\textbf{Piano di contingenza}}& 
	\multicolumn{3}{p{0.7775\textwidth}}{A seconda della gravità del guasto si provvede alla reinstallazione del software, del sistema operativo o alla sostituzione della propria macchina.}
	\tabularnewline		
	
 	
	 %RO1---------------------------------------------------------
	\rowcolordark Calcolo tempistiche \\ RO1 &
	La massiccia presenza di tecnologie nuove per molti dei componenti del team può comportare imprecisioni e variazioni nel calcolo 
	delle tempistiche.&
	Nel corso dello sviluppo verrà assegnata una scadenza a ciascun task; sarà responsibilità dell'owner del task comunicare problematiche, nel rispettare tali scadenze.&	
	Occorrenza: \textbf{Alta} \\
	Pericolosità: \textbf{Alta}
	\tabularnewline
	\rowcolordark\multicolumn{1}{p{0.17\textwidth}}{\centering\textbf{Piano di contingenza}}& 
	\multicolumn{3}{p{0.7775\textwidth}}{All'insorgere di tali problematiche, 
	il \textit{responsabile} in accordo con l'owner del task, provvederà 
	all'assegnazione 
	di maggiori risorse o allo spostamento della scadenza.}
	\tabularnewline	
	
	%R02------------------------------------------------------------
	\rowcolorlight
	Calcolo costi \\ RO2 &
	Considerata l'inesperienza del team 
	nelle valutazioni
	economiche, esse potranno risultare imprecise. &
	Il team ha predisposto specifiche tabelle condivise al fine di permettere 
	al \textit{responsabile} di monitorare le ore di lavoro di ciascun 
	componente.&
	Occorrenza: \textbf{Media} \\
	Pericolosità: \textbf{Alta}
	\tabularnewline
	\rowcolorlight\multicolumn{1}{p{0.17\textwidth}}{\centering\textbf{Piano di contingenza}}& 
	\multicolumn{3}{p{0.7775\textwidth}}{All'insorgere di rilevanti variazioni 
	orarie rispetto al preventivo iniziale, verranno comunicati tempestivamente 
	al committente tali mutamenti.}
	\tabularnewline	
	
	%R03------------------------------------------------------------
	\rowcolordark Impegni accademici \\ RO3 & 
	Possono verificarsi momenti in cui uno o più componenti del team siano meno disponibili a causa di alcuni impegni di tipo accademico. &
	Al fine di prevenire rallentamenti, è stato predisposto un calendario 
	condiviso nel quale segnalare eventuali impegni accademici.&
	Occorrenza: \textbf{Alta} \\
	Pericolosità: \textbf{Bassa}
	\tabularnewline
	\rowcolordark\multicolumn{1}{p{0.17\textwidth}}{\centering\textbf{Piano di contingenza}}& 
	\multicolumn{3}{p{0.7775\textwidth}}{ L'assegnazione di incarchi e scadenze 
	avverrà nel rispetto degli impegni segnalati nel calendario.}
	\tabularnewline	
	
	%R04------------------------------------------------------------
	\rowcolorlight
    Impegni personali \\ RO4 &
	Possono verificarsi momenti in cui uno o più componenti del team siano meno disponibili a causa di alcuni impegni di tipo personale.&
	Ciascun componente del team utilizzerà 
	il calendario già descritto nel caso precedente per segnalare i propri 
	impegni personali. Eventuali impegni imprevisti verranno tempestivamente 
	segnalati al \textit{responsabile}.&
	Occorrenza: \textbf{Media} \\
	Pericolosità: \textbf{Bassa}
	\tabularnewline
	\rowcolorlight\multicolumn{1}{p{0.17\textwidth}}{\centering\textbf{Piano di contingenza}}& 
	\multicolumn{3}{p{0.7775\textwidth}}{L'assegnazione di incarchi e scadenze 
		avverrà nel rispetto degli impegni segnalati nel calendario. 
		All'insorgere di imprevisti, il reponsabile valuterà una riallocazione 
		di risorse oppure una riassegnazione del task.}
	\tabularnewline	
	
	%R05------------------------------------------------------------
	\rowcolordark
	 Ritardi nella pianificazione \\ RO5 &
	Una o più delle problematiche sopracitate (RO1, RO3, RO4) possono 
	comportare ritardi.&
	L'owner di ciascun task segnalerà in modo tempestivo l'impossibiltà di 
	rispettare le proprie scadenze.&
	Occorrenza: \textbf{Media} \\
	Pericolosità: \textbf{Bassa}
	\tabularnewline
	\rowcolordark\multicolumn{1}{p{0.17\textwidth}}{\centering\textbf{Piano di contingenza}}& 
	\multicolumn{3}{p{0.7775\textwidth}}{ Il \textit{responsabile}, se 
	necessario, riassegna le risorse al fine evitare rallentamenti.}
	\tabularnewline	
	
	%R05------------------------------------------------------------
	\rowcolorlight
	Indisponibilità dell'area di lavoro \\ RO6 &
	Il luogo prescelto per lavorare è la Torre Archimede. Spesso capita che, negli orari di punta, non si trovino aule libere o sufficientemente silenziose per lavorare.&
	Assenza di aule consone al lavoro.&
	Occorrenza: \textbf{Alta} \\
	Pericolosità: \textbf{Bassa}
	\tabularnewline
	\rowcolorlight\multicolumn{1}{p{0.17\textwidth}}{\centering\textbf{Piano di contingenza}}& 
	\multicolumn{3}{p{0.7775\textwidth}}{Controlliamo con anticipo di un giorno la disponibilità delle aule. In caso di totale occupazione della Torre, si opta per trasferirsi in una delle strutture vicine (Paolotti, Ex Fiat, Vallisneri).}
	\tabularnewline
	
	%R07------------------------------------------------------------
	\rowcolordark
	Fallimento di consegna \\ RO7 &
	Svariati e concorrenti motivi possono portare al fallimento della consegna.&
	Il fallimento è notificato dai docenti mediante l'applicazione di una penalità o la richiesta di rieffettuare la consegna.&
	Occorrenza: \textbf{Bassa} \\
	Pericolosità: \textbf{Alta}
	\tabularnewline
	\rowcolordark\multicolumn{1}{p{0.17\textwidth}}{\centering\textbf{Piano di contingenza}}& 
	\multicolumn{3}{p{0.7775\textwidth}}{Il \textit{responsabile} monitora l'andamento del progetto per controllare che non manchi nulla; monitora i rischi per ridurre al minimo la possibilità di fallimento.}
	\tabularnewline
	
	%R08------------------------------------------------------------
	\rowcolorlight
	Stallo dei task \\ RO8 &
	I task su Trello possono ritardare di molto il loro completamento. Ciò può dipendere dal task in sè (mancanza di assegnatario, data di scadenza, definizione imprecisa o task monolitico).&
	La bacheca di Trello risulta affollata. Alcuni task sono visibilmente fermi.&
	Occorrenza: \textbf{Media} \\
	Pericolosità: \textbf{Media}
	\tabularnewline
	\rowcolorlight\multicolumn{1}{p{0.17\textwidth}}{\centering\textbf{Piano di contingenza}}& 
	\multicolumn{3}{p{0.7775\textwidth}}{Il \textit{responsabile} ridefinisce i task completandoli, raffinandoli o suddividendoli.}
	\tabularnewline
	
	%R09------------------------------------------------------------
	\rowcolordark
	Inesperienza nei ruoli \\ RO9 &
	Effettuando la rotazione dei ruoli, i membri del team si trovano di fronte a nuovi compiti nei quali sono inesperti, dunque affrontano nuove difficoltà &
	I membri del team faticano ad adattarsi alle nuove mansioni, di conseguenza il ritmo del lavoro è rallentato.&
	Occorrenza: \textbf{Media} \\
	Pericolosità: \textbf{Bassa}
	\tabularnewline
	\rowcolordark\multicolumn{1}{p{0.17\textwidth}}{\centering\textbf{Piano di contingenza}}& 
	\multicolumn{3}{p{0.7775\textwidth}}{Chi ha già eseguito un lavoro affianca inizialmente chi lo esegue per la prima volta.}
	\tabularnewline
	
	%RI1------------------------------------------------------------
	\rowcolorlight
	Comunicazione interna \\ RI1 & 
	Nel corso dello sviluppo del progetto potrebbero verificarsi dei 
	momenti in cui uno più membri del gruppo siano irreperibili. &
	Tutti i componenti del team sono tenuti a segnalare eventuali momenti di 
	irreperibilità e organizzare i propri impegni al fine di poter presenziare 
	alle riunioni del gruppo. &
	Occorrenza: \textbf{Bassa} \\
	Pericolosità: \textbf{Alta}
	\tabularnewline
	\rowcolorlight\multicolumn{1}{p{0.17\textwidth}}{\centering\textbf{Piano di contingenza}}& 
	\multicolumn{3}{p{0.7775\textwidth}}{ Il gruppo ha predisposto molteplici 
	vie di comunicazione interna. Inoltre verranno organizzati incontri a 
	scadenze fissa per discutere dell'avanzamento del progetto.}
	\tabularnewline	
	
	%RI2------------------------------------------------------------
	\rowcolordark
	Comunicazione esterna \\ RI2 &
	Il proponente ha la propria 
	sede all'estero, di conseguenza le comunicazioni saranno più difficili. &
	Come per le comunicazioni interne, sono stati predisposti più canali di 
	comunicazione; le video conferenze con il proponente saranno organizzare 
	con il dovuto preavviso.&
	Occorrenza: \textbf{Bassa} \\
	Pericolosità: \textbf{Media}
	\tabularnewline
	\rowcolordark\multicolumn{1}{p{0.17\textwidth}}{\centering\textbf{Piano di contingenza}}& 
	\multicolumn{3}{p{0.7775\textwidth}}{Il gruppo provvederà a raggruppare 
	quesiti e segnalazioni per il proponente.}
	\tabularnewline	
	
	%RI3------------------------------------------------------------
	\rowcolorlight
	 Contrasti interni \\ RI3 &
	Come in qualsiasi gruppo, nel 
	corso delle varie attività potranno emergere contrasti e tensioni tra i vari componenti. &
	Ciascun membro del team si impegna a limitare tali tensioni e fare in 
	modo che esse non influiscano sul normale svolgersi delle attività. &
	Occorrenza: \textbf{Bassa} \\
	Pericolosità: \textbf{Media}
	\tabularnewline
	\rowcolorlight\multicolumn{1}{p{0.17\textwidth}}{\centering\textbf{Piano di contingenza}}& 
	\multicolumn{3}{p{0.7775\textwidth}}{Il \textit{responsabile} avrà la 
	funzione di 
	mediatore in tali controversie.}
	\tabularnewline	
	
	%RI4------------------------------------------------------------
	\rowcolordark
	Perdita di motivazione \\ RI4 &
	Dopo un lungo periodo di tempo, i membri del gruppo possono tendere a perdere la motivazione necessaria a portare avanti il progetto. &
	Scarsa presenza frontale o remota, bassa produttività. &
	Occorrenza: \textbf{Bassa} \\
	Pericolosità: \textbf{Media}
	\tabularnewline
	\rowcolordark\multicolumn{1}{p{0.17\textwidth}}{\centering\textbf{Piano di contingenza}}& 
	\multicolumn{3}{p{0.7775\textwidth}}{Il gruppo si impegna a lavorare in modo abitudinario rispetto a giorni ed orari. Le abitudini fissano le azioni e suppliscono alla mancanza di motivazione.}
	\tabularnewline
	
	%RI5------------------------------------------------------------
	\rowcolorlight
	Stress mentale \\ RI5 &
	Gli orari di lavoro eccessivi possono portare a frustrazione e forte disagio. Una piccola quantità di stress è normale, ma uno stress eccessivo è segno di un problema da risolvere. &
	Segnali fisici di stress, emotività instabile o assente. &
	Occorrenza: \textbf{Media} \\
	Pericolosità: \textbf{Media}
	\tabularnewline
	\rowcolorlight\multicolumn{1}{p{0.17\textwidth}}{\centering\textbf{Piano di contingenza}}&
	\multicolumn{3}{p{0.7775\textwidth}}{Alcuni dei membri trovano efficace fare attività fisica per rilasciare la tensione, modalità consigliata anche agli altri. In casi più gravi si affronta la questione parlandone con il gruppo, con un familiare o con un supporto psicologico (a tale proposito è stato individuato il servizio gratuito SAP, offerto dall'UniPd).} 
	\tabularnewline
	
	%RI6------------------------------------------------------------
	\rowcolordark
	Stato di malattia \\ RI6 &
	Uno o più membri del team possono ammalarsi. La malattia può influire sul lavoro del membro interessato. Il malato potrebbe contagiare gli altri membri. &
	Il membro malato comunica il proprio stato di poca salute. &
	Occorrenza: \textbf{Media} \\
	Pericolosità: \textbf{Media}
	\tabularnewline
	\rowcolordark\multicolumn{1}{p{0.17\textwidth}}{\centering\textbf{Piano di contingenza}}&
	\multicolumn{3}{p{0.7775\textwidth}}{Solo se la malattia impedisce di lavorare, il membro malato è tenuto a riprendere uno stato di salute ottimale. Il suo lavoro è redistribuito tra gli altri membri con pari ruolo se possibile, o comportando un cambio di ruolo altrimenti.} 
	\tabularnewline

		
	\end{longtable}
\counterwithin{table}{subsection}	
\renewcommand{\arraystretch}{1}
