\subsection*{\quad$G\quad$}
\subsubsection*{Gamification}
\index{Gamification}
Riutilizzo di concetti ed elementi tipici dei giochi in applicazioni di diverso contesto. Utilizzando questi principi si ottengono sia maggiore coinvolgimento degli utenti dell'applicazione, sia maggiore produttività ed organizzazione nel lavoro.

\subsubsection*{Ganache CLI}
\index{Ganache CLI}
Versione da riga di comando di Ganache che simula un client di Ethereum per sviluppare in modo veloce e sicuro. Fa parte della suite di strumenti di Truffle\glo.

\subsubsection*{Gantt}
\index{Gantt}
Il diagramma di Gantt è uno strumento di supporto alla gestione dei progetti. E' usato principalmente nelle attività di project management. L'asse orozzontale rappresenta l'arco temporale totale del progetto, suddiviso in fasi incrementali; l'asse verticale rappresenta le mansioni o attività che costituiscono il progetto.

\subsubsection*{Gas}
\index{Gas}
Il carburante di Ethereum. Ogni operazione eseguita in Ethereum\glosp è caratterizzata da un numero di unità Gas, che indica la quantità di lavoro necessaria per portare l'operazione a termine, e da un nodo della rete, che esegue le computazioni necessarie ad eseguire l'operazione. L'utente definisce un tasso di pagamento in Ether\glo/Gas. L'utente che ha svolto la computazione verrà ricompensato con la quantità di lavoro eseguita (Gas) moltiplicata per il tasso di conversione. Con questo sistema le operazioni con un tasso di conversione più elevato verranno eseguite con priorità maggiore nella rete, in quanto garantiscono una paga più elevata a parità di lavoro svolto.  

\subsubsection*{Gigacore}
\index{Gigacore}
Boilerplate\glosp per lo sviluppo di applicazioni che fanno uso di React\glo, Redux\glosp e Sass\glo.

\subsubsection*{GitHub}
\index{GitHub}
Piattaforma online che permette l'utilizzo semplificato del sistema di versionamento Git. Funge da repository remota, ed è un ottimo strumento per collaborare con i propri compagni di progetto. Fornisce strumenti aggiuntivi rispetto a Git, tra cui issues, analytics e wiki integrata nel sistema.

\subsubsection*{GitLab}
\index{GitLab}
Sistema di gestione repository online, che integra servizi quali issue tracking e l'integrazione di pipeline volte a continuos delivery\glosp e continuos integration\glo.

\subsubsection*{Governo}
\index{Governo}
Termine usato per riferirsi a tutti gli enti governativi interessati nella gestione dell'IVA.

