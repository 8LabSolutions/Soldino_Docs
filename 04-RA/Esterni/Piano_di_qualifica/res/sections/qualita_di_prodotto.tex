\section{Qualità di prodotto}
Per valutare la qualità del prodotto il gruppo ha deciso di far riferimento allo standard ISO/IEC 9126\glosp{} che definisce le caratteristiche di cui tener conto per produrre un prodotto di buona qualità. Le caratteristiche sono descritte attraverso dei parametri, che ne quantificano il grado di raggiungimento. Di seguito vengono citate le voci che il gruppo ha ritenuto rilevanti nel contesto del progetto.
	\subsection{Funzionalità}
	Capacità del prodotto di fornire funzioni che riescano a soddisfare tutti i requisiti, sia espliciti che impliciti, presenti nell'Analisi dei Requisiti.
		\subsubsection{Obiettivi}
		\begin{itemize}
			\item \textbf{Appropriatezza}: il prodotto deve mettere a disposizione un insieme di funzioni conformi agli obiettivi richiesti;
			\item \textbf{Accuratezza}: il prodotto deve fornire risultati attesti con il grado di precisione richiesto;
			\item \textbf{Conformità}: il prodotto deve aderire a determinati standard. %PLACEHOLDER è forse troppo ovvio?
		\end{itemize}
		\subsubsection{Metriche}
			\paragraph{Completezza dell'implementazione}\mbox{}\\
			La completezza del prodotto e il rispetto dei requisiti viene indicato da una percentuale.
			\begin{itemize}
			\item misurazione: si calcola con la seguente formula: \\
			\centerline { C = (1 - \(\frac{N\textsubscript{FNI}}{N\textsubscript{FI}} \))$ \cdot  100$ } \\
			dove N\textsubscript{FNI} indica il numero di funzionalità non implementate e N\textsubscript{FI} indica il numero di funzionalità individuate dall'analisi;
			\item valore preferibile: 100\%;
			\item valore accettabile: 100\%.
			\end{itemize}
			\paragraph{Passed Test Cases Percentage}\mbox{}\\
			Assicurare il superamento dei test è fondamentale per poter verificare la corretta implementazione delle funzionalità previste dai requisiti.
			\begin{itemize}
				\item misurazione: percentuale superamento dei Test (PTCP);
				\item valore preferibile: 90\% $\leq$ PTCP $\leq$ 100\%;
				\item valore accettabile: 80\% $\leq$ PTCP $\leq$ 90\%.
			\end{itemize}
			\paragraph{Failed Test Cases Percentage}\mbox{}\\
			Assicurare che il numero di test falliti sia inferiore a una certa soglia garantisce la corretta implementazione delle funzionalità previste.
			\begin{itemize}
				\item misurazione: percentuale fallimento dei Test (FTCP);
				\item valore preferibile: 0\% $\leq$ FTCP $\leq$ 10\%;
				\item valore accettabile: 10\% $\leq$ FTCP $\leq$ 20\%.
			\end{itemize}
	\subsection{Affidabilità}
	Capacità del prodotto di mantenere prestazioni elevate anche in caso di situazioni anomale o critiche.
		\subsubsection{Obiettivi}
		\begin{itemize}
			\item maturità: il prodotto deve evitare che si verifichino errori e malfunzionamenti;
			\item tolleranza agli errori: il prodotto mantiene alte prestazioni anche in caso di malfunzionamenti o di un uso scorretto.
		\end{itemize}
		\subsubsection{Metriche}
			\paragraph{Densità errori}\mbox{}\\
			L'abilità del prodotto di resistere a malfunzionamenti viene indicata con una percentuale.
			\begin{itemize}
			\item misurazione: si calcola con la seguente formula: \\
			\centerline{ M =  \(\frac{N\textsubscript{ER}}{N\textsubscript{TE}} \)$ \cdot 100$ }
			dove N\textsubscript{ER} indica il numero di errori rilevati durante il testing e N\textsubscript{TE} indica il numero di test eseguiti;
			\item valore preferibile: 0\%;
			\item valore accettabile: $\leq$ 10\%.
			\end{itemize}
	\subsection{Efficienza}
	Capacità del prodotto di ottenere risultati con il minimo consumo di risorse.
		\subsubsection{Obiettivi}
		\begin{itemize}
			\item utilizzo di risorse: il prodotto deve consumare una quantità di risorse proporzionata alle funzionalità che offre.
		\end{itemize}
		\subsubsection{Metriche}
			\paragraph{Deployment cost}
			Il costo di deployment degli smart contracts\glosp.
			\begin{itemize}
				\item misurazione: valore decimale (gas);
				\item valore preferibile: $ \leq 1.500.000$ gas;
				\item valore accettabile: $ \leq 3.000.000$ gas.
			\end{itemize}
			\paragraph{Call cost}
			Il costo di chiamata alle funzioni degli smart contracts\glosp.
			\begin{itemize}
				\item misurazione: valore decimale (gas);
				\item valore preferibile: $ \leq 200.000$ gas;
				\item valore accettabile: $ \leq 400.000$ gas.
			\end{itemize}
	
	\subsection{Usabilità}
	Capacità del prodotto di essere di facile comprensione e utilizzo da parte degli utenti.
		\subsubsection{Obiettivi}
		\begin{itemize}
			\item comprensibilità: l'utente deve essere in grado di comprendere le funzionalità offerte dal prodotto e ad utilizzarle;
			\item apprendibilità: l'utente deve poter imparare facilmente ad utilizzare il prodotto;
			\item attrattività: il prodotto deve essere piacevole da usare.
		\end{itemize}
		\subsubsection{Metriche}
			\paragraph{Facilità di utilizzo}\mbox{}\\
			La facilità con cui l'utente raggiunge ciò che vuole viene rappresentata tramite il numero di click necessari per arrivare al contenuto desiderato.
			\begin{itemize}
			\item misurazione: numero di click per completare il checkout di un acquisto;
			\item valore preferibile: $\leq$ 10;
			\item valore accettabile: $\leq$ 15.
			\end{itemize}
			\paragraph{Facilità di apprendimento}\mbox{}\\
			La facilità con cui l'utente riesce ad imparare ad usare le funzionalità del prodotto viene rappresentata tramite il tempo medio che serve per comprenderle.
			\begin{itemize}
			\item misurazione: tempo necessario a completare il checkout di un acquisto;
			\item valore preferibile: $\leq$ 3 minuti;
			\item valore accettabile: $\leq$ 5 minuti.
			\end{itemize}
		
	\subsection{Manutenibilità}
	Capacità del prodotto di essere modificato, includendo correzioni, miglioramenti o adattamenti.
		\subsubsection{Obiettivi}
		\begin{itemize}
			\item analizzabilità: facilità con la quale è possibile analizzare il codice per localizzare un errore;
			\item modificabilità: capacità del prodotto di permettere l'implementazione di una modifica.
		\end{itemize}
		\subsubsection{Metriche}
			\paragraph{Facilità di comprensione}\mbox{}\\
			La facilità con cui è possibile comprendere cosa fa il codice. Equivale al rapporto tra numero di linee di commento su numero di linee totali di codice. % sarebbe la CCR
			\begin{itemize}
			\item misurazione: si può calcolare con la seguente formula: \\
			$$ R = \frac{N\textsubscript{LCOM}}{N\textsubscript{LCOD}}  $$
		
			dove N\textsubscript{LCOM} indica le linee di commento e N\textsubscript{LCOD} indica le linee di codice;
			\item valore preferibile: $0.10 \leq R \leq 0.20$;
			\item valore accettabile: $R \geq 0.10$ .
			\end{itemize}
			\paragraph{Semplicità delle funzioni}
			La facilità di un metodo può essere rappresentata dal numero di parametri per metodo: meno parametri ha una funzione più è semplice e intuitiva.
			\begin{itemize}
			\item misurazione: numero di parametri per metodo;
			\item valore preferibile $\leq$ 3;
			\item valore accettabile $\leq$ 6.
			\end{itemize}
			\paragraph{Semplicità delle classi}\mbox{}\\
			La facilità di una classe può essere rappresentata dal numero di metodi per classe: una classe con pochi metodi ha uno scopo ben preciso e facilmente comprensibile.
			\begin{itemize}
			\item misurazione: numero di metodi per classe;
			\item valore preferibile $\leq$ 8;
			\item valore accettabile $\leq$ 15.
			\end{itemize}
			\paragraph{Structural fan-in}\mbox{}\\ 
			Indica quante componenti utilizzano un dato modulo. Un alto	valore indica un alto riuso della componente.
			\begin{itemize}
				\item misurazione: conteggio delle componenti;
				\item valore preferibile: $\geq$ 1;
				\item valore accettabile: $\geq$ 0.
			\end{itemize}
			\paragraph{Structural fan-out}\mbox{}\\
			 Indica quante componenti vengono utilizzate dalla componente in esame. Un alto valore indica un alto accoppiamento della componente.
			\begin{itemize}
				\item misurazione: conteggio delle componenti;
				\item valore preferibile: = 0;
				\item valore accettabile: $\leq$ 6.
			\end{itemize}
			\paragraph{SLOC: Source Lines of Code}\mbox{}\\
			Misura le righe (fisiche, non logiche) totali di codice di cui il progetto è composto. Maggiore è la dimensione del progetto, più è difficile manutenerlo
			\begin{itemize}
				\item misurazione: numero di righe di codice;
				\item valore preferibile: $\leq5000$;
				\item valore accettabile: $\leq10000$.
			\end{itemize}
		
\pagebreak

\subsection{Tabella riassuntiva delle metriche adottate}
\begin{longtable}{ >{\centering}p{0.3\textwidth}
		>{\centering}p{0.6\textwidth}}
	\rowcolor{white}\caption{Tabella riassuntiva delle metriche adottate}\\
	\rowcolorhead
	\textbf{\color{white}Caratteristiche}
	& \textbf{\color{white}Metriche}
	\tabularnewline 	
	\endfirsthead
	\rowcolor{white}\caption[]{(continua)} \\
	\rowcolorhead 
	\textbf{\color{white}Caratteristiche}
	& \textbf{\color{white}Metriche} 
	\tabularnewline
	\endhead
	
	Funzionalità & Completezza dell'implementazione\\ Passed Test Cases Percentage \\ Failed Test Cases Percentage
	\tabularnewline
	\rowcolordark
	Affidabilità & Densità errori 
	\tabularnewline
	\rowcolorlight
	Usabilità & Facilità di utilizzo\\ Facilità di apprendimento
	\tabularnewline
	\rowcolordark
	Manutenibilità & Facilità di comprensione\\
	Semplicità delle funzioni \\Semplicità delle classi \\ Structural fan-in \\Structural fan-out \\Source Lines of Code (SLOC)
	\tabularnewline
\end{longtable}
