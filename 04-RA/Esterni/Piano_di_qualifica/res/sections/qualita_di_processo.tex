\section{Qualità di processo}
Per ricercare qualità nello svolgimento del progetto si adoperano dei processi. Inizialmente, tali processi sono stati scelti tra quelli proposti nello standard ISO/IEC/IEEE 12207:1995. Successivamente sono stati semplificati o adattati secondo le esigenze: in alcuni casi (analisi dei requisiti, progettazione dell'architettura, progettazione di dettaglio, pianificazione) le attività sono trattate alla pari di processi, vista la loro importanza nell'esito del progetto.\newline 
Il risultato sono i processi esposti a seguito.

%\subsection{Processo di Costruzione del Software}. Opzione per il futuro.
	\subsection{Processi di sviluppo}
	\subsubsection{Analisi dei requisiti}
		\paragraph{Metriche}
			\paragraph*{PROS: Percentuale di requisiti obbligatori soddisfatti}\mbox{}\\
			Indica appunto la percentuale di requisiti obbligatori soddisfatti.
			\begin{itemize}
				\item misurazione: valore percentuale: $ PROS = \frac{requisiti\ obbligatori\ soddisfatti}{requisiti\ obbligatori\ totali}$;
				\item valore preferibile: $100\%$;
				\item valore accettabile: $100\%$.
			\end{itemize}
			
	\subsubsection{Progettazione di dettaglio}
		\paragraph{Metriche}
			\paragraph*{CBO: Accoppiamento tra le classi di oggetti} \mbox{}\\
			Una classe è accoppiata ad una seconda se usa metodi o variabili definiti nella seconda. 
			\begin{itemize}
				\item misurazione: valore intero: $CBO$;
				\item valore preferibile: $0 \leq CBO \leq 1$;
				\item valore accettabile: $0 \leq CBO \leq 6$.
			\end{itemize}
		
	\subsubsection{Codifica}
		\paragraph{Metriche}
			\paragraph*{Profondità della gerarchia} \mbox{}\\
			La profondità del sito. Un sito per essere facile da utilizzare non deve essere troppo profondo.
			\begin{itemize}
				\item misurazione: livello di profondità della pagine;
				\item valore preferibile: $\leq$ 4;
				\item valore accettabile: $\leq$ 7.
			\end{itemize}
		
			\paragraph*{Rapporto di linee di codice per linee di commento}
			 \mbox{}\\
			Il rapporto tra le linee di codice
			e le linee di commento, escludendo le righe vuote.
			\begin{itemize}
				\item misurazione: $\frac{\mbox{linee di codice}}{\mbox{linee di commento}}$;
				\item valore preferibile: $\geq 0.30$;
				\item valore accettabile: $\geq 0.25$.
			\end{itemize}
		
			\paragraph*{Numero di attributi per contratto}
			\mbox{}\\
			Considera il numero totale di attributi presenti
			all'interno di un contratto.
			\begin{itemize}
				\item misurazione: numero intero;
				\item valore preferibile: $\leq 10$;
				\item valore accettabile: $\leq 15$.
			\end{itemize}
		
			\paragraph*{Livello di annidamento}
			\mbox{}\\
			Livello di annidamento nei vari metodi tenendo conto della
			presenza di strutture di controllo annidate. Il numero di strutture di controllo annidate viene indicato con $x$.
			\begin{itemize}
				\item misurazione: numero intero;
				\item valore preferibile: $ 1 \leq x \leq3$ ;
				\item valore accettabile: $1 \leq x\leq 7$.
			\end{itemize}
	
		
			\paragraph*{Numero di parametri per metodo}
			\begin{itemize}
				\item misurazione: numero intero;
				\item valore preferibile: $ \leq 3$ ;
				\item valore accettabile: $ \leq 5$.
			\end{itemize}
			
\subsection{Processi di supporto}
	\subsubsection{Pianificazione}
		\paragraph{Metriche}			
			\paragraph*{EAC: Estimate At Completion}\mbox{}\\
			Revisione del valore stimato per la realizzazione del progetto, composto da: costo sostenuto più stima del costo ancora da sostenere. In altre parole si tratta del preventivo rivisto allo stato corrente del progetto.
			\begin{itemize}
				\item  misurazione: numero intero;
				\item  valore preferibile: $ EAC \leq preventivo$;
				\item  valore accettabile: $ preventivo -5\% \leq EAC \leq preventivo + 5\%$. 
			\end{itemize}
			
			\paragraph*{VAC: Variance At Completion}\mbox{}\\
			Indica la variazione tra BAC ed EAC e la spesa effettivamente sostenuta.
			\begin{itemize}
				\item  misurazione: percentuale: $\frac{preventivo - EAC}{100}$;
				\item  valore preferibile: $\geq 0$;
				\item  valore accettabile: $\geq 0$.
			\end{itemize}
		
			\paragraph*{EV: Earned Value}\mbox{}\\
			Metrica di utilità per il calcolo di $SV$ e $CV$ (spiegate successivamente). Si tratta del valore del lavoro fatto fino al momento del calcolo; corrisponde al denaro guadagnato fino a quel momento.
			\begin{itemize}
				\item  misurazione: $preventivo \cdot \%\ di\ lavoro\ completato\ $;
				\item  valore preferibile: $ \geq 0$;
				\item  valore accettabile: $ \geq 0$.
			\end{itemize}
		
			\paragraph*{AC: Actual Cost}\mbox{}\\
			Il denaro speso fino al momento del calcolo. Corrisponde al consuntivo di periodo.
			\begin{itemize}
				\item  misurazione: numero intero;
				\item  valore preferibile: $0 \leq AC < PV$;
				\item  valore accettabile: $0 \leq AC \leq budget\ totale\ $.
			\end{itemize}
		
			\paragraph*{PV: Planned Value}\mbox{}\\
			Metrica di utilità per il calcolo di $SV$ e $CV$ (spiegate successivamente). Si tratta del costo pianificato in euro per realizzare le attività di progetto alla data corrente. Corrisponde al preventivo di periodo.
			\begin{itemize}
				\item  misurazione: $preventivo \cdot \%\ di\ lavoro\ pianificato\ $;
				\item  valore preferibile: $ \geq 0$;
				\item  valore accettabile: $ \geq 0$.
			\end{itemize}			
			\paragraph*{SV: Schedule Variance}\mbox{}\\
			Esprime lo stato di anticipo o ritardo nello svolgimento del progetto rispetto alla pianificazione.
			Se $SV > 0$ significa che il progetto viene prodotto più velocemente a quanto pianificato; viceversa se negativo.
			\begin{itemize}
				\item misurazione: $SV = EV - PV$
				\item valore preferibile: $ > 0$;
				\item valore accettabile: $\geq 0$.
			\end{itemize}
			\paragraph*{CV: Cost Variance}\mbox{}\\
			Differenza tra il costo del lavoro effettivamente completato e quello pianificato. Una CV positiva indica che si sta rispettando il budget.
			\begin{itemize}
				\item misurazione: $CV = EV - AC$;
				\item valore preferibile: $ > 0$;
				\item valore accettabile: $ \geq 0$.
			\end{itemize}

	\subsubsection{Verifica}
		\paragraph{Metriche}
			\paragraph*{CC: Code Coverage}\mbox{}\\
				Indica il numero di righe di codice percorse dai test durante la loro esecuzione. Per linee di codice totali si intende tutte quelle appartenenti all'unità in fase di test.\\
				A granularità più fine, il CC si suddivide in quattro aspetti da misurare:
				\begin{itemize}
					\item \textbf{Statement coverage}: misura la percentuale di statement coperti dai test;
					\item \textbf{Branch coverage}: misura la percentuale di ramificazioni del codice coperti dai test;
					\item \textbf{Function coverage}: misura la percentuale di funzioni coperte dai test;
					\item \textbf{Line coverage}: misura la percentuale di linee coperte dai test.
				\end{itemize}
				Ognuno dei quattro aspetti rispetta lo stesso metodo e intervallo di misurazione.
				\begin{itemize}
					\item misurazione: valore percentuale: $CC = \frac{linee\ di\ codice\ percorse}{linee\ di\ codice\ totali}$;
					\item valore preferibile: $100\%$;
					\item valore accettabile: $75\%$.
				\end{itemize}
	\subsubsection{Documentazione}
		\paragraph{Metriche}
			\paragraph*{Indice di Gulpease}\mbox{}\\
			Indice della leggibilità del testo. Valuta la lunghezza delle parole e delle frasi rispetto al numero totale di lettere. 
			\begin{itemize}
				\item misurazione: valore intero da 0 a 100:\newline 	
				$I_G = 89+ \frac{(300 \cdot numero\ di\ frasi - 10 \cdot numero\ di\ lettere)}{numero\ di\ parole}$;	
				\item valore preferibile: $80 < I_G < 100$;
				\item valore accettabile: $40 < I_G < 100$.
			\end{itemize}

	\begin{comment}		
	\subsection{Processi organizzativi}
	
	\subsubsection{Gestione della qualità}
		\paragraph{Metriche}
			\subparagraph{PMS: Percentuale di metriche soddisfatte}
			La percentuale di metriche soddisfatte valuta quante metriche raggiungono soglie accettabili sul numero totale delle metriche calcolate. Una bassa percentuale di soddisfazione può indicare poca qualità, metriche inadeguate o mancata correttezza nel calcolo.
			\begin{itemize}
				\item misurazione: $\frac{numero\ di\ metriche\ soddisfatte}{numero\ di\ metriche\ totali} $;
				\item valore preferibile: $ \geq 80\%$;
				\item valore accettabile: $ \geq 60\%$.
			\end{itemize}
	\end{comment}