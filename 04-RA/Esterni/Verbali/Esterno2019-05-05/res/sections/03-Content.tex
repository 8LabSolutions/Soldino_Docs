\section{Diario della riunione}
Durante l'incontro abbiamo svolto i punti seguenti:
\begin{itemize}
	\item c'è stato uno scambio di opinioni tra il gruppo e i proponenti riguardo lo 
	standard EIP-712: come sospettato dal gruppo e confermato dal proponente, esso 
	risulta già integrato nella nuova versione di MetaMask, la quale è in uso nel 
	progetto. Dunque se MetaMask viene supportato, allora lo standard risulta 	
	rispettato;	
	\item abbiamo mostrato nuovamente il prodotto in esecuzione illustrando i nuovi 
	aggiornamenti, come il rinnovamento del sistema di notifiche e del riepilogo degli 
	ordini; il prodotto ora risulta più stabile, usabile e attrattivo;
	\item i proponenti ci hanno parlato dell'importanza di avere un riferimento per 
	gli errori frequenti che possono accadere durante la codifica e l'esecuzione del 
	prodotto, così abbiamo concordato di riportarli nel readme su GitHub;
	\item infine, i proponenti hanno richiesto le credenziali dell'account governativo
	per poter usufruire del prodotto che verrà ospitato sulla rete di testing Ropsten.
\end{itemize}


\hspace{3cm}

\section{Riepilogo delle decisioni}

	%\renewcommand{\arraystretch}{1.5}
	\rowcolors{2}{pari}{dispari}
	
	\begin{longtable}{ >{\centering}p{0.20\textwidth} >{}p{0.70\textwidth}}
		\caption{Decisioni della riunione interna del 2019-05-05}\\	
		\rowcolorhead
		\textbf{\color{white}Codice} 
		& \centering\textbf{\color{white}Decisione} 
		\tabularnewline 
		
		\endfirsthead
		VE\_9.1 & Essendo EIP-712 già integrato in MetaMask, non si applicheranno ulteriori misure per supportarlo.
		\tabularnewline
		
		VE\_9.2 & Il gruppo si prepara a migrare il prodotto su Ropsten.
		\tabularnewline
		
		VE\_9.3 & Gli errori frequenti saranno riportati nel readme presente nella directory di GitHub.
	\end{longtable}
	




