\section{Procedure per PragmaDB}	
	% PRAGMA
	Per eseguire le seguenti operazioni è necessario eseguire la login al sito 
	\\
	\centerline{\url{http://ec2-52-47-35-145.eu-west-3.compute.amazonaws.com/PragmaDB/PHP/}}
	in cui ogni membro del gruppo deve inserire username e password.
	
	\subsection{Aggiungere un attore} \mbox{}\\ \mbox{}\\
	Dopo aver eseguito l'autenticazione:
	\begin{itemize}
		\item selezionare \texttt{Attori} nel pannello posto a sinistra;
		\item selezionare \texttt{Inserisci Attore} nel panello posto a destra;
		\item compilare il form per l'inserimento di un nuovo attore, inserendo 
		il nome e la descrizione relativi ad esso;
		\item cliccare sul tasto \texttt{Conferma} per aggiungere il nuovo 
		attore 
		creato oppure \texttt{Cancella} per tornare indietro.	
	\end{itemize}
	
	\subsection{Modificare un attore} \mbox{}\\ \mbox{}\\
	Dopo aver eseguito l'autenticazione:
	\begin{itemize}
		\item selezionare \texttt{Attori} nel pannello posto a sinistra;
		\item selezionare \texttt{Modifica} dalla riga della tabella dell'attore
		che si vuole modificare;
		\item modificare i campi del form dell'attore selezionato;
		\item cliccare sul tasto \texttt{Modifica} per salvare le modifiche 
		effettuate
		oppure \texttt{Cancella} per tornare indietro.	
	\end{itemize}
	
	\subsection{Eliminare un attore} \mbox{}\\ \mbox{}\\
	Dopo aver eseguito l'autenticazione:
	\begin{itemize}
		\item selezionare \texttt{Attori} nel pannello posto a sinistra;
		\item selezionare \texttt{Elimina} dalla riga della tabella dell'attore
		che si vuole eliminare;
		\item cliccare sul tasto \texttt{Elimina} per confermare la rimozione
		oppure \texttt{Cancella} per tornare indietro.	
	\end{itemize}
	
	\subsection{Aggiungere una classe} \mbox{}\\ \mbox{}\\
	Dopo aver eseguito l'autenticazione:
	\begin{itemize}
		\item selezionare \texttt{Classi} nel pannello posto a sinistra;
		\item selezionare \texttt{Inserisci Classe} nel panello posto a destra;
		\item compilare il form per l'inserimento di una nuova classe;
		\item cliccare sul tasto \texttt{Inserisci} per aggiungere la nuova 
		classe 
		creata oppure \texttt{Cancella} per tornare indietro.	
	\end{itemize}
	
	\subsection{Modificare una classe} \mbox{}\\ \mbox{}\\
	Dopo aver eseguito l'autenticazione:
	\begin{itemize}
		\item selezionare \texttt{Classi} nel pannello posto a sinistra;
		\item selezionare \texttt{Modifica} dalla riga della tabella della 
		classe
		che si vuole modificare;
		\item modificare i campi del form della classe selezionata;
		\item cliccare sul tasto \texttt{Modifica} per salvare le modifiche 
		effettuate
		oppure \texttt{Cancella} per tornare indietro.
	\end{itemize}
	
	\subsection{Eliminare una classe} \mbox{}\\ \mbox{}\\
	Dopo aver eseguito l'autenticazione:
	\begin{itemize}
		\item selezionare \texttt{Classi} nel pannello posto a sinistra;
		\item selezionare \texttt{Eliminare} dalla riga della tabella della 
		classe
		che si vuole eliminare;
		\item cliccare sul tasto \texttt{Elimina} per confermare la rimozione 
		della classe
		oppure \texttt{Annulla} per tornare indietro.
	\end{itemize}
	
	\subsection{Aggiungere un attributo alla classe} \mbox{}\\ \mbox{}\\
	Dopo aver eseguito l'autenticazione:
	\begin{itemize}
		\item selezionare \texttt{Classi} nel pannello posto a sinistra;
		\item selezionare \texttt{Attributi} dalla riga della tabella della 
		classe
		a cui si vuole aggiungere un attributo;
		\item selezionare \texttt{Inserisci Attributo} dal pannello posto a 
		destra;
		\item compilare il form per l'inserimento di un nuovo attributo;
		\item cliccare sul tasto \texttt{Inserisci} per aggiungere il nuovo 
		attributo 
		creato oppure \texttt{Cancella} per tornare indietro.	
	\end{itemize}
	
	\subsection{Modificare un attributo alla classe} \mbox{}\\ \mbox{}\\
	Dopo aver eseguito l'autenticazione:
	\begin{itemize}
		\item selezionare \texttt{Classi} nel pannello posto a sinistra;
		\item selezionare \texttt{Attributi} dalla riga della tabella della 
		classe
		a cui si vuole modificare un attributo;\
		\item selezionare \texttt{Modifica} dalla riga della tabella 
		dell'attributo
		che si vuole modificare;
		\item modificare i campi del form dell'attributo selezionato;
		\item cliccare sul tasto \texttt{Modifica} per salvare le modifiche 
		effettuate
		oppure \texttt{Cancella} per tornare indietro.
	\end{itemize}
	
	\subsection{Eliminare un attributo alla classe} \mbox{}\\ \mbox{}\\
	Dopo aver eseguito l'autenticazione:
	\begin{itemize}
		\item selezionare \texttt{Classi} nel pannello posto a sinistra;
		\item selezionare \texttt{Attributi} dalla riga della tabella della 
		classe
		a cui si vuole eliminare un attributo;\
		\item selezionare \texttt{Eliminare} dalla riga della tabella 
		dell'attributo 
		che si vuole eliminare;
		\item cliccare sul tasto \texttt{Elimina} per confermare la rimozione 
		dell'attributo
		oppure \texttt{Annulla} per tornare indietro.
	\end{itemize}
	
	\subsection{Aggiungere un metodo alla classe} \mbox{}\\ \mbox{}\\
	Dopo aver eseguito l'autenticazione:
	\begin{itemize}
		\item selezionare \texttt{Classi} nel pannello posto a sinistra;
		\item selezionare \texttt{Metodi} dalla riga della tabella della classe
		a cui si vuole aggiungere un metodo;
		\item selezionare \texttt{Inserisci Metodo} dal pannello posto a destra;
		\item compilare il form per l'inserimento di un nuovo metodo;
		\item cliccare sul tasto \texttt{Inserisci} per aggiungere il nuovo 
		metodo 
		creato oppure \texttt{Cancella} per tornare indietro.	
	\end{itemize}
	
	\subsection{Modificare un metodo alla classe} \mbox{}\\ \mbox{}\\
	Dopo aver eseguito l'autenticazione:
	\begin{itemize}
		\item selezionare \texttt{Classi} nel pannello posto a sinistra;
		\item selezionare \texttt{Metodi} dalla riga della tabella della classe
		a cui si vuole modificare un metodo;
		\item selezionare \texttt{Inserisci Metodo} dal pannello posto a destra;
		\item modificare i campi del form del metodo selezionato;
		\item cliccare sul tasto \texttt{Modifica} per salvare le modifiche 
		effettuate
		oppure \texttt{Cancella} per tornare indietro.	
	\end{itemize}
	
	\subsection{Eliminare un metodo alla classe} \mbox{}\\ \mbox{}\\
	Dopo aver eseguito l'autenticazione:
	\begin{itemize}
		\item selezionare \texttt{Classi} nel pannello posto a sinistra;
		\item selezionare \texttt{Attributi} dalla riga della tabella della 
		classe
		a cui si vuole eliminare un metodo;\
		\item selezionare \texttt{Eliminare} dalla riga della tabella del metodo
		che si vuole eliminare;
		\item cliccare sul tasto \texttt{Elimina} per confermare la rimozione 
		del metodo
		oppure \texttt{Annulla} per tornare indietro.
	\end{itemize}
	
	\subsection{Aggiungere un requisito alla classe} \mbox{}\\ \mbox{}\\
	Dopo aver eseguito l'autenticazione:
	\begin{itemize}
		\item selezionare \texttt{Classi} nel pannello posto a sinistra;
		\item selezionare \texttt{Requisiti} dalla riga della tabella della 
		classe
		a cui si vuole aggiungere un requisito;
		\item selezionare \texttt{Inserisci Requisito} dal pannello posto a 
		destra;
		\item compilare il form per l'inserimento di un nuovo metodo;
		\item cliccare sul tasto \texttt{Inserisci} per aggiungere il nuovo 
		requisito 
		creato oppure \texttt{Cancella} per tornare indietro.	
	\end{itemize}
	
	\subsection{Modificare un requisito alla classe} \mbox{}\\ \mbox{}\\
	Dopo aver eseguito l'autenticazione:
	\begin{itemize}
		\item selezionare \texttt{Classi} nel pannello posto a sinistra;
		\item selezionare \texttt{Requisiti} dalla riga della tabella della 
		classe
		a cui si vuole modificare un requisito;
		\item selezionare \texttt{Inserisci Requisito} dal pannello posto a 
		destra;
		\item modificare i campi del form del requisito selezionato;
		\item cliccare sul tasto \texttt{Modifica} per salvare le modifiche 
		effettuate
		oppure \texttt{Cancella} per tornare indietro.
	\end{itemize}
	
	\subsection{Eliminare un requisito alla classe} \mbox{}\\ \mbox{}\\
	Dopo aver eseguito l'autenticazione:
	\begin{itemize}
		\item selezionare \texttt{Classi} nel pannello posto a sinistra;
		\item selezionare \texttt{Requisiti} dalla riga della tabella della 
		classe
		a cui si vuole eliminare un requisito;\
		\item selezionare \texttt{Eliminare} dalla riga della tabella del 
		requisito
		che si vuole eliminare;
		\item cliccare sul tasto \texttt{Elimina} per confermare la rimozione 
		del requisito
		oppure \texttt{Annulla} per tornare indietro.
	\end{itemize}
	
	\subsection{Aggiungere una fonte} \mbox{}\\ \mbox{}\\
	Dopo aver eseguito l'autenticazione:
	\begin{itemize}
		\item selezionare \texttt{Fonti} nel pannello posto a sinistra;
		\item selezionare \texttt{Inserisci Fonte} dal pannello posto a destra;
		\item compilare il form per l'inserimento di una nuova fonte;
		\item cliccare sul tasto \texttt{Inserisci} per aggiungere la nuova 
		fonte 
		creata oppure \texttt{Cancella} per tornare indietro.	
	\end{itemize}
	
	\subsection{Modificare una fonte} \mbox{}\\ \mbox{}\\
	Dopo aver eseguito l'autenticazione:
	\begin{itemize}
		\item selezionare \texttt{Fonti} nel pannello posto a sinistra;
		\item selezionare \texttt{Modifica} dalla riga della tabella della fonte
		che si vuole modificare;
		\item modificare i campi del form della fonte selezionata;
		\item cliccare sul tasto \texttt{Modifica} per salvare le modifiche 
		effettuate
		oppure \texttt{Cancella} per tornare indietro.
	\end{itemize}
	
	\subsection{Eliminare una fonte} \mbox{}\\ \mbox{}\\\\
	Dopo aver eseguito l'autenticazione:
	\begin{itemize}
		\item selezionare \texttt{Fonti} nel pannello posto a sinistra;
		\item selezionare \texttt{Elimina} dalla riga della tabella della fonte 
		che si vuole eliminare;\
		\item cliccare sul tasto \texttt{Elimina} per confermare la rimozione 
		della fonte
		oppure \texttt{Annulla} per tornare indietro.
	\end{itemize}
	
	\subsection{Aggiungere un package} \mbox{}\\ \mbox{}\\
	Dopo aver eseguito l'autenticazione:
	\begin{itemize}
		\item selezionare \texttt{Package} nel pannello posto a sinistra;
		\item selezionare \texttt{Inserisci Package} dal pannello posto a 
		destra;
		\item compilare il form per l'inserimento di un nuovo package;
		\item cliccare sul tasto \texttt{Inserisci} per aggiungere il nuovo 
		package
		creato oppure \texttt{Cancella} per tornare indietro.	
	\end{itemize}
	
	\subsection{Modificare un package} \mbox{}\\ \mbox{}\\
	Dopo aver eseguito l'autenticazione:
	\begin{itemize}
		\item selezionare \texttt{Package} nel pannello posto a sinistra;
		\item selezionare \texttt{Modifica} dalla riga della tabella del package
		che si vuole modificare;
		\item modificare i campi del form del package selezionato;
		\item cliccare sul tasto \texttt{Modifica} per salvare le modifiche 
		effettuate
		oppure \texttt{Cancella} per tornare indietro.
	\end{itemize}
	
	\subsection{Eliminare un package} \mbox{}\\ \mbox{}\\
	Dopo aver eseguito l'autenticazione:
	\begin{itemize}
		\item selezionare \texttt{Package} nel pannello posto a sinistra;
		\item selezionare \texttt{Elimina} dalla riga della tabella del package 
		che si vuole eliminare;\
		\item cliccare sul tasto \texttt{Elimina} per confermare la rimozione 
		del package
		oppure \texttt{Annulla} per tornare indietro.
	\end{itemize}
	
	\subsection{Aggiungere un test} \mbox{}\\ \mbox{}\\
	Dopo aver eseguito l'autenticazione:
	\begin{itemize}
		\item selezionare \texttt{Test} nel pannello posto a sinistra;
		\item selezionare \texttt{Inserisci Test} dal pannello posto a destra;
		\item compilare il form per l'inserimento di un nuovo test;
		\item cliccare sul tasto \texttt{Inserisci} per aggiungere il nuovo test
		creato oppure \texttt{Cancella} per tornare indietro.	
	\end{itemize}
	
	\subsection{Modificare un test} \mbox{}\\ \mbox{}\\
	Dopo aver eseguito l'autenticazione:
	\begin{itemize}
		\item selezionare \texttt{Test} nel pannello posto a sinistra;
		\item selezionare \texttt{Modifica} dalla riga della tabella del test
		che si vuole modificare;
		\item modificare i campi del form del test selezionato;
		\item cliccare sul tasto \texttt{Modifica} per salvare le modifiche 
		effettuate
		oppure \texttt{Cancella} per tornare indietro.
	\end{itemize}
	
	\subsection{Eliminare un test} \mbox{}\\ \mbox{}\\
	Dopo aver eseguito l'autenticazione:
	\begin{itemize}
		\item selezionare \texttt{Test} nel pannello posto a sinistra;
		\item selezionare \texttt{Elimina} dalla riga della tabella del test
		che si vuole eliminare;\
		\item cliccare sul tasto \texttt{Elimina} per confermare la rimozione 
		del test
		oppure \texttt{Annulla} per tornare indietro.
	\end{itemize}		
	
	\subsection{Aggiungere un requisito} \mbox{}\\ \mbox{}\\
	Dopo aver eseguito l'autenticazione:
	\begin{itemize}
		\item selezionare \texttt{Requisiti} nel pannello posto a sinistra;
		\item selezionare \texttt{Inserisci Requisiti} dal pannello posto a 
		destra;
		\item compilare il form per l'inserimento di un nuovo requisito;
		\item cliccare sul tasto \texttt{Inserisci} per aggiungere il nuovo 
		requisito
		creato oppure \texttt{Cancella} per tornare indietro.	
	\end{itemize}
	
	\subsection{Modificare un requisito} \mbox{}\\ \mbox{}\\
	Dopo aver eseguito l'autenticazione:
	\begin{itemize}
		\item selezionare \texttt{Requisiti} nel pannello posto a sinistra;
		\item selezionare \texttt{Modifica} dalla riga della tabella del 
		requisito
		che si vuole modificare;
		\item modificare i campi del form del requisito selezionato;
		\item cliccare sul tasto \texttt{Modifica} per salvare le modifiche 
		effettuate
		oppure \texttt{Cancella} per tornare indietro.
	\end{itemize}
	
	\subsection{Eliminare un requisito} \mbox{}\\ \mbox{}\\
	Dopo aver eseguito l'autenticazione:
	\begin{itemize}
		\item selezionare \texttt{Requisiti} nel pannello posto a sinistra;
		\item selezionare \texttt{Elimina} dalla riga della tabella del 
		requisito 
		che si vuole eliminare;\
		\item cliccare sul tasto \texttt{Elimina} per confermare la rimozione 
		del requisito
		oppure \texttt{Annulla} per tornare indietro.
	\end{itemize}
	
	\subsection{Aggiungere un use case} \mbox{}\\ \mbox{}\\
	Dopo aver eseguito l'autenticazione:
	\begin{itemize}
		\item selezionare \texttt{Use Case} nel pannello posto a sinistra;
		\item selezionare \texttt{Inserisci Use Case} dal pannello posto a 
		destra;
		\item compilare il form per l'inserimento di un nuovo use case;
		\item cliccare sul tasto \texttt{Inserisci} per aggiungere il nuovo use 
		case
		creato oppure \texttt{Cancella} per tornare indietro.	
	\end{itemize}
	
	\subsection{Modificare un use case} \mbox{}\\ \mbox{}\\
	Dopo aver eseguito l'autenticazione:
	\begin{itemize}
		\item selezionare \texttt{Use Case} nel pannello posto a sinistra;
		\item selezionare \texttt{Modifica} dalla riga della tabella del use 
		case
		che si vuole modificare;
		\item modificare i campi del form del use case selezionato;
		\item cliccare sul tasto \texttt{Modifica} per salvare le modifiche 
		effettuate
		oppure \texttt{Cancella} per tornare indietro.
	\end{itemize}
	
	\subsection{Eliminare un use case} \mbox{}\\ \mbox{}\\
	Dopo aver eseguito l'autenticazione:
	\begin{itemize}
		\item selezionare \texttt{Use Case} nel pannello posto a sinistra;
		\item selezionare \texttt{Elimina} dalla riga della tabella del use 
		case 
		che si vuole eliminare;\
		\item cliccare sul tasto \texttt{Elimina} per confermare la rimozione 
		del use case
		oppure \texttt{Annulla} per tornare indietro.
	\end{itemize}