\subsection*{\quad$R\quad$}
\subsubsection*{Raiden}
\index{Raiden}
È una tecnologia off-chain che consente pagamenti quasi istantanei, a basso costo e scalabili. È stata ideata per interagire con la rete Ethereum\glosp e può essere utilizzata per le transazioni di token\glosp compatibili con lo standard ECR20\glo.

\subsubsection*{React}
\index{React}
Libreria open source per JavaScript per la creazione di interfacce grafiche e la gestione delle interazioni in ambito web.

\subsubsection*{Redmine}
\index{Redmine}
Piattaforma open source che permette la gestione di più progetti contemporaneamente, offrendo funzionalità di project management e issue tracking.

\subsubsection*{Redux}
\index{Redux}
Libreria open source JavaScript per la gestione degli stati di React\glo.

\subsubsection*{Repository}
\index{Repository}
In generale, locazione di salvataggio dei dati. Nei sistemi di versionamento è una struttura dati più complessa, contenente metadati e operazioni per maneggiarla.

\subsubsection*{REST}
\index{REST}
Representational State Transfer, stile architetturale atto alla creazione di servizi web. Si espongono operazioni per manipolare i servizi, appoggiandosi a protocolli web esistenti, tipicamente HTTP.

\subsubsection*{Rete Bayesiana}
\index{Rete Bayesiana}
Modello grafico probabilistico che rappresenta un insieme di variabili stocastiche con le loro dipendenze condizionali.

\subsubsection*{Ropsten}
\index{Ropsten}
Rete di test ufficiale per Ethereum per uso pubblico. Funziona in modo simile alla main net. In essa le operazioni di scrittura sono gratuite. Si usa per testare e trovare bug prima del rilascio sulla rete principale.

