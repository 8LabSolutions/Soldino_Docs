\subsection*{\quad$T\quad$}
\subsubsection*{Technology Baseline}
\index{Technology Baseline}
Definisce le tecnologie, i framework e le librerie selezionate per lo sviluppo del prodotto.

\subsubsection*{Telegram}
\index{Telegram}
Servizio di messaggistica istantanea  e VoIP cloud-based.

\subsubsection*{Testrpc}
\index{Testrpc}
Ambiente di test per Ethereum, permette di testare gli smart contract senza overhead. Una versione più recente che lo ha incorporato è Ganache CLI\glo.

\subsubsection*{Time Series}
\index{Time Series}
Insieme di variabili casuali ordinate rispetto al tempo, esprimendo quindi la dinamica di un certo fenomeno nel tempo.

\subsubsection*{Token}
\index{Token}
Rappresentazione di una particolare risorsa o utilità che opera su una blockchain\glo. Un token può rappresentare un qualsiasi bene commerciabile, da materie prime fino alle criptovalute. Si differenzia dagli alctoin per il fatto che usa una blockchain già esistente.

\subsubsection*{Travis}
\index{Travis}
Servizio di continuous integration usato per fare build e test di progetti presenti in GitHub.

\subsubsection*{Truffle}
\index{Truffle}
Framework per lo sviluppo ed il testing di codice in una blockchain\glo. Gestisce l'environment necessario per eseguire e testare gli smart contracts\glo.

\subsubsection*{Twelve-Factor App}
\index{Twelve-Factor App}
Metodologia per la creazione di software come applicazioni di servizio. Queste best practice sono progettate per consentire alle applicazioni di essere costruite con portabilità e resilienza una volta implementate sul Web.

