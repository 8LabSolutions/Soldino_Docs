\section{Test}
I test sono attività che servono a verificare che il software prodotto implementi le funzionalità richieste.
\subsection{Test di accettazione} 
	\renewcommand{\arraystretch}{1.5}
	\rowcolors{2}{pari}{dispari}
	
	\begin{longtable}{ >{\centering}p{0.10\textwidth} >{}p{0.70\textwidth}
			>{\centering}p{0.10\textwidth} >{\centering}p{0.10\textwidth}}
			
		%\tabularnewline
		\caption{Riepilogo Test di Accettazione}\\	
		\rowcolorhead
		\textbf{\color{white}Codice} 
		& \centering\textbf{\color{white}Descrizione} 
		& \centering\textbf{\color{white}Stato}
		& \centering\textbf{\color{white}Esito} 
		\tabularnewline %\tabularnewline 
		\endfirsthead	
		
		\rowcolor{white}\caption[]{(continua)}\\	
		\rowcolorhead
		\textbf{\color{white}Codice} 
		& \centering\textbf{\centering\color{white}Descrizione} 
		& \centering\textbf{\color{white}Stato}
		& \centering\textbf{\color{white}Esito} 
		\tabularnewline %\tabularnewline 
		\endhead	
		
\hypertarget{TAFD1}{TAFD1} & L'utente deve poter accedere alla guida dell'applicazione. All'utente viene
		 chiesto di:
		 \begin{itemize}
		 	\item verificare che sia possibile accedere alla guida;
		 	\item verificare che la guida riporti informazioni riguardanti l'utilizzo di
		 	MetaMask\glo{};
		 	\item verificare che la guida riporti informazioni riguardanti il pagamento
		 	delle operazioni. 
		 \end{itemize} & \textit{NI} & \textit{NS}\\ \tabularnewline
\hypertarget{TAFO2}{TAFO2} & L'utente non ancora autenticato deve poter effettuare la registrazione al
		 sistema. All'utente viene chiesto di:
		 \begin{itemize}
		 	\item accedere alla pagina di registrazione;
		 	\item selezionare il tipo di utente;
		 	\item inserire i dati in base al tipo scelto.
		 \end{itemize} & \textit{NI} & \textit{NS}\\ \tabularnewline
\hypertarget{TAFO2.1}{TAFO2.1} & L'utente deve potersi registrare come cittadino. All'utente viene chiesto di:
		 \begin{itemize}
		 	\item accedere alla pagina di registrazione e scegliere la registrazione
		 	come cittadino;
		 	\item inserire l'email;
		 	\item inserire l'indirizzo;
		 	\item inserire il nome;
		 	\item inserire il cognome.
		 \end{itemize} & \textit{NI} & \textit{NS}\\ \tabularnewline
\hypertarget{TAFO2.2}{TAFO2.2} & L'utente deve potersi registrare come azienda. All'utente viene chiesto di:
		 \begin{itemize}
		 	\item accedere alla pagina di registrazione e scegliere la registrazione
		 	come azienda;
		 	\item inserire l'email;
		 	\item inserire la sede;
		 	\item inserire la partita IVA;
		 	\item inserire il nome.
		 \end{itemize} & \textit{NI} & \textit{NS}\\ \tabularnewline
\hypertarget{TAFO2.3}{TAFO2.3} & L'utente deve poter visualizzare un messaggio di errore nel caso in cui la
		 chiave reperita da MetaMask\glo{} durante la registrazione sia già presente
		 nel sistema. & \textit{NI} & \textit{NS}\\ \tabularnewline
\hypertarget{TAFO3.1}{TAFO3.1} & L'utente deve poter autenticarsi nel sistema in modo automatico attraverso
		 MetaMask\glo{}. All'utente è chiesto di:
		 \begin{itemize}
		 	\item accedere alla sezione per il login;
		 	\item verificare l'avvenuta autenticazione.
		 \end{itemize} & \textit{NI} & \textit{NS}\\ \tabularnewline
\hypertarget{TAFO4}{TAFO4} & L'utente deve potersi disconnettere dal sistema. All'utente viene chiesto di:
		 \begin{itemize}
		 	\item autenticarsi correttamente nel sistema;
		 	\item trovarsi in una qualsiasi delle pagine di navigazione destinate 
		 	all'utente;
		 	\item premere il tasto di logout;
		 	\item confermare di voler effettuare la disconnessione premendo il tasto 
		 	di conferma;
		 	\item verificare che la disconnessione sia avvenuta correttamente.
		 \end{itemize} & \textit{NI} & \textit{NS}\\ \tabularnewline
\hypertarget{TAFO5}{TAFO5} & L'utente autenticato come governo deve poter coniare Cubit\glo{}.
		 All'utente viene chiesto di:
		 \begin{itemize}
		 	\item autenticarsi correttamente nel sistema;
		 	\item trovarsi nella pagina per la coniazione della moneta;
		 	\item selezionare una quantità positiva di Cubit\glo{};
		 	\item dare conferma dell'operazione. 
		 \end{itemize} & \textit{NI} & \textit{NS}\\ \tabularnewline
\hypertarget{TAFO6}{TAFO6} & L'utente autenticato come governo deve poter distribuire Cubit\glo{}
		 agli utenti registrati come aziende o cittadini. All'utente viene chiesto di:
		 \begin{itemize}
		 	\item autenticarsi correttamente nel sistema;
		 	\item trovarsi nella pagina per la distribuzione della moneta;
		 	\item selezionare una quantità positiva di Cubit da 
		 	distribuire;
		 	\item selezionare la lista di utenti a cui trasferire i Cubit; 
		 	\item dare conferma dell'operazione. 
		 \end{itemize} & \textit{NI} & \textit{NS}\\ \tabularnewline
\hypertarget{TAFO8}{TAFO8} & L'utente autenticato come azienda o cittadino deve poter visualizzare i
		 prodotti in vendita nel sito. All'utente viene chiesto di:
		 \begin{itemize}
		 	\item accedere alla pagina dei prodotti in vendita nel sito;
		 	\item verificare che sia visualizzato il nome del prodotto;
		 	\item verificare che sia visualizzato il prezzo lordo\glo{} del prodotto;
		 	\item verificare che sia visualizzata la descrizione del prodotto;
		 	\item verificare che sia visualizzata la quantità relativa ad un certo
		 	prodotto in vendita nel sito;
		 	\item verificare che sia possibile cambiare la quantità selezionata relativa ad un prodotto in
		 	vendita nel sito.
		 \end{itemize} & \textit{NI} & \textit{NS}\\ \tabularnewline
\hypertarget{TAFO9}{TAFO9} & L'utente autenticato come azienda o cittadino deve poter aggiungere
		 al proprio carrello prodotti in vendita nel sito. All'utente viene chiesto di:
		 \begin{itemize}
		 	\item accedere alla pagina dei prodotti in vendita nel sito;
		 	\item aggiungere un prodotto al carrello.
		 \end{itemize} & \textit{NI} & \textit{NS}\\ \tabularnewline
\hypertarget{TAFO10}{TAFO10} & L'utente autenticato come azienda o cittadino deve poter visualizzare i
		 prodotti aggiunti nel proprio carrello. All'utente viene chiesto di:
		 \begin{itemize}
		 	\item accedere alla pagina dei prodotti in vendita nel sito;
		 	\item aggiungere un prodotto al carrello;
		 	\item accedere alla pagina dedicata al carrello;
		 	\item verificare che il prodotto sia stato aggiunto al carrello;
		 	\item verificare che sia visualizzato il nome del prodotto nel carrello;
		 	\item verificare che sia visualizzato il prezzo unitario del prodotto nel carrello;
		 	\item verificare che sia visualizzato il prezzo totale dei prodotti nel carrello.
		 \end{itemize} & \textit{NI} & \textit{NS}\\ \tabularnewline
\hypertarget{TAFO11}{TAFO11} & L'utente autenticato come azienda o cittadino deve poter modificare la 
		 quantità di un certo prodotto aggiunto nel proprio carrello. All'utente viene
		 chiesto di:
		 \begin{itemize}
		 	\item accedere alla pagina dei prodotti in vendita nel sito;
		 	\item aggiungere un prodotto al carrello;
		 	\item accedere alla pagina dedicata al carrello;
		 	\item modificare la quantità di prodotto aggiunta al carrello;
		 	\item verificare che la quantità di prodotto aggiunta al carrello sia
		 	effettivamente cambiata.
		 \end{itemize} & \textit{NI} & \textit{NS}\\ \tabularnewline
\hypertarget{TAFO12}{TAFO12} & L'utente autenticato come azienda o cittadino deve poter rimuovere un
		 certo prodotto aggiunto nel proprio carrello. All'utente viene chiesto di:
		 \begin{itemize}
		 	\item accedere alla pagina dei prodotti in vendita nel sito;
		 	\item aggiungere un prodotto al carrello;
		 	\item accedere alla pagina dedicata al carrello;
		 	\item rimuovere il prodotto aggiunto al carrello;
		 	\item verificare che il prodotto non sia più presente nel carrello.
		 \end{itemize} & \textit{NI} & \textit{NS}\\ \tabularnewline
\hypertarget{TAFD13}{TAFD13} & L'utente autenticato come azienda o cittadino deve poter resettare il
		 contenuto del carrello. All'utente viene chiesto di:
		 \begin{itemize}
		 	\item accedere alla pagina dei prodotti in vendita nel sito;
		 	\item aggiungere un prodotto al carrello;
		 	\item accedere alla pagina dedicata al carrello;
		 	\item premere il tasto per resettare il contenuto del carrello;
		 	\item verificare che il carrello sia vuoto.
		 \end{itemize} & \textit{NI} & \textit{NS}\\ \tabularnewline
\hypertarget{TAFO14.1}{TAFO14.1} & L'utente autenticato come azienda o cittadino deve poter effettuare il
		 checkout dei prodotti nel carrello. All'utente viene chiesto di:
		 \begin{itemize}
		 	\item accedere alla pagina dei prodotti in vendita nel sito;
		 	\item aggiungere un prodotto al carrello;
		 	\item accedere alla pagina dedicata al carrello;
		 	\item cliccare sul pulsante dedicato alla fase di checkout;
		 	\item accedere alla pagina dedicata al checkout;
		 	\item verificare che sia possibile procedere al pagamento.
		 \end{itemize} & \textit{NI} & \textit{NS}\\ \tabularnewline
\hypertarget{TAFO14.1.1}{TAFO14.1.1} & L'utente autenticato come azienda o cittadino deve poter visualizzare un
		 messaggio di errore nel caso in cui si proceda al checkout e il carrello sia
		 vuoto. All'utente viene chiesto di:
		 \begin{itemize}
		 	\item accedere alla pagina dedicata al carrello;
		 	\item verificare che non siano presenti prodotti nel carrello;
		 	\item procedere alla fase di checkout;
		 	\item verificare che sia visualizzato il messaggio di errore.
		 \end{itemize} & \textit{NI} & \textit{NS}\\ \tabularnewline
\hypertarget{TAFO14.2}{TAFO14.2} & L'utente autenticato come azienda o cittadino deve selezionare l'indirizzo di
		 spedizione per un certo prodotto. All'utente viene chiesto di:
		 \begin{itemize}
		 	\item accedere alla pagina dei prodotti in vendita nel sito;
		 	\item aggiungere un prodotto al carrello;
		 	\item procedere al checkout;
		 	\item verificare che sia possibile selezionare come indirizzo di
		 	spedizione il proprio indirizzo di residenza;
		 	\item verificare che sia possibile inserire un nuovo indirizzo di
		 	spedizione.
		 \end{itemize} & \textit{NI} & \textit{NS}\\ \tabularnewline
\hypertarget{TAFO14.3}{TAFO14.3} & L'utente autenticato come azienda o cittadino deve poter visualizzare un
		 messaggio di errore nel caso in cui si proceda al pagamento e tale operazione non vada a buon fine. All'utente viene chiesto di:
		 \begin{itemize}
		 	\item accedere alla pagina dedicata al carrello;
		 	\item procedere alla fase di checkout;
		 	\item procedere al pagamento cliccando sull'apposito pulsante;
		 	\item verificare che sia visualizzato il messaggio di errore.
		 \end{itemize} & \textit{NI} & \textit{NS}\\ \tabularnewline
\hypertarget{TAFO14.4}{TAFO14.4}	&	L'utente autenticato come azienda o cittadino deve procedere all'acquisto di un prodotto e completare la procedura di pagamento. All'utente viene chiesto di:
		 \begin{itemize}
		 	\item accedere alla pagina dei prodotti in vendita nel sito;
		 	\item aggiungere un prodotto al carrello;
		 	\item procedere al checkout;
		 	\item confermare l'ordine quindi procedere al pagamento;
		 	\item verificare che sia visibile la conferma d'acquisto nella pagina dedicata.
		 \end{itemize} & \textit{NI} & \textit{NS}\\ \tabularnewline
\hypertarget{TAFF15}{TAFF15} & L'utente autenticato come azienda o cittadino deve poter visualizzare lo
		 storico degli acquisiti effettuati. All'utente viene chiesto di:
		 \begin{itemize}
		 	\item accedere alla pagina dedicata agli acquisti effettuati nel sito;
		 	\item verificare che per ogni acquisto sia visualizzata la data di quando
		 	è stato effettuato;
		 	\item verificare che per ogni acquisto sia visualizzato il numero
		 	dell'acquisto;
		 	\item verificare che per ogni acquisto siano visualizzati i prodotti
		 	acquistati;
		 	\item verificare che per ogni acquisto sia visualizzato il totale dell'IVA
		 	dell'acquisto;
		 	\item verificare che per ogni acquisto sia visualizzato il prezzo
		 	lordo\glo{} dell'acquisto;
		 	\item verificare che per ogni acquisto sia visualizzato l'indirizzo di 
		 	spedizione dell'acquisto;
		 \end{itemize} & \textit{NI} & \textit{NS}\\ \tabularnewline
\hypertarget{TAFF15.7}{TAFF15.7} & L'utente autenticato come azienda o cittadino deve poter visualizzare la lista delle conferme d'acquisto\glosp che necessitano
		 di conferma. All'utente viene chiesto di:
		 \begin{itemize}
		 	\item accedere alla pagina dedicata alle conferme d'acquisto;
		 	\item verificare che per ogni conferma d'acquisto siano visualizzati i dettagli del prodotto;
		 	\item verificare che per ogni conferma d'acquisto sia visualizzato il prezzo netto del prodotto;
		 	\item verificare che per ogni conferma d'acquisto sia visualizzata l'aliquota IVA applicata al prodotto;
		 \end{itemize} & \textit{NI} & \textit{NS}\\ \tabularnewline
\hypertarget{TAFF15.7.1}{TAFF15.7.1} & L'utente autenticato come azienda o cittadino deve poter rifiutare una conferma d'acquisto\glosp che necessita
		 di conferma. All'utente viene chiesto di:
		 \begin{itemize}
		 	\item accedere alla pagina dedicata alle conferme d'acquisto;
		 	\item cliccare sul pulsante per rifiutare una conferma d'ordine;
		 	\item verificare che tale conferma d'ordine venga rifiutata;
		 \end{itemize} & \textit{NI} & \textit{NS}\\ \tabularnewline
\hypertarget{TAFF15.7.2}{TAFF15.7.2} & L'utente autenticato come azienda o cittadino deve poter approvare una conferma d'acquisto\glosp che necessita
		 di conferma. All'utente viene chiesto di:
		 \begin{itemize}
		 	\item accedere alla pagina dedicata alle conferme d'acquisto;
		 	\item cliccare sul pulsante per approvare una conferma d'ordine;
		 	\item verificare che tale conferma d'ordine venga approvata.
		 \end{itemize} & \textit{NI} & \textit{NS}\\ \tabularnewline
\hypertarget{TAFO16.1}{TAFO16.1} & L'utente autenticato come azienda deve poter mettere in vendita nuovi
		 prodotti. All'utente viene chiesto di:
		 \begin{itemize}
		 	\item accedere alla pagina dedicata all'aggiunta di un prodotto in
		 	vendita;
		 	\item inserire il nome del nuovo prodotto in vendita;
		 	\item inserire la descrizione del nuovo prodotto in vendita;
		 	\item inserire il prezzo lordo\glo{} del nuovo prodotto in vendita;
		 	\item inserire l'aliquota IVA del nuovo prodotto in vendita;
		 	\item verificare che il prodotto sia stato correttamente aggiunto al sistema.
		 \end{itemize} & \textit{NI} & \textit{NS}\\ \tabularnewline
\hypertarget{TAFO16.2}{TAFO16.2} & L'utente autenticato come azienda deve poter modificare i dati relativi ad
		 un prodotto in vendita. All'utente viene chiesto di:
		 \begin{itemize}
		 	\item accedere alla pagina dedicata ai propri prodotti in vendita;
		 	\item accedere alla pagina per la modifica di uno dei prodotti;
		 	\item modificare i dati del prodotto in vendita;
		 	\item verificare che alla conferma i dati del prodotto siano effettivamente
		 	modificati.
		 \end{itemize} & \textit{NI} & \textit{NS}\\ \tabularnewline
\hypertarget{TAFO16.2.1}{TAFO16.2.1} & L'utente autenticato come azienda deve poter verificare un errore nel caso
		 in cui non sia effettuata alcuna modifica ad un prodotto in vendita dopo
		 l'accesso alla pagina di modifica. All'utente viene chiesto di:
		 \begin{itemize}
		 	\item accedere alla pagina dedicata ai propri prodotti in vendita;
		 	\item accedere alla pagina per la modifica di uno dei prodotti;
		 	\item non effettuare alcuna modifica quindi confermare la modifica;
		 	\item verificare che alla conferma sia visualizzato l'errore.
		 \end{itemize} & \textit{NI} & \textit{NS}\\ \tabularnewline
\hypertarget{TAFO16.3}{TAFO16.3} & L'utente autenticato come azienda deve poter eliminare
		 un prodotto in vendita. All'utente viene chiesto di:
		 \begin{itemize}
		 	\item accedere alla pagina dedicata ai propri prodotti in vendita;
		 	\item accedere alla pagina per la modifica di uno dei prodotti;
		 	\item rimuovere il prodotto;
		 	\item verificare che alla conferma il prodotto non sia più presente nella
		 	lista dei prodotti in vendita.
		 \end{itemize} & \textit{NI} & \textit{NS}\\ \tabularnewline
\hypertarget{TAFF17}{TAFF17} & L'utente autenticato come azienda deve poter visualizzare lo storico delle
		 vendite. All'utente viene chiesto di:
		 \begin{itemize}
		 	\item accedere alla pagina dedicata alle vendite nel sito;
		 	\item verificare che per ogni vendita sia visualizzata la data di quando
		 	è stato effettuata;
		 	\item verificare che per ogni vendita sia visualizzato il numero
		 	della vendita;
		 	\item verificare che per ogni vendita siano visualizzati i prodotti
		 	venduti;
		 	\item verificare che per ogni vendita sia visualizzato il totale dell'IVA
		 	della vendita;
		 	\item verificare che per ogni vendita sia visualizzato il prezzo
		 	lordo\glo{} della vendita;
		 	\item verificare che per ogni vendita sia visualizzato l'indirizzo di 
		 	spedizione della vendita;
		 	\item verificare che per ogni vendita sia visualizzato il nome
		 	dell'acquirente qualora quest'ultimo sia un cittadino;
		 	\item verificare che per ogni vendita sia visualizzato il cognome
		 	dell'acquirente qualora quest'ultimo sia un cittadino;
		 	\item verificare che per ogni vendita sia visualizzato il nome
		 	dell'acquirente qualora quest'ultimo sia un'azienda;
		 	\item verificare che per ogni vendita sia visualizzata la partita IVA
		 	dell'acquirente qualora quest'ultimo sia un'azienda.
		 \end{itemize} & \textit{NI} & \textit{NS}\\ \tabularnewline
\hypertarget{TAFO18}{TAFO18} & L'utente autenticato come azienda deve procedere alla vendita di un prodotto e attendere la conferma d'ordine da parte dell'acquirente. All'utente viene chiesto di:
		 \begin{itemize}
		 	\item mettere in vendita un prodotto;
		 	\item attendere la conferma d'ordine da parte dell'acquirente;
		 	\item verificare che nel proprio wallet\glosp sia presente l'ammontare dell'ordine;
		 \end{itemize} & \textit{NI} & \textit{NS}\\ \tabularnewline
\hypertarget{TAFO19.1}{TAFO19.1} & L'utente autenticato come azienda deve poter visualizzare il saldo dell'IVA
		 relativo al trimestre corrente. All'utente viene chiesto di:
		 \begin{itemize}
		 	\item accedere alla pagina dedicata al saldo dell'IVA;
		 	\item verificare sia visualizzato il saldo dell'IVA relativo al trimestre
		 	corrente.
		 \end{itemize} & \textit{NI} & \textit{NS}\\ \tabularnewline
\hypertarget{TAFO19.2}{TAFO19.2} & L'utente autenticato come azienda deve poter visualizzare il saldo dell'IVA
		 relativo ad un trimestre concluso. All'utente viene chiesto di:
		 \begin{itemize}
		 	\item accedere alla pagina dedicata al saldo dell'IVA;
		 	\item selezionare un trimestre concluso;
		 	\item verificare sia visualizzato il saldo dell'IVA relativo al trimestre
		 	selezionato.
		 \end{itemize} & \textit{NI} & \textit{NS}\\ \tabularnewline
\hypertarget{TAFO19.3}{TAFO19.3} & L'utente autenticato come azienda deve poter visualizzare la lista delle
		 fatture relative ad un trimestre selezionato. All'utente viene chiesto di:
		 \begin{itemize}
		 	\item accedere alla pagina dedicata al saldo dell'IVA;
		 	\item selezionare un trimestre;
		 	\item verificare sia visualizzato per ogni fattura la data in cui è stata
		 	effettuata;
		 	\item verificare sia visualizzato per ogni fattura il numero identificativo;
		 	\item verificare sia visualizzato per ogni fattura il nome dell'azienda che
		 	l'ha emessa;
		 	\item verificare sia visualizzato per ogni fattura le informazioni relative
		 	all'acquirente;
		 	\item verificare sia visualizzato per ogni fattura l'importo totale
		 	dell'ordine;
		 	\item verificare sia visualizzato per ogni fattura il totale in debito/credito
		 	dell'IVA.
		 \end{itemize} & \textit{NI} & \textit{NS}\\ \tabularnewline
\hypertarget{TAFO19.4}{TAFO19.4} & L'utente autenticato come azienda deve poter effettuare il versamento 
		 dell'IVA di un trimestre nel quale il saldo di questa risulti in debito. All'utente 
		 viene chiesto di:
		 \begin{itemize}
		 	\item accedere alla pagina dedicata al saldo dell'IVA;
		 	\item selezionare un trimestre;
		 	\item verificare sia visualizzato il saldo dell'IVA relativo al trimestre
		 	selezionato;
		 	\item confermare il versamento dell'IVA da effettuare;
		 	\item verificare che sia possibile procedere al versamento.
		 \end{itemize} & \textit{NI} & \textit{NS}\\ \tabularnewline
\hypertarget{TAFO19.5}{TAFO19.5} & L'utente autenticato come azienda deve poter effettuare la dilazionare del versamento 
		 dell'IVA di un trimestre nel quale il saldo di questa risulti in debito. All'utente 
		 viene chiesto di:
		 \begin{itemize}
		 	\item accedere alla pagina dedicata al saldo dell'IVA;
		 	\item selezionare un trimestre;
		 	\item verificare sia visualizzato il saldo dell'IVA relativo al trimestre
		 	selezionato;
		 	\item selezionare la dilazione del versamento con cui si vuole procedere;
		 	\item confermare la dilazione da effettuare;
		 	\item verificare che il pagamento sia stato dilazionato.
		 \end{itemize} & \textit{NI} & \textit{NS}\\ \tabularnewline
\hypertarget{TAFO19.6}{TAFO19.6} & L'utente autenticato come azienda deve poter scaricare in formato PDF 
		 la lista dei movimenti dell'IVA del trimestre desiderato. All'utente viene chiesto 
		 di:
		 \begin{itemize}
		 	\item accedere alla pagina dedicata al saldo dell'IVA;
		 	\item selezionare il trimestre che desidera;
		 	\item verificare sia presente l'opzione di scaricare il PDF della lista
		 	dei movimenti dell'IVA;
		 	\item verificare che sia possibile possibile scaricare il PDF.
		 \end{itemize} & \textit{NI} & \textit{NS}\\ \tabularnewline
\hypertarget{TAFO19.7}{TAFO19.7} & L'utente autenticato come azienda deve poter 
		 visualizzare in
		 dettaglio una particolare fattura. All'utente viene chiesto di:
		 \begin{itemize}
		 	\item accedere alla pagina dedicata al saldo dell'IVA;
		 	\item selezionare un trimestre;
		 	\item verificare sia visualizzato per ogni fattura la data in cui è stata
		 	effettuata;
		 	\item verificare sia visualizzato per ogni fattura il numero identificativo;
		 	\item verificare sia visualizzata la data dell'ordine relativo alla fattura;
		 	\item verificare sia visualizzato in numero identificativo dell'ordine 
		 	relativo alla fattura;
		 	\item verificare siano visualizzati tutti i prodotti acquistati;
		 	\item verificare sia visualizzato l'importo totale dell'IVA relativa alla fattura;
		 	\item verificare sia visualizzato l'importo totale della fattura;
		 	\item verificare sia visualizzato il nome dell'azienda emittente;
		 	\item verificare sia visualizzata la partita IVA dell'azienda emittente;
		 	\item verificare sia visualizzato il nome dell'acquirente se è un cittadino;
		 	\item verificare sia visualizzato il cognome dell'acquirente se è un cittadino;
		 	\item verificare sia visualizzata la partita IVA dell'acquirente se è un'azienda;
		 	\item verificare sia visualizzato l'indirizzo di spedizione dell'ordine.
		 \end{itemize} & \textit{NI} & \textit{NS}\\ \tabularnewline
\hypertarget{TAFO19.8}{TAFO19.8} & L'utente autenticato come azienda deve poter scaricare in formato PDF 
		 la lista dei movimenti dell'IVA del trimestre desiderato. All'utente viene chiesto 
		 di:
		 \begin{itemize}
		 	\item accedere alla pagina dedicata al saldo dell'IVA;
		 	\item selezionare il trimestre che desidera;
		 	\item verificare sia presente l'opzione di scaricare il PDF della lista
		 	dei movimenti dell'IVA;
		 	\item verificare che sia possibile possibile scaricare il PDF.
		 \end{itemize} & \textit{NI} & \textit{NS}\\ \tabularnewline
\hypertarget{TAFF19.9}{TAFF19.9} & L'utente autenticato come azienda deve poter scaricare in formato PDF la
		 fattura relativa ad un acquisto. All'utente viene chiesto di:
		 \begin{itemize}
		 	\item di accedere alla pagina dedicata alle fatture;
		 	\item verificare di poter selezionare la fattura da scaricare in formato PDF;
		 	\item verificare di riuscire a scaricare la fattura in formato PDF.
		 \end{itemize} & \textit{NI} & \textit{NS}\\ \tabularnewline
\end{longtable}

\subsection{Test di sistema}
	\renewcommand{\arraystretch}{1.5}
	\rowcolors{2}{pari}{dispari}
	
	\begin{longtable}{ >{\centering}p{0.10\textwidth} >{}p{0.70\textwidth}
			>{\centering}p{0.10\textwidth} >{\centering}p{0.10\textwidth}}
			
		%\tabularnewline
		\caption{Riepilogo Test di Sistema}\\	
		\rowcolorhead
		\textbf{\color{white}Codice} 
		& \centering\textbf{\color{white}Descrizione} 
		& \centering\textbf{\color{white}Stato}
		& \centering\textbf{\color{white}Esito} 
		\tabularnewline %\tabularnewline 
		\endfirsthead	
		
		\rowcolor{white}\caption[]{(continua)}\\	
		\rowcolorhead
		\textbf{\color{white}Codice} 
		& \centering\textbf{\centering\color{white}Descrizione} 
		& \centering\textbf{\color{white}Stato}
		& \centering\textbf{\color{white}Esito} 
		\tabularnewline %\tabularnewline 
		\endhead	
		

		\hypertarget{TSFD1}{TSFD1} & Viene verificato che il sistema permetta all'utente 
		di leggere una breve guida riguardante l'utilizzo di MetaMask ed il pagamento 
		delle operazioni all'interno della piattaforma. & \textit{NI} & \textit{NS}\\ 

		\tabularnewline
		\hypertarget{TSFO2}{TSFO2} & Viene verificato che il sistema permetta la 
		registrazione di un nuovo utente. & \textit{I} & \textit{S}\\ 

		\tabularnewline
		\hypertarget{TSFO2.1}{TSFO2.1} & Viene verificato che il sistema permetta la 
		registrazione come cittadino. & \textit{I} & \textit{S}\\ 

		\tabularnewline
		\hypertarget{TSFO2.2}{TSFO2.2} & Viene verificato che il sistema permetta la 
		registrazione come azienda. & \textit{I} & \textit{S}\\ 

		\tabularnewline
		\hypertarget{TSFO2.3}{TSFO2.3} & Viene verificato che la registrazione non vada 
		a buon fine se la chiave di MetaMask è già presente nel sistema. & 
		\textit{NI} & \textit{NS}\\  

		\tabularnewline
		\hypertarget{TSFO3}{TSFO3} & Viene verificato che il sistema permetta di 
		autenticarsi. & \textit{I} & \textit{S}\\ 

		\tabularnewline
		\hypertarget{TSFO3.1}{TSFO3.1} & Viene verificato che nel sistema il login 
		avvenga automaticamente tramite Metamask. & \textit{I} & \textit{S}\\ 

		\tabularnewline
		\hypertarget{TSFO3.2}{TSFO3.2} & Viene verificato che nel sistema il login e/o 
		registrazione non vadano a buon fine se MetaMask non è presente. & 
		\textit{NI} & \textit{NS}\\  

		\tabularnewline
		\hypertarget{TSFO3.3}{TSFO3.3} & Viene verificato che il processo di login e/o 
		registrazione non vada a buon fine se non vi è alcuna chiave disponibile nel 
		plug-in. & \textit{NI} & \textit{NS}\\ 

		\tabularnewline
		\hypertarget{TSFO3.4}{TSFO3.4} & Viene verificato che il processo di login si 
		interrompa se la chiave non è registrata nel sistema. & 
		\textit{NI} & \textit{NS}\\  

		\tabularnewline
		\hypertarget{TSFO3.5}{TSFO3.5} & Viene verificato che il processo di login e/o 
		registrazione non vada a buon fine se l'utente risulta "disabilitato". & 
		\textit{NI} & \textit{NS}\\ 

		\tabularnewline
		\hypertarget{TSFO4}{TSFO4} & Viene verificato che il sistema permetta di 
		eseguire il logout. & \textit{I} & \textit{S}\\ 

		\tabularnewline
		\hypertarget{TSFO5}{TSFO5} & Viene verificato che governo possa coniare Cubit. & 
		\textit{NI} & \textit{NS}\\ 

		\tabularnewline
		\hypertarget{TSFO6}{TSFO6} & Viene verificato che il governo possa distribuire 
		Cubit a cittadini ed aziende. & \textit{NI} & \textit{NS}\\ 

		\tabularnewline
		\hypertarget{TSFO7}{TSFO7} & Viene verificato che il governo posso visualizzare 
		e gestire la lista degli utenti registrati. & \textit{NI} & \textit{NS}\\ 

		\tabularnewline
		\hypertarget{TSFO8}{TSFO8} & Viene verificato che aziende e cittadini possano 
		visualizzare i prodotti in vendita nel sito. & \textit{I} & 
		\textit{S}\\ 

		\tabularnewline
		\hypertarget{TSFO9}{TSFO9} & Viene verificato che aziende e cittadini possano 
		aggiungere prodotti nel carrello. & \textit{I} & \textit{S}\\ 

		\tabularnewline
		\hypertarget{TSFO10}{TSFO10} & Viene verificato che il sistema permetta di 
		visualizzare i prodotti nel carrello. & \textit{I} & \textit{S}\\ 

		\tabularnewline
		\hypertarget{TSFO11}{TSFO11} & Viene verificato che il sistema permetta di 
		modificare la quantità di un bene presente nel carrello. & 
		\textit{I} & \textit{S}\\ 

		\tabularnewline
		\hypertarget{TSFO12}{TSFO12} & Viene verificato che il sistema permetta di 
		rimuovere prodotti dal carrello. & \textit{I} & \textit{S}\\ 

		\tabularnewline
		\hypertarget{TSFD13}{TSFD13} & Viene verificato che il sistema permetta di 
		resettare il contenuto del carrello. & \textit{I} & \textit{S}\\ 

		\tabularnewline
		\hypertarget{TSFO14}{TSFO14} & Viene verificato che il sistema permetta 
		a 
		cittadini ed aziende di acquistare prodotti venduti nella piattaforma. & 
		\textit{I} & \textit{S}\\ 

		\tabularnewline
		\hypertarget{TSFF15}{TSFF15} & Viene verificato che il sistema permetta a 
		cittidani ed aziende di visualizzare la lista degli acquisti effettuati. & 
		\textit{I} & \textit{S}\\ 

		\tabularnewline
		\hypertarget{TSFO16}{TSFO16} & Viene verificato che un'azienda possa gestire i 
		prodotti in vendita. & \textit{NI} & \textit{NS}\\ 

		\tabularnewline
		\hypertarget{TSFO16.1}{TSFO16.1} & Viene verificato che il sistema permetta ad 
		un'azienda di mettere in vendita beni e servizi. & \textit{NI} & \textit{NS}\\ 

		\tabularnewline
		\hypertarget{TSFO16.2}{TSFO16.2} & Viene verificato che un'azienda possa 
		modificare i dati relativi ad un proprio bene o servizio in vendita. & 
		\textit{NI} & \textit{NS}\\ 

		\tabularnewline
		\hypertarget{TSFO16.3}{TSFO16.3} & Verificare che un'azienda possa eliminare un 
		proprio bene o servizio in vendita. & \textit{NI} & \textit{NS}\\ 

		\tabularnewline
		\hypertarget{TSFF17}{TSFF17} & Viene verificato che un'azienda possa 
		visualizzare la lista delle vendite. & \textit{NI} & \textit{NS}\\ 

		\tabularnewline
		\hypertarget{TSFO18}{TSFO18} & Viene verificato che un'azienda riceva il 
		pagamento a seguito della conferma dell'ordine. & \textit{NI} & \textit{NS}\\ 

		\tabularnewline
		\hypertarget{TSFO19.1}{TSFO19.1} & Viene verificato che un'azienda possa 
		visualizzare il saldo IVA relativo al trimestre corrente. & 
		\textit{NI} & \textit{NS}\\ 

		\tabularnewline
		\hypertarget{TSFO19.2}{TSFO19.2} & Viene verificato che un'azienda possa 
		visualizzare il saldo IVA relativo ad un trimestre concluso selezionato. & 
		\textit{NI} & \textit{NS}\\ 

		\tabularnewline
		\hypertarget{TSFO19.4}{TSFO19.4} & Viene verificato che un'azienda possa 
		effettuare il versamento IVA di un trimestre nel quale il saldo IVA risulti a 
		debito. & \textit{NI} & \textit{NS}\\ 

		\tabularnewline
		\hypertarget{TSFO19.5}{TSFO19.5} & Viene verificato che un'azienda possa 
		effettuare la dilazione del versamento IVA di un trimestre nel quale il saldo 
		IVA risulti a debito. & \textit{NI} & \textit{NS}\\ 

		\tabularnewline
		\hypertarget{TSFO19.7}{TSFO19.7} & Viene verificato che un'azienda possa 
		visualizzare in dettaglio una particolare fattura. & \textit{NI} & \textit{NS}\\ 

		\tabularnewline
		\hypertarget{TSFO20}{TSFO20} & Viene verificato che gli smart contracts siano 
		upgradable. & \textit{NI} & \textit{NS}\\ 

		\tabularnewline
		\hypertarget{TSFO21}{TSFO21} & Viene verificato che il sistema impedisca 
		all'utente di compiere alcuna azione prima di essersi autenticato al sistema con 
		MetaMask. & \textit{NI} & \textit{NS}\\ 

		\tabularnewline
		\hypertarget{TSVO9}{TSVO9} & Viene verificato che la Dapp sia accessibile 
		utilizzando il browser Google Chrome, dalla versione. 71 & 
		\textit{I} & \textit{S}\\

		\tabularnewline
		\hypertarget{TSVO10}{TSVO10} & Viene verificato che la Dapp sia accessibile 
		utilizzando il browser Mozilla Firefox, dalla versione. 64 & \textit{I} 
		& \textit{S}\\  
		\tabularnewline
		\end{longtable}


\subsection{Test di integrazione}
	\renewcommand{\arraystretch}{1.5}
	\rowcolors{2}{pari}{dispari}	
		
		\begin{longtable}{ >{\centering}p{0.10\textwidth} >{}p{0.70\textwidth}
				>{\centering}p{0.10\textwidth} >{\centering}p{0.10\textwidth}}
			
			%\tabularnewline
			\caption{Riepilogo Test di Integrazione}\\	
			\rowcolorhead
			\textbf{\color{white}Codice} 
			& \centering\textbf{\color{white}Descrizione} 
			& \centering\textbf{\color{white}Stato}
			& \centering\textbf{\color{white}Esito} 
			\tabularnewline %\tabularnewline 
			\endfirsthead	
			
			\rowcolor{white}\caption[]{(continua)}\\	
			\rowcolorhead
			\textbf{\color{white}Codice} 
			& \centering\textbf{\centering\color{white}Descrizione} 
			& \centering\textbf{\color{white}Stato}
			& \centering\textbf{\color{white}Esito} 
			\tabularnewline %\tabularnewline 
			\endhead	
			
			
			\hypertarget{TI1}{TI1} & Viene verificato che sia lanciato un messaggio di 
			errore se un utente è già presente nel sistema. & \textit{I} & \textit{S}\\ 
			
			\tabularnewline
			\hypertarget{TI2}{TI2} & Viene verificato che un utente venga rimosso 
			correttamente. & \textit{I} & \textit{S}\\
			
			\tabularnewline
			\hypertarget{TI3}{TI3} & Viene verificato che sia lanciato un messaggio di 
			errore se si vuole eliminare un utente non presente nel sistema. & \textit{I} & 
			\textit{S}\\
			
			\tabularnewline
			\hypertarget{TI4}{TI4} & Viene verificato che venga ritornata la lista di 
			utenti. & \textit{I} & \textit{S}\\
			
			\tabularnewline
			\hypertarget{TI5}{TI5} & Viene verificato che venga ritornata la lista dei 
			prodotti correttamente. & \textit{I} & \textit{S}\\
			
			\tabularnewline
			\hypertarget{TI6}{TI6} & Viene verificato che venga fatto il login di un 
			cittadino dopo aver effettuato la registrazione. & \textit{I} & \textit{S}\\
			
			\tabularnewline
			\hypertarget{TI7}{TI7} & Viene verificato che venga fatto il login di 
			un'azienda dopo aver effettuato la registrazione. & \textit{I} & \textit{S}\\
			
			\tabularnewline
			\hypertarget{TI8}{TI8} & Viene verificato che i dati di un cittadino siano 
			correttamente visualizzati nel suo profilo. & \textit{I} & \textit{S}\\
			
			\tabularnewline
			\hypertarget{TI9}{TI9} & Viene verificato che i dati di un'azienda siano 
			correttamente visualizzati nel suo profilo. & \textit{I} & \textit{S}\\
			
			\tabularnewline
			\hypertarget{TI10}{TI10} & Viene verificato che sia visualizzata la lista dei 
			prodotti di una specifica azienda. & \textit{I} & \textit{S}\\
			
			\tabularnewline
			\hypertarget{TI11}{TI11} & Viene verificato che sia visualizzata la lista dei 
			prodotti vuota di una specifica azienda che non ha ancora inserito dei prodotti. 
			& \textit{I} & \textit{S}\\
			
			\tabularnewline
			\hypertarget{TI12}{TI12} & Viene verificato che la lista degli utenti sia 
			aggiornata dopo aver inserito un nuovo utente. & \textit{I} & \textit{S}\\
			
			\tabularnewline
			\hypertarget{TI13}{TI13} & Viene verificato che la lista dei prodotti sia 
			aggiornata dopo che un'azienda aggiunge un nuovo prodotto. & \textit{I} & 
			\textit{S}\\
			
			\tabularnewline
			\hypertarget{TI14}{TI14} & Viene verificato che la lista dei prodotti di una 
			specifica azienda venga aggiornata dopo l'inserimento di un nuovo prodotto. & 
			\textit{I} & \textit{S}\\
			
			\tabularnewline
			\hypertarget{TI15}{TI15} & Viene verificato che venga ritornata la lista 
			delle aziende che vende uno specifico prodotto. & \textit{I} & \textit{S}\\
			
			\tabularnewline
			\hypertarget{TI16}{TI16} & Viene verificato che l'utente possa visualizzare 
			la lista degli ordini. & \textit{I} & \textit{S}\\
			
			\tabularnewline
			\hypertarget{TI17}{TI17} & Viene verificato che l'azienda possa visualizzare 
			la lista delle fatture. & \textit{I} & \textit{S}\\
			
			\tabularnewline
			\hypertarget{TI18}{TI18} & Viene verificato che sia lanciato un messaggio di 
			errore se un utente cerca di eseguire un operazione a cui non è abilitato. & 
			\textit{I} & \textit{S}\\
			
			\tabularnewline
			\hypertarget{TI19}{TI19} & Viene verificato che l'utente possa modificare i 
			propri dati personali. & \textit{I} & \textit{S}\\
			
			\tabularnewline
			\hypertarget{TI20}{TI20} & Viene verificato che l'azienda possa visualizzare 
			la lista del rimborso dell'IVA. & \textit{I} & \textit{S}\\

				
			\tabularnewline
		\end{longtable}

\subsection{Test di unità}
	\renewcommand{\arraystretch}{1.5}
	\rowcolors{2}{pari}{dispari}	
		
		\begin{longtable}{ >{\centering}p{0.10\textwidth} >{}p{0.70\textwidth}
				>{\centering}p{0.10\textwidth} >{\centering}p{0.10\textwidth}}
			
			%\tabularnewline
			\caption{Riepilogo Test di Unità}\\	
			\rowcolorhead
			\textbf{\color{white}Codice} 
			& \centering\textbf{\color{white}Descrizione} 
			& \centering\textbf{\color{white}Stato}
			& \centering\textbf{\color{white}Esito} 
			\tabularnewline %\tabularnewline 
			\endfirsthead	
			
			\rowcolor{white}\caption[]{(continua)}\\	
			\rowcolorhead
			\textbf{\color{white}Codice} 
			& \centering\textbf{\centering\color{white}Descrizione} 
			& \centering\textbf{\color{white}Stato}
			& \centering\textbf{\color{white}Esito} 
			\tabularnewline %\tabularnewline 
			\endhead	
			
			
			\hypertarget{TU1}{TU1} & Viene verificata la corretta 
			registrazione di un cittadino. & \textit{I} & \textit{S}\\ 
			
			\tabularnewline
			\hypertarget{TU2}{TU2} & Viene verificata la corretta 
			registrazione di un'azienda. & \textit{I} & \textit{S}\\
			
			\tabularnewline
			\hypertarget{TU3}{TU3} & Viene verificato che il login di un utente 
			avvenga correttamente se esso è già registrato nel sistema. & 
			\textit{I} & \textit{S}\\ 
			
			\tabularnewline
			\hypertarget{TU4}{TU4} & Viene verificato che il login di un utente 
			non avvenga correttamente se esso non è già registrato nel sistema. 
			& \textit{I} & \textit{S}\\ 
			
			\tabularnewline
			\hypertarget{TU5}{TU5} & Viene verificata la possibilità di un 
			utente registrato come azienda di inserire correttamente un 
			prodotto. & \textit{I} & \textit{S}\\
			
			\tabularnewline
			\hypertarget{TU6}{TU6} & Viene verificato che un'azienda possa 
			modificare correttamente un proprio prodotto. & \textit{I} & 
			\textit{S}\\
			
			\tabularnewline
			\hypertarget{TU7}{TU7} & Viene verificato che un'azienda possa 
			eliminare un proprio prodotto. & \textit{I} & \textit{S}\\ 
			
			\tabularnewline
			\hypertarget{TU8}{TU8} &Viene verificato che un'azienda non possa 
			modificare un prodotto inserito da un'altra azienda. & \textit{I} & 
			\textit{S}\\ 
			
			\tabularnewline
			\hypertarget{TU9}{TU9} & Viene verificato che un'azienda non possa 
			eliminare un prodotto inserito da un'altra azienda. & \textit{I} & 
			\textit{S}\\ 
			
			\tabularnewline
			\hypertarget{TU10}{TU10} & Viene verificato che un cittadino non 
			possa aggiungere, eliminare, modificare un prodotto. & \textit{I} & 
			\textit{S}\\
			
			\tabularnewline
			\hypertarget{TU11}{TU11} & Viene verificato che il calcolo del 
			valore IVA su un prodotto avvenga correttamente. & \textit{I} 
			& \textit{S}\\
			
			\tabularnewline
			\hypertarget{TU12}{TU12} & Viene verificato che il calcolo del 
			prezzo lordo di un prodotto sia corretto. & \textit{I} 
			& \textit{S}\\
			
			\tabularnewline
			\hypertarget{TU13}{TU13} & Viene verificato che sia possibile 
			effettuare un ordine. & \textit{I} & \textit{S}\\ 
			
			
			\tabularnewline
			\hypertarget{TU14}{TU14} & Viene verificato che la creazione 
			dell'ordine non avvenga correttamente se esso non contiene almeno 
			un prodotto. & \textit{I} & 
			\textit{S}\\
			
			\tabularnewline
			\hypertarget{TU15}{TU15} & Viene verificato che la creazione 
			dell'ordine non avvenga correttamente se gli indirizzi del 
			compratore e del venditore non sono validi o se essi coincidono. & 
			\textit{I} & 
			\textit{S}\\ 
			
			\tabularnewline
			\hypertarget{TU16}{TU16} & Viene verificato che il calcolo del 
			totale relativo ad un ordine avvenga correttamente. & \textit{I} & 
			\textit{S}\\ 
			
			\tabularnewline
			\hypertarget{TU17}{TU17} & Viene verificato che il calcolo dello 
			stato IVA corrisponda alla situazione attuale. & \textit{I} & 
			\textit{S}\\
			
			\tabularnewline
			\hypertarget{TU18}{TU18} & Viene verificato il corretto rimborso 
			dell'IVA da parte del governo. & \textit{I} & \textit{S}\\ 
				
			\tabularnewline
			\hypertarget{TU19}{TU19} & Viene verificato che il rimborso 
			dell'IVA da parte del governo avvenga solamente se lo stato della 
			stessa risulta a credito. & \textit{I} & \textit{S}\\ 
			
			\tabularnewline
			\hypertarget{TU20}{TU20} & Viene verificato il corretto pagamento 
			dell'IVA da parte di un'azienda. & \textit{I} & \textit{S}\\
			
			\tabularnewline
			\hypertarget{TU21}{TU21} & Viene verificato il corretto 
			trasferimento di Cubit. & \textit{I} & 
			\textit{S}\\ 
			
			\tabularnewline
			\hypertarget{TU22}{TU22} & Viene verificato che il trasferimento di 
			Cubit avvenga solo se l'indirizzo di destinazione è valido. & 
			\textit{I} & \textit{S}\\
			
			\tabularnewline
			\hypertarget{TU23}{TU23} & Viene verificato che il trasferimento di 
			Cubit avvenga solo se il mittente possiede tale somma. & \textit{I} 
			& \textit{S}\\ 
			
			\tabularnewline
			\hypertarget{TU24}{TU24} & Viene verificata la corretta coniazione 
			di un certo quantitativo di Cubit. & \textit{I} & \textit{S}\\  
			
			\tabularnewline
			\hypertarget{TU25}{TU25} &Viene verificato che il metodo 
			\texttt{render()} del componente \texttt{AddProductsManager} 
			comporti la corretta renderizzazione del componente. & \textit{I} & 
			\textit{S}\\
			
			\tabularnewline
			\hypertarget{TU26}{TU26} & Viene verificato che venga costruito 
			correttamente il componente \texttt{AddProductManager} per la 
			gestione da parte di una azienda dei prodotti in vendita. & 
			\textit{I} & 
			\textit{S}\\
			
			\tabularnewline
			\hypertarget{TU27}{TU27} & Viene verificato che il metodo 
			\texttt{handleChange()} catturi correttamente l'evento scatenato e 
			setti di conseguenza lo stato del componente. & 
			\textit{I} & 
			\textit{S}\\
			
			\tabularnewline
			\hypertarget{TU28}{TU28} & Viene verificato che il metodo 
			\texttt{render()} del componente \texttt{BusinessProduct} comporti 
			la corretta renderizzazione del componente. & 
			\textit{I} & 
			\textit{S}\\
			
			\tabularnewline
			\hypertarget{TU29}{TU29} & Viene verificato che il metodo 
			\texttt{render()} del componente \texttt{Button} comporti la 
			corretta renderizzazione del componente. & 
			\textit{I} & 
			\textit{S}\\
			
			\tabularnewline
			\hypertarget{TU30}{TU30} & Viene verificato che il metodo 
			\texttt{printProduct()} ritorni correttamente il componente 
			\texttt{CartProduct} con le informazioni relative al prodotto. & 
			\textit{I} & 
			\textit{S}\\
			
			\tabularnewline
			\hypertarget{TU31}{TU31} & Viene verificato che il metodo 
			\texttt{render()} del componente \texttt{Cart} comporti la corretta 
			renderizzazione del componente. & 
			\textit{I} & 
			\textit{S}\\
			
			\tabularnewline
			\hypertarget{TU32}{TU32} & Viene verificato che il metodo 
			\texttt{enableRemove()} visualizzi correttamente il bottone per la 
			diminuzione della quantità di un prodotto presente nel carrello. & 
			\textit{I} & 
			\textit{S}\\
			
			\tabularnewline
			\hypertarget{TU33}{TU33} & Viene verificato che il metodo 
			\texttt{disableRemove()} disabiliti correttamente il bottone per la 
			diminuzione della quantità di prodotto. & 
			\textit{I} & 
			\textit{S}\\
			
			\tabularnewline
			\hypertarget{TU34}{TU34} & Viene verificato che venga costruito 
			correttamente il componente \texttt{Checkout} per la gestione del 
			checkout dell'acquisto. & 
			\textit{I} & 
			\textit{S}\\
			
			\tabularnewline
			\hypertarget{TU35}{TU35} & Viene verificato che il metodo 
			\texttt{handleOptionChange()} catturi correttamente l'evento 
			scatenato e setti di conseguenza lo stato del componente. & 
			\textit{I} & 
			\textit{S}\\
			
			\tabularnewline
			\hypertarget{TU36}{TU36} & Viene verificato che il metodo 
			\texttt{oldShipment()} visualizzi l' impossibilità di inserire dati 
			per l'indirizzo di spedizione(utilizzo dell'indirizzo di 
			iscrizione). & 
			\textit{I} & 
			\textit{S}\\
			
			\tabularnewline
			\hypertarget{TU37}{TU37} & Viene verificato che il metodo 
			\texttt{newShipment()} visualizzi correttamente la possibilità di 
			inserire dati per il nuovo indirizzo di spedizione. & 
			\textit{I} & 
			\textit{S}\\
			
			\tabularnewline
			\hypertarget{TU38}{TU38} & Viene verificato che il metodo 
			\texttt{handleInputChange()} catturi correttamente l'evento 
			scatenato e setti di conseguenza lo stato del componente. & 
			\textit{I} & 
			\textit{S}\\
			
			\tabularnewline
			\hypertarget{TU39}{TU39} & Viene verificato che il metodo 
			\texttt{getAddress()} restituisca correttamente l'indirizzo 
			corretto. & 
			\textit{I} & 
			\textit{S}\\
			
			\tabularnewline
			\hypertarget{TU40}{TU40} & Viene verificato che il metodo 
			\texttt{render()} del componente \texttt{Checkout} comporti la 
			corretta renderizzazione del componente. & 
			\textit{I} & 
			\textit{S}\\
			
			\tabularnewline
			\hypertarget{TU41}{TU41} & Viene verificato che il metodo 
			\texttt{render()} del componente \texttt{CubitManager} comporti la 
			corretta renderizzazione del componente. & 
			\textit{I} & 
			\textit{S}\\
			
			\tabularnewline
			\hypertarget{TU42}{TU42} & Viene verificato che venga costruito 
			correttamente il componente \texttt{EditProductsManager} per la 
			modifica da parte di una azienda dei prodotti in vendita. & 
			\textit{I} & 
			\textit{S}\\
			
			\tabularnewline
			\hypertarget{TU43}{TU43} & Viene verificato che il metodo 
			\texttt{handleChange()} catturi correttamente l'evento scatenato e 
			setti di conseguenza lo stato del componente. & 
			\textit{I} & 
			\textit{S}\\                    
			
			\tabularnewline
			\hypertarget{TU44}{TU44} & Viene verificato che il metodo 
			\texttt{render()} del componente \texttt{EdiProductsManager} 
			comporti la corretta renderizzazione del componente. & 
			\textit{I} & 
			\textit{S}\\
			
			\tabularnewline
			\hypertarget{TU45}{TU45} & Viene verificato che il metodo 
			\texttt{render()} del componente \texttt{Error} comporti la 
			corretta renderizzazione del componente. & 
			\textit{I} & 
			\textit{S}\\
			
			\tabularnewline
			\hypertarget{TU46}{TU46} & Viene verificato che il metodo 
			\texttt{render()} del componente \texttt{Footer} comporti la 
			corretta renderizzazione del componente. & 
			\textit{I} & 
			\textit{S}\\
			
			\tabularnewline
			\hypertarget{TU47}{TU47} & Viene verificato che il metodo 
			\texttt{render()} del componente \texttt{FormCubitManager} comporti 
			la corretta renderizzazione del componente. & 
			\textit{I} & 
			\textit{S}\\
			
			\tabularnewline
			\hypertarget{TU48}{TU48} & Viene verificato che venga costruito 
			correttamente il componente \texttt{FormRegistration} per la 
			registrazione di un utente. & 
			\textit{I} & 
			\textit{S}\\ 
			
			\tabularnewline
			\hypertarget{TU49}{TU49} & Viene verificato che il metodo 
			\texttt{citizenFormMaker()} restituisca correttamente il form per 
			la registrazione del cittadino. & 
			\textit{I} & 
			\textit{S}\\
			
			\tabularnewline
			\hypertarget{TU50}{TU50} & Viene verificato che il metodo 
			\texttt{businessFormMaker()} restituisca correttamente il form per  
			la registrazione dell'azienda. & 
			\textit{I} & 
			\textit{S}\\        
			     
			\tabularnewline
			\hypertarget{TU51}{TU51} & Viene verificato che il metodo 
			\texttt{handleOptionChange()} catturi correttamente l'evento scatenato 
			e setti di conseguenza lo stato del componente. & 
			\textit{I} & 
			\textit{S}\\        
						     
			\tabularnewline
			\hypertarget{TU52}{TU52} & Viene verificato che il metodo 
			\texttt{handleChange()} catturi correttamente l'evento scatenato e setti 
			di conseguenza lo stato del componente. & 
			\textit{I} & 
			\textit{S}\\        
						     
			\tabularnewline
			\hypertarget{TU53}{TU53} & Viene verificato che il metodo 
			\texttt{render()} del componente \texttt{FormRegistration} comporti la 
			corretta renderizzazione del componente & 
			\textit{I} & 
			\textit{S}\\        
						     
			\tabularnewline
			\hypertarget{TU54}{TU54} & Viene verificato che il metodo 
			\texttt{render()} del componente \texttt{Guide} comporti la corretta 
			renderizzazione del componente. & 
			\textit{I} & 
			\textit{S}\\        
						     
			\tabularnewline
			\hypertarget{TU55}{TU55} & Viene verificato che il metodo 
			\texttt{render()} del componente \texttt{Home} comporti la corretta 
			renderizzazione del componente. & 
			\textit{I} & 
			\textit{S}\\        
						     
			\tabularnewline
			\hypertarget{TU56}{TU56} & Viene verificato che il metodo 
			\texttt{render()} del componente \texttt{NavBar} comporti la corretta 
			renderizzazione del componente. & 
			\textit{I} & 
			\textit{S}\\        
						     
			\tabularnewline
			\hypertarget{TU57}{TU57} & Viene verificato che il metodo 
			\texttt{printSeller()} restituisca correttamente i dati relativi 
			al venditore. & 
			\textit{I} & 
			\textit{S}\\        
						     
			\tabularnewline
			\hypertarget{TU58}{TU58} & Viene verificato che il metodo 
			\texttt{printProduct()} restituisca correttamente i dati relativi 
			al prodotto. & 
			\textit{I} & 
			\textit{S}\\        
						     
			\tabularnewline
			\hypertarget{TU59}{TU59} & Viene verificato che il metodo 
			\texttt{printOrder()} restituisca correttamente i dati relativi 
			all'ordine. & 
			\textit{I} & 
			\textit{S}\\        
						     
			\tabularnewline
			\hypertarget{TU60}{TU60} & Viene verificato che il metodo 
			\texttt{render()} del componente \texttt{Orders} comporti la 
			corretta renderizzazione del componente. & 
			\textit{I} & 
			\textit{S}\\        
						     
			\tabularnewline
			\hypertarget{TU61}{TU61} & Viene verificato che il metodo 
			\texttt{render()} del componente \texttt{PendingOrder} comporti 
			la corretta renderizzazione del componente. & 
			\textit{I} & 
			\textit{S}\\        
						     
			\tabularnewline
			\hypertarget{TU62}{TU62} & Viene verificato che venga costruito 
			correttamente il componente \texttt{Product} per la visualizzazione 
			di un prodotto nello store. & 
			\textit{I} & 
			\textit{S}\\        
						     
			\tabularnewline
			\hypertarget{TU63}{TU63} & Viene verificato che il metodo 
			\texttt{handleChange()} catturi correttamente l'evento scatenato 
			e setti di conseguenza lo stato del componente. & 
			\textit{I} & 
			\textit{S}\\        
						     
			\tabularnewline
			\hypertarget{TU64}{TU64} & Viene verificato che il metodo 
			\texttt{render()} del componente \texttt{Product} comporti la 
			corretta renderizzazione del componente. & 
			\textit{I} & 
			\textit{S}\\        
						     
			\tabularnewline
			\hypertarget{TU65}{TU65} & Viene verificato che il metodo 
			\texttt{printProduct()} restituisca correttamente i dati 
			relativi al prodotto. & 
			\textit{I} & 
			\textit{S}\\        
						     
			\tabularnewline
			\hypertarget{TU66}{TU66} & Viene verificato che il metodo 
			\texttt{render()} del componente \texttt{ProductManager} 
			comporti la corretta renderizzazione del componente. & 
			\textit{I} & 
			\textit{S}\\        
						     
			\tabularnewline
			\hypertarget{TU67}{TU67} & Viene verificato che il metodo 
			\texttt{printProduct()} restituisca correttamente i dati 
			relativi al prodotto. & 
			\textit{I} & 
			\textit{S}\\        
						     
			\tabularnewline
			\hypertarget{TU68}{TU68} & Viene verificato che il metodo 
			\texttt{printOrder()} restituisca correttamente i dati relativi 
			all'ordine. & 
			\textit{I} & 
			\textit{S}\\        
			
			\tabularnewline
			\hypertarget{TU69}{TU69} & Viene verificato che il metodo 
			\texttt{render()} del componente \texttt{PurchaseConfirmation} 
			comporti la corretta renderizzazione del componente. & 
			\textit{I} & 
			\textit{S}\\
						
			\tabularnewline
			\hypertarget{TU70}{TU70} & Viene verificato che il metodo 
			\texttt{render()} del componente \texttt{Search} comporti 
			la corretta renderizzazione del componente. & 
			\textit{I} & 
			\textit{S}\\
						
			\tabularnewline
			\hypertarget{TU71}{TU71} & Viene verificato che il metodo 
			\texttt{printProduct()} restituisca correttamente i dati 
			relativi al prodotto. & 
			\textit{I} & 
			\textit{S}\\
						
			\tabularnewline
			\hypertarget{TU72}{TU72} & Viene verificato che il metodo 
			\texttt{render()} del componente \texttt{Search} comporti 
			la corretta renderizzazione del componente. & 
			\textit{I} & 
			\textit{S}\\
						
			\tabularnewline
			\hypertarget{TU73}{TU73} & Viene verificato che il metodo 
			\texttt{render()} del componente \texttt{Success} comporti 
			la corretta renderizzazione del componente. & 
			\textit{I} & 
			\textit{S}\\
						
			\tabularnewline
			\hypertarget{TU74}{TU74} & Viene verificato che venga 
			costruito correttamente il componente \texttt{TransactionsManager} 
			per la gestione delle fatture e del credito/debito IVA. & 
			\textit{I} & 
			\textit{S}\\
						
			\tabularnewline
			\hypertarget{TU75}{TU75} & Viene verificato che il metodo 
			\texttt{downloadPDF()} permetta di scaricare il pdf delle 
			fatture. & 
			\textit{I} & 
			\textit{S}\\
						
			\tabularnewline
			\hypertarget{TU76}{TU76} & Viene verificato che il metodo 
			\texttt{printQuarters()} ritorni il trimestre correttamente. & 
			\textit{I} & 
			\textit{S}\\
						
			\tabularnewline
			\hypertarget{TU77}{TU77} & Viene verificato che il metodo 
			\texttt{handleChange()} catturi correttamente l'evento scatenato 
			e setti di conseguenza lo stato del componente. & 
			\textit{I} & 
			\textit{S}\\
						
			\tabularnewline
			\hypertarget{TU78}{TU78} & Viene verificato che il metodo 
			\texttt{printDebitButtons()} mostri correttamente la possibilità 
			di gestione del pagamento nel caso di debito nel trimestre. & 
			\textit{I} & 
			\textit{S}\\
						
			\tabularnewline
			\hypertarget{TU79}{TU79} & Viene verificato che il metodo 
			\texttt{printStatus()} mostri correttamente la stato del saldo 
			IVA. & 
			\textit{I} & 
			\textit{S}\\
						
			\tabularnewline
			\hypertarget{TU80}{TU80} & Viene verificato che il metodo 
			\texttt{printInvoices()} mostri correttamente i dati relativi 
			alle fatture. & 
			\textit{I} & 
			\textit{S}\\

			\tabularnewline
			\hypertarget{TU81}{TU81} & Viene verificato che il metodo 
			\texttt{render()} del componente \texttt{TransactionsManager} 
			comporti la corretta renderizzazione del componente. & 
			\textit{I} & 
			\textit{S}\\
			
			\tabularnewline
			\hypertarget{TU82}{TU82} & Viene verificato che il metodo 
			\texttt{printUsers()} mostri correttamente la lista degli 
			utenti. & 
			\textit{I} & 
			\textit{S}\\
			
			\tabularnewline
			\hypertarget{TU83}{TU83} & Viene verificato che il metodo 
			\texttt{render()} del componente \texttt{UsersList} comporti la 
			corretta renderizzazione del componente. & 
			\textit{I} & 
			\textit{S}\\
			
			\tabularnewline
			\hypertarget{TU84}{TU84} & Viene verificato che il metodo 
			\texttt{printBusiness()} mostri correttamente i dati relativi 
			alle aziende e alle loro situazioni di rimborso IVA. & 
			\textit{I} & 
			\textit{S}\\

			\tabularnewline
			\hypertarget{TU85}{TU85} & Viene verificato che il metodo 
			\texttt{render()} del componente \texttt{VATRefund} comporti 
			la corretta renderizzazione del componente. & 
			\textit{I} & 
			\textit{S}\\

			\tabularnewline
		\end{longtable}

%TRACCIAMENTO TEST-METODI
		\renewcommand{\arraystretch}{1.5}
		\rowcolors{2}{pari}{dispari}
		
		\begin{longtable}{ >{\centering}p{0.10\textwidth} >{}p{0.95\textwidth}}
			
			%\tabularnewline
			\caption{Tracciamento Test di Unità-Metodi}\\	
			\rowcolorhead
			\textbf{\color{white}Codice} 
			& \centering\textbf{\color{white}Metodo}  
			\tabularnewline %\tabularnewline 
			\endfirsthead	
			
			\rowcolor{white}\caption[]{(continua)}\\	
			\rowcolorhead
			\textbf{\color{white}Codice} 
			& \centering\textbf{\centering\color{white}Metodo} 
			\tabularnewline %\tabularnewline 
			\endhead	
			
			
			
			\hypertarget{TU1}{TU1} & \texttt{UserLogic::addCitizen(bytes32 \_hashIpfs, uint8 \_hashSize, uint8 \_hashFun)}\\ 
			
			%\tabularnewline
			\hypertarget{TU2}{TU2} & \texttt{UserLogic::addBusiness(bytes32 
			\_hashIpfs,uint8 \_hashSize,uint8 \_hashFun)}\\
			
			\hypertarget{TU3}{TU3} & \texttt{UserLogic::login(address 
			\_userAddress)}\\
			
			\hypertarget{TU4}{TU4} & \texttt{UserLogic::login(address 
			\_userAddress)}\\
			
			\hypertarget{TU5}{TU5} & \texttt{ProductLogic::addProduct(bytes32 
			\_hashIPFS, uint8 \_hashSize, uint8 \_hashFunction, uint8 
			\_vatPercentage, uint256 \_netPrice)}\\
			
			\hypertarget{TU6}{TU6} & \texttt{ProductLogic::function 
			modifyProduct(bytes32 \_keyHash, bytes32 \_hashIPFS, uint8 
			\_hashSize, uint8 \_hashFunction, uint8 \_vatPercentage, uint256 
			\_netPrice)}\\
			
			\hypertarget{TU7}{TU7} & 
			\texttt{ProductLogic::deleteProduct(bytes32 \_keyHash)}\\
			
			\hypertarget{TU8}{TU8} & \texttt{ProductLogic::function 
			modifyProduct(bytes32 \_keyHash, bytes32 \_hashIPFS, uint8 
			\_hashSize, uint8 \_hashFunction, uint8 \_vatPercentage, uint256 
			\_netPrice)}\\
			
			\hypertarget{TU9}{TU9} & 
			\texttt{ProductLogic::deleteProduct(bytes32 \_keyHash)}\\
			
			\hypertarget{TU10}{TU10} & \texttt{ 
			ProductLogic::addProduct(bytes32 \_hashIPFS, uint8 \_hashSize, 
			uint8 \_hashFunction, uint8 \_vatPercentage, uint256 
			\_netPrice) \newline \newline
			ProductLogic::modifyProduct(bytes32 \_keyHash, 
			bytes32 \_hashIPFS, uint8 \_hashSize, uint8 \_hashFunction, uint8 
			\_vatPercentage, uint256 \_netPrice)\newline \newline
			ProductLogic::deleteProduct(bytes32 \_keyHash)}\\
			
			\hypertarget{TU11}{TU11} & 
			\texttt{ProductLogic::calculateProductVat(bytes32 \_keyHash)}\\
			
			\hypertarget{TU12}{TU12} & 
			\texttt{ProductLogic::calculateProductGrossPrice(bytes32 
			\_keyHash)}\\
		
			\hypertarget{TU13}{TU13} & 
			\texttt{OrderLogic::registerOrder(bytes32 \_hashIpfs, uint8 
			\_hashFun, uint8 \_hashSize, address \_buyer, string calldata 
			\_period, bytes32[] calldata \_productsHash)}\\
		
			\hypertarget{TU14}{TU14} & 
			\texttt{OrderLogic::registerOrder(bytes32 \_hashIpfs, uint8 
				\_hashFun, uint8 \_hashSize, address \_buyer, string calldata 
				\_period, bytes32[] calldata \_productsHash)}\\
			
			\hypertarget{TU15}{TU15} & 
			\texttt{OrderLogic::registerOrder(bytes32 \_hashIpfs, uint8 
				\_hashFun, uint8 \_hashSize, address \_buyer, string calldata 
				\_period, bytes32[] calldata \_productsHash)}\\
			
			\hypertarget{TU16}{TU16} & 
			\texttt{OrderLocic::calculateOrderTotal(bytes32[] memory 
			\_productsHash)}\\
			
			\hypertarget{TU17}{TU17} & \texttt{VatLogic::registerVat(address 
			\_business, int256 \_vatAmount, string calldata \_period)}\\
			
			\hypertarget{TU18}{TU18} & \texttt{VatLogic::refundVat(bytes32 
			\_key)}\\
			
			\hypertarget{TU19}{TU19} & \texttt{VatLogic::refundVat(bytes32 
				\_key)}\\
			
			\hypertarget{TU20}{TU20} & \texttt{VatLogic::payVat(address \_from, 
			bytes32 \_key)}\\
			
			\hypertarget{TU21}{TU21} & \texttt{TokenCubit::transfer(address 
			\_to, uint256 \_value)}\\
			
			\hypertarget{TU22}{TU22} & \texttt{TokenCubit::transfer(address 
				\_to, uint256 \_value)}\\
			
			\hypertarget{TU23}{TU23} & \texttt{TokenCubit::transfer(address 
				\_to, uint256 \_value)}\\
			
			\hypertarget{TU24}{TU24} & \texttt{TokenCubit::mintToken(address 
			target, uint256 mintedAmount)}\\
		
			\hypertarget{TU25}{TU25} & \texttt{AddProductsManager::render()}\\
			
			\hypertarget{TU26}{TU26} & 
			\texttt{AddProductsManager::constructor(props: object)}\\
			
			\hypertarget{TU27}{TU27} & 
			\texttt{AddProductsManager::handleChange(event: event)}\\
			
			\hypertarget{TU28}{TU28} & 
			\texttt{BusinessProduct::render()}\\
			
			\hypertarget{TU29}{TU29} & 
			\texttt{Button::render()}\\
			
			\hypertarget{TU30}{TU30} & 
			\texttt{CartProduct::printProduct(product: object)}\\
			
			\hypertarget{TU31}{TU31} & 
			\texttt{Cart::render()}\\
			
			\hypertarget{TU32}{TU32} & 
			\texttt{CartProduct::enableRemove(title: string, quantity: int, 
			price: double)}\\
		
			\hypertarget{TU33}{TU33} & 
			\texttt{CartProduct::disableRemove()}\\
			
			\hypertarget{TU34}{TU34} & 
			\texttt{Checkout::constructor(props: object)}\\
			
			\hypertarget{TU35}{TU35} & 
			\texttt{Checkout::handleOptionChange(event: event)}\\
			
			\hypertarget{TU36}{TU36} & 
			\texttt{Checkout::oldShipment()}\\
			
			\hypertarget{TU37}{TU37} & 
			\texttt{Checkout::newShipment()}\\
			
			\hypertarget{TU38}{TU38} & 
			\texttt{Checkout::handleInputChange(event: event)}\\
			
			\hypertarget{TU39}{TU39} & 
			\texttt{Checkout::getAddress()}\\
			
			\hypertarget{TU40}{TU40} & 
			\texttt{Checkout::render()}\\
			
			\hypertarget{TU41}{TU41} & 
			\texttt{CubitManager::render()}\\
			
			\hypertarget{TU42}{TU42} & 
			\texttt{EditProductsManager::constructor(props: object)}\\
			
			\hypertarget{TU43}{TU43} & 
			\texttt{EditProductsManager::handleChange(event: event)}\\
			
			\hypertarget{TU44}{TU44} & 
			\texttt{EditProductsManager::render()}\\
			
			\hypertarget{TU45}{TU45} & 
			\texttt{Error::render()}\\
			
			\hypertarget{TU46}{TU46} & 
			\texttt{Footer::render()}\\
			
			\hypertarget{TU47}{TU47} & 
			\texttt{FormCubitManager::render()}\\
			
			\hypertarget{TU48}{TU48} & 
			\texttt{FormRegistration::constructor(props: object)}\\
			
			\hypertarget{TU49}{TU49} & 
			\texttt{FormRegistration::citizenFormMaker()}\\
			
			\hypertarget{TU50}{TU50} & 
			\texttt{FormRegistration::businessFormMaker()}\\
			%\tabularnewline

			\hypertarget{TU51}{TU51} & 
			\texttt{FormRegistration::handleOptionChange(event: event)}\\
			
			\hypertarget{TU52}{TU52} & 
			\texttt{FormRegistration::handleChange(event: event)}\\

			\hypertarget{TU53}{TU53} & 
			\texttt{FormRegistration::render()}\\

			\hypertarget{TU54}{TU54} & 
			\texttt{Guide::render()}\\

			\hypertarget{TU55}{TU55} & 
			\texttt{Home::render()}\\

			\hypertarget{TU56}{TU56} & 
			\texttt{NavBar::render()}\\

			\hypertarget{TU57}{TU57} & 
			\texttt{Orders::printSeller(product: object)}\\

			\hypertarget{TU58}{TU58} & 
			\texttt{Orders::printProduct(product: object)}\\

			\hypertarget{TU59}{TU59} & 
			\texttt{Orders::printOrder(order: object)}\\

			\hypertarget{TU60}{TU60} & 
			\texttt{Orders::render()}\\

			\hypertarget{TU61}{TU61} & 
			\texttt{PendingOrder::render()}\\

			\hypertarget{TU62}{TU62} & 
			\texttt{Product::constructor(props: object)}\\

			\hypertarget{TU63}{TU63} & 
			\texttt{Product::handleChange(event: event)}\\

			\hypertarget{TU64}{TU64} & 
			\texttt{Product::render()}\\

			\hypertarget{TU65}{TU65} & 
			\texttt{ProductManager::printProduct(product: object)}\\

			\hypertarget{TU66}{TU66} & 
			\texttt{ProductManager::render()}\\

			\hypertarget{TU67}{TU67} & 
			\texttt{PurchaseConfirmation::printProduct(product: object)}\\

			\hypertarget{TU68}{TU68} & 
			\texttt{PurchaseConfirmation::printOrder(order: object)}\\

			\hypertarget{TU69}{TU69} & 
			\texttt{PurchaseConfirmation::render()}\\

			\hypertarget{TU70}{TU70} & 
			\texttt{Search::render()}\\

			\hypertarget{TU71}{TU71} & 
			\texttt{Store::printProduct(product: object)}\\

			\hypertarget{TU72}{TU72} & 
			\texttt{Store::render()}\\

			\hypertarget{TU73}{TU73} & 
			\texttt{Success::render()}\\

			\hypertarget{TU74}{TU74} & 
			\texttt{TransactionsManager::constructor(props: object)}\\

			\hypertarget{TU75}{TU75} & 
			\texttt{TransactionsManager::downloadPDF()}\\

			\hypertarget{TU76}{TU76} & 
			\texttt{TransactionsManager::printQuarters(quarter: string)}\\

			\hypertarget{TU77}{TU77} & 
			\texttt{TransactionsManager::handleChange(event: event)}\\

			\hypertarget{TU78}{TU78} & 
			\texttt{TransactionsManager::printDebitButtons()}\\

			\hypertarget{TU79}{TU79} & 
			\texttt{TransactionsManager::printStatus()}\\

			\hypertarget{TU80}{TU80} & 
			\texttt{TransactionsManager::printInvoices()}\\

			\hypertarget{TU81}{TU81} & 
			\texttt{TransactionsManager::render()}\\

			\hypertarget{TU82}{TU82} & 
			\texttt{UsersList::printUser(user: object)}\\

			\hypertarget{TU83}{TU83} & 
			\texttt{UsersList::render()}\\

			\hypertarget{TU84}{TU84} & 
			\texttt{VATRefund::printBusiness(business: object)}\\

			\hypertarget{TU85}{TU85} & 
			\texttt{VATRefund::render()}\\

		\end{longtable}
