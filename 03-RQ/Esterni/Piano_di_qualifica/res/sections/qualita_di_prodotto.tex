\section{Qualità di prodotto}
Per valutare la qualità del prodotto il gruppo ha deciso di far riferimento allo standard ISO/IEC 9126\glosp{} che definisce le caratteristiche di cui tener conto per produrre un prodotto di buona qualità. Le caratteristiche sono descritte attraverso dei parametri, che ne quantificano il grado di raggiungimento. Di seguito vengono citate le voci che il gruppo ha ritenuto rilevanti nel contesto del progetto.
	\subsection{Funzionalità}
	Capacità del prodotto di fornire funzioni che riescano a soddisfare tutti i requisiti, sia espliciti che impliciti, presenti nell'Analisi dei Requisiti.
		\subsubsection{Obiettivi}
		\begin{itemize}
			\item \textbf{Appropriatezza}: il prodotto deve mettere a disposizione un insieme di funzioni conformi agli obiettivi richiesti;
			\item \textbf{Accuratezza}: il prodotto deve fornire risultati attesti con il grado di precisione richiesto;
			\item \textbf{Conformità}: il prodotto deve aderire a determinati standard. %PLACEHOLDER è forse troppo ovvio?
		\end{itemize}
		\subsubsection{Metriche}
			\paragraph{Completezza dell'implementazione}\mbox{}\\
			La completezza del prodotto e il rispetto dei requisiti viene indicato da una percentuale.
			\begin{itemize}
			\item misurazione: si calcola con la seguente formula: \\
			\centerline { C = (1 - \(\frac{N\textsubscript{FNI}}{N\textsubscript{FI}} \))$ \cdot  100$ } \\
			dove N\textsubscript{FNI} indica il numero di funzionalità non implementate e N\textsubscript{FI} indica il numero di funzionalità individuate dall'analisi;
			\item valore preferibile: 100\%;
			\item valore accettabile: 100\%.
			\end{itemize}
	\subsection{Affidabilità}
	Capacità del prodotto di mantenere prestazioni elevate anche in caso di situazioni anomale o critiche.
		\subsubsection{Obiettivi}
		\begin{itemize}
			\item maturità: il prodotto deve evitare che si verifichino errori e malfunzionamenti;
			\item tolleranza agli errori: il prodotto mantiene alte prestazioni anche in caso di malfunzionamenti o di un uso scorretto.
		\end{itemize}
		\subsubsection{Metriche}
			\paragraph{Densità errori}\mbox{}\\
			L'abilità del prodotto di resistere a malfunzionamenti viene indicata con una percentuale.
			\begin{itemize}
			\item misurazione: si calcola con la seguente formula: \\
			\centerline{ M =  \(\frac{N\textsubscript{ER}}{N\textsubscript{TE}} \)$ \cdot 100$ }
			dove N\textsubscript{ER} indica il numero di errori rilevati durante il testing e N\textsubscript{TE} indica il numero di test eseguiti;
			\item valore preferibile: 0\%;
			\item valore accettabile: $\leq$ 10\%.
			\end{itemize}
	\subsection{Usabilità}
	Capacità del prodotto di essere di facile comprensione e utilizzo da parte degli utenti.
		\subsubsection{Obiettivi}
		\begin{itemize}
			\item comprensibilità: l'utente deve essere in grado di comprendere le funzionalità offerte dal prodotto e ad utilizzarle;
			\item apprendibilità: l'utente deve poter imparare facilmente ad utilizzare il prodotto;
			\item attrattività: il prodotto deve essere piacevole da usare.
		\end{itemize}
		\subsubsection{Metriche}
			\paragraph{Facilità di utilizzo}\mbox{}\\
			La facilità con cui l'utente raggiunge ciò che vuole viene rappresentata tramite il numero di click necessari per arrivare al contenuto desiderato.
			\begin{itemize}
			\item misurazione: click per raggiungere la schermata di checkout;
			\item valore preferibile: $\leq$ 10;
			\item valore accettabile: $\leq$ 15.
			\end{itemize}
			\paragraph{Facilità di apprendimento}\mbox{}\\
			La facilità con cui l'utente riesce ad imparare ad usare le funzionalità del prodotto viene rappresentata tramite il tempo medio che serve per comprenderle.
			\begin{itemize}
			\item misurazione: minuti per raggiungere pagina di checkout;
			\item valore preferibile: $\leq$ 3;
			\item valore accettabile: $\leq$ 5.
			\end{itemize}
\pagebreak
	\subsection{Manutenibilità}
	Capacità del prodotto di essere modificato, includendo correzioni, miglioramenti o adattamenti.
		\subsubsection{Obiettivi}
		\begin{itemize}
			\item analizzabilità: facilità con la quale è possibile analizzare il codice per localizzare un errore;
			\item modificabilità: capacità del prodotto di permettere l'implementazione di una modifica.
		\end{itemize}
		\subsubsection{Metriche}
			\paragraph{Facilità di comprensione}\mbox{}\\
			La facilità con cui è possibile comprendere cosa fa il codice può rappresentata dal numero di linee di commento nel codice.
			\begin{itemize}
			\item misurazione: si può calcolare con la seguente formula: \\
			$$ R = \frac{N\textsubscript{LCOM}}{N\textsubscript{LCOD}}  $$
		
			dove N\textsubscript{LCOM} indica le linee di commento e N\textsubscript{LCOD} indica le linee di codice;
			\item valore preferibile: $\geq$ 0.20;
			\item valore accettabile: $\geq$ 0.10.
			\end{itemize}
			\paragraph{Semplicità delle funzioni}
			La facilità di un metodo può essere rappresentata dal numero di parametri per metodo: meno parametri ha una funzione più è semplice e intuitiva.
			\begin{itemize}
			\item misurazione: numero di parametri per metodo;
			\item valore preferibile $\leq$ 3;
			\item valore accettabile $\leq$ 6.
			\end{itemize}
			\paragraph{Semplicità delle classi}\mbox{}\\
			La facilità di una classe può essere rappresentata dal numero di metodi per classe: una classe con pochi metodi ha uno scopo ben preciso e facilmente comprensibile.
			\begin{itemize}
			\item misurazione: numero di metodi per classe;
			\item valore preferibile $\leq$ 8;
			\item valore accettabile $\leq$ 15.
			\end{itemize}
			\paragraph{Structural fan-in}\mbox{}\\ 
			Indica quante componenti utilizzano un dato modulo. Un alto	valore indica un alto riuso della componente.
			\begin{itemize}
				\item misurazione: conteggio delle componenti;
				\item valore preferibile: $\geq$ 1;
				\item valore accettabile: $\geq$ 0.
			\end{itemize}
			\paragraph{Structural fan-out}\mbox{}\\
			 Indica quante componenti vengono utilizzate dalla componente in esame. Un alto valore indica un alto accoppiamento della componente.
			\begin{itemize}
				\item misurazione: conteggio delle componenti;
				\item valore preferibile: = 0;
				\item valore accettabile: $\leq$ 6.
			\end{itemize}
			\paragraph{SLOC: Source Lines of Code}\mbox{}\\
			Misura le righe (fisiche, non logiche) totali di codice di cui il progetto è composto. Maggiore è la dimensione del progetto, più è difficile manutenerlo
			\begin{itemize}
				\item misurazione: numero di righe di codice;
				\item valore preferibile: $\leq5000$;
				\item valore accettabile: $\leq10000$.
			\end{itemize}

\subsection{Specifica dei test}
Capacità del prodotto di soddisfare le richieste. \\
Per assicurare la qualità del software prodotto, il gruppo \textit{8Lab
	Solutions} adotta come modello di sviluppo del software il
\textbf{Modello a V\glo}, il quale prevede lo sviluppo dei test in parallelo alle
attività di analisi e progettazione. In questo modo i test permetteranno di
verificare sia la correttezza delle parti di programma sviluppati, sia che
tutti gli aspetti del progetto siano implementati e corretti. 
Segue quindi l'esito dei test per mezzo di tabelle che ne
semplificheranno la consultazione e che potranno fornire una precisa indicazione 
degli output prodotti, specificando se il risultato ottenuto sia quello atteso, errato
oppure non coerente a quanto fissato in precedenza.
Per definire lo stato dei test, vengono utilizzate le seguenti sigle:
\begin{itemize}
	\item \textbf{I}: per indicare che il test è stato implementato;
	\item \textbf{NI}: per indicare che il test non è stato implementato.
\end{itemize}
Inoltre per l'esito dei test si usano le seguenti abbreviazioni:
\begin{itemize}
	\item \textbf{S}: per indicare che il test ha soddisfatto la richiesta;
	\item \textbf{NS}: per indicare che il test non ha soddisfatto la richiesta.
\end{itemize}

\noindent La codifica dei seguenti test è presente nel documento \textit{Norme di Progetto v3.0.0}.
\subsubsection{Obiettivi}
	\begin{itemize}
		\item accettazione: dimostrazione della conformità del prodotto sulla base di casi di prova;
		\item sistema: accertamento della totale copertura dei requisiti software;
		\item integrazione: costruzione incrementale del sistema e rilevamento dei problemi derivanti dall'integrazione e dal riutilizzo dei vari componenti;
		\item unità: verifica della correttezza del codice implementato.
	\end{itemize}
\subsubsection{Metriche}
\paragraph{Passed Test Cases Percentage}\mbox{}\\
Assicurare il superamento dei test è fondamentale per poter verificare la corretta implementazione delle funzionalità previste dai requisiti.
\begin{itemize}
	\item misurazione: percentuale superamento dei Test (PTCP);
	\item valore preferibile: 90\% $\leq$ PTCP $\leq$ 100\%;
	\item valore accettabile: 80\% $\leq$ PTCP $\leq$ 90\%.
\end{itemize}
\paragraph{Failed Test Cases Percentage}\mbox{}\\
Assicurare che il numero di test falliti sia inferiore a una certa soglia, in modo da garantire la corretta implementazione delle funzionalità previste.
\begin{itemize}
	\item misurazione: percentuale fallimento dei Test (FTCP);
	\item valore preferibile: 0\% $\leq$ FTCP $\leq$ 10\%;
	\item valore accettabile: 10\% $\leq$ FTCP $\leq$ 20\%.
\end{itemize}
		
\pagebreak
