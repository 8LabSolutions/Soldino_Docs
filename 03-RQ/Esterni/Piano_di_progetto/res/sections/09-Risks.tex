\section{Attualizzazione dei rischi}

\rowcolors{2}{pari}{dispari}
\renewcommand{\arraystretch}{1.5}
\begin{longtable}{
		 >{\centering}p{0.10\textwidth}
		 >{}p{0.4\textwidth}
		 >{}p{0.4\textwidth}
	 }
 	\caption{Tabella attuazione dei rischi }\\
 	
	\rowcolorhead 
		\textbf{\color{white}Rischio}	&
		\centering\textbf{\color{white}Descrizione} &
		\centering\textbf{\color{white}Contromisura adottata}
		\tabularnewline 		
	\endhead
	RT1 & 
	I programmatori hanno riscontrato alcuni bug nell'ultima release di un framework\glosp utilizzato. & 
	È stato deciso di utilizzare una versione precedente di tale framework\glosp che non presentava i bug individuati.
	\tabularnewline
	
	RO1 & 
	Modifiche in itinere applicate al PdP hanno modificato considerevolmente le task e gli assegnatari. & 
	Abbiamo deciso di pianificare a grana più fine per avere una maggiore corrispondenza tra PdP e task da assegnare. Così facendo, le task risultano più stabili e coerenti con gli obiettivi di progetto.
	\tabularnewline
	
	RO3 &
	Tutti i componenti del gruppo hanno dovuto sostenere alcuni esami. Pertanto, in alcuni periodi non tutti i componenti hanno potuto dare la propria disponibilità. &
	Le task più onerose sono state assegnate alle persone momentaneamente più disponibili.
	\tabularnewline
	
	RO5 &
	Si sono verificati ritardi nell'esecuzione dei task, a causa di un cambio delle priorità nel progetto. &
	Le priorità sono state definite più precisamente e task sono stati riorganizzati di conseguenza.
	\tabularnewline
	
	RO6 &
	Sovente è accaduto di non trovare un'idoneo luogo di lavoro in Torre Archimede.&
	La maggior parte delle volte il gruppo si è spostato in plesso Paolotti.
	\tabularnewline
	
	RO8 &
	La bacheca di Trello è risultata talvolta sovraffollata. &
	Le task sono state suddivise in modo da facilitarne il completamento. Se le task sono sufficientemente piccole e facili non risulta necessario imporre delle scadenze temporali. 
	\tabularnewline
	
	RI2 &
	Il rischio si è verificato in un ambito imprevisto, ovvero non in contatto con il proponente, bensì con il prof. Cardin. Ci siamo trovati impreparati in una chiamata esterna in cui non eravamo tutti fisicamente assieme.&
	Da parte del gruppo ora ci si trova nello stesso luogo per effettuare chiamate con docente o proponente, per ridurre ritardi e problemi di comunicazione.
	\tabularnewline
	
	RI5 &
	Alcuni componenti del gruppo sono talvolta visibilmente stressati e improduttivi. &
	Le soluzioni variano a seconda del membro, e includono: il supporto dei compagni del gruppo, lavorare da casa anziché in locale, ridurre le ore di lavoro (se possibile), richiedere task precisi su cui concentrarsi, fare attività sportiva, alternare periodo di lavoro al computer a periodi all'aria aperta.
	\tabularnewline
	
		
\end{longtable}
