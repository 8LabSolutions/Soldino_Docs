\section{Attualizzazione dei rischi}

\rowcolors{2}{pari}{dispari}
\renewcommand{\arraystretch}{1.5}
\begin{longtable}{
		 >{\centering}p{0.10\textwidth}
		 >{}p{0.4\textwidth}
		 >{}p{0.4\textwidth}
	 }
 	\caption{Tabella attuazione dei rischi }\\
 	
	\rowcolorhead 
		\textbf{\color{white}Rischio}	&
		\centering\textbf{\color{white}Descrizione} &
		\centering\textbf{\color{white}Contromisura adottata}
		\tabularnewline 		
	\endfirsthead
	
	\rowcolor{white}\caption[]{(continua)} \\
	\rowcolorhead 
		\textbf{\color{white}Rischio}	&
		\centering\textbf{\color{white}Descrizione} &
		\centering\textbf{\color{white}Contromisura adottata}
		\tabularnewline 	
	\endhead
		
	RT2 & 
	I programmatori hanno riscontrato alcuni bug nell'ultima release della 
	libreria Web3 utilizzata. & 
	È stato deciso di utilizzare una versione precedente che non presentava 
	i bug individuati.
	\tabularnewline
	
	RO1 & 
	Le modifiche in itinere applicate al PdP hanno portato a dei cambiamenti sui 
	task assegnati. & 
	Abbiamo deciso di pianificare a grana più fine per avere una maggiore corrispondenza 
	tra PdP e task. Così facendo, i task risultano più stabili e coerenti con gli 
	obiettivi del progetto.
	\tabularnewline
	
	RO1 & 
	Lo studio del linguaggio Solidity e della libreria Web3, tecnologie recenti e di cui 
	le ultime versioni sono poco documentate, ha richiesto più tempo del dovuto. & 
	Le tempistiche sono state ricalcolate e il piano di progetto è stato modificato 
	di conseguenza.
	\tabularnewline

	RO3 &
	Tutti i componenti del gruppo hanno dovuto sostenere alcuni esami. Pertanto, in alcuni 
	periodi non tutti i componenti hanno potuto dare la propria disponibilità. &
	Le task più onerose sono state assegnate alle persone momentaneamente più disponibili.
	\tabularnewline
	
	RO5 &
	Si sono verificati ritardi nell'esecuzione dei task, a causa di un cambio delle 
	priorità nel progetto. &
	Le priorità sono state definite più precisamente e i task sono stati riorganizzati
	di conseguenza.
	\tabularnewline
	
	RO6 &
	\'E accaduto di non trovare un luogo idoneo in Torre Archimede per lavorare in
	gruppo. &
	La maggior parte delle volte il gruppo si è spostato nelle aule del plesso Paolotti.
	\tabularnewline
	
	RO8 &
	La bacheca di Trello è risultata talvolta sovraffollata. &
	I task sono stati inseriti in modo generale contenendo al suo interno task più
	piccoli e precisi.
	\tabularnewline
	
	RI2 &
	Durante una video chiamata con il committente Cardin, non tutto il gruppo era 
	riunito fisicamente e il collegamento tra parte del gruppo, il committente e 
	i membri del gruppo non presenti ha comportato dei problemi e dei ritardi nella
	comunicazione. &
	Si è deciso di contattare i committenti e i proponenti solo quando il gruppo 
	è al completo e fisicamente nello stesso luogo.	
	\tabularnewline
	
	RI5 &
	Alcuni componenti del gruppo sono talvolta visibilmente stressati e improduttivi. &
	Le soluzioni variano a seconda del membro e includono: il supporto dei compagni del gruppo,
	lavorare da casa anziché riunirsi nella sede di lavoro per evitare lo stress 
	dei mezzi di trasporto, richiedere task più piccolo e precisi su cui concentrarsi rispetto 
	a un unico task generale e impegnativo, fissare dei momenti di pausa per riprendere poi il
	lavoro con più lucidità.
	\tabularnewline
	
	RI6 &
	Un membro è stato assente alcuni giorni per malattia in prossimità di una consegna. & 
	Il lavoro del componente che si è assentato è stato redistribuito, ma essendo già
	ben avviato non ha comportato ritardi.
	\tabularnewline
	
		
\end{longtable}
