\section{Modello di sviluppo}
La scelta di un modello di sviluppo è fondamentale per la pianificazione di 
progetto e a tal fine è stato adottato il \textbf{modello incrementale}.

\subsection{Modello incrementale}
Il modello di sviluppo incrementale permette di suddividere lo sviluppo del sistema per 
incrementi, ognuno dei quali incorpora una funzionalità. \\
L'aggiunta, modifica e cancellazione di requisiti sono consentite previa discussione con il proponente e sua 
approvazione; tuttavia non sono permesse durante la fase di sviluppo dell'incremento corrente.\\
Questo modello di sviluppo si combina bene con il versionamento del sistema, che traccia le modifiche rendendole più visibili.\\
L'adozione di questo modello comporta i seguenti vantaggi:
\begin{itemize}
	\item le funzionalità primarie hanno priorità nello sviluppo, così che il proponente possa subito valutarle;
	\item si può ricevere il feedback del cliente frequentemente, anche ad ogni incremento;
	\item sviluppare per incrementi successivi limita gli errori al singolo incremento;
	\item le modifiche, l'individuazione e la correzione degli errori sono più economiche;
	\item anche le fasi di verifica e test sono facilitate, perché più mirate.
\end{itemize}
\subsubsection{Incrementi individuati} Durante i periodi di Progettazione e Codifica per la Technology Baseline e Progettazione di Dettaglio sono stati individuati alcuni incrementi. Di seguito è riportato il tracciamento incremento-requisiti, in maniera tale da comprendere meglio quali requisiti vengono soddisfatti in ciascun incremento. \\ \\
I requisiti riportati nella tabella includono tutti i requisiti figli. Tutti i requisiti non riportati nella tabella sono da intendersi soddisfatti, in parte, da ogni incremento. Tali requisiti sono reperibili all'interno del documento \textit{Analisi dei Requisiti}.

%tabella tracciamento incremento-requisiti

\rowcolors{2}{pari}{dispari}
\renewcommand{\arraystretch}{1.5}
\begin{longtable}{ >{\centering}p{0.5\textwidth}
		>{\centering}p{0.5\textwidth}}
	\caption{Tabella di tracciamento incremento-requisiti}\\
	\rowcolorhead 
	\textbf{\color{white}Incremento}
	& \textbf{\color{white}Requisiti} 
	\tabularnewline 	
	\endfirsthead
	\rowcolor{white}\caption[]{(continua)} \\
	\rowcolorhead 
	\textbf{\color{white}Incremento}
	& \textbf{\color{white}Requisiti} 
	\tabularnewline 
	\endhead

	Incremento 1 - Registrazione	&	R2F1 \\
	R1F2	\tabularnewline
	Incremento 2 - Login	&	R1F3 \\
	R1F4	\tabularnewline
	Incremento 3 - Home	&	R1F8 \\
	R1F9 \\
	R1F10 \\
	R1F11 \\
	R1F12 \\
	R2F13 \\
	R1F14 \\
	R1F18 	\tabularnewline
	Incremento 4 - Pagina Governativa	&	R1F5 \\
	R1F6 \\
	R1F7	\tabularnewline
	Incremento 5 - Pagina Azienda	&	R3F15 \\
	R1F16 \\
	R3F17 \\
	R1F19	\tabularnewline
	Incremento 6 - Pagina Cittadino	&	R3F15	\tabularnewline
	
\end{longtable}

Durante il periodo di Progettazione di dettaglio e codifica, i casi d'uso 
analizzati in precedenza e riportati nel documento \textit{Analisi dei Requisiti}
sono stati assegnati al relativo incremento individuato. Quest'operazione è stata 
ritenuta utile al fine di organizzare meglio il lavoro. Viene riportata di seguito 
la tabella del tracciamento incremento-casi d'uso, in cui i casi d'uso riportati 
includono tutti i casi d'uso figli.

%tabella tracciamento incremento-casi d'uso

\rowcolors{2}{pari}{dispari}
\renewcommand{\arraystretch}{1.5}
\begin{longtable}{ >{\centering}p{0.5\textwidth}
		>{\centering}p{0.5\textwidth}}
	\caption{Tabella di tracciamento incremento-casi d'uso}\\
	\rowcolorhead 
	\textbf{\color{white}Incremento}
	& \textbf{\color{white}Casi d'uso} 
	\tabularnewline 	
	\endfirsthead
	\rowcolor{white}\caption[]{(continua)} \\
	\rowcolorhead 
	\textbf{\color{white}Incremento}
	& \textbf{\color{white}Casi d'uso} 
	\tabularnewline 
	\endhead

	Incremento 1 - Registrazione	&	
	UC1 \\
	UC2	\\
	UC3 \\
	UC4 \\
	UC5
	\tabularnewline
	
	Incremento 2 - Login	&	
	UC6 \\
	UC7 \\
	UC8 
	\tabularnewline
	
	Incremento 3 - Home	&
	UC9 \\
	UC10 \\
	UC11 \\
	UC12 \\
	UC13 \\
	UC14 	
	\tabularnewline
	
	Incremento 4 - Pagina Governativa	&
	UC9 \\
	UC16 \\
	UC17 \\
	UC18 \\
	UC19 \\
	UC20 \\
	UC21 \\
	UC22
	\tabularnewline
	
	Incremento 5 - Pagina Azienda	&	
	UC9 \\
	UC15 \\
	UC21 \\
	UC22 		
	\tabularnewline
	
	Incremento 6 - Pagina Cittadino	&
	UC9 \\
	UC21
		
	\tabularnewline
	
\end{longtable}




























