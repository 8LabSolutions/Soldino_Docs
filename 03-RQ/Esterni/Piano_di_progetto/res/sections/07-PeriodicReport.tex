\section{Consuntivi di periodo}
Di seguito verranno indicate le spese effettivamente sostenute, considerando sia quelle per ruolo sia quelle per persona. Il bilancio potrà risultare:
\begin{itemize}
	\item \textbf{Negativo:} se il consuntivo supera il preventivo;
	\item \textbf{Pari:} se il consuntivo e il preventivo sono pari;
	\item \textbf{Positivo:} se il preventivo supera il consuntivo.
\end{itemize}

\subsection{Periodo di analisi}
Le ore di lavoro sostenute in questa fase sono da considerarsi come ore di investimento per l'approfondimento personale. Esse sono quindi non rendicontate.

\begin{table}[H]
				\centering\renewcommand{\arraystretch}{1.5}
				\caption{Consuntivo di periodo della fase di Analisi}
				\vspace{0.2cm}
                \begin{tabular}{c c c}
                               
                \rowcolorhead
                 {\colorhead \textbf{Ruolo}} &
                 {\colorhead \textbf{Ore}} & 
                 {\colorhead \textbf{Costo}} \\
				
                \rowcolorlight
                 {\colorbody Responsabile} & {\colorbody 31 (+0)} & 
                 {\colorbody \EUR{930,00} (+\EUR{0,00})}  
				\\
				
				\rowcolordark
                 {\colorbody Amministratore} & {\colorbody 44 (+19)} & 
                 {\colorbody \EUR{880,00} (+\EUR{380,00})}
				\\	
				
				\rowcolorlight
                 {\colorbody Analista} & {\colorbody 96 (+7)} & 
                 {\colorbody \EUR{2.400,00} (+\EUR{175,00})} 
				\\
				
				\rowcolordark
                 {\colorbody Progettista} & {\colorbody 19
                 (+7)} & 
                 {\colorbody \EUR{418,00} (+\EUR{154,00})} 
				\\
				
				\rowcolorlight
                 {\colorbody Programmatore} & {\colorbody -} & 
                 {\colorbody -} 
				\\
				
				\rowcolordark
                 {\colorbody Verificatore} & {\colorbody 90 (+7)} & 
                 {\colorbody \EUR{1.350,00} (+\EUR{105,00})} 
				\\
				
				\rowcolorlight
                 {\colorbody \textbf{Totale Preventivo}} & {\colorbody \textbf{280}} & 
                 {\colorbody \textbf{\EUR{5.978,00}}} 
				\\
				
				
				\rowcolordark
                 {\colorbody \textbf{Totale Consuntivo}} & {\colorbody \textbf{320}} & 
                 {\colorbody \textbf{\EUR{6.792,00}}} 
				\\
				
				
				\rowcolorlight
                 {\colorbody \textbf{Differenza}} & {\colorbody \textbf{40}} & 
                 {\colorbody \textbf{\EUR{+814,00}}} 
				\\
				
                

                \end{tabular}
                
\end{table}

\subsubsection{Conclusioni}
Come emerge dai dati riportati nella tabella soprastante, che presenta le ore relative al consuntivo della fase di Analisi, è stato necessario investire più tempo del previsto nei ruoli di \textit{Amministratore}, \textit{Analista}, \textit{Progettista} e \textit{Verificatore}. Per questo motivo il bilancio risultante è negativo. Di seguito sono riportate le cause di tali ritardi:
\begin{itemize}
	\item \textbf{\textit{Amministratori}}: è servito più tempo del previsto per riuscire ad individuare i software più adatti per la gestione del progetto, e per la loro corretta configurazione. Inoltre sono state aggiunte ed aggiornate alcune sezioni nelle \textit{Norme di Progetto}, necessarie al chiarimento di alcune problematiche sorte durante la stesura dei documenti;
	\item \textbf{\textit{Analisti}}: alcuni requisiti si sono rivelati di non facile comprensione, e sono state necessarie più ore di lavoro per la discussione interna (tra gli \textit{Analisti)} ed esterna (con i proponenti); 
	\item \textbf{\textit{Progettisti}}: l'elevato numero di requisiti individuati nell'\textit{Analisi dei Requisiti} ha comportato un aumento del tempo necessario alla stesura dei test;
	\item \textbf{\textit{Verificatori}}: l'aggiunta di nuove sezioni nelle Norme di Progetto e l'elevato numero di requisiti individuati hanno implicato un maggiore lavoro anche per questo ruolo. 
\end{itemize}
Il notevole quantitativo di ore che il gruppo ha dovuto impiegare nel primo periodo non deve ripetersi durante il lavoro rendicontato. Per le problematiche riscontrate verranno adottate le seguenti contromisure:
\begin{itemize}
	\item \textbf{configurazione della strumentazione}: buona parte degli strumenti sono già stati configurati. La configurazione della rimanente strumentazione verrà effettuata all'inizio della seconda fase, al fine di evitare un successivo consumo di ore per uniformare la strumentazione stessa ed i prodotti dei diversi elementi del gruppo;
	\item \textbf{comprensione dei requisiti}: i requisiti sono stati ampiamente discussi con il proponente durante questa fase, non si prevede di incorrere ulteriormente in tale problema;
	\item \textbf{scorretta implementazione ed utilizzo iniziale delle Norme di Progetto}: il gruppo ha finito la stesura del documento ed ha imparato ad applicare le norme in esso definite.
\end{itemize}

\subsubsection{Preventivo a finire} 
Essendo il primo periodo non rendicontato, non è necessario prevedere alcuna contromisura dal punto di vista del monte ore totale nonché del preventivo economico. I componenti del gruppo dovranno adottare le contromisure sopra descritte e lavorare con il metodo acquisito durante questo primo periodo.

\subsection{Periodo di consolidamento dei requisiti}
Le ore di lavoro sostenute in questa periodo sono relative al consolidamento dei requisiti, successivo al periodo di analisi. Alcune ore sono state dedicate allo studio per l'approfondimento personale e sono da considerarsi come ore non rendicontate. Per tale motivo tali ore non sono state riportate nella seguente tabella.

\begin{table}[H]
	\centering\renewcommand{\arraystretch}{1.5}
	\caption{Consuntivo di periodo della fase di Consolidamento dei requisiti}
	\vspace{0.2cm}
	\begin{tabular}{c c c}
		
		\rowcolorhead
		{\colorhead \textbf{Ruolo}} &
		{\colorhead \textbf{Ore}} & 
		{\colorhead \textbf{Costo}} \\
		
		\rowcolorlight
		{\colorbody Responsabile} & {\colorbody 3 (+0)} & 
		{\colorbody \EUR{90,00} (+\EUR{0,00})}  
		\\
		
		\rowcolordark
		{\colorbody Amministratore} & {\colorbody 5 (+0)} & 
		{\colorbody \EUR{100,00} (+\EUR{0,00})}
		\\	
		
		\rowcolorlight
		{\colorbody Analista} & {\colorbody 15 (+0)} & 
		{\colorbody \EUR{375,00} (+\EUR{0,00})} 
		\\
		
		\rowcolordark
		{\colorbody Progettista} & {\colorbody -
			} & 
		{\colorbody -} 
		\\
		
		\rowcolorlight
		{\colorbody Programmatore} & {\colorbody -} & 
		{\colorbody -} 
		\\
		
		\rowcolordark
		{\colorbody Verificatore} & {\colorbody 17 (+0)} & 
		{\colorbody \EUR{255,00} (+\EUR{0,00})} 
		\\
		
		\rowcolorlight
		{\colorbody \textbf{Totale Preventivo}} & {\colorbody \textbf{40}} & 
		{\colorbody \textbf{\EUR{820,00}}} 
		\\
		
		
		\rowcolordark
		{\colorbody \textbf{Totale Consuntivo}} & {\colorbody \textbf{40}} & 
		{\colorbody \textbf{\EUR{820,00}}} 
		\\
		
		
		\rowcolorlight
		{\colorbody \textbf{Differenza}} & {\colorbody \textbf{-}} & 
		{\colorbody \textbf{-}} 
		\\
		
		
		
	\end{tabular}
	
\end{table}

\subsubsection{Conclusioni}
Come emerge dai dati riportati nelle tabella soprastante, le ore di lavoro effettivamente impiegate dai diversi ruoli coincidono con quanto riportato nel preventivo. Il gruppo, diversamente dal periodo di Analisi dei requisiti, è riuscito a rientrare nelle ore di lavoro preventivate. Tale miglioramento nella gestione della mole di lavoro è causato, oltre alla ridotta durata di tale periodo, all'esperienza guadagnata dai membri nel primo periodo.

\subsubsection{Preventivo a finire} 
Considerando che le ore effettivamente utilizzate coincidono con quelle preventivate, ed essendo il periodo in considerazione non rendicontato, non è necessario prendere alcun accorgimento e/o modificare i preventivi riguardanti i periodi rimanenti.

\subsection{Periodo di progettazione e codifica per la Technology Baseline}
Le ore sostenute durante questo periodo sono relative alla progettazione ed alla codifica del Proof of Concept\glo, necessario al soddisfacimento della Techology Baseline\glo. Tale periodo, successivo al Consolidamento dei requisiti, è da considerarsi rendicontato, in quanto il capitolato d'appalto è stato aggiudicato e quindi il lavoro è svolto con lo scopo di sviluppare il prodotto finale. 

\begin{table}[H]
	\centering\renewcommand{\arraystretch}{1.5}
	\caption{Consuntivo di periodo della fase di Progettazione e Codifica per la Technology Baseline}
	\vspace{0.2cm}
	\begin{tabular}{c c c}
		
		\rowcolorhead
		{\colorhead \textbf{Ruolo}} &
		{\colorhead \textbf{Ore}} & 
		{\colorhead \textbf{Costo}} \\
		
		\rowcolorlight
		{\colorbody Responsabile} & {\colorbody 10 (+2)} & 
		{\colorbody \EUR{300,00} (+\EUR{60,00})}  
		\\
		
		\rowcolordark
		{\colorbody Amministratore} & {\colorbody 22 (+10)} & 
		{\colorbody \EUR{440,00} (+\EUR{200,00})}
		\\	
		
		\rowcolorlight
		{\colorbody Analista} & {\colorbody 30 (+0)} & 
		{\colorbody \EUR{750,00} (+\EUR{0,00})} 
		\\
		
		\rowcolordark
		{\colorbody Progettista} & {\colorbody 67
			(-32)} & 
		{\colorbody \EUR{1474,00} (-\EUR{704,00})} 
		\\
		
		\rowcolorlight
		{\colorbody Programmatore} & {\colorbody 30 (+50)} & 
		{\colorbody \EUR{450,00} (+\EUR{750,00})} 
		\\
		
		\rowcolordark
		{\colorbody Verificatore} & {\colorbody 65 (-30)} & 
		{\colorbody \EUR{975,00} (-\EUR{450,00})} 
		\\
		
		\rowcolorlight
		{\colorbody \textbf{Totale Preventivo}} & {\colorbody \textbf{224}} & 
		{\colorbody \textbf{\EUR{4.389,00}}} 
		\\
		
		
		\rowcolordark
		{\colorbody \textbf{Totale Consuntivo}} & {\colorbody \textbf{224}} & 
		{\colorbody \textbf{\EUR{4.245,00}}} 
		\\
		
		
		\rowcolorlight
		{\colorbody \textbf{Differenza}} & {\colorbody \textbf{0}} & 
		{\colorbody \textbf{\EUR{-144,00}}} 
		\\
		
		
		
	\end{tabular}
	
\end{table}

\subsubsection{Conclusioni}
Dai dati riportati nella tabella soprastante si evince che la progettazione riguardante tale periodo ha subito una sostanziale modifica. Il gruppo aveva infatti pianificato di redarre una completa progettazione architetturale del prodotto, e di riportarla all'interno di un documento formale. Tuttavia il lavoro svolto durante questo periodo si è concentrato maggiormente sul verificare, attraverso la progettazione e la codifica del Proof of Concept\glo, che le tecnologie scelte si integrassero efficacemente tra loro, e che grazie al loro utilizzo i requisiti potessero essere soddisfatti. Di seguito sono riportate le motivazioni delle variazioni del monte ore di lavoro ricoperto dai diversi ruoli:

\begin{itemize}
	\item \textbf{\textit{Amministratori}}: dopo aver testato l'editor precedentemente stabilito, sono stati riscontrati alcuni problemi riguardanti dei plug-in\glosp necessari allo sviluppo del software. La conseguente decisione di cambiare l'editor ha comportato la necessità di ore aggiuntive di lavoro da parte degli amministratori. Un'altra causa di tale scostamento dalle ore preventivate sono alcune difficoltà riscontrate nel set-up del sistema di Continuous Integration\glo;
	\item \textbf{\textit{Progettisti}}: visto il cambiamento dell'obiettivo finale di tale periodo, i progettisti hanno dovuto occuparsi solamente di alcune componenti del prodotto finale. Ciò ha richiesto molto meno tempo rispetto a quanto preventivato;
	\item \textbf{\textit{Programmatori}}: visto il cambiamento dell'obiettivo finale di tale periodo, la quantità di ore assegnate al ruolo di programmatore ha subito un notevole aumento. Parte dell'ammontare è stato aggiunto per risolvere alcune problematiche nate dall'utilizzo delle ultime versioni di alcuni framework\glosp utilizzati. Questi infatti contenevano dei bug non risolti e/o scarsa documentazione;
	\item \textbf{\textit{Verificatori}}: il cambiamento dell'obiettivo di tale periodo ha comportato una diminuzione delle ore dedicate alla verifica. Era stata preventivata infatti la redazione di un documento aggiuntivo, che non è stato più redatto.
\end{itemize}
Le ore di lavoro relative ai ruoli che hanno subito delle variazioni rispetto a quanto pianificato sono state distribuite in maniera tale che ogni elemento del gruppo svolgesse lo stesso monte ore di lavoro complessivo.
\subsubsection{Preventivo a finire} 
Il bilancio economico risultante è positivo, ovvero sono stati risparmiati {\EUR{144,00}. Tali fondi verranno impiegati nei prossimi periodi per far fronte ad eventuali ritardi o per implementare i requisiti opzionali.
	
\subsection{Periodo di progettazione di dettaglio e codifica}
Le ore sostenute durante questo periodo sono relative alla redazione alla codifica necessaria 
per la realizzazione della Product Baseline\glo. Tale periodo è da considerarsi 
rendicontato in quanto il lavoro è svolto con lo scopo di sviluppare il prodotto finale.

\begin{table}[H]
	\centering\renewcommand{\arraystretch}{1.5}
	\caption{Consuntivo di periodo della fase di progettazione di dettaglio e codifica}
	\vspace{0.2cm}
	\begin{tabular}{c c c}
		\rowcolorhead
		{\colorhead \textbf{Ruolo}} &
		{\colorhead \textbf{Ore}} & 
		{\colorhead \textbf{Costo}} \\
		
		\rowcolorlight
		{\colorbody Responsabile} & {\colorbody 21(+0)} & 
		{\colorbody \EUR{630,00} (+\EUR{0,00})}  
		\\
		
		\rowcolordark
		{\colorbody Amministratore} & {\colorbody 29(+0)} & 
		{\colorbody \EUR{580,00} (+\EUR{0,00})}
		\\	
		
		\rowcolorlight
		{\colorbody Analista} & {\colorbody 0(+4)} & 
		{\colorbody \EUR{0,00} (+\EUR{100,00})} 
		\\
		
		\rowcolordark
		{\colorbody Progettista} & {\colorbody 90(+30)} & 
		{\colorbody \EUR{1980,00} (+\EUR{660,00})} 
		\\
		
		\rowcolorlight
		{\colorbody Programmatore} & {\colorbody 157(-27)} & 
		{\colorbody \EUR{2355,00} (-\EUR{405,00})} 
		\\
		
		\rowcolordark
		{\colorbody Verificatore} & {\colorbody 103(+0)} & 
		{\colorbody \EUR{1545,00} (+\EUR{0,00})} 
		\\
		
		\rowcolorlight
		{\colorbody \textbf{Totale Preventivo}} & {\colorbody \textbf{400}} & 
		{\colorbody \textbf{\EUR{7.090,00}}} 
		\\
		
		\rowcolordark
		{\colorbody \textbf{Totale Consuntivo}} & {\colorbody \textbf{407}} & 
		{\colorbody \textbf{\EUR{7.445,00}}} 
		\\
		
		\rowcolorlight
		{\colorbody \textbf{Differenza}} & {\colorbody \textbf{7}} & 
		{\colorbody \textbf{\EUR{+355,00}}} 
		\\
		
		\rowcolordark
		{\colorbody \textbf{Totale con risparmio(-\EUR{144,00})}} &  & 
		{\colorbody \textbf{\EUR{211,00}}} 
		\\	
	\end{tabular}
\end{table}

\subsubsection{Conclusioni}
Dai dati riportati nella tabella soprastante si evince che la progettazione ha subito una sostanziale modifica, di conseguenza le ore del programmatore hanno subito un notevole ridimensionamento. 
Di seguito sono riportate le problematiche sorte durante questa fase di lavoro e le motivazioni degli scostamenti del mantenimento del monte ore di 
lavoro preventivato:
\begin{itemize}
	\item \textbf{\textit{Amministratori}}: gli amministratori hanno riscontrato alcuni problemi durante la configurazione dell'utility \texttt{solidity-coverage}, ma li hanno risolti in tempi contenuti, quindi le ore dedicate all'amministrazione non sono aumentate;
	\item \textbf{\textit{Analisti}}: sono state aggiunte alcune ore all'Analista non preventivate, in quanto c'è stato bisogno di effettuare le correzioni segnalate sull'\textit{Analisi dei Requisiti} e, come indicato nel \textit{Piano di Qualifica v3.0.0}, alcune ore sono state dedicate al supporto dei Progettisti nella stesura di diagrammi relativi alla progettazione;
	\item \textbf{\textit{Progettisti}}: alcuni problemi si sono riscontrati con l'identificazione dei design pattern da applicare al backend, in quanto per incompatibilità, si è dovuto procedere con la ricerca di ulteriori pattern, oltre a quelli studiati a lezione. Inoltre si sono riscontrate difficoltà nella progettazione dei moduli per l'ottimizzazione delle transazioni e nella sicurezza legata all'accesso ai dati. Nella definizione dei prodotti si sono riscontrati problemi legati a limiti del linguaggio Solidity, dunque si è deciso di utilizzare nuovi costrutti e di portare delle modifica della logica. L'introduzione di IPFS\glo{} ha richiesto un numero di ore necessario allo studio della tecnologia;
	\item \textbf{\textit{Programmatori}}: a seguito di quanto specificato relativamente alla progettazione, la programmazione ha subito un periodo di stallo in attesa del termine della rivisitazione della progettazione e la risoluzione dei problemi con Solidity. D'altra parte, una volta terminata la rivisitazione della progettazione, è stato possibile procedere più rapidamente nel lavoro di implementazione.
\end{itemize}
\subsubsection{Preventivo a finire}
Il bilancio economico risultante è negativo, mitigato dal fatto di aver un fondo cuscinetto dalla fase precedente di \EUR{144,00} che porta ad avere una perdita di \EUR{211,00}. Tale perdita sarà tenuta in considerazione nella fase successiva, in quanto si prevede di recuperare il debito grazie alla rivisitazione della progettazione effettuata che ci consentirà di risparmiare ore nella fase di testing e codifica in generale che attualmente si trova ad uno stato avanzato.
\\
