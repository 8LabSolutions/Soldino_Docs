
\documentclass[a4paper]{article}
\usepackage{pdflscape}
%Tutti gli usepackage vanno qui

\usepackage{geometry}
\usepackage[italian]{babel}
\usepackage[utf8]{inputenc}
\usepackage[T1]{fontenc}
\usepackage{tabularx}
\usepackage{longtable}
\usepackage{fancyhdr}
\usepackage{titlesec}
\setcounter{secnumdepth}{4}
\usepackage{amsmath, amssymb}
%\usepackage{caption}
\usepackage{graphicx}
%\usepackage{float}
\usepackage{layouts}
%per mettere le note in fondo alla pagina
\usepackage[bottom]{footmisc}
\usepackage{hyperref}

%Comandi di impaginazione uguale per tutti i documenti
\pagestyle{fancy}
\lhead{\includegraphics[scale=0.07]{res/images/logo8_crop.png}}
%Titolo del documento
\rhead{\doctitle{}}
%\lfoot{\rightmark}
\rfoot{\thepage}
\cfoot{}
\setlength{\headheight}{35pt}
\setcounter{tocdepth}{4}
\setcounter{secnumdepth}{4}
\renewcommand{\footrulewidth}{0.4pt}
%Lista dei comandi personalizzati
\newcommand{\doctitle}{Studio di Fattibilità}
\newcommand{\rev}{0.0.3}
\newcommand{\approv}{ }
\newcommand{\ver}{ }
\newcommand{\red}{ }
\newcommand{\stato}{ }
\newcommand{\uso}{Interno}
\newcommand{\describedoc}{Studio di fattibilità dei capitolati proposti.}
%Comandi per mettere a pedice la G
%Con questo comando si pone la G e non viene aggiunto uno spazio (da usare per la punteggiatura)
\newcommand{\glo}{$_{G}$}
%Con questo comando si pone la G e viene aggiunto lo spazio (da usare tra parole)
\newcommand{\glosp}{$_{G}$ }



\usepackage{listings}
\usepackage{color}

\definecolor{dkgreen}{rgb}{0,0.6,0}
\definecolor{gray}{rgb}{0.5,0.5,0.5}
\definecolor{mauve}{rgb}{0.58,0,0.82}
%define Javascript language
\lstdefinelanguage{JavaScript}{
	keywords={typeof, new, true, false, catch, function, return, null, catch, switch, var, if, in, while, do, else, case, break, then, deployer, deploy, artifacts, require},
	keywordstyle=\color{blue}\bfseries,
	ndkeywords={class, export, boolean, throw, implements, import, this},
	ndkeywordstyle=\color{darkgray}\bfseries,
	identifierstyle=\color{black},
	sensitive=false,
	comment=[l]{//},
	morecomment=[s]{/*}{*/},
	commentstyle=\color{purple}\ttfamily,
	stringstyle=\color{red}\ttfamily,
	morestring=[b]',
	morestring=[b]"
}
\lstset{
	language=JavaScript,
	extendedchars=true,
	basicstyle=\footnotesize\ttfamily,
	showstringspaces=false,
	showspaces=false,
	numbers=left,
	numberstyle=\footnotesize,
	numbersep=9pt,
	tabsize=2,
	breaklines=true,
	showtabs=false,
	captionpos=b
}



\makeindex
\begin{document}
\thispagestyle{empty}
\begin{titlepage}
	\begin{center}
		\includegraphics[scale = 0.3]{res/images/logo8_crop.png}\\
		\large \textbf{8Lab Solutions - Progetto "Soldino"} \\
		\vfill
		\Huge \textbf{Norme di Progetto}
		\vspace*{\fill} 
        \vfill
        \large
        \begin{tabular}{r|l}
                        \textbf{Versione} & 1.0.0\\
                        \textbf{Approvazione} &\\
                        \textbf{Redazione} &\\
                        \textbf{Verifica} &\\
                        \textbf{Stato} &\\
                        \textbf{Uso} & Interno\\
                        \textbf{Destinato a} & \parbox[t]{5cm}{8Lab Solutions\\Prof. Tullio Vardanega\\Prof. Riccardo Cardin}
                \end{tabular}
                \vfill
                \normalsize
                \textbf{Descrizione}\\
                Questo documento descrive le regole, gli strumenti e le convenzioni adottate dal gruppo 8Lab Solutions e necessariamente rispettate durante la realizzazione del progetto Soldino.\\
                \vfill
                \small
                \texttt{8labsolutions@gmail.com}
	\end{center}
\end{titlepage}

\pagebreak
\section*{Changelog}
\renewcommand{\arraystretch}{1.5}
\rowcolors{2}{pari}{dispari}
	\begin{longtable}{ 
			>{\centering}p{0.09\textwidth} 
			>{\centering}p{0.13\textwidth}
			>{\centering}p{0.2\textwidth} 
			>{\centering}p{0.16\textwidth} 
			>{}p{0.2775\textwidth} }
		
		\rowcolorhead
		\textbf{\color{white}Version} & 
		\textbf{\color{white}Date} & 
		\textbf{\color{white}Name} & 
		\textbf{\color{white}Role} &
		\centering \textbf{\color{white}Description} 
		\tabularnewline  
		\endfirsthead
		\rowcolorhead
		\textbf{\color{white}Version} & 
		\textbf{\color{white}Date} & 
		\textbf{\color{white}Name} & 
		\textbf{\color{white}Role} &
		\centering \textbf{\color{white}Description} 
		\tabularnewline  
		\endhead
		
		1.0.0 & 2019-04-29 & Francesco Donè & \textit{Project Manager} &
		Approval
		\tabularnewline
		
		0.2.0 & 2019-04-27 & Matteo Santinon & \textit{Verifier} &
		Verification
		\tabularnewline
		
		0.1.4 & 2019-04-26 & Giacomo Greggio & \textit{Developer} &
		Expanded §5.6
		\tabularnewline
		
		0.1.3 & 2019-04-24 & Samuele Giuliano Piazzetta & \textit{Developer} &
		Expanded §5.4, §5.5
		\tabularnewline
		
		0.1.2 & 2019-04-23 & Giacomo Greggio & \textit{Developer} &
		Expanded §4.6
		\tabularnewline
		
		0.1.1 & 2019-04-21 & Giacomo Greggio & \textit{Developer} &
		Corrected §3.5 and §4.5
		\tabularnewline
%		roles suitable: Developer, verifier, project manager
		
		0.1.0 & 2019-04-06 & Paolo Pozzan & \textit{Verifier} &
		Document checked.
		\tabularnewline
		
		0.0.7 & 2019-04-05 & Sara Feltrin & \textit{Developer} &
		Written §5.3, §5.4 and §6.
		\tabularnewline
		
		0.0.6 & 2019-04-04 & Francesco Don\`e & \textit{Developer} &
		Written §4.4 and §4.5.
		\tabularnewline
		
		0.0.5 & 2019-04-03 & Sara Feltrin & \textit{Developer} &
		Written §3.3 and §4.3.
		\tabularnewline
		
		0.0.4 & 2019-03-30 & Francesco Don\`e & \textit{Developer} &
		Written §3.2, §4.2 and §5.2.
		\tabularnewline
		
		0.0.3 & 2019-03-28 & Sara Feltrin & \textit{Developer} &
		Written §3.1, §4.1 and §5.1.
		\tabularnewline
		
		0.0.2 & 2019-03-28 & Francesco Don\`e & \textit{Developer} &
		Written §1 and §2.
		\tabularnewline
		
		0.0.1 & 2019-03-27 & Francesco Don\`e & 
		\textit{Developer} & Created the structure of the document.
		\tabularnewline
		
	
	\end{longtable}
\renewcommand{\arraystretch}{1} 


\pagebreak
\tableofcontents

\pagebreak
\listoffigures

\pagebreak
\listoftables

\pagebreak
\section{Introduction} 
This document is the user manual of \textit{Soldino}, a project by 
	\textit{8Lab Solutions} based on the Erhereum infrastructure. 
	\subsection{What is \textit{Soldino}}
	\textit{Soldino} is a platform that allows 
	users to buy goods and services, allows business to manage the VATs 
	related to those products and allows the Government to mint and distribute
	Cubits\glo, reimburse businesses of their VAT credit and manage users 
	registered on the platform.
	More precisely:
	\begin{itemize}
		\item citizens can:
		\begin{itemize}
			\item  buy goods and services;
		\end{itemize}
		\item business can:
		\begin{itemize}
			\item sell goods and services;
			\item buy goods and services;
			\item manage VAT;
		\end{itemize}
		\item Government can:
		\begin{itemize}
			\item mint Cubits\glo;
			\item distribute Cubits\glo;
			\item activate disabled users;
			\item deactivate active users;
			\item reimburse businesses of their VAT credit;
		\end{itemize}
	\end{itemize}
	\textit{Soldino} is powered by the Ethereum\glosp network, allowing a high
	degree of security due to the distributed nature of the network.
	\subsection{Glossary}
	At the end of this document there is an appendix A where you can find 
	the definitions of words that may be ambiguous or new to the users of 
	\textit{Soldino}. This words are marked with a subscript \textit{G} 
	(e.g. example\glo).
	\subsection{Requirements}
	To be able to use \textit{Soldino} you need to have the desktop version 
	one of the following browsers:
	\begin{itemize}
		\item Google Chrome version 71 or newer;
		\item Mozilla Firefox version 64 or newer.
	\end{itemize}
	You also must have the MetaMask\glosp plug-in in your browser. More 
	information will follow. 

\pagebreak
\section{Setup} 
\subsection{Requirements}
In this section all of the requirements needed are described.
\subsubsection{Browser}
\textit{Soldino} is accessible through a web interface. The currently most recent versions of the following broswers are supported:
\begin{itemize}
	\item \textbf{Mozilla Firefox}: version 64 or later;
	\item \textbf{Google Chrome} version 71 or later.
\end{itemize}

\subsubsection{Tools}
The following tools are needed:
\begin{itemize}
	\item \textbf{Git}: a famous control version system: \textit{Soldino} is hosted on GitHub;
	\item \textbf{Node.js}: a framework needed for installing dependencies;
	\item \textbf{Truffle}: needed to write and deploy contracts with ease;
	\item \textbf{Ganache}: needed to put up a local Ethereum network and check transactions in it;
	\item \textbf{Metamask}: a browser plugin used as a virtual wallet;
	\item \textbf{Surge.sh}: a web platform chosen for hosting the website interface of Soldino.
\end{itemize}

\subsubsection{Dependencies}
Soldino depends on many different packages, some for use and others for development.\\
All these packages are located in the file \texttt{package.json} which is in the root folder of the project.\\
The packages required to execute the software \textit{Soldino} are listed below.\\

\renewcommand{\arraystretch}{1.5}
\rowcolors{2}{pari}{dispari}
\begin{longtable}{ 
		>{\centering}p{0.4\textwidth} 
		>{\centering}p{0.4\textwidth}
	}
	\caption{Packages required for software usage}\\
	\rowcolorhead
	\textbf{\color{white}Software} & 
	\textbf{\color{white}Version}
	\tabularnewline  
	\endhead	
	
	% voci prese da package.json, repo "Soldino-PoC, ramo "develop"


	react-text-mask & $\geq$5.4.4
	\tabularnewline
	commondir &$\geq$1.0.1
	\tabularnewline
	history &$\geq$4.7.2
	\tabularnewline
	prop-types &$\geq$15.7.2\tabularnewline
	react &$\geq$16.8.3\tabularnewline
	react-dom &$\geq$16.8.3\tabularnewline
	react-number-format &$\geq$4.0.6\tabularnewline
	react-redux &$\geq$6.0.1\tabularnewline
	react-router &$\geq$4.3.1\tabularnewline
	react-router-dom &$\geq$4.3.1\tabularnewline
	react-router-redux &$\geq$4.0.8\tabularnewline
	react-scripts &$\geq$2.1.8\tabularnewline
	redux &$\geq$4.0.1\tabularnewline
	redux-thunk &$\geq$2.3.0\tabularnewline
	web3 & 1.0.0-beta.37\tabularnewline
	
\end{longtable}

Other packages, listed below, are required for the development.
\renewcommand{\arraystretch}{1.5}
\rowcolors{2}{pari}{dispari}
\begin{longtable}{ 
		>{\centering}p{0.4\textwidth} 
		>{\centering}p{0.4\textwidth}
	}
	\caption{Packages required for development}\\
	\rowcolorhead
	\textbf{\color{white}Software} & 
	\textbf{\color{white}Version}
	\tabularnewline  
	\endhead	
	
	% voci prese da package.json, repo "Soldino-PoC, ramo "develop"
	eslint & 5.12.0\tabularnewline
	eslint-config-airbnb &$\geq$17.1.0\tabularnewline
	eslint-loader & $\geq$2.1.2\tabularnewline
	eslint-plugin-import & $\geq$2.16.0\tabularnewline
	eslint-plugin-jsx-a11y & $\geq$6.2.1\tabularnewline
	pre-commit & $\geq$1.2.2\tabularnewline
	truffle-contract & $\geq$4.0.6
	
\end{longtable}

\subsection{Installing}
\subsubsection{Browser}
The first thing is to have your browser installed. You can get the latest chrome version  \href{https://www.google.com/chrome/}{here}.\\

\subsubsection{Git}
You should type this script in the shell for installing Git packet: \texttt{sudo apt install git}.

\subsubsection{Node}
For installing Node.js you have to digit in the shell the following commands:
\begin{enumerate}
	\item \texttt{curl -sL https://deb.nodesource.com/setup\_11.x | sudo -E bash -}
	\item \texttt{sudo apt install -y nodejs}
	\item check that \texttt{node} have been installed correctly with \texttt{node -v}.
\end{enumerate}
There is no need to install npm separately, since it is automatically installed with Node.

\subsubsection{Truffle}
You should check you have the truffle requirements:
\begin{itemize}
	\item an OS among Linux, Windows and MacOS (prefer Linux);
	\item NodeJS v8.9.4 or later (we picked version 11);
	\item Node Package Manager (npm).
\end{itemize}
then you can install Truffle by running the command: \texttt{npm install -g truffle}

\subsubsection{Ganache}
There are three step to install Ganache:
\begin{enumerate}
	\item you can download the Ganache executable at this \href{https://truffleframework.com/ganache}{link}, clicking on the download button;
	\item if you have a Linux operating system, you have to give the permissions to make the Ganache file executable. This can be done with the command \\\texttt{chmod +x path-of-the-appimage/name-of-downloaded-file.AppImage};
\end{enumerate}

\subsubsection{MetaMask}
You can add it to your browser in this way:
\begin{itemize}
	\item Chrome:  \href{https://chrome.google.com/webstore/search/metamask?hl=it}{https://chrome.google.com/webstore/search/metamask?hl=it};
	\item Firefox: \href{https://addons.mozilla.org/it/firefox/addon/ether-metamask/?src=search}{https://addons.mozilla.org/it/firefox/addon/ether-metamask/?src=search}.
\end{itemize}


\subsubsection{Surge}
You have to install Surge by executing the shell command: \texttt{npm install -g surge}.


\subsection{Configuration}
This section shows how to configure your work environment, so that it's the same as ours, in order to minimize the number and entity of troubles you will occur in.\\
Make sure to configure the tools in order. In particular, Truffle requires that Ganache is opened and is configured to execute successfully.
% clona la repo; apri truffle, apri ganache, apri browser con metamask, setta
\subsubsection{Cloning the repo}
You have to clone the \textit{Soldino} repository on GitHub: open the shell, move to the directory where you want Soldino to be, then use \texttt{git clone https://github.com/8LabSolutions/Soldino-PoC}. \textbf{LINK DA MODIFICARE CON IL LINK ALLA REPOSITORY FINALE}

\subsubsection{Ganache}
You have to go to the folder where you've put Ganache, and open it with double click.
\begin{figure}
	\centering
	\caption{Ganache UI: from top to bottom you can see the menu bar, the current configuration, the mnemonic, and the interface of the selected menu option}
	\includegraphics[scale=0.25]{res/images/ganache-ui.png}
\end{figure}
Then you have to click on the cog at the top left corner to access Ganache settings and make sure you've matched the following settings (most of which are defined in \texttt{truffle-config.js}) on each respective window:
\begin{itemize}
	\item Server:
	\begin{itemize}
		\item hostname: \texttt{127.0.0.1};
		\item port number: \texttt{9545};
		\item network it: any (you can keep the default one).
	\end{itemize}
	\item Account \& keys:
	\begin{itemize}
		\item nothing to configure here, but you should have a look at the \textbf{mnemonic}: it will help you later.
	\end{itemize}
\end{itemize}

\subsubsection{Truffle}
The configuration of Truffle is defined in the file \texttt{truffle-config.js} in the root directory of \textit{Soldino}.
On your shell type the following shell commands:
\begin{itemize}
	\item \texttt{truffle console}\\
	opens the truffle environment in the shell under the configuration defined in truffle-config.js. From now on, every command is executed in the truffle environment, from which you can exit double typing \texttt{ctrl + C}.
	\item \texttt{compile}\\
	to compile the contracts: these are compiled in in a .json format (which enables interaction with the frontend) and put into the folder defined in \texttt{truffle-config.js} at \texttt{contracts\_build\_directory} (in our case the location is \texttt{./src/contracts\_build})
	\item \texttt{migrate -{}-network development}\\
	puts the contracts runnning on the blockchain. 
\end{itemize}

\subsubsection{MetaMask}
It's necessary to create an account.
\begin{itemize}
	\item open MetaMask on your browser
	\item select get started
	\item select import wallet
	\item use the seed frase (AKA the mnemonic) copy pasting the one \item from ganache and put your password
\end{itemize}
Now your account is up and synchronized with Ganache.
Now you have to connect the wallet to Ganache network. Ganache settings are exposed in the Ganache UI just above the mnemonic.
Let's synchronize MetaMask to the same network. On the top right corner there is a drop-down menu to select the network. Select \texttt{Custom RPC} to set your own local network.\\
You are now in the MetaMask advanced settings screen. In the field \texttt{Net Network} click on \texttt{Show Advanced Options} to open the form, then insert these data:
\begin{enumerate}
	\item \textbf{New RPC URL}: http://127.0.0.1:9545 (the port number matches the one in truffle-config.js);
	\item \textbf{Nickname}: the name you wanna give to your network;
	\item \textbf{save} when you're done.
\end{enumerate}
MetaMask is now connected to Ganache.\\

To enable transactions on your local test network, you can find free ether for testing networks on this site: \href{https://faucet.metamask.io/}{https://faucet.metamask.io/}.

% PUT CUBIT TO METAMASK

\subsection{Running}
Now that you have all the required software installed and configured, it's time to get it up running.\\
All you have to do is moving to the root directory of Soldino and prompt
\begin{itemize}
	\item[]\texttt{npm run start}
\end{itemize}
This will open the website on your browser at the address \texttt{http://127.0.0.1:3000} and allow you to explore it. 


\subsection{Deploying}
This part shows you how to deploy contracts with Truffle and have them up running on Soldino.
% qua c'è surge, il deploy su ropsten, etc. VAnno cambiati alcuni file nei parametri di configurazione
% 

% La struttura interna delle sezioni di React, Redux, Web3 e Solidity e` indicativa, non tassativa. Specie per Web3, che e` quello meno strutturato, e che i 353 non hanno descritto in questo documento

\pagebreak
\section{React} 

\subsection{Overview}
React is a JavaScript library, based on npm and made by Facebook, for building user intrerfaces by assembling user-defined components.
\subsection{Components}
Components are the key of this npm module, \textit{Soldino} is made by two types of component:
\begin{itemize}
	\item Presentational components;
	\item Container components.
\end{itemize}
\subsubsection{Presentational}
Each presentational component implements \textit{Component} interface provided by the library. According to this interface, some methods are inherited to the presentational components, in particular our components uses \textit{Render()} method for rendering themselves.
Most components can be customized when they are created, with different parameters. These creation parameters are called \textit{props}. Each presentational component returns a single HTML tag, here we can customize the returned tag with some Bootstrap classes. The props are accessible, by components which own them, referencing to them with \textit{this.props.propsName}.
\subsubsection{Containers} 
This type of component is like a wrapper of presentational components. A function called \textit{Connect(arg1, arg2)} is needed for connecting a container component to a presentational component, in this way we are able to map inside the presentational component some actions and some application state variables passing them through props. The \textit{arg1} is a function called \textit{mapStateToProps()} and the \textit{arg2} is another function called \textit{mapDispatchToProps()}.
\subsection{React-Router} 
React-router-dom is a npm module used for rendering different pages without reloading the entire website, this module works with \textit{Render()} method provided by each presentational component. \textit{Soldino} is a single page application, so the router, implemented into a JavaScript file called \textit{App.js}, is responsible to manage which components should be rendered.
% \subsection{Collaborations % } if there is one, sequence diag
% \subsection{How to extend} Super optional


\pagebreak
\section{Redux} 
\subsection{Overview}
\subsection{Unidirectional flux} %pattern
\subsection{Components} %store and state, action, reducer n dispatch
\subsection{How to extend} 

\pagebreak
\section{Web3} 
% section names are optional, not mandatory
\subsection{Overview}
\subsection{Collaborations} % connecting with MM / with Contracts / with Redux / making jsons 

\pagebreak
\section{IPFS} 

It is well known that the Ethereum blockchain is not suited for saving data. You just need to know that for hypothetically uploading a file on Ethereum, you would have to pay about 2\$ per kilobyte. 
\\
For these reasons, we adopted the InterPlanetary File System (IPFS) protocol in the homonymous peer-to-peer network to store all the information that is not involved in VAT management. Specifically, for security purposes, all the data used to make payments between the clients are exclusively stored in the Ethereum blockchain. All the remaining data, such as not sensitive clients' data, products' data, orders' data, are stored into IPFS since it is a free and distributed technology born with the aim of sharing every kind of content.

\subsection{Architecture overview}

To establish a connection to the IPFS network we developed a small package that is responsible for hiding the IPFS connection implementation (\texttt{ipfs-mini} API\glo). The available interface contains two methods: one for adding new data into the blockchain, and the other one for retrieving them starting from the IPFS content identifier (CID).
\begin{figure}[h]
	\centering
	\includegraphics[scale=0.6]{res/images/IPFS.png}
	\caption{Class diagram of the IPFS package}
\end{figure}
\subsection{Methods}
The package provides the following methods:
\begin{itemize}
	\item \textbf{getJSONfromHash}: it uses the \texttt{ipfs-mini} API to retreive the JSON object corresponding to the passed IPFS CID;
	\item \textbf{insertJSONintoIPFS}: it uses the \texttt{ipfs-mini} API to upload the JSON passed to the function, then returns the IPFS CID.
\end{itemize}



\pagebreak
\pagebreak
\section{Facade} 

\subsection{Architecture overview}

Since most of the operations include the interaction with both web3funcions and IPFS packages, we provide a third package, facade, with the aim of hiding the web3 and IPFS layers. This package executes function calls in the suitable order to the other two packages, saving and retrieving data from both IPFS and Ethereum.

\begin{figure}[h]
	\centering
	\includegraphics[scale=0.6]{res/images/facade.png}
	\caption{Package diagram with the interaction between facade-web3-IPFS}
\end{figure}

\subsection{Methods}

All the methods described in the picture below consist of two different steps:
\begin{itemize}
	\item \textbf{setter}: first the data is stored to IPFS, then it is stored to Ethereum with the related IPFS CID;
	\item \textbf{getter}: the IPFS CID is retrived from web3. The CID is then used to get the the data from IPFS. 
\end{itemize}


\noindent Since the methods are just a combination of the already described methods of the web3 and IPFS packages, we omit their description.

\subsection{How to extend with new features}
This package should be used to organize information retrieved from different sources before passing them to the front-end. This package hides the interactions between the different technologies, but should not be used to interact directly with the blockchain or IPFS. Two important aspects the developer must keep in consideration are:
\begin{itemize}
	\item if new functionality is added in the Solidity back-end and in the web3 package, the facade package should be updated only if the user could interact directly with this functionality;
	\item even if the purpose of this package is to transform and organize data to make them more readable and usable, it is not responsible for preparing data as they will be shown/used by the front-end, that is containers/actionCreators' responsibility.
\end{itemize}
\begin{figure}[H]
	\centering
	\includegraphics[scale=0.55]{res/images/facade-package.png}
	\caption{Class diagram of the facade package}
\end{figure}



\pagebreak
\pagebreak
\section{Solidity}
% i nomi delle sezioni sono indicativi, non tassativi
\subsection{Architecture overview}
As you know the main feature of the blockchain is its immutability. It means that once a contract is 
deployed, it's not possible to modify or update it. In order to update a contract you need to deploy 
a new contract on the blockchain.\\
In view of this fact and the proponent's request to develop upgradeable smart contracts, the architecture ideated and developed reflects these two aspects. In fact the smarts contracts in Soldino
are essentially of three types:
\begin{enumerate}
	\item\textbf{Storage contract}: this type of contracts aren't upgreadable because they are used
		to save all the critical data like products, orders, vat movements. 
\end{enumerate}


\pagebreak
\subsection{Contracts}
\textit{Disclamer: the class diagram for each contract may be updated in the future to better represent the final product. Updates concern cost or performance optimization.}

\subsubsection{Generic contracts}
\paragraph{Security contracts}
\subparagraph{Owned}
This contract defines the owner of a contract.In Soldino the owner is the address which deploy all contracts. The contract defines a modifier, \texttt{onlyOwner}, used to allow only the owner to call a function which uses 
the above-mentioned modifier.
\begin{figure}[H]
	\centering
	\frame{\includegraphics[scale=0.132]{res/images/solidity/owned.png}}
	\caption{class diagram of the Owned contract}
\end{figure}

\subparagraph{Authorizable}
This contract is used by all storage contracts. It's a derivate contract of \texttt{Owned} and stores 
all address authorized to call the functions that use the modifier \texttt{onlyAuthorized} defined in this contract.
To authorize a contract the function \texttt{addAuthorized} needs to be called from the owner.
\begin{figure}[H]
	\centering
	\frame{\includegraphics[scale=0.2]{res/images/solidity/authorizable.png}}
	\caption{class diagram of the Authorizable contract}
\end{figure}
\pagebreak
\paragraph{Token ERC20}\mbox{}\\

\noindent The contract \texttt{TokenCubit} implements the custom token \textit{Cubit}, this contract stores all the users' balance and defines all the methods in order to be ERC20 compliant.
\begin{figure}[H]
	\centering
	\frame{\includegraphics[scale=0.25]{res/images/solidity/tokencubit.png}}
	\caption{class diagram of the TokenCubit contract}
\end{figure}

\paragraph{ContractManager}\mbox{}\\ 

\noindent In order to achieve simple and cost effective upgradeability of the logic contracts \texttt{ContractManager} is used. This contract store a map in which the entries are composed in the following way:
\begin{itemize}
	\item\textbf{Key}: the contract name.
	\item\textbf{Value}: the last version of the contract deployment address.
\end{itemize}
When contract $A$ needs to communicate with another contract $B$, contract $A$ get the address of the last version of the contract $B$ from \texttt{ContractManager}. In this way there's no need to manage all references in the contracts. 
\begin{figure}[H]
	\centering
	\frame{\includegraphics[scale=0.2]{res/images/solidity/contractmanager.png}}
	\caption{class diagram of the ContractManager contract}
\end{figure}
\pagebreak
\paragraph{Purchase}\mbox{}\\ 

\noindent The \texttt{Purchase} contract acts as a façade when it comes to buy products on Soldino.
Usually for every order (intended as one order for each seller) the user needs to confirm $n$ transaction, $n =$ number of orders, to buy all the products in his cart.\\
With the \texttt{Purchase} contract, the user has to confirm only two transaction independently how many products are in his cart. 
\begin{figure}[H]
	\centering
	\frame{\includegraphics[scale=0.25]{res/images/solidity/purchase.png}}
	\caption{class diagram of the Purchase contract}
\end{figure}
\hypertarget{st}{\subsubsection{Storage contracts}}
In this section we will illustrate the storage contracts. As mentioned before the storage contracts are immutable because they store all the critical data. In fact if a contract is upgraded, which means that a new version(using inheritance) is deployed, then all its state variable are new and the data of the previous version should be copied into the new contract, which translate in high cost transaction.\\
To avoid that, storage contracts implement no business logic of any kind. Their purpose is to store data and allow specific contract to modify their state. For data retrieval, on the other hand, the are no limitation because getter methods don't modify the storage contract's state.
All storage contracts inherit from \texttt{Authorizable} in order to use the \texttt{onlyAuthorized} modifier in the setter methods.
\paragraph*{Note: struct in the storage contracts}
All contracts, except for \texttt{VatStorage}, in their struct have three state variables: \texttt{hashIpfs, hashFunction} and \texttt{hashSize}, which represent the \texttt{IPFS CID} used to locate additional data on the \texttt{IPFS} network such as name and surname for users. The choice to separate critical data from additional data has been made in order to optimize storage costs.
\pagebreak
\paragraph{UserStorage}\mbox{}\\

\noindent This contract stores all the critical data of the users. 
\begin{figure}[H]
	\centering
	\includegraphics[scale=0.25]{res/images/solidity/userstorage.png}
	\caption{class diagram of the UserStorage contract}
\end{figure}
\pagebreak
\paragraph{ProductStorage}\mbox{}\\

\noindent The \texttt{ProductStorage} contract stores all the data needed to be secured (e.g. net price). Furthermore, the contract defines all methods used to maintain the data. 
\begin{figure}[H]
	\centering
	\includegraphics[scale=0.45]{res/images/solidity/productstorage.png}
	\caption{class diagram of the ProductStorage contract}
\end{figure}
\pagebreak
\paragraph{VatStorage}\mbox{}\\
This contract maintain the VAT amount of every business for every quarter. Each time a business buy or sell some products, the vat amount of each product is added or subtracted to the amount in this contract. If the product is sold, its VAT is added, whereas if the product is sold the vat is subtracted. \\
\texttt{VatStorage} has an array of map keys in order to iterate the map and enumerate it.

\begin{figure}[H]
	\centering
	\includegraphics[scale=0.30]{res/images/solidity/vatstorage.png}
	\caption{class diagram of the VatStorage contract}
\end{figure}
\pagebreak

\paragraph{OrderStorage}\mbox{}\\

\noindent This contract stores all the information of an order. In its struct are saved also the products' key (\texttt{productsHash}) of the order.
\begin{figure}[H]
	\centering
	\includegraphics[scale=0.35]{res/images/solidity/orderstorage.png}
	\caption{class diagram of the OrderStorage contract}
\end{figure}
\subsubsection{Logic contracts}
In this section we show how the logic contracts of \textit{Soldino} work. These contracts implement a major part of the business logic, defining:
\begin{itemize}
	\item inputs validation mechanics;
	\item events to be emitted on the blockchain when a state change is made;
	\item communication with each other.
\end{itemize}
Since logic contracts are upgradeable, every logic contract has a \texttt{ContractManager} state variable, used to get the latest address of another contract. In this way the coupling between logic contract is minimum and there's no need to maintain correct references in each logic contract.\\
However each \texttt{xLogic} contract has a direct reference to its \texttt{xStorage} contract because the latter is immutable.
\pagebreak
\paragraph{UserLogic}\mbox{}\\

\noindent This contract provides an interface to comunicate with the \texttt{UserStorage} contract. \texttt{addCitizen} and \texttt{addBusiness} functions add a new user and set its type depending on which function is called.
\begin{figure}[H]
	\centering
	\includegraphics[scale=0.20]{res/images/solidity/userlogic.png}
	\caption{class diagram of the UserLogic contract}
\end{figure}

\paragraph{ProductLogic}\mbox{}\\
\noindent This contract allows business to insert new products, modify or delete exiting ones. Obviously, only businesses can insert products in \textit{Soldino}: to comply this requirement \texttt{ProductLogic} defines the modifier \texttt{onlyBusiness} which checks if the address of the contract that is trying to insert a product is a business. Furthermore, products can be modified and deleted only by their sellers and the modifier \texttt{onlyProductOwner} is responsible for that.
\begin{figure}[H]
	\centering
	\includegraphics[scale=0.25]{res/images/solidity/productlogic.png}
	\caption{class diagram of the ProductLogic contract}
\end{figure}

\paragraph{VatLogic}\mbox{}\\

\noindent This contract defines the logic to interact with the \texttt{VatStorage} contract. 
Its function \texttt{refundVat} makes sure that only the government can refund the VAT to a business, thanks to the attached modifier \texttt{onlyGovernment}.\\
In order to store VAT movements broken down by business and quarter, the function \texttt{createVatKey} computes a valid key for the map stored in \texttt{VatStorage} using the address of the business and the string representing the quarter in format \texttt{"year-quarterNumber"}. 
\begin{figure}[H]
	\centering
	\includegraphics[scale=0.20]{res/images/solidity/vatlogic.png}
	\caption{class diagram of the VatLogic contract}
\end{figure}
\pagebreak
\paragraph{OrderLogic}\mbox{}\\

\noindent This contract allows to register an order. The function \texttt{registerOrder} can be called only by the \texttt{Purchase} contract because when a purchase is made on \textit{Soldino} the latter contract is called and forwards the request to this contract. In other words, if a purchase contains products from different sellers, the \texttt{Purchase} contract calls \texttt{registerOrder} for every different seller. \\
\texttt In addition, {registerOrder}  registers the VAT movements related to the businesses involved in the purchase.
\begin{figure}[H]
	\centering
	\includegraphics[scale=0.25]{res/images/solidity/orderlogic.png}
	\caption{class diagram of the OrderLogic contract}
\end{figure}
%\pagebreak
%\input{res/sections/solidity/collabs.tex}
\pagebreak
\subsection{How to extend}
As described above, smart contracts have been developed with the aim of being upgradable. This means that all the logic contracts can be upgraded without losing the stored data. In order to upgrade a logic contract you have to deploy the new contract, and update the contract manager, changing the address associated with the upgraded contract:
\begin{lstlisting}[language=JavaScript]
//import the new version of the contract
var Contract_v2 = artifacts.require("Name_of_the_contract");
//deploy the new contract
deployer.deploy(Contract_v2, contractManagerInstance.address)
.then((newContractInstance)=>{
	//change the deployment address of the upgraded contract
	//in the contract manager
	return contractManagerInstance.setContractAddress("Name_of_the_contract", newContractInstance.address);
});
\end{lstlisting}
At line 4 the object \texttt{deployer}, provided by Truffle framework, is used to deploy the contract on the blockchain$^{1}$. \texttt{deployer.deploy} returns a JavaScript Promise\glosp which if resolved returns an instance of the contract deployed. In fact at line 5 \texttt{then} is used to construct an anonymous function with the contract instance as parameter. The parameter is used to get its address used as parameter for the \texttt{setContractAddress} function. \\
As mentioned in  




 % described with class diagrams
% significative interactions described through sequence diagrams
 % upgradability; manager attachs and detachs contracts


\pagebreak
\section{Testing}
% questo cap effettivamente ha questo scopo nei 353 (2 pagine). In sostanza parla piu` di metriche che di test
This chapter shows how to
\begin{itemize}
	\item test the javascript and solidity code automatically;
	\item check if the code syntax is complied to the rules given.
\end{itemize}



\appendix
\section{Glossary}

\subsection*{A}
\addcontentsline{toc}{subsection}{A}

\subsubsection*{Add-on}
\index{Add-on}
Piece of software that attaches to another software extending its functionalities. Usually the application and its addons are executed together as a whole.

\subsubsection*{API}
\index{API}
Abbreviation for Application Programming Interface, it's a set of procedures and functions available to the developers to ease the development process. APIs expose chuncks of code that comes from libraries fostering the reuse of library code.

\subsubsection*{Authentication}
\index{Autenticazione}
It's an essential feature of security. It means to demonstrate one's own identity. In other words, the one who declares to be somebody must be able to proof it, e.g. showing some ID, or exchanging cryptographic keys.

\subsection*{B}
\addcontentsline{toc}{subsection}{B}

\subsubsection*{Backend}
\index{Back end}
It's the part of a computer program that provides core functionalities, but does not expose them in a fancy way to the users.
The user can access these functionalities from an interface provided by the frontend part of the application.


\subsubsection*{Blockchain}
\index{Blockchain}
Immutable and shared data structure shaped as a registister: the register is made of chronologically-ordered chained blocks 
Once a block has been added to the structure it can't be changed. 
This technology, similar to a distributed database, is managed by a network of nodes, each possessing a copy of all data. Each transaction is controlled by a protocol that verifies if the transaction is valid, hence approves it or not. Every approved transactions is added to the blockchain and can't be modified, just like any other block. This way there's no need of external authorities to approve transactions and grant for their validity.

\subsection*{C}
\addcontentsline{toc}{subsection}{C}

\subsubsection*{Purchase confirmation}
\index{Conferma d'acquisto}
It's a document in use in the \textit{Soldino} platform.
Its purpose is supplying an estimate of the VAT invoice, so that a client-business can verify that all products and applications of the IVA invoices applied upon itself are correct. Confirming a purchase is equal to carrying out the effective payment to the seller-business and receiving the actual invoice.


\subsubsection*{Contract}
\index{Contract}
See Smart Contract. 
% add label

\subsubsection*{Cubit}
\index{Cubit}
Custom token of \textit{Soldino}\glosp forked from the Ether\glosp currency, imposed by the \glosp C6 specification.

\subsection*{D}
\addcontentsline{toc}{subsection}{D}

\subsubsection*{DApp}
\index{DApp}

Known also as ÐApp, it stands for Decentralized App, running on a blockchain and managing smart contracts. Transactions on DApps are enrichened by user-defined rules that guarantee more security and control over them. Usually DApps are open-source and operates with cripted data.

\subsubsection*{Design pattern}
\index{Design pattern}
General solution to a common problem. It comes in the form of a model to apply to solve this problem. Each problem can be solved with different patterns, but every pattern is only good for to certain type of problems.

\subsection*{E}
\addcontentsline{toc}{subsection}{E}

\subsubsection*{ECR20}
\index{ECR20}
Abbreviation for Ethereum Request for Comment. It's the standard to adopt to implement tokens on the Ethereum platform.
It defines rules concerning token transfer and data access. Every token that follows the standard has well defined and predictable operations.

%\subsubsection*{EIP-712}
%\index{EIP-712}
%Standard per le transazioni Ethereum. Implementa la firma digitale delle transazioni semplificando la procedura all'utente. In questa maniera l'utente utilizzerà tool come MetaMask\glosp per gestire le transazioni, senza la necessità di inserire le chiavi esadecimali a mano.

\subsubsection*{ESlint}
\index{ESlint}
Tool for code static analysis and pattern identification in JavaScript. Used to achieve correctness, readability and conformity to given rules on your codebase.

\subsubsection*{Ether}
\index{Ether}
Main currency of the Ethereum platform\glo.

\subsubsection*{Ethereum}
\index{Ethereum}
Platform\glosp for the creation and spread of smart contracts\glo. Contracts must pay Ethereum in Ether\glosp to be run on it for the computational power provided by the platform. The data are saved on Ethereum on a blockchain data structure. Every person can access to Ethereum from a node, so the platform can also be seen as a network. 
Like other blockchains, Ethereum allows money exchange.

%\subsubsection*{Ethereum Virtual Machine}
%\index{Ethereum Virtual Machine}
%Macchina virtuale decentralizzata sulla rete Ethereum, che permette di eseguire ÐApps\glo.

\subsection*{F}
\addcontentsline{toc}{subsection}{F}

\subsubsection*{Framework}
\index{Framework}
Un framework è un'astrazione software, in cui è scritto del codice che fornisce funzionalità generiche e atte a essere estese, scrivendo così del software specifico. Può includere vari componenti, come librerie, compilatori ed API\glo, tutte atte a migliorare il processo di sviluppo software.

\subsubsection*{Front end}
\index{Front end}
Il front end è la parte di un'applicazione con la quale l'utente interagisce direttamente, responsabile dell'acquisizione dei dati di ingresso e per la loro elaborazione. Tali dati sono poi utilizzabili dal back end\glo. 

\subsection*{G}
\addcontentsline{toc}{subsection}{G}

\subsubsection*{Gamification}
\index{Gamification}
Riutilizzo di concetti ed elementi tipici dei giochi in applicazioni di diverso contesto. Utilizzando questi principi si ottengono sia maggiore coinvolgimento degli utenti dell'applicazione, sia maggiore produttività ed organizzazione nel lavoro.

\subsubsection*{Ganache}
\index{Ganache}

\subsubsection*{Ganache CLI}
\index{Ganache CLI}
Versione da riga di comando di Ganache che simula un client di Ethereum per sviluppare in modo veloce e sicuro. Fa parte della suite di strumenti di Truffle\glo.


\subsubsection*{Gigacore}
\index{Gigacore}
Boilerplate\glosp per lo sviluppo di applicazioni che fanno uso di React\glo, Redux\glosp e Sass\glo.

\subsubsection*{GitHub}
\index{GitHub}

\subsubsection*{GitHub}
\index{GitHub}
Sistema di gestione repository online, che integra servizi quali issue tracking e l'integrazione di pipeline volte a continuos delivery\glosp e continuos integration\glo.

%\subsubsection*{Gross price}
%\index{Gross price}
%Prezzo del prodotto dopo l'applicazione dell'imposta IVA.


\subsubsection*{Government}
\index{Government}
Termine usato per riferirsi a tutti gli enti governativi interessati nella gestione dell'IVA.


\subsection*{I}
\addcontentsline{toc}{subsection}{I}

\subsubsection*{IPFS}
\index{IPFS}
Protocollo progettato per sostenere la diffusione e l'utilizzo di una rete che possa facilitare la la condivisione di file e documenti attraverso un file system distribuito e con un approccio peer-to-peer\glo.

\subsection*{J}
\addcontentsline{toc}{subsection}{J}

\subsubsection*{JSON}
\index{JSON}
JavaScript Object Notation. \'E una sintassi per il salvataggio e scambio di dati.


\subsection*{K}
\addcontentsline{toc}{subsection}{K}

\subsubsection*{Key}
\index{Key}
Il termine viene utilizzato per indicare la chiave pubblica di un wallet\glosp Ethereum\glosp. Una chiave pubblica identifica un wallet\glosp ed è nota a tutti. Può essere utilizzata per identificare un account destinatario di un versamento. Una chiave privata invece è conosciuta solamente dall'utente proprietario del wallet ed è utilizzata per verificare una transazione effettuata dall'account stesso. 

%\subsection*{L}
%\addcontentsline{toc}{subsection}{L}

\subsection*{M}
\addcontentsline{toc}{subsection}{M}

\subsubsection*{MetaMask}
\index{MetaMask}
Add-on\glosp disponibile nei browser Chrome, Firefox, Opera e Brave che permette agli utenti di interfacciarsi con la rete Ethereum\glosp senza ospitare un nodo della rete. Permette di autenticarsi in maniera sicura e di eseguire ÐApps\glosp sul proprio browser. Fornisce un'interfaccia utente per gestire più wallet/account e salvare tutti i dati direttamente nel browser. 

\subsection*{N}
\addcontentsline{toc}{subsection}{N}

\subsubsection*{Net price}
\index{Net price}
Prezzo del prodotto esente da imposta IVA.


\subsubsection*{Node.js}
\index{Node.js}
Ambiente open-source per l'esecuzione di codice Javascript a run-time, al di fuori dei browser. Ciò permette l'esecuzione di codice Javascript server-side.

\subsubsection*{NPM}
\index{NPM}
NPM è quindi un package manager (Node.js\glosp Package Manager) lo strumento che permette di includere, rimuovere e aggiornare le librerie all'interno di un proprio progetto.

%\subsection*{O}
%\addcontentsline{toc}{subsection}{O}

\subsection*{P}
\addcontentsline{toc}{subsection}{P}


\subsubsection*{Peer-to-peer}
\index{Peer-to-peer}
Nelle reti telematiche, architettura in cui tutti i computer connessi svolgono la funzione sia di client che di server.

\subsubsection*{Plugin}
\index{Plug-in}
Componente aggiuntivo che può essere aggiunto a un'altro software per ampliarne le funzionalità. Di solito può essere eseguito in modo indipendente.



\subsection*{Q}
\addcontentsline{toc}{subsection}{Q}

\subsection*{R}
\addcontentsline{toc}{subsection}{R}

\subsubsection*{React}
\index{React}
Libreria open source per JavaScript per la creazione di interfacce grafiche e la gestione delle interazioni in ambito web.

\subsubsection*{Redux}
\index{Redux}
Libreria open source JavaScript per la gestione degli stati di React\glo.

\subsubsection*{Repository}
\index{Repository}
In generale, locazione di salvataggio dei dati. Nei sistemi di versionamento è una struttura dati più complessa, contenente metadati e operazioni per maneggiarla.


\subsection*{S}
\addcontentsline{toc}{subsection}{S}



\subsubsection*{Smart contract}
\index{Smart contract}
Protocolli per facilitare, attuare e verificare la negoziazione di un contratto in versione digitale. Permettono di ottenere lo stesso valore di un contratto reale senza l'ausilio di un garante esterno. Le transazioni che avvengono con questo protocollo sono tracciabili e irreversibili. Uno smart contract rappresenta del codice che può essere eseguito.

\subsubsection*{Solidity}
\index{Solidity}
Linguaggio di programmazione utilizzato per lo sviluppo di smart contracts\glosp eseguibili in diverse blockchain\glo.

\subsubsection*{Surge}
\index{Surge}
Short for Surge.sh
% add label here

\subsubsection*{Surge.sh}
\index{Surge.sh}
Web server che offre il servizio di hosting per siti web statici.


\subsection*{T}
\addcontentsline{toc}{subsection}{T}

\subsubsection*{Token}
\index{Token}
Rappresentazione di una particolare risorsa o utilità che opera su una blockchain\glo. Un token può rappresentare un qualsiasi bene commerciabile, da materie prime fino alle criptovalute. Si differenzia dagli alctoin per il fatto che usa una blockchain già esistente.

\subsubsection*{Truffle}
\index{Truffle}
Framework per lo sviluppo ed il testing di codice in una blockchain\glo. Gestisce l'environment necessario per eseguire e testare gli smart contracts\glo.


\subsection*{U}
\addcontentsline{toc}{subsection}{U}

%\subsubsection*{UML}
%\index{UML}
%Unified Modeling Language è un linguaggio standard per specififcare, visualizzare, costruire e documentare gli artefatti\glosp di un sistema software.

\subsection*{V}
\addcontentsline{toc}{subsection}{V}

\subsection*{Z}
\addcontentsline{toc}{subsection}{Z}



\end{document}
