\section{Setup} 
\subsection{Requirements}
In this section we describe all the requirements needed.
\subsubsection{Browser}
Soldino is accessible through a web interface. The currently most recent versions of the following broswers are supported:
\begin{itemize}
	\item \textbf{Mozilla Firefox}: version 66.0.1;
	\item \textbf{Google Chrome} version 73.0.3683.86.
\end{itemize}

\subsubsection{Tools}
The following tools are needed:
\begin{itemize}
	\item \textbf{Node.js}: you need it to run commands and as a Truffle requirement;
	\item \textbf{Truffle}: you need it to write and deploy contracts with ease;
	\item \textbf{Ganache}: you need it to put up a local Ethereum network and check transactions on it;
	\item \textbf{Metamask}: it's used as virtual wallet.
\end{itemize}

\subsubsection{Dependencies}
Soldino depends on many different packages, some for use and others for development.\\
All these packages are located in the file \texttt{package.json} which is in the root folder of the project.\\
The packages required to execute the software \textit{Soldino} are listed below.\\

\renewcommand{\arraystretch}{1.5}
\rowcolors{2}{pari}{dispari}
\begin{longtable}{ 
		>{\centering}p{0.4\textwidth} 
		>{\centering}p{0.4\textwidth}
	}
	\caption{Packages required for software usage}\\
	\rowcolorhead
	\textbf{\color{white}Software} & 
	\textbf{\color{white}Versione}
	\tabularnewline  
	\endhead	
	
	% voci prese da package.json, repo "Soldino-PoC, ramo "develop"


	react-text-mask & $\geq$5.4.4
	\tabularnewline
	commondir &$\geq$1.0.1
	\tabularnewline
	history &$\geq$4.7.2
	\tabularnewline
	prop-types &$\geq$15.7.2\tabularnewline
	react &$\geq$16.8.3\tabularnewline
	react-dom &$\geq$16.8.3\tabularnewline
	react-number-format &$\geq$4.0.6\tabularnewline
	react-redux &$\geq$6.0.1\tabularnewline
	react-router &$\geq$4.3.1\tabularnewline
	react-router-dom &$\geq$4.3.1\tabularnewline
	react-router-redux &$\geq$4.0.8\tabularnewline
	react-scripts &$\geq$2.1.8\tabularnewline
	redux &$\geq$4.0.1\tabularnewline
	redux-thunk &$\geq$2.3.0\tabularnewline
	web3 & 1.0.0-beta.37\tabularnewline
	
\end{longtable}

Other packages, listed below, are required for the development.
\renewcommand{\arraystretch}{1.5}
\rowcolors{2}{pari}{dispari}
\begin{longtable}{ 
		>{\centering}p{0.4\textwidth} 
		>{\centering}p{0.4\textwidth}
	}
	\caption{Packages required for development}\\
	\rowcolorhead
	\textbf{\color{white}Software} & 
	\textbf{\color{white}Versione}
	\tabularnewline  
	\endhead	
	
	% voci prese da package.json, repo "Soldino-PoC, ramo "develop"
	eslint & 5.12.0\tabularnewline
	eslint-config-airbnb &$\geq$17.1.0\tabularnewline
	eslint-loader & $\geq$2.1.2\tabularnewline
	eslint-plugin-import & $\geq$2.16.0\tabularnewline
	eslint-plugin-jsx-a11y & $\geq$6.2.1\tabularnewline
	pre-commit & $\geq$1.2.2\tabularnewline
	truffle-contract & $\geq$4.0.6
	
\end{longtable}

\subsection{Installing}
\subsubsection{Browser}
The first thing is to have your browser installed. You can get the latest chrome version  \href{https://www.google.com/chrome/}{here}.\\

\subsubsection{Node}
Install Node.js. Digit on the shell the following commands:
\begin{enumerate}
	\item \texttt{curl -sL https://deb.nodesource.com/setup\_11.x | sudo -E bash -}
	\item \texttt{sudo apt-get install -y nodejs}
	\item check that \texttt{node} have been installed correctly with \texttt{node -v}.
\end{enumerate}
There is no need to install npm separately, since it is automatically installed with Node.

\subsubsection{Truffle}
Third thing: install Truffle.\\
Truffle requirements are:
\begin{itemize}
	\item an OS among Linux, Windows and MacOS (prefer Linux);
	\item NodeJS v8.9.4 or later (we picked version 11);
	\item Node Package Manager (npm).
\end{itemize}
You can then install Truffle with the command:
\begin{itemize}
	\item[] \texttt{npm install -g truffle}
\end{itemize}

\subsubsection{Ganache}
Fourth step: installing Ganache. There are three step to install Ganache:
\begin{enumerate}
	\item you can download the Ganache executable at this \href{https://truffleframework.com/ganache}{link}, clicking on the download button.;
	\item give the permissions to make the Ganache file executable. This can be done on Linux with the command \texttt{chmod +x path-of-the-appimage/ganache-1.3.0-x86\_64.AppImage};
	\item eventually, run it double clicking on the icon.
\end{enumerate}

\subsection{Running}
Now that you have all the required software installed, it's time to get it up running.\\

\subsection{Deploying}
