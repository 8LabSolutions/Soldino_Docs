\subsection{Capitolato C5 - P2PCS}
\begin{itemize}
\item \textbf{Nome:} P2PCS: piattaforma di peer-to-peer car sharing
\item \textbf{Proponente:} Prof.Tullio Vardanega e Prof.
Riccardo Cardin
	\item \textbf{Committente:} GaiaGo S.r.l
\end{itemize}
\subsubsection{Descrizione}
Creare un'applicazione per car sharing condominiale che permetta ad utenti che posseggono un'auto di prestarla a vicini che ne fanno richiesta, permettendo così di evitare che il veicolo resti un peso economico per chi la possiede ma non la utilizza spesso.
\subsubsection{Obiettivo Finale}
Un utente registrato potrà cercare un'auto libera nella zona interessata, prenotarla per quando ne avrà bisogno e andare a ritirare le chiavi. I proprietari invece potranno offrire la propria macchina nei giorni in cui segnaleranno che non è inutilizzata.
\subsubsection{Dominio tecnologico}
\begin{itemize}
	\item \textbf{JavaScript:} per la stesura del codice;
	\item \textbf{Node.js:} framework impiegato per la scrittura di applicazioni JavaScript dal lato server con un modello asincrono di I/O basato su eventi, permettendo un'ottimizzazione di tempi e risorse;
	\item \textbf{Google Cloud:} per la gestione del database;
	\item \textbf{Octalysis:} framework per ottimizzare il sistema più verso la motivazione delle persone che verso la pura efficienza;
	\item \textbf{Henshin:} nello specifico si parla di Movens ovvero una piattaforma software open source, specifica nell'impiego di mobilità, gestione dell'IoT e smart cities. Tra servizi principali offerti ci sono la completa gestione della condivisione dei veicoli, la gestione dell'assicurazione e la gestione della connessione peer-to-peer.
	\item \textbf{Android:} per lo sviluppo di un'app.
\end{itemize}

\subsubsection{Aspetti Positivi}
\begin{itemize}
	\item Possibilità di imparare linguaggi e tecnologie molto utilizzati e richiesti;
	\item Possibilità di capire come funziona uno standup di una metodologia Agile all'interno di un'azienda;
	\item Possibilità di comprendere la teoria del Gamification;
	\item L'azienda fornirebbe parte di ciò che è richiesto per lo sviluppo dell'applicazione.

\end{itemize}

\subsubsection{Criticità e fattori di rischio}
\begin{itemize}
	\item Recentemente simili progetti italiani per car sharing si sono rivelati fallimentari raccoglimento uno scarso numero di utenti.
\end{itemize}
\subsubsection{Conclusioni}
Il gruppo ha espresso un giudizio principalmente negativo su questo capitolato soprattutto considerando i fallimenti dei altre compagnie in questo campo.