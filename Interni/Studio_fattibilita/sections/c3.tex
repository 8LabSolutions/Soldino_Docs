% NOTE: Descrizione e dominio applicativo, che differenza?
% La descrizione va bene?
\subsection{Capitolato C3 - C\&B}
\subsubsection{Informazioni sul capitolato}
\begin{itemize}
	\item \textbf{Nome}: G\&B: monitoraggio intelligente di processi DevOps
	\item \textbf{Proponente}: Zucchetti
	\item \textbf{Committente}: Prof. Tullio Vardanega e Prof. Riccardo Cardin
\end{itemize}
\subsubsection{Descrizione}
Il capitolato prevede la costruzione di un software per monitorare un sistema 
DevOps, cioè un sistema in cui, a livello aziendale, chi produce il software 
e chi lo usa collaborano strettamente. Per migliorare ulteriormente il servizio 
erogato si richiede un secondo software che supporti il primo visualizzando, 
analizzando, misurando e controllando i dati forniti dal primo.
\subsubsection{Studio del dominio}
\paragraph{Dominio applicativo} \mbox{}\\
La struttura del software da realizzare è conforme ai seguenti punti: 
\begin{itemize}
	\item un flusso di dati in input viene associato a una rete Bayesiana: 
	questa rete è composta di nodi contenenti informazioni di probabilità
	\item la rete riceve il flusso e lo usa per fare dei calcoli, aggiornando
	 quindi le probabilità dei propri nodi
	\item sia il flusso dati che la rete sono monitorati in un cruscotto;
	\item l'andamento dei dati determina l'eventuale generazione di allarmi
	 e notifiche.
\end{itemize}
\paragraph {Dominio tecnologico} \mbox{} \\
Le tecnologie proposte per lo sviluppo del progetto sono:
\begin{itemize}
	\item Grafana: software \textit{open-source} che, ricevuti i dati in input,
	 consente di raccoglierli in un cruscotto, visualizzarli, analizzarli, 
	 misurarli e controllarli;
	\item InfluxDB: database di tipo \textit{Time Series}, generati con continuità
	 temporale e atti a essere letti e monitorati costantemente per misurarne 
	 le variazioni;
	\item JavaScript: Linguaggio di programmazione richiesto per costruire i 
	\textit{plug-in} di Grafana e per definire la rete di Bayes in formato .json;
	\item Rete di Bayes: rete di nodi che contengono informazioni di probabilità;
	 quando un evento significativo si verifica, le probabilità dei nodi si aggiornano 
	 conseguentemente.
\end{itemize}
\subsubsection{Aspetti positivi}
\begin{itemize}
	\item L'azienda è grande e possiede esperienza utile al nostro apprendimento;
	\item Il dominio del problema è chiaro e circoscritto;
	\item I requisiti sono ben manifesti e comprensibili nel capitolato;
	\item Le tecnologie riguardano vari ambiti (database, linguaggi, probabilità, 
	monitoraggio) e sono in numero ragionevole da apprendere.
\end{itemize}
\subsubsection{Criticità e fattori di rischio}
\begin{itemize}
	\item il capitolato è fortemente conteso tra i gruppi appaltatori: la probabilità
	 di aggiudicarselo è inferiore a quella desiderata.
\end{itemize}
\subsubsection{Conclusioni}
Il capitolato è stato escluso dalle preferenze. Ci sono dei fattori positivi riguardo
 questo capitolato: in particolare, la definizione chiara del dominio e dei requisiti 
 del software.
Tuttavia, questi non sono sufficienti.
Si ritiene che l'insieme delle tecnologie non sia il migliore disponibile: alcune di 
esse (es. Reti di Bayes) sono circoscritte, altre (Grafana) sono di dubbia diffusione.
Inoltre, il capitolato è fortemente conteso tra i gruppi. 