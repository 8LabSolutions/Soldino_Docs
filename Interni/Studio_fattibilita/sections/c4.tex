\section{Capitolato C4 - MegAlexa}
\subsection{Informazioni sul capitolato}
\begin{itemize}
    \item \textbf{Nome}: MegAlexa: arricchitore di skill di Amazon Alexa
	\item \textbf{Proponente}: ZERO12
	\item \textbf{Committente}: Prof.Tullio Vardanega e Prof. Riccardo Cardin
\end{itemize}
\subsubsection{Descrizione}
Creare una skill per Alexa di Amazon in grado di avviare dei workflow creati dagli utenti tramite interfaccia web o
mobile app per iOS e Android.
\subsubsection{Dominio tecnologico}
\begin{itemize}
    \item \textbf{Funzionamento Alexa}
    \item \textbf{Lambda (AWS)}
    \item \textbf{API gateway (AWS)}: Servizio API per la comunicazione con Lambda
    \item \textbf{Aurora Serverless (AWS)}: offre capacità di database senza dover allocare e gestire server
    \item \textbf{Javascript}
    \item \textbf{NodeJS}: piattaforma event-driven per esecuzione di codice JavaScript server-side
    \item \textbf{HTML5}
    \item \textbf{CSS3}
    \item \textbf{Bootstrap}: Framework per front-end più utilizzato (solamente consigliato)
\end{itemize}
\subsubsection{Aspetti positivi}
\begin{itemize}
    \item Il proponente offre delle lezioni al fine di introdurre al gruppo le nuove tecnologie da utilizzare nello sviluppo del progetto e dirigire lo studio autonomo;
    \item La massiccia presenza nel web di documentazione dettagliata, esempi e strumenti rende relativamnete semplice l'apprendimento di tali tecnologie, in particolare Amazon fornisce Alexa Skills Kit (raccolta di API self-service, strumenti, documentazioni, esempi);
    \item Amazon ed il mercato in generale sembrano, per ora, molto interessati agli Assistenti vocali quindi la conscenza di tali tecnologie può essere una nota rilevante a livello curriculare.
\end{itemize}
\subsubsection{Aspetti negativi, Criticità e Fattori di Rischio}
\begin{itemize}
    \item É obbligatoro che le shortcut siano multilingua. Echo al momento supporta: Inglese, Francese Tedesco, Italiano, Giapponese, Spagnolo; tattavia le nostre conoscenze in ambito linguistico ci permettono di realizzare in modo esaustivo solamente le versione italiana e inglese.
    \item Sono già presenti nel web tecnologie per realizzare, anche se in modo piuttosto grezzo ciò che viene richiesto dal capitolato; la stessa applicazione di Alexa permette di creare sequenze di azioni precedentemente selezionate.
    \item Il proponente offre l'opportunità a solamente due gruppi di aggiudicarsi il capitolato.
\end{itemize}
\subsubsection{Conclusioni}
Nonostante tale capitolato abbia destato particolare interesse all'interno del gruppo, sia a livello tecnologico che di competenze curriculari, l'esiguo numero di postidisponibili e la presenza di capitolati non mneo interessanti, ha comportato lo spostamente in secondo piano del suddetto progetto.

