%Ð
\section{Capitolato scelto C6 - Soldino}
\subsection{Informazioni generali}
% ------ Inserire qui testo ------
% Eliminare parte dei controlli e burocrazia (Guardia di Finanza -> controlli a campione)
% 
\begin{itemize}
\item
\textbf{Nome:} Soldino
\item
\textbf{Proponente:} \textit{Red Babel} 
\item
\textbf{Committente:} Prof.Tullio Vardanega e Prof. Riccardo Cardin
\end{itemize}
\subsection{Descrizione}
Attualmente il sistema di pagamento dell'IVA prevede che siano le aziende a registrare i loro acquisti/vendite, ed a cadenza trimestrale viene calcolato il saldo IVA. Se l'azienda risulta debitrice allora deve versare all'ente governativo il rispettivo ammontare, altrimenti ottiene un credito IVA da poter utilizzare successivamente. \\Soldino nasce per automatizzare questo processo , proponendo una piattaforma di e-commerce gestita dall'Agenzia delle entrate, nella quale le aziende e i cittadini possono comprare e/o vendere beni e servizi per mezzo di una criptovaluta.

\subsection{Obiettivo finale}

L'obiettivo di Soldino è sviluppare una piattaforma di e-commerce, in cui i pagamenti sono gestiti per mezzo della blockchain "Ethereum", che sfrutta l'omonima criptovaluta. Le transazioni sfrutteranno il meccanismo degli Smart Contracts$_{G}$ per garantirne la sicurezza, e saranno sviluppate come  ÐApps$_{G}$

L'obiettivo di Soldino è quello di costruire il sistema con un insieme di ÐApps 
che girano su EVM.

\subsection{Studio del dominio}
\subsubsection{Dominio applicativo}
Si individuano tre attori:
\begin{itemize}	
	\item \textbf{Ente governativo: }\`E in grado di coniare e distribuire la 
	criptovaluta utilizzata in Soldino. Trimestralmente riceve il saldo IVA 
	delle aziende iscritte.
	\item \textbf{Azienda: } L'azienda iscritta al sito può comprare e vendere 
	beni e servizi. %Ogni tre mesi versa automaticamente attraverso l'ausilio di Smart Contract
	% l'IVA a debito al netto dell'IVA a credito (attendere risposta red babel)
	\item \textbf{Persona fisica$_{G}$: }Può convertire Euro in Cubit e successivamente acquistare
	i beni e servizi offerti sulla piattaforma.
\end{itemize}
\subsubsection{Dominio tecnologico}
Per lo sviluppo del lato backend si individuano le seguenti tecnologie:
\begin{itemize}
	\item \textbf{Ethereum:}
	\item \textbf{Metamask:}
	\item \textbf{Smart Contracts:}
	\item \textbf{ÐApps:}
\end{itemize}
Per lo sviluppo del lato front-end:
\begin{itemize}
	\item \textbf{Javascript:} linguaggio di scripting client-side. Il proponente impone di sottostare allo standard esposto nella "Airbnb Javascript style guide", e di utilizzare un toll di analisi statica del codice e syntax checking, "ESlint";
	\item \textbf{React/Redux:} framework utilizzati per sviluppare l'interfaccia dell'applicativo;
	\item \textbf{SCSS:} linguaggio-estensione per CSS, che ne aumenta le funzionalità ed espressività.
\end{itemize}


\subsubsection{Aspetti positivi}

\subsubsection{Criticità e fattori di rischio}

\subsubsection{Conclusioni}
