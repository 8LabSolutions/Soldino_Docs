\subsection{Capitolato C1 - Butterfly}
\subsubsection{Informazioni generali}
% ------ Inserire qui testo ------
\begin{itemize}
\item
\textbf{Nome:} Butterfly 
\item
\textbf{Proponente:} \textit{Imola Informatica} 
\item
\textbf{Committente:} Prof.Tullio Vardanega e Prof. Riccardo Cardin
\end{itemize}
\subsubsection{Descrizione}
\textit{Butterfly} mira allo sviluppo di una piattaforma che permette di
 accentrare, standardizzare, automatizzare e personalizzare le segnalazioni di
 diversi strumenti di versionamento, di continuos integration e continuos
 delivery, così da permettere all'utente di interfacciarsi ad un unica 
 dashboard.
\subsubsection{Studio del dominio}
\paragraph{Dominio applicativo}
Il prodotto finale, integrando al suo interno le segnalazioni delle diverse 
applicazioni, semplifica e organizza il lavoro dell'utente. L'azienda propone,
per la realizzazione di questa soluzione, l'utilizzo di quattro componenti:
\begin{itemize}
	\item \textbf{Producers}, che hanno la funzionalità di recuperare le
	segnalazioni e mostrarle come messaggi nei rispettivi Topic;
	\item \textbf{Broker}, come strumento per istanziare e gestire i Topic;
	\item \textbf{Consumers}, che hanno il compito di allacciarsi ai Topic
    specifici così da "smistare" i messaggi verso gli utenti finali;
    \item \textbf{Componente custom specifico}, inteso come un componente da
    implementare per l'azienda che permetta di indirizzare la notifica alla
    persona più idonea.
\end{itemize}
\paragraph{Dominio tecnologico}
L'azienda lascia liberà scelta su alcune preferenze tecniche come l'utilizzo di
Java, python o nedejs per lo sviluppo dei componenti applicativi e Apache Kafka 
come Broker, anche se ne consiglio l'uso.
Altre preferenze tecniche sono, invece, obbligatoriamente richieste, come
\begin{itemize}
	\item rispettare i 12 fattori presenti in "The Twelve-Factor App" nelle 
	applicazioni sviluppate;
	\item utilizzare Docker come container per l'istanziazione dei componenti;
	\item esporre le API Rest dei componenti per l'utilizzo dell'applicazione; 
	\item utilizzare test unitari e d'integrazione per ogni componente 
	realizzato.

\end{itemize}
\subsubsection{Aspetti positivi}

\subsubsection{Criticità}
\begin{itemize}
	\item L'apprendimento delle tecnologie coinvolte nel lato 
	Producer coprirebbe solo aspetti marginali (le segnalazioni) delle
	suddette, senza conferire capacità estese. 
	\item Il lavoro per la raccolta dati appare ripetitivo e le
	 API da imparare ad usare sono molto specifiche, quindi sarebbero 
	 circoscritte al progetto e non più utilizzate in futuro$_{G}$. 
	 

\end{itemize}
\subsubsection{Conclusioni}