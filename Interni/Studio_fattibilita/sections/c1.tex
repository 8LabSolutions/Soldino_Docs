\subsection{Capitolato C1 - Butterfly}
\subsubsection{Informazioni generali}
% ------ Inserire qui testo ------
\begin{itemize}
\item
\textbf{Nome:} Butterfly \texttt{da verificare}
\item
\textbf{Proponente:} \textit{Imola Informatica} 
\item
\textbf{Committente:} Prof.Tullio Vardanega e Prof. Riccardo Cardin
\end{itemize}
\subsubsection{Descrizione}
\textit{Butterfly} mira allo sviluppo di una piattaforma che permette di
 accentrare, standardizzare, automatizzare e personalizzare le segnalazioni di
 diversi strumenti di versionamento, di continuos integration e continuos delivery.
\subsubsection{Studio del dominio}
\paragraph{Dominio applicativo}
\paragraph{Dominio tecnologico}
\begin{itemize}
	\item 
	\textbf{GitLab: }
	\item
	\textbf{RedMine:}
	\item
	\textbf{SonarQube:}
	\item
	\textbf{Docker: }
	\item
	\textbf{A scelta tra:}
		\begin{itemize}
			\item\textbf{Java: }
			\item\textbf{Python: }
			\item\textbf{NodeJS: }
		\end{itemize}
\end{itemize}
\subsubsection{Aspetti positivi}

\subsubsection{Criticità}
\begin{itemize}
	\item L'apprendimento delle tecnologie coinvolte nel lato 
	Producer coprirebbe solo aspetti marginali (le segnalazioni) delle
	suddette, senza conferire capacità estese. 
	\item Il lavoro per la raccolta dati appare ripetitivo e le
	 API da imparare ad usare sono molto specifiche, quindi sarebbero 
	 circoscritte al progetto e non più utilizzate in futuro$_{G}$. 
	 

\end{itemize}
\subsubsection{Conclusioni}