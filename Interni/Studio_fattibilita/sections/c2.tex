\subsection{Capitolato C2 - Colletta}
\subsubsection{Informazioni sul capitolato}
\begin{itemize}
	\item \textbf{Nome}: Colletta;
	\item \textbf{Proponente}: Mivoq s.r.l.;
	\item \textbf{Committente}: Prof. Tullio Vardanega e Prof. Riccardo Cardin.
\end{itemize}
\subsubsection{Descrizione}
Lo scopo del progetto è la raccolta di dati relativi alla 
classificazione grammaticale di parole nel contesto in cui vengono utilizzate e 
la possibilità di rendere facilmente disponibili ed esportabili tali 
informazioni.
La raccolta dati non deve avvenire in modo esplicito, ma gli utenti 
devono trovare un'utilità intrinseca nell'utilizzo della piattaforma. A tal fine,
il proponente suggerisce l'implementazione di un sistema predisposto alla gestione
di esercizi di grammatica, come l'analisi grammaticale. Il prodotto finale deve 
permettere sia di osservare le prove sottoposte dagli insegnanti, che il loro 
svolgimento da parte degli studenti



 
\subsubsection{Studio del dominio}
\paragraph{Studio del dominio applicativo} \mbox{} \\ 

Vengono delineati tre principali attori: insegnanti, allievi e sviluppatori.
\begin{itemize}
	\item \textbf{Insegnanti}. L'insegnante dovrà poter creare esercizi
	 di analisi in modo agevole. Dopo l'inserimento di nuove frasi
	 nel sistema, un tool integrato provvederà automaticamente allo svolgimento
	 dell'esercizio, proponendo una soluzione. L'insegnante dovrà 
	 successivamente correggere e/o validare il risultato proposto, al fine di
	 garantire che i propri allievi ricevano il materiale controllato e corretto.
	\item \textbf{Allievi}.
	L'allievo che accede al sistema dovrà poter svolgere gli esercizi proposti
	dall'insegnante e ricevere una valutazione immediata. La piattaforma
	deve dare la possibilità di esprimere una preferenza sui propri insegnanti in modo da
	privilegiare le versioni di un insegnante rispetto ad un altro. La scelta
	dell'esercizio da svolgere avverrà tramite un elenco di frasi proposte o
	inserendo autonomamente una frase nel sistema. Verrà fornita la possibilità
	di selezionare l'insegnante per la valutazione tra quelli che hanno
	predisposto quel determinato esercizio. Nel caso non ce ne fossero, il
	sistema automatico provvederà alla correzione con valutazione. \`E previsto
    anche uno storico dei progressi nel tempo e un sistema di ricompensa.
	\item \textbf{Sviluppatori}.
	Gli sviluppatori sono interessati prevalentemente ad accedere ai dati
	raccolti degli utenti al fine di utilizzarli nella fase di addestramento di
	sistemi di apprendimento automatico. Allo sviluppatore dovrà essere fornita
	più di una versione dell'annotazione di ogni frase, in modo tale da dedurre
	quale sia quella più corretta. \`E dunque importante che venga fornito loro
	uno storico dei dati per poter estrarre solo i dati d'interesse ed
	escludere le correzioni di alcuni utenti. 
\end{itemize}
\paragraph{Studio del dominio tecnologico}
\begin{itemize}
	\item \textbf{Hunpos/Freeling}: Hunpos [Mivoq(2014-2018)] e FreeLing [TALP
	Research Center, UPC(2008-2018a)] sono due software specializzati nel Part
	of Speech (PoS) tagging, che consiste nell'interpretare un testo
	etichettando ciascuna parola con il relativo significato grammaticale. I
	due software sfruttano delle tecniche di apprendimento automatico
	supervisionato per tale classificazione.  
	
	\item \textbf{Firebase}: è una piattaforma per sviluppatori web e mobile
	 offerta da Google dove è presente FireBase(FB) Storage. Quest'ultimo è un
	 servizio per la gestione dei dati che consente, in particolare, l'upload e
	 il download sicuri, anche con una connessione di scarsa qualità. Inoltre 
	 permette di salvare immagini, video, audio ed ogni contenuto generato
	 dall'utente. I dati relativi a ciascun individuo sono successivamente
	 sfruttati da altri servizi (per esempio, FB Analytics) per offrire una versione
	 personalizzata dell'applicazione.
	 
	\item \textbf{Web/Mobile programming}: il proponente richiede che la
	piattaforma sia sviluppata sotto forma di pagina web oppure come
	applicazione mobile. L'azienda non ha imposto l'adozione di nessuna
	tecnologia specifica per quanto riguarda questa parte del progetto, quindi
	la scelta spetta agli sviluppatori.		
\end{itemize}
\subsubsection{Aspetti positivi}
A favore di questo capitolato sono emersi i seguenti punti:
\begin{itemize}
	\item Il proponente non ha specificato nessuna tecnologia con la quale
	 sviluppare la piattaforma, viene per cui lasciata agli sviluppatori totale
	 libertà di scelta;
	\item Nel capitolato i requisiti obbligatori, sia espliciti che impliciti,
	 sono in numero inferiore rispetto agli opzionali rendendo maggiormente
	 flessibile la quantità di requisiti da gestire;
	\item La piattaforma Google FireBase potrebbe risultare una conoscenza
	 utile da applicare successivamente nel mondo del lavoro;
	\item L'azienda rimane aperta a proposte differenti dall'analisi
	 grammaticale, purché venga mantenuto l'obiettivo finale.
	
\end{itemize}
\subsubsection{Criticità}
Gli aspetti negativi riscontrati sono invece:
\begin{itemize}
	\item Uno dei requisiti opzionali di maggior interesse da parte del
	 proponente consiste nel multilinguismo della piattaforma e il tempo
	 necessario da dedicare all'analisi grammaticale di lingue straniere è
	 complesso da quantificare;
	
	\item Nel progetto sono presenti temi già ampiamente studiati nel corso di
	 studi universitario, per cui non si amplierebbe il bagaglio di
	 tecnologie conosciute.
 	
\end{itemize}
\subsubsection{Conclusioni}
Sebbene il gruppo abbia trovato interessante questa proposta, ha deciso
di orientarsi verso progetti rivolti a nuove tecnologie, considerate più
stimolanti e che potranno arricchire maggiormente le abilità di ogni 
componente. 


