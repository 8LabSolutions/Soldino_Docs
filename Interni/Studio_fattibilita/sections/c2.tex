\section{Capitolato C2}
\subsection{Descrizione generale}
L’obiettivo del progetto è realizzare una piattaforma collaborativa di raccolta dati in cui gli utenti possano predisporre e/o svolgere piccoli esercizi di grammatica. La raccolta dati è incentrata sia gli esercizi predisposti che sul loro svolgimento da parte degli utenti.

\subsection{Finalità del progetto}
La raccolta dati non deve avvenire in
modo esplicito, piuttosto gli utenti della piattaforma devono trovare una utilità intrinseca nel fornire i dati. A tal fine il pronenente suggerisce l'implementazione di un sistema predisposto alla gestione di esercizi di grammatica, come l'analisi grammaticale. Vengono delineati tre principali attori: insegnanti, allievi e sviluppatori.
\subsubsection{Insegnanti}
L’insegnante dovrà poter predisporre esercizi di analisi grammaticale in modo agevole. Dopo l'inserimento di nuove frasi nel sistema, un tool integrato provvederà automaticamente allo svolgimento dell'esercizio, proponendo una soluzione. L'insegnante dovrà successivamente correggere e/o validare il risultato proposto, al fine di garantire che i propri allievi ricevano del materiale controllato.
\subsubsection{Allievi}
L’allievo che accede al sistema dovrà poter svolgere gli esercizi proposti dall’insegnante e ricevere una valutazione immediata. Gli verrà data la possibilità di esprimere una preferenza sui propri insegnanti in modo da privilegiare le versioni di un insegnante rispetto ad un altro. La scelta dell'esercizio da svolgere avverrà tramite un elenco di frasi proposte o inserendo autonomamente una frase nel sistema. Verrà fornita la possibilità di selezionare l'insegnante per la valutazione tra quelli che hanno predisposto quel determinato esercizio. Nel caso non ce ne fossero, il sistema automatico provvederà alla correzione con valutazione. é previsto anche uno storico dei progressi nel tempo e un sistema di ricompensa.
\subsubsection{Sviluppatori}
Gli sviluppatori sono interessati prevalentemente ad accedere ai dati raccolti dagli utenti al fine di utilizzarli nella fase di addestramento di sistemi di apprendimento automatico. Allo sviluppatore dovrà essere fornita più di una versione dell’annotazione di ogni frase, in modo tale da dedurre quale sia quella più corretta. é dunque importante che venga loro fornito uno storico dei dati per poter estrarre solo i dati d'interesse ed escludere le correzioni di alcuni utenti. 
\subsection{Tecnologie interessate}
\begin{itemize}
	\item \textbf{Hunpos}: esempio
	\item aa
\end{itemize}
\subsubsection{Conclusioni}
\textit{Questo è un paragrafo di prova del contenuto per il capitolato c2}
\paragraph{Note}
prova (Aggiungo delle piccole annotazioni riguardo a ciò che abbiamo scritto in modo tale da rivederle lunedì insieme:\\ 
1) uso di "predisporre" eccessivo;\\
2) errori grammaticali;\\
3)il paragrafo "Finalità di progetto" comincia con: "La raccolta dati"(manca l'obiettivo subito specificato);\\
4)2.1 termina con "La raccolta" 2.2 inizia con "La raccolta";\\
5)1) uso di "raccolta dati" eccessivo;\\