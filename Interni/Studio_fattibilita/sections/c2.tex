\section{Capitolato C2}
\subsection{Descrizione generale}
L’obiettivo del progetto è realizzare una piattaforma collaborativa di raccolta dati in cui gli utenti possano predisporre
e/o svolgere piccoli esercizi di grammatica. La raccolta dati è incentrata sia gli esercizi predisposti che sul loro svolgimento da parte degli utenti.

\subsection{Finalità del progetto}
La raccolta dati non deve avvenire in
modo esplicito, piuttosto gli utenti della piattaforma devono trovare una utilità intrinseca nel fornire i dati. A tal fine il pronenente suggerisce l'implementazione di un sistema predisposto alla gestione di esercizi di grammatica, come l'analisi grammaticale. Vengono delineati tre principali attori: insegnanti, allievi e sviluppatori.
\subsubsection{Insegnanti}
L’insegnante dovrà poter predisporre esercizi di analisi grammaticale in modo agevole. Dopo l'inserimento di nuove frasi nel sistema, un tool integrato provvederà automaticamente allo svolgimento dell'esercizio, proponendo una soluzione. L'insegnante dovrà successivamente correggere e/o validare il risultato proposto, al fine di garantire che i propri allievi ricevano del materiale controllato.
\subsubsection{Allievi}
\subsubsection{Sviluppatori}

\subsection{Tecnologie interessate}
\begin{itemize}
	\item \textbf{Hunpos}: esempio
	\item aa
\end{itemize}
\subsubsection{Conclusioni}
\textit{Questo è un paragrafo di prova del contenuto per il capitolato c2}
\paragraph{Note}
prova