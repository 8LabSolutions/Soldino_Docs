\section{Capitolato C2 - Colletta}
\subsection{Informazioni sul capitolato}
\begin{itemize}
	\item \textbf{Nome}: Colletta: piattaforma per raccolta dati
	mediante esercizi di grammatica
	\item \textbf{Proponente}: Mivoq s.r.l.
	\item \textbf{Committente}: Prof.Tullio Vardanega e Prof. Riccardo Cardin
\end{itemize}
\subsection{Descrizione e Obiettivo finale}
\subsubsection{Descrizione}

Il progetto consiste nella realizzazione di una piattaforma collaborativa di raccolta dati in cui gli utenti possano predisporre e/o svolgere piccoli esercizi di grammatica. Verranno osservati sia le prove proposte dagli insegnanti, sia i loro svolgimenti da parte degli allievi
\subsubsection{Obiettivo finale}

Lo scopo del progetto è la raccolta di dati relativi alla classificazione grammaticale di parole nel contesto in cui vengono utilizzate e la possibilità di rendere facilmente disponibili ed esportabili tali informazioni. La raccolta dati non deve avvenire in modo esplicito, piuttosto gli utenti devono trovare un'utilità intrinseca nell'utilizzo della piattaforma. A tal fine il proponente suggerisce l'implementazione di un sistema predisposto alla gestione di esercizi di grammatica, come l'analisi grammaticale. 
\subsection{Studio del dominio}
\subsubsection{Studio del dominio applicativo}

	Vengono delineati tre principali attori: insegnanti, allievi e sviluppatori.
\begin{itemize}
	\item \textbf{Insegnanti}:
	L'insegnante dovrà poter creare esercizi di analisi grammaticale in modo agevole. Dopo l'inserimento di nuove frasi nel sistema, un tool integrato provvederà automaticamente allo svolgimento dell'esercizio, proponendo una soluzione. L'insegnante dovrà successivamente correggere e/o validare il risultato proposto, al fine di garantire che i propri allievi ricevano del materiale controllato.
	\item \textbf{Allievi}:
	L'allievo che accede al sistema dovrà poter svolgere gli esercizi proposti dall'insegnante e ricevere una valutazione immediata. Gli verrà data la possibilità di esprimere una preferenza sui propri insegnanti in modo da privilegiare le versioni di un insegnante rispetto ad un altro. La scelta dell'esercizio da svolgere avverrà tramite un elenco di frasi proposte o inserendo autonomamente una frase nel sistema. Verrà fornita la possibilità di selezionare l'insegnante per la valutazione tra quelli che hanno predisposto quel determinato esercizio. Nel caso non ce ne fossero, il sistema automatico provvederà alla correzione con valutazione. E' previsto anche uno storico dei progressi nel tempo e un sistema di ricompensa.
	\item \textbf{Sviluppatori}:
	Gli sviluppatori sono interessati prevalentemente ad accedere ai dati raccolti dagli utenti al fine di utilizzarli nella fase di addestramento di sistemi di apprendimento automatico. Allo sviluppatore dovrà essere fornita più di una versione dell’annotazione di ogni frase, in modo tale da dedurre quale sia quella più corretta. E' dunque importante che venga loro fornito uno storico dei dati per poter estrarre solo i dati d'interesse ed escludere le correzioni di alcuni utenti. 
\end{itemize}
\subsubsection{Studio del dominio tecnologico}
\begin{itemize}
	\item \textbf{Hunpos/Freeling}: Hunpos [Mivoq(2014-2018)] e FreeLing [TALP Research Center, UPC(2008-2018a)] sono due software specializzati nel Part of Speech (PoS) tagging, che consiste nell'interpretare un testo etichettando ciascuna parola con il relativo significato grammaticale. I due software sfruttano delle tecniche di apprendimento automatico supervisionato per tale classificazione.  
	
	\item \textbf{Firebase}: è una piattaforma per sviluppatori web e mobile offerta da Google trai quali è presente FireBase(FB) Storage. Quest'ultimo è un servizio per la gestione dei dati che consente in particolare upload e download sicuri anche con una connessione di scarsa qualità. Può salvare immagini, video, audio ed ogni contenuto generato dall'utente. I dati relativi a ciascun individuo sono successivamente sfruttati da altri servizi (e.g. FB Analytics) per offrire una versione personalizzata dell'applicazione.
	\item \textbf{Web/Mobile programming}: il proponente richiede che la piattaforma sia sviluppata sotto forma di pagina web oppure come applicazione mobile. L'azienda non ha imposto l'adozione di nessuna tecnologia specifica per quanto riguarda questa parte del progetto, quindi la scelta spetta agli sviluppatori.		
\end{itemize}
\subsubsection{Aspetti positivi}
\subsubsection{Criticità e fattori di rischio}
\subsection{Conclusioni}
\textit{Questo è un paragrafo di prova del contenuto per il capitolato c2}
\paragraph{Note}
prova (Aggiungo delle piccole annotazioni riguardo a ciò che abbiamo scritto in modo tale da rivederle lunedì insieme:\\ 
1) uso di "predisporre" eccessivo;\\
2) errori grammaticali;\\
3)il paragrafo "Finalità di progetto" comincia con: "La raccolta dati"(manca l'obiettivo subito specificato);\\
4)2.1 termina con "La raccolta" 2.2 inizia con "La raccolta";\\
5)1) uso di "raccolta dati" eccessivo;\\