\section{Qualità di processo}
Per ricercare qualità nello svolgimento del progetto si adoperano dei processi. Inizialmente, tali processi sono stati scelti tra quelli proposti nello standard ISO/IEC/IEEE 12207:1995. Successivamente sono stati semplificati o adattati secondo le esigenze: in alcuni casi (analisi dei requisiti, progettazione dell'architettura, progettazione di dettaglio, pianificazione) le attività sono trattate alla pari di processi, vista la loro importanza nell'esito del progetto.\newline 
Il risultato sono i processi esposti a seguito.

%\subsection{Processo di Costruzione del Software}. Opzione per il futuro.
	\subsection{Processi di sviluppo}
	\subsubsection{Analisi dei requisiti}
		\paragraph{Metriche}
			\subparagraph{PROS: Percentuale di requisiti obbligatori soddisfatti} Indica appunto la percentuale di requisiti obbligatori soddisfatti.
			\begin{itemize}
				\item misurazione: valore percentuale: $ PROS = \frac{requisiti\ obbligatori\ soddisfatti}{requisiti\ obbligatori\ totali}$;
				\item valore preferibile: $100\%$;
				\item valore accettabile: $100\%$.
			\end{itemize}
			
	\subsubsection{Progettazione di dettaglio}
		\paragraph{Metriche}
			\subparagraph{CBO: Accoppiamento tra le classi di oggetti} 
			Una classe è accoppiata ad una seconda se usa metodi o variabili definiti nella seconda. 
			\begin{itemize}
				\item misurazione: valore intero: $CBO$;
				\item valore preferibile: $0 \leq CBO \leq 1$;
				\item valore accettabile: $0 \leq CBO \leq 6$.
			\end{itemize}
		
\subsection{Processi di supporto}
	\subsubsection{Pianificazione}
		\paragraph{Metriche}
			%BAC RIMOSSO: equivale al preventivo. Non è una metrica.
			\begin{comment}
			\subparagraph{BAC: Budget At Completion}
			Budget totale allocato per il progetto. Detto anche preventivo.
			\begin{itemize}
			\item misurazione: numero intero;
			% PLACEHOLDER: Aggiornare i 2 item qui sotto
			\item valore preferibile: $15132 euro$;
			\item valore accettabile: il valore del preventivo con un errore massimo del 5\%, equivale a dire $preventivo -5\% \leq BAC \leq preventivo + 5\%$. 
			\end{itemize}
			\end{comment}
			
			\subparagraph{EAC: Estimate At Completion}
			Rappresenta il nuovo preventivo sul totale alla fine di un periodo.
			\begin{itemize}
				\item  misurazione: numero intero;
				\item  valore preferibile: $0 \leq preventivo$;
				\item  valore accettabile: $ preventivo -5\% \leq EAC \leq preventivo + 5\%$. 
			\end{itemize}
			
			\subparagraph{VAC: Variance At Completion}
			Indica la variazione tra BAC ed EAC e la spesa effettivamente sostenuta.
			\begin{itemize}
				\item  misurazione: percentuale: $\frac{preventivo - EAC}{100}$;
				\item  valore preferibile: $\geq 0$;
				\item  valore accettabile: $\geq 0$.
			\end{itemize}
				
			\subparagraph{AC: Actual Cost}
			Il denaro speso fino al momento del calcolo.
			\begin{itemize}
				\item  misurazione: numero intero;
				\item  valore preferibile: $0 \leq AC < PV$;
				\item  valore accettabile: $0 \leq AC \leq budget\ totale\ $.
			\end{itemize}
		
			\subparagraph{EV: Earned Value}
			Metrica di utilità per il calcolo di $SV$ e $CV$ (spiegate successivamente). Si tratta del valore del lavoro fatto fino al momento del calcolo; corrisponde al denaro guadagnato fino a quel momento.
			\begin{itemize}
				\item  misurazione: $BAC \cdot \%\ di\ lavoro\ completato\ $;
				\item  valore preferibile: $ \geq 0$;
				\item  valore accettabile: $ \geq 0$.
			\end{itemize}
			\subparagraph{PV: Planned Value}
			Metrica di utilità per il calcolo di $SV$ e $CV$ (spiegate successivamente). Si tratta del valore del lavoro pianificato al momento del calcolo: corrisponde al denaro che si dovrebbe aver guadagnato in quel momento.
			\begin{itemize}
				\item  misurazione: $BAC \cdot \%\ di\ lavoro\ pianificato\ $;
				\item  valore preferibile: $ \geq 0$;
				\item  valore accettabile: $ \geq 0$.
			\end{itemize}			
			\subparagraph{SV: Schedule Variance}
			Esprime lo stato di anticipo o ritardo nello svolgimento del progetto rispetto alla pianificazione.
			\begin{itemize}
				\item misurazione: $SV = EV - PV$
				\item valore preferibile: $ > 0$;
				\item valore accettabile: 0.
			\end{itemize}
			\subparagraph{CV: Cost Variance}
			Differenza tra il costo del lavoro effettivamente completato e quello pianificato. Una CV positiva indica che si sta rispettando il budget.
			\begin{itemize}
				\item misurazione: $CV = EV - AC$;
				\item valore preferibile: $ > 0$;
				\item valore accettabile: $ \geq 0$.
			\end{itemize}

	\subsubsection{Verifica}
		\paragraph{Metriche}
			\subparagraph{CC: Code Coverage}
				Indica il numero di righe di codice percorse dai test durante la loro esecuzione. Per linee di codice totali si intende tutte quelle appartenenti all'unità in fase di test.
				\begin{itemize}
					\item misurazione: valore percentuale: $CC = \frac{linee\ di\ codice\ percorse}{linee\ di\ codice\ totali}$;
					\item valore preferibile: $100\%$;
					\item valore accettabile: $75\%$.
				\end{itemize}
	\subsubsection{Documentazione}
		\paragraph{Metriche}
			\subparagraph{Indice di Gulpease}
			Indice della leggibilità del testo. Valuta la lunghezza delle parole e delle frasi rispetto al numero totale di lettere. 
			\begin{itemize}
				\item misurazione: valore intero da 0 a 100:\newline 	
				$I_G = 89+ \frac{(300 \cdot numero\ di\ frasi - 10 \cdot numero\ di\ lettere)}{numero\ di\ parole}$;	
				\item valore preferibile: $80 < I_G < 100$;
				\item valore accettabile: $40 < I_G < 100$.
			\end{itemize}
			
\subsection{Processi organizzativi}
	\subsubsection{Gestione della qualità}
		\paragraph{Metriche}
			\subparagraph{PMS: Percentuale di metriche soddisfatte}
			La percentuale di metriche soddisfatte valuta quante metriche raggiungono soglie accettabili sul numero totale delle metriche calcolate. Una bassa percentuale di soddisfazione può indicare poca qualità, metriche inadeguate o mancata correttezza nel calcolo.
			\begin{itemize}
				\item misurazione: $\frac{numero\ di\ metriche\ soddisfatte}{numero\ di\ metriche\ totali} $;
				\item valore preferibile: $ \geq 80\%$;
				\item valore accettabile: $ \geq 60\%$.
			\end{itemize}
		