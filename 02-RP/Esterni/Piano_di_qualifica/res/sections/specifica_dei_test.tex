\section{Specifica dei test}
Per assicurare la qualità del software prodotto, il gruppo \textit{8Lab
Solutions} adotta come modello di sviluppo del software il
\textbf{Modello a V\glo}, il quale prevede lo sviluppo dei test in parallelo alle
attività di analisi e progettazione. In questo modo i test permetteranno di
verificare sia la correttezza delle parti di programma sviluppati, sia che
tutti gli aspetti del progetto siano implementati e corretti. 
Segue quindi l'esito dei test per mezzo di tabelle che ne
semplificheranno la consultazione e che potranno fornire una precisa indicazione 
degli output prodotti, specificando se il risultato ottenuto sia quello atteso, errato
oppure non coerente a quanto fissato in precedenza.
Per definire lo stato dei test, vengono utilizzate le seguenti sigle:
\begin{itemize}
	\item \textbf{I}: per indicare che il test è stato implementato;
	\item \textbf{NI}: per indicare che il test non è stato implementato.
\end{itemize}
Inoltre per lo stato dei test si usano le seguenti abbreviazioni:
\begin{itemize}
	\item \textbf{S}: per indicare che il test ha soddisfatto la richiesta;
	\item \textbf{NS}: per indicare che il test non ha soddisfatto la richiesta.
\end{itemize}

\subsection{Test di Accettazione} 
	\renewcommand{\arraystretch}{1.5}
	\rowcolors{2}{pari}{dispari}
	
	\begin{longtable}{ >{\centering}p{0.10\textwidth} >{\centering}p{0.655\textwidth}
			>{\centering}p{0.15\textwidth}}% >{\centering}p{0.14\textwidth}}
			
		%\hline
		\caption{Riepilogo Test di Accettazione}\\	
		\rowcolorhead
		\textbf{\color{white}Requisito} 
		& \textbf{\color{white}Descrizione} 
		& \centering\textbf{\color{white}Esito}
	%	& \textbf{\color{white}Fonti} 
		\tabularnewline %\hline 
		\endfirsthead	
		
		\rowcolor{white}\caption[]{(continua)}\\	
		\rowcolorhead
		\textbf{\color{white}Requisito} 
		& \textbf{\color{white}Descrizione} 
		& \centering\textbf{\color{white}Esito}
		%	& \textbf{\color{white}Fonti} 
		\tabularnewline %\hline 
		\endhead	
		
		 \texttt{TSDF1}	&	L'utente deve poter accedere alla guida dell'applicazione. All'utente viene
		 chiesto di:
		 \begin{itemize}
		 	\item verificare che sia possibile accedere alla guida;
		 	\item verificare che la guida riporti informazioni riguardanti l'utilizzo di
		 	MetaMask\glo{};
		 	\item verificare che la guida riporti informazioni riguardanti il pagamento
		 	delle operazioni. 
		 \end{itemize}	&	NI	\tabularnewline
		 \texttt{TSOF2}	&	L'utente non ancora autenticato deve poter effettuare la registrazione al
		 sistema. All'utente viene chiesto di:
		 \begin{itemize}
		 	\item accedere alla pagina di registrazione;
		 	\item selezionare il tipo di utente;
		 	\item inserire i dati in base al tipo scelto.
		 \end{itemize}	&	NI	\tabularnewline
		 \texttt{TSOF2.1}	&	L'utente deve potersi registrare come cittadino. All'utente viene chiesto di:
		 \begin{itemize}
		 	\item accedere alla pagina di registrazione e scegliere la registrazione
		 	come cittadino;
		 	\item inserire l'email;
		 	\item inserire l'indirizzo;
		 	\item inserire il nome;
		 	\item inserire il cognome.
		 \end{itemize}	&	NI	\tabularnewline
		 \texttt{TSOF2.2}	&	L'utente deve potersi registrare come azienda. All'utente viene chiesto di:
		 \begin{itemize}
		 	\item accedere alla pagina di registrazione e scegliere la registrazione
		 	come azienda;
		 	\item inserire l'email;
		 	\item inserire la sede;
		 	\item inserire la partita IVA;
		 	\item inserire il nome.
		 \end{itemize}	&	NI	\tabularnewline
		 \texttt{TSOF2.3}	&	L'utente deve poter visualizzare un messaggio di errore nel caso in cui la
		 chiave reperita da MetaMask\glo{} durante la registrazione sia già presente
		 nel sistema.	&	NI	\tabularnewline
		 \texttt{TSOF3.1}	&	L'utente deve poter autenticarsi nel sistema in modo automatico attraverso
		 MetaMask\glo{}. All'utente è chiesto di:
		 \begin{itemize}
		 	\item accedere alla sezione per il login;
		 	\item verificare l'avvenuta autenticazione.
		 \end{itemize}	&	NI	\tabularnewline
		 TSOF4	&	L'utente deve potersi disconnettere dal sistema. All'utente viene chiesto di:
		 \begin{itemize}
		 	\item autenticarsi correttamente nel sistema;
		 	\item trovarsi in una qualsiasi delle pagine di navigazione destinate 
		 	all'utente;
		 	\item premere il tasto di logout;
		 	\item confermare di voler effettuare la disconnessione premendo il tasto 
		 	di conferma;
		 	\item verificare che la disconnessione sia avvenuta correttamente.
		 \end{itemize}	&	NI	\tabularnewline
		 TSOF5	&	L'utente autenticato come governo deve poter coniare Cubit\glo{}.
		 All'utente viene chiesto di:
		 \begin{itemize}
		 	\item autenticarsi correttamente nel sistema;
		 	\item trovarsi nella pagina per la coniazione della moneta;
		 	\item selezionare una quantità positiva di Cubit\glo{};
		 	\item dare conferma dell'operazione. 
		 \end{itemize}	&	NI	\tabularnewline
		 TSOF6	&	L'utente autenticato come governo deve poter distribuire Cubit\glo{}
		 agli utenti registrati come aziende o cittadini. All'utente viene chiesto di:
		 \begin{itemize}
		 	\item autenticarsi correttamente nel sistema;
		 	\item trovarsi nella pagina per la distribuzione della moneta;
		 	\item selezionare una quantità positiva di Cubit da 
		 	distribuire;
		 	\item selezionare la lista di utenti a cui trasferire i Cubit; 
		 	\item dare conferma dell'operazione. 
		 \end{itemize}	&	NI	\tabularnewline
		 TSOF8	&	L'utente autenticato come azienda o cittadino deve poter visualizzare i
		 prodotti in vendita nel sito. All'utente viene chiesto di:
		 \begin{itemize}
		 	\item accedere alla pagina dei prodotti in vendita nel sito;
		 	\item verificare che sia visualizzato il nome del prodotto;
		 	\item verificare che sia visualizzato il prezzo lordo\glo{} del prodotto;
		 	\item verificare che sia visualizzata la descrizione del prodotto;
		 	\item verificare che sia visualizzata la quantità relativa ad un certo
		 	prodotto in vendita nel sito;
		 	\item verificare che sia possibile cambiare la quantità selezionata relativa ad un prodotto in
		 	vendita nel sito.
		 \end{itemize}	&	NI	\tabularnewline
		 TSOF9	&	L'utente autenticato come azienda o cittadino deve poter aggiungere
		 al proprio carrello prodotti in vendita nel sito. All'utente viene chiesto di:
		 \begin{itemize}
		 	\item accedere alla pagina dei prodotti in vendita nel sito;
		 	\item aggiungere un prodotto al carrello.
		 \end{itemize}	&	NI	\tabularnewline
		 TSOF10	&	L'utente autenticato come azienda o cittadino deve poter visualizzare i
		 prodotti aggiunti nel proprio carrello. All'utente viene chiesto di:
		 \begin{itemize}
		 	\item accedere alla pagina dei prodotti in vendita nel sito;
		 	\item aggiungere un prodotto al carrello;
		 	\item accedere alla pagina dedicata al carrello;
		 	\item verificare che il prodotto sia stato aggiunto al carrello;
		 	\item verificare che sia visualizzato il nome del prodotto nel carrello;
		 	\item verificare che sia visualizzato il prezzo unitario del prodotto nel carrello;
		 	\item verificare che sia visualizzato il prezzo totale dei prodotti nel carrello.
		 \end{itemize}	&	NI	\tabularnewline
		 TSOF11	&	L'utente autenticato come azienda o cittadino deve poter modificare la 
		 quantità di un certo prodotto aggiunto nel proprio carrello. All'utente viene
		 chiesto di:
		 \begin{itemize}
		 	\item accedere alla pagina dei prodotti in vendita nel sito;
		 	\item aggiungere un prodotto al carrello;
		 	\item accedere alla pagina dedicata al carrello;
		 	\item modificare la quantità di prodotto aggiunta al carrello;
		 	\item verificare che la quantità di prodotto aggiunta al carrello sia
		 	effettivamente cambiata.
		 \end{itemize}	&	NI	\tabularnewline
		 TSOF12	&	L'utente autenticato come azienda o cittadino deve poter rimuovere un
		 certo prodotto aggiunto nel proprio carrello. All'utente viene chiesto di:
		 \begin{itemize}
		 	\item accedere alla pagina dei prodotti in vendita nel sito;
		 	\item aggiungere un prodotto al carrello;
		 	\item accedere alla pagina dedicata al carrello;
		 	\item rimuovere il prodotto aggiunto al carrello;
		 	\item verificare che il prodotto non sia più presente nel carrello.
		 \end{itemize}	&	NI	\tabularnewline
		 TSDF13	&	L'utente autenticato come azienda o cittadino deve poter resettare il
		 contenuto del carrello. All'utente viene chiesto di:
		 \begin{itemize}
		 	\item accedere alla pagina dei prodotti in vendita nel sito;
		 	\item aggiungere un prodotto al carrello;
		 	\item accedere alla pagina dedicata al carrello;
		 	\item premere il tasto per resettare il contenuto del carrello;
		 	\item verificare che il carrello sia vuoto.
		 \end{itemize}	&	NI	\tabularnewline
		 TSOF14.1	&	L'utente autenticato come azienda o cittadino deve poter effettuare il
		 checkout dei prodotti nel carrello. All'utente viene chiesto di:
		 \begin{itemize}
		 	\item accedere alla pagina dei prodotti in vendita nel sito;
		 	\item aggiungere un prodotto al carrello;
		 	\item accedere alla pagina dedicata al carrello;
		 	\item cliccare sul pulsante dedicato alla fase di checkout;
		 	\item accedere alla pagina dedicata al checkout;
		 	\item verificare che sia possibile procedere al pagamento.
		 \end{itemize}	&	NI	\tabularnewline
		 TSOF14.1.1	&	L'utente autenticato come azienda o cittadino deve poter visualizzare un
		 messaggio di errore nel caso in cui si proceda al checkout e il carrello sia
		 vuoto. All'utente viene chiesto di:
		 \begin{itemize}
		 	\item accedere alla pagina dedicata al carrello;
		 	\item verificare che non siano presenti prodotti nel carrello;
		 	\item procedere alla fase di checkout;
		 	\item verificare che sia visualizzato il messaggio di errore.
		 \end{itemize}	&	NI	\tabularnewline
		 TSOF14.2	&	L'utente autenticato come azienda o cittadino deve selezionare l'indirizzo di
		 spedizione per un certo prodotto. All'utente viene chiesto di:
		 \begin{itemize}
		 	\item accedere alla pagina dei prodotti in vendita nel sito;
		 	\item aggiungere un prodotto al carrello;
		 	\item procedere al checkout;
		 	\item verificare che sia possibile selezionare come indirizzo di
		 	spedizione il proprio indirizzo di residenza;
		 	\item verificare che sia possibile inserire un nuovo indirizzo di
		 	spedizione.
		 \end{itemize}	&	NI	\tabularnewline
		 TSOF14.3	&	L'utente autenticato come azienda o cittadino deve poter visualizzare un
		 messaggio di errore nel caso in cui si proceda al pagamento e tale operazione non vada a buon fine. All'utente viene chiesto di:
		 \begin{itemize}
		 	\item accedere alla pagina dedicata al carrello;
		 	\item procedere alla fase di checkout;
		 	\item procedere al pagamento cliccando sull'apposito pulsante;
		 	\item verificare che sia visualizzato il messaggio di errore.
		 \end{itemize}	&	NI	\tabularnewline
		 TSOF14.4	&	L'utente autenticato come azienda o cittadino deve procedere all'acquisto di un prodotto e completare la procedura di pagamento. All'utente viene chiesto di:
		 \begin{itemize}
		 	\item accedere alla pagina dei prodotti in vendita nel sito;
		 	\item aggiungere un prodotto al carrello;
		 	\item procedere al checkout;
		 	\item confermare l'ordine quindi procedere al pagamento;
		 	\item verificare che sia visibile la conferma d'acquisto nella pagina dedicata.
		 \end{itemize}	&	NI	\tabularnewline
		 TSOF15	&	L'utente autenticato come azienda o cittadino deve poter visualizzare lo
		 storico degli acquisiti effettuati. All'utente viene chiesto di:
		 \begin{itemize}
		 	\item accedere alla pagina dedicata agli acquisti effettuati nel sito;
		 	\item verificare che per ogni acquisto sia visualizzata la data di quando
		 	è stato effettuato;
		 	\item verificare che per ogni acquisto sia visualizzato il numero
		 	dell'acquisto;
		 	\item verificare che per ogni acquisto siano visualizzati i prodotti
		 	acquistati;
		 	\item verificare che per ogni acquisto sia visualizzato il totale dell'IVA
		 	dell'acquisto;
		 	\item verificare che per ogni acquisto sia visualizzato il prezzo
		 	lordo\glo{} dell'acquisto;
		 	\item verificare che per ogni acquisto sia visualizzato l'indirizzo di 
		 	spedizione dell'acquisto;
		 \end{itemize}	&	NI	\tabularnewline
		 TSOF15.7	&	L'utente autenticato come azienda o cittadino deve poter visualizzare la lista delle conferme d’acquisto\glosp che necessitano
		 di conferma. All'utente viene chiesto di:
		 \begin{itemize}
		 	\item accedere alla pagina dedicata alle conferme d'acquisto;
		 	\item verificare che per ogni conferma d'acquisto siano visualizzati i dettagli del prodotto;
		 	\item verificare che per ogni conferma d'acquisto sia visualizzato il prezzo netto del prodotto;
		 	\item verificare che per ogni conferma d'acquisto sia visualizzata l'aliquota IVA applicata al prodotto;
		 \end{itemize}	&	NI	\tabularnewline
		 TSOF15.7.1	&	L'utente autenticato come azienda o cittadino deve poter rifiutare una conferma d’acquisto\glosp che necessita
		 di conferma. All'utente viene chiesto di:
		 \begin{itemize}
		 	\item accedere alla pagina dedicata alle conferme d'acquisto;
		 	\item cliccare sul pulsante per rifiutare una conferma d'ordine;
		 	\item verificare che tale conferma d'ordine venga rifiutata;
		 \end{itemize}	&	NI	\tabularnewline
		 TSOF15.7.2	&	L'utente autenticato come azienda o cittadino deve poter approvare una conferma d’acquisto\glosp che necessita
		 di conferma. All'utente viene chiesto di:
		 \begin{itemize}
		 	\item accedere alla pagina dedicata alle conferme d'acquisto;
		 	\item cliccare sul pulsante per approvare una conferma d'ordine;
		 	\item verificare che tale conferma d'ordine venga approvata.
		 \end{itemize}	&	NI	\tabularnewline
		 TSOF16.1	&	L'utente autenticato come azienda deve poter mettere in vendita nuovi
		 prodotti. All'utente viene chiesto di:
		 \begin{itemize}
		 	\item accedere alla pagina dedicata all'aggiunta di un prodotto in
		 	vendita;
		 	\item inserire il nome del nuovo prodotto in vendita;
		 	\item inserire la descrizione del nuovo prodotto in vendita;
		 	\item inserire il prezzo lordo\glo{} del nuovo prodotto in vendita;
		 	\item inserire l'aliquota IVA del nuovo prodotto in vendita;
		 	\item verificare che il prodotto sia stato correttamente aggiunto al sistema.
		 \end{itemize}	&	NI	\tabularnewline
		 TSOF16.2	&	L'utente autenticato come azienda deve poter modificare i dati relativi ad
		 un prodotto in vendita. All'utente viene chiesto di:
		 \begin{itemize}
		 	\item accedere alla pagina dedicata ai propri prodotti in vendita;
		 	\item accedere alla pagina per la modifica di uno dei prodotti;
		 	\item modificare i dati del prodotto in vendita;
		 	\item verificare che alla conferma i dati del prodotto siano effettivamente
		 	modificati.
		 \end{itemize}	&	NI	\tabularnewline
		 TSOF16.2.1	&	L'utente autenticato come azienda deve poter verificare un errore nel caso
		 in cui non sia effettuata alcuna modifica ad un prodotto in vendita dopo
		 l'accesso alla pagina di modifica. All'utente viene chiesto di:
		 \begin{itemize}
		 	\item accedere alla pagina dedicata ai propri prodotti in vendita;
		 	\item accedere alla pagina per la modifica di uno dei prodotti;
		 	\item non effettuare alcuna modifica quindi confermare la modifica;
		 	\item verificare che alla conferma sia visualizzato l'errore.
		 \end{itemize}	&	NI	\tabularnewline
		 TSOF16.3	&	L'utente autenticato come azienda deve poter eliminare
		 un prodotto in vendita. All'utente viene chiesto di:
		 \begin{itemize}
		 	\item accedere alla pagina dedicata ai propri prodotti in vendita;
		 	\item accedere alla pagina per la modifica di uno dei prodotti;
		 	\item rimuovere il prodotto;
		 	\item verificare che alla conferma il prodotto non sia più presente nella
		 	lista dei prodotti in vendita.
		 \end{itemize}	&	NI	\tabularnewline
		 TSOF17	&	L'utente autenticato come azienda deve poter visualizzare lo storico delle
		 vendite. All'utente viene chiesto di:
		 \begin{itemize}
		 	\item accedere alla pagina dedicata alle vendite nel sito;
		 	\item verificare che per ogni vendita sia visualizzata la data di quando
		 	è stato effettuata;
		 	\item verificare che per ogni vendita sia visualizzato il numero
		 	della vendita;
		 	\item verificare che per ogni vendita siano visualizzati i prodotti
		 	venduti;
		 	\item verificare che per ogni vendita sia visualizzato il totale dell'IVA
		 	della vendita;
		 	\item verificare che per ogni vendita sia visualizzato il prezzo
		 	lordo\glo{} della vendita;
		 	\item verificare che per ogni vendita sia visualizzato l'indirizzo di 
		 	spedizione della vendita;
		 	\item verificare che per ogni vendita sia visualizzato il nome
		 	dell'acquirente qualora quest'ultimo sia un cittadino;
		 	\item verificare che per ogni vendita sia visualizzato il cognome
		 	dell'acquirente qualora quest'ultimo sia un cittadino;
		 	\item verificare che per ogni vendita sia visualizzato il nome
		 	dell'acquirente qualora quest'ultimo sia un'azienda;
		 	\item verificare che per ogni vendita sia visualizzata la partita IVA
		 	dell'acquirente qualora quest'ultimo sia un'azienda.
		 \end{itemize}	&	NI	\tabularnewline
		 TSOF18	&	L'utente autenticato come azienda deve procedere alla vendita di un prodotto e attendere la conferma d'ordine da parte dell'acquirente. All'utente viene chiesto di:
		 \begin{itemize}
		 	\item mettere in vendita un prodotto;
		 	\item attendere la conferma d'ordine da parte dell'acquirente;
		 	\item verificare che nel proprio wallet\glosp sia presente l'ammontare dell'ordine;
		 \end{itemize}	&	NI	\tabularnewline
		 TSOF19.1	&	L'utente autenticato come azienda deve poter visualizzare il saldo dell'IVA
		 relativo al trimestre corrente. All'utente viene chiesto di:
		 \begin{itemize}
		 	\item accedere alla pagina dedicata al saldo dell'IVA;
		 	\item verificare sia visualizzato il saldo dell'IVA relativo al trimestre
		 	corrente.
		 \end{itemize}	&	NI	\tabularnewline
		 TSOF19.2	&	L'utente autenticato come azienda deve poter visualizzare il saldo dell'IVA
		 relativo ad un trimestre concluso. All'utente viene chiesto di:
		 \begin{itemize}
		 	\item accedere alla pagina dedicata al saldo dell'IVA;
		 	\item selezionare un trimestre concluso;
		 	\item verificare sia visualizzato il saldo dell'IVA relativo al trimestre
		 	selezionato.
		 \end{itemize}	&	NI	\tabularnewline
		 TSOF19.3	&	L'utente autenticato come azienda deve poter visualizzare la lista delle
		 fatture relative ad un trimestre selezionato. All'utente viene chiesto di:
		 \begin{itemize}
		 	\item accedere alla pagina dedicata al saldo dell'IVA;
		 	\item selezionare un trimestre;
		 	\item verificare sia visualizzato per ogni fattura la data in cui è stata
		 	effettuata;
		 	\item verificare sia visualizzato per ogni fattura il numero identificativo;
		 	\item verificare sia visualizzato per ogni fattura il nome dell'azienda che
		 	l'ha emessa;
		 	\item verificare sia visualizzato per ogni fattura le informazioni relative
		 	all'acquirente;
		 	\item verificare sia visualizzato per ogni fattura l'importo totale
		 	dell'ordine;
		 	\item verificare sia visualizzato per ogni fattura il totale in debito/credito
		 	dell'IVA.
		 \end{itemize}	&	NI	\tabularnewline
		 TSOF19.4	&	L'utente autenticato come azienda deve poter effettuare il versamento 
		 dell'IVA di un trimestre nel quale il saldo di questa risulti in debito. All'utente 
		 viene chiesto di:
		 \begin{itemize}
		 	\item accedere alla pagina dedicata al saldo dell'IVA;
		 	\item selezionare un trimestre;
		 	\item verificare sia visualizzato il saldo dell'IVA relativo al trimestre
		 	selezionato;
		 	\item confermare il versamento dell'IVA da effettuare;
		 	\item verificare che sia possibile procedere al versamento.
		 \end{itemize}
		 	&	NI	\tabularnewline
		 TSOF19.5	&	L'utente autenticato come azienda deve poter effettuare la dilazionare del versamento 
		 dell'IVA di un trimestre nel quale il saldo di questa risulti in debito. All'utente 
		 viene chiesto di:
		 \begin{itemize}
		 	\item accedere alla pagina dedicata al saldo dell'IVA;
		 	\item selezionare un trimestre;
		 	\item verificare sia visualizzato il saldo dell'IVA relativo al trimestre
		 	selezionato;
		 	\item selezionare la dilazione del versamento con cui si vuole procedere;
		 	\item confermare la dilazione da effettuare;
		 	\item verificare che il pagamento sia stato dilazionato.
		 \end{itemize}
		 	&	NI	\tabularnewline
		 TSOF19.6	&	L'utente autenticato come azienda deve poter scaricare in formato PDF 
		 la lista dei movimenti dell'IVA del trimestre desiderato. All'utente viene chiesto 
		 di:
		 \begin{itemize}
		 	\item accedere alla pagina dedicata al saldo dell'IVA;
		 	\item selezionare il trimestre che desidera;
		 	\item verificare sia presente l'opzione di scaricare il PDF della lista
		 	dei movimenti dell'IVA;
		 	\item verificare che sia possibile possibile scaricare il PDF.
		 \end{itemize}
		 	&	NI	\tabularnewline
		 TSOF19.7	&	L'utente autenticato come azienda deve poter visualizzare la lista delle
		 fatture relative ad un trimestre selezionato. All'utente viene chiesto di:
		 \begin{itemize}
		 	\item accedere alla pagina dedicata al saldo dell'IVA;
		 	\item selezionare un trimestre;
		 	\item verificare sia visualizzato per ogni fattura la data in cui è stata
		 	effettuata;
		 	\item verificare sia visualizzato per ogni fattura il numero identificativo;
		 	\item verificare sia visualizzata la data dell'ordine relativo alla fattura;
		 	\item verificare sia visualizzato in numero identificativo dell'ordine 
		 	relativo alla fattura;
		 	\item verificare siano visualizzati tutti i prodotti acquistati;
		 	\item verificare sia visualizzato l'importo totale dell'IVA relativa alla fattura;
		 	\item verificare sia visualizzato l'importo totale della fattura;
		 	\item verificare sia visualizzato il nome dell'azienda emittente;
		 	\item verificare sia visualizzata la partita IVA dell'azienda emittente;
		 	\item verificare sia visualizzato il nome dell'acquirente se è un cittadino;
		 	\item verificare sia visualizzato il cognome dell'acquirente se è un cittadino;
		 	\item verificare sia visualizzata la partita IVA dell'acquirente se è un'azienda;
		 	\item verificare sia visualizzato l'indirizzo di spedizione dell'ordine.
		 \end{itemize}
		 	&	NI	\tabularnewline
		 TSOF19.8	&	L'utente autenticato come azienda deve poter scaricare in formato PDF 
		 la lista dei movimenti dell'IVA del trimestre desiderato. All'utente viene chiesto 
		 di:
		 \begin{itemize}
		 	\item accedere alla pagina dedicata al saldo dell'IVA;
		 	\item selezionare il trimestre che desidera;
		 	\item verificare sia presente l'opzione di scaricare il PDF della lista
		 	dei movimenti dell'IVA;
		 	\item verificare che sia possibile possibile scaricare il PDF.
		 \end{itemize}
		 	&	NI	\tabularnewline
		 TSOF19.9	&	L'utente autenticato come azienda deve poter scaricare in formato PDF la
		 fattura relativa ad un acquisto. All'utente viene chiesto di:
		 \begin{itemize}
		 	\item di accedere alla pagina dedicata alle fatture;
		 	\item verificare di poter selezionare la fattura da scaricare in formato PDF;
		 	\item verificare di riuscire a scaricare la fattura in formato PDF.
		 \end{itemize}	&	NI	
	
	
	\end{longtable}

\subsection{Test di Sistema}

\subsection{Test di Integrazione}

\subsection{Test di Unità}
