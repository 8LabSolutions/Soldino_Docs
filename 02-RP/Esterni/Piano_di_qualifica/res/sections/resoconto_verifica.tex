\section{Resoconto attività di verifica}
In questa sezione vengono descritti e analizzati gli esiti delle attività
di verifica svolte su tutti i documenti che vengono consegnati nelle varie 
revisioni di avanzamento del progetto.

\subsection{Revisione dei requisiti}
\subsubsection{Tracciamento dei casi d'uso e dei requisiti}
Per facilitare il tracciamento delle relazioni fra casi d'uso e requisiti che
fra requisiti e fonti, il gruppo ha deciso di utilizzare il software PragmaDB.

\subsubsection{Analisi statica dei documenti}
L'analisi dei documenti mediante \textit{Walkthrough}\glo{} ha portato 
all'individuazione di alcuni errori frequenti a partire dai quali è stata 
stilata una lista di controllo interna. In questo modo sarà possibile applicare
l'\textit{Inspection}\glo{} per le future attività di verifica.

\subsection{Revisione di progettazione}
\subsubsection{Analisi statica del codice}
La stesura del codice è supportata dai plugin dell'editor Visual Studio Code. I plugin rilevanti in ambito di verifica sono due:
\begin{itemize}
	\item \textbf{ESLint v1.8.1}: linter per il linguaggio Javascript;
	\item \textbf{solidity v0.0.49}: linter per il linguaggio Solidity.
\end{itemize}
 I linter hanno aiutato notevolmente la stesura di codice sintatticamente corretto e alla standardizzazione del codice scritto, che risulta più uniforme; inoltre hanno facilitato parzialmente il processo di debugging.
