\section{Resoconto attività di verifica}
In questa sezione vengono descritti e analizzati gli esiti delle attività
di verifica svolte su tutti i documenti che vengono consegnati nelle varie 
revisioni di avanzamento del progetto.

\subsection{Revisione dei requisiti}
\subsubsection{Tracciamento dei casi d'uso e dei requisiti}
Per facilitare il tracciamento delle relazioni fra casi d'uso e requisiti che
fra requisiti e fonti, il gruppo ha deciso di utilizzare il software PragmaDB.

\subsubsection{Analisi statica dei documenti}
L'analisi dei documenti mediante \textit{Walkthrough}\glo{} ha portato 
all'individuazione di alcuni errori frequenti a partire dai quali è stata 
stilata una lista di controllo interna. In questo modo sarà possibile applicare
l'\textit{Inspection}\glo{} per le future attività di verifica.

\subsubsection{Esiti verifiche automatizzate}
Nella tabella seguente vengono riportati gli indici Gulpease\glo{} di tutti
i documenti prodotti finora.

	\rowcolors{2}{pari}{dispari}
	
		
	\begin{longtable}{ >{\centering}p{0.40\textwidth} >{\centering}p{0.25\textwidth}
			 >{\centering}p{0.2075\textwidth}}
		\caption{Esiti verifiche automatizzate - Revisione dei requisiti} \\
		%\hline
		\rowcolorhead
		\centering\textbf{\color{white}Documento} 
		& \centering\textbf{\color{white}Indice Gulpease} 
		& \centering\textbf{\color{white}Esito}
		\tabularnewline %\hline 
		\endfirsthead
			
		
		
		\textit{Analisi dei Requisiti v1.0.0} & 67 & Superato
		
		\tabularnewline 
		\textit{Glossario v1.0.0} & 71 & Superato
				
		\tabularnewline 
		\textit{Norme di Progetto v1.0.0} & 65 & Superato
		
		\tabularnewline 
		\textit{Piano di Progetto v1.0.0} & 68 & Superato
		
		\tabularnewline 
		\textit{Piano di Qualifica v1.0.0} & 70 & Superato	
		
		\tabularnewline 
		\textit{Studio di Fattibilità v1.0.0} & 73 & Superato
		
		\tabularnewline 
		\textit{Verbale Interno 2018-12-04 v1.0.0} & 75 & Superato
		
		\tabularnewline 
		\textit{Verbale Interno 2018-12-10 v1.0.0} & 74 & Superato
		
		\tabularnewline 
		\textit{Verbale Interno 2018-12-20 v1.0.0} & 70 & Superato
		
		\tabularnewline 
		\textit{Verbale Interno 2018-12-24 v1.0.0} & 71 & Superato
		
		\tabularnewline 
		\textit{Verbale Esterno 2018-12-07 v1.0.0} & 74 & Superato
		
		\tabularnewline 
		\textit{Verbale Esterno 2018-12-21 v1.0.0} & 70 & Superato
		
		\tabularnewline 
		\textit{Verbale Esterno 2019-01-04 v1.0.0} & 73 & Superato
	
	\end{longtable}

\subsection{Revisione di progettazione}
\subsubsection{Analisi statica del codice}
La stesura del codice è supportata dai plugin dell'editor Visual Studio Code. I plugin in uso sono molti: i più importanti sono due:
\begin{itemize}
	\item \textbf{ESLint v1.8.1}: linter per il linguaggio Javascript;
	\item \textbf{solidity v0.0.49}: linter per il linguaggio Solidity.
\end{itemize}
 I plugin hanno aiutato fortemente alla stesura di codice sintatticamente corretto e alla standardizzazione del codice scritto, che risulta più uniforme. Inoltre hanno facilitato in parte il processo di debugging.

\subsubsection{Esiti verifiche automatizzate}
Nella tabella seguente vengono riportati gli indici Gulpease\glo{} di tutti
i documenti aggiornati o prodotti durante la revisione di progettazione.

\rowcolors{2}{pari}{dispari}


\begin{longtable}{ >{\centering}p{0.40\textwidth} >{\centering}p{0.25\textwidth}
		>{\centering}p{0.2075\textwidth}}
	\caption{Esiti verifiche automatizzate - Revisione di progettazione} \\
	%\hline
	\rowcolorhead
	\centering\textbf{\color{white}Documento} 
	& \centering\textbf{\color{white}Indice Gulpease} 
	& \centering\textbf{\color{white}Esito}
	\tabularnewline %\hline 
	\endfirsthead
	
	\textit{Analisi dei Requisiti v2.0.0} & 63 & Superato
	
	\tabularnewline 
	\textit{Glossario v2.0.0} & 71 & Superato
	
	\tabularnewline 
	\textit{Norme di Progetto v2.0.0} & 65 & Superato
	
	\tabularnewline 
	\textit{Piano di Progetto v2.0.0} & 66 & Superato
	
	\tabularnewline 
	\textit{Piano di Qualifica v2.0.0} & 67 & Superato	

	\tabularnewline 
	\textit{Verbale esterno 2019-02-19} & 63 & Superato
	
	\tabularnewline 
	\textit{Verbale esterno 2019-03-06} & 72 & Superato
	
	\tabularnewline 
	\textit{Verbale interno 2019-01-31} & 70 & Superato
	
	\tabularnewline 
	\textit{verbale interno 2019-02-20} & 69 & Superato
	
	\tabularnewline 
	\textit{verbale interno 2019-03-05} & 71 & Superato
		
\end{longtable}
