\section{Resoconto attività di verifica}
In questa sezione vengono descritti e analizzati gli esiti delle attività
di verifica svolte su tutti i documenti che vengono consegnati nelle varie 
revisioni di avanzamento del progetto.

\subsection{Revisione dei requisiti}
\subsubsection{Tracciamento dei casi d'uso e dei requisiti}
Per facilitare il tracciamento delle relazioni fra casi d'uso e requisiti che
fra requisiti e fonti, il gruppo ha deciso di utilizzare il software PragmaDB.

\subsubsection{Analisi statica dei documenti}
L'analisi dei documenti mediante \textit{Walkthrough}\glo{} ha portato 
all'individuazione di alcuni errori frequenti a partire dai quali è stata 
stilata una lista di controllo interna. In questo modo sarà possibile applicare
l'\textit{Inspection}\glo{} per le future attività di verifica.

\subsubsection{Esiti verifiche automatizzate}
Nella tabella seguente vengono riportati gli indici Gulpease\glo{} di tutti
i documenti prodotti finora.

	\rowcolors{2}{pari}{dispari}
	
		
	\begin{longtable}{ >{\centering}p{0.40\textwidth} >{\centering}p{0.25\textwidth}
			 >{\centering}p{0.2075\textwidth}}
		\caption{Esiti verifiche automatizzate - Revisione dei Requisiti} \\
		%\hline
		\rowcolorhead
		\centering\textbf{\color{white}Documento} 
		& \centering\textbf{\color{white}Indice Gulpease} 
		& \centering\textbf{\color{white}Esito}
		\tabularnewline %\hline 
		\endfirsthead
			
		
		
		\textit{Analisi dei Requisiti v1.0.0} & 67 & Superato
		
		\tabularnewline 
		\textit{Glossario v1.0.0} & 71 & Superato
				
		\tabularnewline 
		\textit{Norme di Progetto v1.0.0} & 65 & Superato
		
		\tabularnewline 
		\textit{Piano di Progetto v1.0.0} & 68 & Superato
		
		\tabularnewline 
		\textit{Piano di Qualifica v1.0.0} & 70 & Superato	
		
		\tabularnewline 
		\textit{Studio di Fattibilità v1.0.0} & 73 & Superato
		
		\tabularnewline 
		\textit{Verbale Interno 2018-12-04 v1.0.0} & 75 & Superato
		
		\tabularnewline 
		\textit{Verbale Interno 2018-12-10 v1.0.0} & 74 & Superato
		
		\tabularnewline 
		\textit{Verbale Interno 2018-12-20 v1.0.0} & 70 & Superato
		
		\tabularnewline 
		\textit{Verbale Interno 2018-12-24 v1.0.0} & 71 & Superato
		
		\tabularnewline 
		\textit{Verbale Esterno 2018-12-07 v1.0.0} & 74 & Superato
		
		\tabularnewline 
		\textit{Verbale Esterno 2018-12-21 v1.0.0} & 70 & Superato
		
		\tabularnewline 
		\textit{Verbale Esterno 2019-01-04 v1.0.0} & 73 & Superato
	
	\end{longtable}

\subsection{Revisione di progettazione}
\subsubsection{Correzione dei documenti}
Abbiamo svolto un'attività di verifica guidata dalle correzioni proposte dai professori alla RR. Gli errori presenti nei documenti, segnalati dai professori, sono stati corretti. Si elencano qui le più importanti modifiche:
\begin{itemize}
		\item In tutti i documenti sono stati rivisti lo stile di scrittura e la formattazione del testo, cercando di ottenere maggior chiarezza e uniformità;	
		\item le \textit{Norme di Progetto}
		\item 
		\item 
		\item 
\end{itemize}

\subsubsection{Esiti verifiche automatizzate}
Nella tabella seguente vengono riportati gli indici Gulpease\glo{} di tutti
i documenti aggiornati o prodotti durante la revisione di progettazione.

\rowcolors{2}{pari}{dispari}


\begin{longtable}{ >{\centering}p{0.40\textwidth} >{\centering}p{0.25\textwidth}
		>{\centering}p{0.2075\textwidth}}
	\caption{Esiti verifiche automatizzate - Revisione dei Requisiti} \\
	%\hline
	\rowcolorhead
	\centering\textbf{\color{white}Documento} 
	& \centering\textbf{\color{white}Indice Gulpease} 
	& \centering\textbf{\color{white}Esito}
	\tabularnewline %\hline 
	\endfirsthead
	
	
	
	\textit{Analisi dei Requisiti v2.0.0} & 67 & Superato
	
	\tabularnewline 
	\textit{Glossario v2.0.0} & 71 & Superato
	
	\tabularnewline 
	\textit{Norme di Progetto v2.0.0} & 65 & Superato
	
	\tabularnewline 
	\textit{Piano di Progetto v2.0.0} & 68 & Superato
	
	\tabularnewline 
	\textit{Piano di Qualifica v2.0.0} & 70 & Superato	
	
	\tabularnewline 
	\textit{Verbale Interno 2018-12-10 v1.0.0} & 74 & Superato

	\tabularnewline 
	\textit{Verbale Esterno 2019-01-04 v1.0.0} & 73 & Superato
	
\end{longtable}
