\section{Esiti delle revisioni}
	\subsection{Revisione dei Requisiti}
	Successivamente alla prima revisione, il gruppo, basandosi sulle 
	segnalazioni ricevute, ha apportato varie modifiche ai documenti. 
	Di seguito vengono descritte brevemente tali modifiche:
	\begin{itemize}
		\item in ogni documento è stato cambiato il nome della sezione 
		riportante le modifiche effettuate da "Tabella delle modifiche" a 
		"Registro delle modifiche";
		\item in ogni documento è stato inserito il numero di pagine totali 
		affianco al numero di pagina corrente;
		\item è stato cambiato il nome di alcune sezioni per renderle uniformi 
		con gli altri documenti;
		\item \textbf{Norme di Progetto} è stata aggiunta una sezione dove 
		vengono presentate le metriche relative ai processi. È stata aggiunta
		 una sezione riguardante la formazione dei membri del gruppo;
		\item \textbf{Analisi dei Requisiti} seguendo le indicazioni del 
		professor Cardin sono state apportate modifiche ai casi d'uso e ai requisiti;
		\item \textbf{Piano di Progetto} è stata aggiunta una sezione per il 
		preventivo a finire. È stato spostato il focus dalla documentazione per concentrarsi di più sui processi e sulla loro applicazione secondo il modello incrementale; %sto sparando un 
		%po' a caso modificate se serve
		\item \textbf{Piano di Qualifica} le definizioni 
		delle metriche dei processi sono state spostate nelle \textit{Norme di Progetto}, mantenendo nel \textit{Piano di Qualifica} solamente le formule di calcolo, le soglie e gli intervalli.
	\end{itemize}
	