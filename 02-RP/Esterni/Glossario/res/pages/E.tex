\subsection*{\quad$E\quad$}
\subsubsection*{ECR20}
\index{ECR20}
Sigla per "Ethereum Request for Comment". È lo standard per implementare token\glosp su Ethereum. Definisce regole sul trasferimento di token e su come accedere ai dati contenuti in essi, allo scopo di rendere queste operazioni ben definite e prevedibili.

\subsubsection*{EIP-712}
\index{EIP-712}
Stadard per le transazioni Ethereum. Implementa la firma digitale delle transazioni semplificando la procedura all'utente. In questa maniera l'utente utilizzerà tool come MetaMask\glosp per gestire le transazioni, senza la necessità di inserire le chiavi esadecimali a mano.

\subsubsection*{Escrow}
\index{Escrow}
Individua un accordo fra due soggetti in forza del quale somme di denaro o titoli di proprietà, oggetti del contratto, vengono depositati presso una terza parte a titolo di garanzia, e rilasciate poi all'avveramento di determinate condizioni espressamente stabilite dalle parti. In Soldino la terza parte garante è sostituita dagli smart contracts\glo.

\subsubsection*{ESlint}
\index{ESlint}
Linter\glo per l'analisi statica del codice e per l'identificazione di pattern in JavaScript. Si usa per ottenere codice corretto, comprensibile, conforme a delle regole date.

\subsubsection*{Ether}
\index{Ether}
Valuta in uso nella piattaforma Ethereum\glo.

\subsubsection*{Ethereum}
\index{Ethereum}
Blockchain\glosp per la creazione e pubblicazione di smart contracts\glo. Uno smart contract\glosp che vuole girare su questa rete paga la sua potenza computazionale tramite la valuta Ether. Come le altre blockchain\glosp permette anche lo scambio di denaro.

\subsubsection*{Ethereum Virtual Machine}
\index{Ethereum Virtual Machine}
Macchina virtuale decentralizzata sulla rete Ethereum, che permette di eseguire ÐApps\glo.

\subsubsection*{Event-driven}
\index{Event-driven}
Un programma si dice event-driven quando il flusso di esecuzione è determinato dal verificarsi di eventi esterni, ad esempio input dati dall'utente.

