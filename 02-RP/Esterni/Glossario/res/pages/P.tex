\subsection*{\quad$P\quad$}
\subsubsection*{Partizione di equivalenza}
\index{Partizione di equivalenza}
Nei test case, insieme di valori che sono trattati uniformemente: valori validi, valori di contorno della partizione o valoro non validi.

\subsubsection*{Peer-to-peer}
\index{Peer-to-peer}
Nelle reti telematiche, architettura in cui tutti i computer connessi svolgono la funzione sia di client che di server.

\subsubsection*{Plug-in}
\index{Plug-in}
Componente aggiuntivo che può essere aggiunto a un'altro software per ampliarne le funzionalità. Di solito può essere eseguito in modo indipendente.

\subsubsection*{PoS tagging}
\index{PoS tagging}
Interpretazione di un testo etichettando ciascuna parola con il suo significato grammaticale.

\subsubsection*{Prezzo lordo}
\index{Prezzo lordo}
Prezzo del prodotto dopo l'applicazione dell'imposta IVA.

\subsubsection*{Prezzo netto}
\index{Prezzo netto}
Prezzo del prodotto esente da imposta IVA.

\subsubsection*{Product Baseline}
\index{Product Baseline}
Baseline del progetto di ingegneria del software in cui il prodotto è stato realizzato e funziona, ma non è ancora pronto per il rilascio.

\subsubsection*{Proof of Concept}
\index{Proof of Concept}
È un prototipo software che dimostra che il progetto è fattibile conformemente alle richieste.

\subsubsection*{Pull request}
\index{Pull request}
Funzionalità di GitHub. Permette di comunicare agli altri membri del team i cambiamenti fatti, i  quali sono stati pushati sulla repository\glosp di GitHub.

\subsubsection*{Python}
\index{Python}
Linguaggio di programmazione general purpose, interpretato, ad alto livello. Gode di ampia diffusione per la sua semplicità e l'ampia collezione di librerie che lo supportano.

