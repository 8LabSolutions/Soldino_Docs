\subsection*{\quad$B\quad$}
\subsubsection*{Back end}
\index{Back end}
Back end è un termine largamente utilizzato per caratterizzare le interfacce che hanno come destinatario un programma. Una applicazione back end è un programma con il quale l’utente interagisce indirettamente, in generale attraverso l’utilizzo di una applicazione front end.

\subsubsection*{Blockchain}
\index{Blockchain}
Struttura dati condivisa ed immutabile, composta da un registro di blocchi concatenati in ordine cronologico. I blocchi una volta inseriti non sono più modificabili. Questa tecnologia, assimilabile ad un database distribuito, è gestito da una rete di nodi, ognuno dei quali possiede una copia dei dati.  Ogni transazione è regolata da un protocollo che ne verifica l'autenticità e la approva o meno. Una transazione approvata non è più modificabile e viene aggiunta alla blockchain. Con questo sistema non è necessaria la presenza di un'autorità esterna alla transazione che faccia da garante.

\subsubsection*{Body of knowledge}
\index{Body of knowledge}
Insieme di conoscenze di uno stesso dominio. Le conoscenze fanno parte del dominio e allo stesso tempo lo costituiscono.

\subsubsection*{Boilerplate}
\index{Boilerplate}
Pratica di riuso del codice in cui un frammento di codice viene replicato in più punti di un file, o in più file nello stesso punto. Solitamente rappresenta la base da cui cominciare per la stesura di un'applicazione. 

