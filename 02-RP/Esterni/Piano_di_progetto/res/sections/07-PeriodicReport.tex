\section{Consuntivo}
Di seguito verranno indicate le spese effettivamente sostenute, considerando sia quelle per ruolo sia quelle per persona. Il bilancio potrà risultare:
\begin{itemize}
	\item \textbf{Negativo:} se il consuntivo supera il preventivo;
	\item \textbf{Pari:} se il consuntivo e il preventivo sono pari;
	\item \textbf{Positivo:} se il preventivo supera il consuntivo.
\end{itemize}

\subsection{Periodo di Analisi}
Le ore di lavoro sostenute in questa fase sono da considerarsi come ore di investimento per l'approfondimento personale. Esse sono quindi non rendicontate.

\begin{table}[H]
				\centering\renewcommand{\arraystretch}{1.5}
				\caption{Consuntivo della fase di Analisi}
				\vspace{0.2cm}
                \begin{tabular}{c c c}
                               
                \rowcolorhead
                 {\colorhead \textbf{Ruolo}} &
                 {\colorhead \textbf{Ore}} & 
                 {\colorhead \textbf{Costo}} \\
				
                \rowcolorlight
                 {\colorbody Responsabile} & {\colorbody 31 (+0)} & 
                 {\colorbody \EUR{930,00} (+\EUR{0,00})}  
				\\
				
				\rowcolordark
                 {\colorbody Amministratore} & {\colorbody 44 (+19)} & 
                 {\colorbody \EUR{880,00} (+\EUR{380,00})}
				\\	
				
				\rowcolorlight
                 {\colorbody Analista} & {\colorbody 96 (+7)} & 
                 {\colorbody \EUR{2.400,00} (+\EUR{175,00})} 
				\\
				
				\rowcolordark
                 {\colorbody Progettista} & {\colorbody 19
                 (+7)} & 
                 {\colorbody \EUR{418,00} (+\EUR{154,00})} 
				\\
				
				\rowcolorlight
                 {\colorbody Programmatore} & {\colorbody -} & 
                 {\colorbody -} 
				\\
				
				\rowcolordark
                 {\colorbody Verificatore} & {\colorbody 90 (+7)} & 
                 {\colorbody \EUR{1.350,00} (+\EUR{105,00})} 
				\\
				
				\rowcolorlight
                 {\colorbody \textbf{Totale Preventivo}} & {\colorbody \textbf{280}} & 
                 {\colorbody \textbf{\EUR{5.978,00}}} 
				\\
				
				
				\rowcolordark
                 {\colorbody \textbf{Totale Consuntivo}} & {\colorbody \textbf{320}} & 
                 {\colorbody \textbf{\EUR{6.792,00}}} 
				\\
				
				
				\rowcolorlight
                 {\colorbody \textbf{Differenza}} & {\colorbody \textbf{40}} & 
                 {\colorbody \textbf{\EUR{+814,00}}} 
				\\
				
                

                \end{tabular}
                
\end{table}

\subsubsection{Conclusioni}
Come emerge dai dati riportati nella tabella soprastante, che presenta le ore relative al consuntivo della fase di Analisi, è stato necessario investire più tempo del previsto nei ruoli di \textit{Amministratore}, \textit{Analista}, \textit{Progettista} e \textit{Verificatore}. Per questo motivo il bilancio risultante è negativo. Di seguito sono riportate le cause di tali ritardi:
\begin{itemize}
	\item \textbf{\textit{Amministratori}}: è servito più tempo del previsto per riuscire ad individuare i software più adatti per la gestione del progetto, e per la loro corretta configurazione. Inoltre sono state aggiunte ed aggiornate alcune sezioni nelle \textit{Norme di Progetto}, necessarie al chiarimento di alcune problematiche sorte durante la stesura dei documenti;
	\item \textbf{\textit{Analisti}}: alcuni requisiti si sono rivelati di non facile comprensione, e sono state necessarie più ore di lavoro per la discussione interna (tra gli \textit{Analisti)} ed esterna (con i proponenti); 
	\item \textbf{\textit{Progettisti}}: l'elevato numero di requisiti individuati nell'\textit{Analisi dei Requisiti} ha comportato un aumento del tempo necessario alla stesura dei test;
	\item \textbf{\textit{Verificatori}}: l'aggiunta di nuove sezioni nelle Norme di Progetto e l'elevato numero di requisiti individuati hanno implicato un maggiore lavoro anche per questo ruolo. 
\end{itemize}
Il notevole quantitativo di ore che il gruppo ha dovuto impiegare nel primo periodo non deve ripetersi durante il lavoro rendicontato. Per le problematiche riscontrate verranno adottate le seguenti contromisure:
\begin{itemize}
	\item \textbf{configurazione della strumentazione}: buona parte degli strumenti sono già stati configurati. La configurazione della rimanente strumentazione verrà effettuata all'inizio della seconda fase, al fine di evitare un successivo consumo di ore per uniformare la strumentazione stessa ed i prodotti dei diversi elementi del gruppo;
	\item \textbf{comprensione dei requisiti}: i requisiti sono stati ampiamente discussi con il proponente durante questa fase, non si prevede di incorrere ulteriormente in tale problema;
	\item \textbf{scorretta implementazione ed utilizzo iniziale delle Norme di Progetto}: il gruppo ha finito la stesura del documento ed ha imparato ad applicare le norme in esso definite.
\end{itemize}

\subsubsection{Preventivo a finire} 
Essendo il primo periodo non rendicontato, non è necessario prevedere alcuna contromisura dal punto di vista del monte ore totale nonché del preventivo economico. I componenti del gruppo dovranno adottare le contromisure sopra descritte e lavorare con il metodo acquisito durante questo primo periodo.

\subsection{Periodo di Consolidamento dei requisiti}
Le ore di lavoro sostenute in questa fase sono relative al consolidamento dei requisiti, successivo al periodo di analisi. Alcune ore sono dedicate allo studio per l'approfondimento personale e sono da considerarsi come ore non rendicontate.

\begin{table}[H]
	\centering\renewcommand{\arraystretch}{1.5}
	\caption{Consuntivo della fase di Analisi}
	\vspace{0.2cm}
	\begin{tabular}{c c c}
		
		\rowcolorhead
		{\colorhead \textbf{Ruolo}} &
		{\colorhead \textbf{Ore}} & 
		{\colorhead \textbf{Costo}} \\
		
		\rowcolorlight
		{\colorbody Responsabile} & {\colorbody 31 (+0)} & 
		{\colorbody \EUR{930,00} (+\EUR{0,00})}  
		\\
		
		\rowcolordark
		{\colorbody Amministratore} & {\colorbody 44 (+19)} & 
		{\colorbody \EUR{880,00} (+\EUR{380,00})}
		\\	
		
		\rowcolorlight
		{\colorbody Analista} & {\colorbody 96 (+7)} & 
		{\colorbody \EUR{2.400,00} (+\EUR{175,00})} 
		\\
		
		\rowcolordark
		{\colorbody Progettista} & {\colorbody 19
			(+7)} & 
		{\colorbody \EUR{418,00} (+\EUR{154,00})} 
		\\
		
		\rowcolorlight
		{\colorbody Programmatore} & {\colorbody -} & 
		{\colorbody -} 
		\\
		
		\rowcolordark
		{\colorbody Verificatore} & {\colorbody 90 (+7)} & 
		{\colorbody \EUR{1.350,00} (+\EUR{105,00})} 
		\\
		
		\rowcolorlight
		{\colorbody \textbf{Totale Preventivo}} & {\colorbody \textbf{280}} & 
		{\colorbody \textbf{\EUR{5.978,00}}} 
		\\
		
		
		\rowcolordark
		{\colorbody \textbf{Totale Consuntivo}} & {\colorbody \textbf{320}} & 
		{\colorbody \textbf{\EUR{6.792,00}}} 
		\\
		
		
		\rowcolorlight
		{\colorbody \textbf{Differenza}} & {\colorbody \textbf{40}} & 
		{\colorbody \textbf{\EUR{+814,00}}} 
		\\
		
		
		
	\end{tabular}
	
\end{table}

\subsubsection{Conclusioni}
Come emerge dai dati riportati nella tabella soprastante, che presenta le ore relative al consuntivo della fase di Analisi, è stato necessario investire più tempo del previsto nei ruoli di \textit{Amministratore}, \textit{Analista}, \textit{Progettista} e \textit{Verificatore}. Per questo motivo il bilancio risultante è negativo. Di seguito sono riportate le cause di tali ritardi:
\begin{itemize}
	\item \textbf{\textit{Amministratori}}: è stato speso la maggior parte del tempo per la decisione degli strumenti relativi alla codifica e all'ambiente di lavoro; il rimanente è stato dedicato all'aggiornamento delle \textit{Norme di Progetto} in merito agli strumenti scelti e relativi alla codifica da applicare;
	\item \textbf{\textit{Analisti}}: è stata modificata la gerarchia dei casi d'uso sullo strumento di tracciamento dei requisiti, quindi aggiornati alcuni di questi in maniera più dettagliata laddove fosse stato necessario e riportandoli sul documento col fine di finalizzare lo stesso;
	\item \textbf{\textit{Progettisti}}: l'elevato numero di requisiti individuati nell'\textit{Analisi dei Requisiti} ha comportato un aumento del tempo necessario alla stesura dei test;
	\item \textbf{\textit{Verificatori}}: l'aggiunta di nuove sezioni nelle Norme di Progetto e l'elevato numero di requisiti individuati hanno implicato un maggiore lavoro anche per questo ruolo. 
\end{itemize}
Il notevole quantitativo di ore che il gruppo ha dovuto impiegare nel primo periodo non deve ripetersi durante il lavoro rendicontato. Per le problematiche riscontrate verranno adottate le seguenti contromisure:
\begin{itemize}
	\item \textbf{configurazione della strumentazione}: buona parte degli strumenti sono già stati configurati. La configurazione della rimanente strumentazione verrà effettuata all'inizio della seconda fase, al fine di evitare un successivo consumo di ore per uniformare la strumentazione stessa ed i prodotti dei diversi elementi del gruppo;
	\item \textbf{comprensione dei requisiti}: i requisiti sono stati ampiamente discussi con il proponente durante questa fase, non si prevede di incorrere ulteriormente in tale problema;
	\item \textbf{scorretta implementazione ed utilizzo iniziale delle Norme di Progetto}: il gruppo ha finito la stesura del documento ed ha imparato ad applicare le norme in esso definite.
\end{itemize}

\subsubsection{Preventivo a finire} 
Essendo il primo periodo non rendicontato, non è necessario prevedere alcuna contromisura dal punto di vista del monte ore totale nonché del preventivo economico. I componenti del gruppo dovranno adottare le contromisure sopra descritte e lavorare con il metodo acquisito durante questo primo periodo.

\subsection{Periodo di Progettazione architetturale}
Le ore di lavoro sostenute in questa fase sono da considerarsi come ore di investimento per l'approfondimento personale. Esse sono quindi non rendicontate.

\begin{table}[H]
	\centering\renewcommand{\arraystretch}{1.5}
	\caption{Consuntivo della fase di Analisi}
	\vspace{0.2cm}
	\begin{tabular}{c c c}
		
		\rowcolorhead
		{\colorhead \textbf{Ruolo}} &
		{\colorhead \textbf{Ore}} & 
		{\colorhead \textbf{Costo}} \\
		
		\rowcolorlight
		{\colorbody Responsabile} & {\colorbody 31 (+0)} & 
		{\colorbody \EUR{930,00} (+\EUR{0,00})}  
		\\
		
		\rowcolordark
		{\colorbody Amministratore} & {\colorbody 44 (+19)} & 
		{\colorbody \EUR{880,00} (+\EUR{380,00})}
		\\	
		
		\rowcolorlight
		{\colorbody Analista} & {\colorbody 96 (+7)} & 
		{\colorbody \EUR{2.400,00} (+\EUR{175,00})} 
		\\
		
		\rowcolordark
		{\colorbody Progettista} & {\colorbody 19
			(+7)} & 
		{\colorbody \EUR{418,00} (+\EUR{154,00})} 
		\\
		
		\rowcolorlight
		{\colorbody Programmatore} & {\colorbody -} & 
		{\colorbody -} 
		\\
		
		\rowcolordark
		{\colorbody Verificatore} & {\colorbody 90 (+7)} & 
		{\colorbody \EUR{1.350,00} (+\EUR{105,00})} 
		\\
		
		\rowcolorlight
		{\colorbody \textbf{Totale Preventivo}} & {\colorbody \textbf{280}} & 
		{\colorbody \textbf{\EUR{5.978,00}}} 
		\\
		
		
		\rowcolordark
		{\colorbody \textbf{Totale Consuntivo}} & {\colorbody \textbf{320}} & 
		{\colorbody \textbf{\EUR{6.792,00}}} 
		\\
		
		
		\rowcolorlight
		{\colorbody \textbf{Differenza}} & {\colorbody \textbf{40}} & 
		{\colorbody \textbf{\EUR{+814,00}}} 
		\\
		
		
		
	\end{tabular}
	
\end{table}

\subsubsection{Conclusioni}
Come emerge dai dati riportati nella tabella soprastante, che presenta le ore relative al consuntivo della fase di Analisi, è stato necessario investire più tempo del previsto nei ruoli di \textit{Amministratore}, \textit{Analista}, \textit{Progettista} e \textit{Verificatore}. Per questo motivo il bilancio risultante è negativo. Di seguito sono riportate le cause di tali ritardi:
\begin{itemize}
	\item \textbf{\textit{Amministratori}}: è servito più tempo del previsto per riuscire ad individuare i software più adatti per la gestione del progetto, e per la loro corretta configurazione. Inoltre sono state aggiunte ed aggiornate alcune sezioni nelle \textit{Norme di Progetto}, necessarie al chiarimento di alcune problematiche sorte durante la stesura dei documenti;
	\item \textbf{\textit{Analisti}}: alcuni requisiti si sono rivelati di non facile comprensione, e sono state necessarie più ore di lavoro per la discussione interna (tra gli \textit{Analisti)} ed esterna (con i proponenti); 
	\item \textbf{\textit{Progettisti}}: l'elevato numero di requisiti individuati nell'\textit{Analisi dei Requisiti} ha comportato un aumento del tempo necessario alla stesura dei test;
	\item \textbf{\textit{Verificatori}}: l'aggiunta di nuove sezioni nelle Norme di Progetto e l'elevato numero di requisiti individuati hanno implicato un maggiore lavoro anche per questo ruolo. 
\end{itemize}
Il notevole quantitativo di ore che il gruppo ha dovuto impiegare nel primo periodo non deve ripetersi durante il lavoro rendicontato. Per le problematiche riscontrate verranno adottate le seguenti contromisure:
\begin{itemize}
	\item \textbf{configurazione della strumentazione}: buona parte degli strumenti sono già stati configurati. La configurazione della rimanente strumentazione verrà effettuata all'inizio della seconda fase, al fine di evitare un successivo consumo di ore per uniformare la strumentazione stessa ed i prodotti dei diversi elementi del gruppo;
	\item \textbf{comprensione dei requisiti}: i requisiti sono stati ampiamente discussi con il proponente durante questa fase, non si prevede di incorrere ulteriormente in tale problema;
	\item \textbf{scorretta implementazione ed utilizzo iniziale delle Norme di Progetto}: il gruppo ha finito la stesura del documento ed ha imparato ad applicare le norme in esso definite.
\end{itemize}

\subsubsection{Preventivo a finire} 
Essendo il primo periodo non rendicontato, non è necessario prevedere alcuna contromisura dal punto di vista del monte ore totale nonché del preventivo economico. I componenti del gruppo dovranno adottare le contromisure sopra descritte e lavorare con il metodo acquisito durante questo primo periodo.

