\subsection{Capitolato C4 - MegAlexa}
\subsubsection{Informazioni generali}
\begin{itemize}
    \item \textbf{Nome}: MegAlexa, arricchitore di skill\glo{} di Amazon Alexa;
	\item \textbf{Proponente}: ZERO12; 
	\item \textbf{Committente}: Prof. Tullio Vardanega e Prof. Riccardo Cardin.
\end{itemize}
\subsubsection{Descrizione}
La sfida proposta dall'azienda proponente consiste nel progettare una skill\glo{} per 
Alexa, l'assistente virtuale prodotto da Amazon,
in grado di avviare dei workflow\glo{} creati dagli utenti tramite interfaccia web o
mobile app per iOS e Android.

\subsubsection{Finalità del progetto}
Un qualsiasi utente dotato di Amazon Alexa sarà in grado di creare alcuni comandi vocali personalizzati, al di fuori degli attuali schemi imposti da tale tecnologia. 
\subsubsection{Tecnologie interessate}
\begin{itemize}
    \item \textbf{Amazon Alexa}: l'assistente digitale di Amazon;
    \item \textbf{Lambda (AWS\glo)}: servizio di elaborazione serverless per 
l'esecuzione del proprio codice;
    \item \textbf{API gateway (AWS\glo)}: servizio API per la comunicazione con 
Lambda;
    \item \textbf{Aurora Serverless (AWS\glo)}: offre capacità di database senza 
dover allocare e gestire il server;
    \item \textbf{Node.js\glo}: piattaforma event-driven\glo{} per esecuzione di codice 
JavaScript server-side\glo;
    \item \textbf{HTML5}, \textbf{CSS3} e \textbf{JavaScript}: linguaggi da 
utilizzare per l'implementazione
    dell'interfaccia web;
    \item \textbf{Bootstrap}: uno dei framework\glosp più utilizzati per sviluppare front end\glo, consigliato dal proponente;
	\item \textbf{Android} e \textbf{iOS}: studio di questi sistemi operativi e 
	dei relativi framework\glosp per lo sviluppo dell'applicazione. 

\end{itemize}
\subsubsection{Aspetti positivi}
\begin{itemize}
    \item Il proponente offre delle lezioni al fine di introdurre il gruppo alle 
nuove tecnologie da utilizzare nello sviluppo del progetto e dirigere lo studio 
autonomo;
    \item La massiccia presenza nel web di documentazione dettagliata, esempi e 
strumenti può semplificare l'apprendimento di tali tecnologie. In 
particolare Amazon fornisce Alexa Skills Kit (raccolta di API\glo, 
strumenti, documentazioni ed esempi);
    \item Amazon ed il mercato in generale sembrano, al momento, molto interessati 
agli assistenti vocali, quindi la conoscenza di tali tecnologie può essere una 
nota rilevante a livello curricolare.
\end{itemize}
\subsubsection{Criticità e fattori di rischio}
\begin{itemize}
    \item \`E obbligatoro che le shortcuts\glosp siano multilingua. Echo al momento 
supporta: inglese, francese, tedesco, italiano, giapponese e spagnolo. Tuttavia, 
le nostre conoscenze in ambito linguistico ci permettono di realizzare in modo 
esaustivo solamente le versione italiana ed inglese;
    \item Sono già presenti, nel web, tecnologie per la realizzazione di skills\glo{} in 
    grado di avviare dei workflow\glo{} personalizzati, anche se in modo piuttosto grezzo.
    Infatti la stessa applicazione di Alexa permette di creare sequenze di azioni precedentemente selezionate.
 
\end{itemize}
\subsubsection{Conclusioni}
Nonostante tale capitolato\glosp abbia destato particolare interesse all'interno del 
team di lavoro, sia a livello tecnologico che di competenze curricolari, il 
gruppo si è mostrato più stimolato verso un altro progetto non meno allettante.


