\section{Casi d'uso} 
\subsection{Attori dei casi d'uso}
\subsubsection{Attori primari}
\begin{figure}[h]
	\includegraphics[width=7cm]{res/images/attori_primari.png}
	\centering
	\caption{Gerarchia degli attori primari}
\end{figure}
\begin{description}[style=nextline]
	\item[Utente generico]
	Si riferisce ad un utente generico che accede alla piattaforma dal sito web.
	\item[Utente non autenticato]
	Si riferisce ad un utente generico che non ha ancora effettuato l'autenticazione alla piattaforma.
	\item[Utente autenticato]
	Si riferisce ad un utente generico che si è autenticato nel sistema con la procedura di login. Ciò implica che sia in possesso di una chiave\glosp valida sulla rete. Ethereum\glosp con la quale, precedentemente, ha portato a termine la procedura di autenticazione.
	\item[Cittadino] Si riferisce ad un utente che si è autenticato nel sistema con il ruolo di cittadino.
	\item[Azienda] Si riferisce ad un utente che si è autenticato nel sistema con il ruolo di azienda. Le azioni sono eseguite considerando l'azienda come persona giuridica, nonostante le azioni vengano eseguite da un suo rappresentante.
	\item[Governo\glosp] Si riferisce ad un utente che si è autenticato al sistema con il ruolo di governo\glo. Nelle parti successive del documento tale parola verrà intesa con tale significato.
\end{description}
\subsubsection{Attori secondari}
\begin{description}[style=nextline]
	\item[MetaMask]
	Plug-in\glosp per browser che permette di interfacciarsi con la rete Ethereum\glosp e di validare le transazioni con la propria chiave privata.
	
\end{description}

\subsection{Elenco dei casi d'uso}
In questa sezione vi sono elencati tutti i casi d'uso individuati. Ogni caso d'uso rappresenta uno scenario per uno o più attori, ovviamente applicabile anche ad eventuali attori derivati. Ogni caso d'uso, inoltre, viene descritto tramite diagrammi dei casi d'uso e possiede una precondizione seguita da una postcondizione.
\subsubsection*{Operazioni utenti}
Di seguito sono riportati tutti i casi d'uso che coinvolgono come attore primario l'utente generico, l'utente non autenticato, l'utente autenticato, il cittadino e parte dei casi d'uso che coinvolgono l'azienda.

\begin{figure}[H]
	\includegraphics[width=14cm]{res/images/UseCase1-8.png}
	\centering
	\caption{Use Cases che interessano gli utenti}
\end{figure}

\subsubsection{UC1 - Breve Guida}
\begin{itemize}
	\item \textbf{Attori Primari}: utente generico;
	\item \textbf{Descrizione}: l'utente visualizza una guida riguardante l'installazione ed il funzionamento del plug-in Metamask\glosp. Viene spiegato come impostare le chiavi del plug-in e come utilizzarlo per accedere al sistema;
	\item \textbf{Scenario principale}: l'utente accede alla guida;
	\item \textbf{Precondizione}: il sistema è raggiungibile e funzionante, l'utente accede alla pagina iniziale del sito della piattaforma;
	\item \textbf{Postcondizione}: il sistema fornisce all'utente, attraverso la lettura della guida, tutte le istruzioni necessarie alla registrazione ed all'autenticazione.
	
	
\end{itemize}
\subsubsection{UC2 - Registrazione}
\begin{figure}[h]
	\includegraphics[width=16cm]{res/images/UC2Registrazione.png}
	\centering
	\caption{UC2 - Registrazione}
\end{figure}
\begin{itemize}
	\item \textbf{Attori Primari}: utente non autenticato, eventualmente proprietario d'azienda;
	\item \textbf{Attori Secondari}: MetaMask\glosp;
	\item \textbf{Descrizione}: l'utente non autenticato compila tutti i campi richiesti al fine di registrarsi sulla piattaforma, successivamente dovrà aspettare che il proprio account venga verificato da parte del governo\glo;
	\item \textbf{Scenario principale}: 
	
	\begin{enumerate}[label=\alph*.]
		
		\item l'utente vuole registrarsi come normale \textbf{cittadino}: 
		\begin{enumerate}[label=\roman*.]
			\item l'utente inserisce i dati anagrafici [UC2.3];
			\item l'utente conferma ed invia i dati inseriti [UC2.8].
		\end{enumerate}
		
		
		\item l'utente vuole registrare la propria \textbf{azienda}:
		\begin{enumerate}[label=\roman*.]
			\item l'utente inserisce i dati aziendali [UC2.4];
			\item l'utente conferma ed invia i dati inseriti [UC2.8].
		\end{enumerate}
		
		
		
	\end{enumerate}
	\item \textbf{Precondizione}: l'utente ha il plug-in MetaMask\glosp installato e correttamente impostato sul proprio browser, viene considerato dal sistema come un utente non autenticato ed intende registrarsi come cittadino o registrare la propria azienda; 
	\item \textbf{Postcondizione}: il sistema riceve le informazioni dell'utente, le salva ed inoltra la richiesta di verifica ed approvazione al governo\glo.
	
\end{itemize}
\subsubsection{UC2.1 - Identificazione utente attraverso MetaMask}
\begin{itemize}
	\item \textbf{Attori Primari}: utente non autenticato;
	\item \textbf{Attori Secondari}: MetaMask\glo;
	\item \textbf{Descrizione}: il sistema verifica la presenza del plug-in MetaMask\glo, al quale richiede la chiave pubblica che l'utente desidera usare come identificativo e metodo di pagamento. Essendo la chiave univoca, il sistema controlla che nessun utente registrato stia attualmente utilizzandola. Il sistema mostra infine l'interfaccia contente il form di registrazione;
	\item \textbf{Scenario principale}: il sistema, prima di mostrare il form per la registrazione, controlla la presenza di una chiave\glosp utilizzabile attraverso il plug-in MetaMask\glosp.
	\item \textbf{Estensioni}:
	\begin{itemize}
		\item \textbf{UC2.5}: se il plug-in MetaMask\glosp non risulta installato nel browser dell'utente, o è stato disabilitato, esso viene avvisato tramite l'apposito messaggio di errore;
		\item \textbf{UC2.6}: se l'utente non possiede una chiave\glosp su MetaMask\glosp, esso viene avvisato tramite l'apposito messaggio di errore;
		\item \textbf{UC2.7}: se è presente una chiave su MetaMask\glosp, ma il sistema rileva che essa è già stata utilizzata per la registrazione sulla piattaforma, allora l'utente viene avvisato attraverso l'apposito messaggio di errore.
	\end{itemize}
	\item \textbf{Precondizione}: l'utente ha cliccato sul link per accedere al form di registrazione di un account cittadino o aziendale;
	\item \textbf{Postcondizione}: il sistema ha verificato che la chiave\glosp con la quale l'utente sta cercando di registrarsi non risulti già utilizzata nel sistema. La chiave è salvata e verrà utilizzata durante la registrazione. All'utente è permesso compilare i dati relativi alla tipologia di registrazione richiesta.
	
\end{itemize}
\subsubsection{UC2.2 - Inserimento dati account}
\begin{figure}[h]
	\includegraphics[width=8cm]{res/images/UC2-2RegistrazioneGenerale.png}
	\centering
	\caption{UC2.2 - Inserimento dati account}
\end{figure}
\begin{itemize}
	\item \textbf{Attori Primari}: utente non autenticato;
	\item \textbf{Descrizione}: l'utente compila il form contenente i dati relativi all'account;
	\item \textbf{Scenario principale}: l'utente compila tutti i campi del form riguardanti l'account, ovvero:
	\begin{enumerate}[label=\alph*.]
		\item l'utente inserisce l'email da associare all'account [UC2.2.1];
		\item l'utente inserisce l'indirizzo di residenza da associare all'account [UC2.2.2];
		\item l'utente inserisce la password da associare all'account [UC2.2.3].
	\end{enumerate}
	\item \textbf{Specializzazione}:
	\begin{itemize}
		\item \textbf{UC2.3}: l'utente inserisce i dati relativi alla registrazione di un cittadino;
		\item \textbf{UC2.4}: l'utente inserisce i dati relativi alla registrazione di un'azienda.
		
	\end{itemize}
	\item \textbf{Precondizione}: l'utente ha espresso la volontà di iscriversi alla piattaforma cliccando uno dei link per la registrazione. Il sistema ha rilevato e salvato una chiave\glosp valida attraverso il plug-in MetaMask\glosp;
	\item \textbf{Postcondizione}: l'utente ha compilato i campi relativi ai dati dell'account.
	
\end{itemize}
\subsubsection{UC2.2.1 - Inserimento e-mail}
\begin{itemize}
	\item \textbf{Attori Primari}: utente non autenticato;
	\item \textbf{Descrizione}: al fine di portare a termine il processo di registrazione l'utente deve inserire un indirizzo e-mail, campo ritenuto obbligatorio;
	\item \textbf{Scenario principale}: l'utente compila il campo relativo all'indirizzo e-mail;
	\item \textbf{Precondizione}: il sistema ha reso disponibile il campo per l'inserimento dell'indirizzo e-mail;
	\item \textbf{Postcondizione}: l'utente ha compilato il campo con la propria e-mail.
	
\end{itemize}
\subsubsection{UC2.2.2 - Inserimento indirizzo di residenza}
\begin{itemize}
	\item \textbf{Attori Primari}: utente non autenticato;
	\item \textbf{Descrizione}: al fine di portare a termine il processo di registrazione l'utente deve inserire un indirizzo di residenza, campo ritenuto obbligatorio;
	\item \textbf{Scenario principale}: l'utente compila il campo relativo all'indirizzo di residenza;
	\item \textbf{Precondizione}: il sistema ha reso disponibile il campo per l'inserimento dell'indirizzo di residenza;
	\item \textbf{Postcondizione}: l'utente ha compilato il campo con l'indirizzo relativo alla residenza.
\end{itemize}
\subsubsection{UC2.2.3 - Inserimento password}
\begin{itemize}
	\item \textbf{Attori Primari}: utente non autenticato;
	\item \textbf{Descrizione}: al fine di portare a termine il processo di registrazione l'utente deve inserire una password, campo ritenuto obbligatorio;
	\item \textbf{Scenario principale}: l'utente compila il campo relativo alla password;
	\item \textbf{Precondizione}: il sistema ha reso disponibile il campo per l'inserimento della password;
	\item \textbf{Postcondizione}: l'utente ha compilato il campo con la password scelta.
\end{itemize}
\subsubsection{UC2.3 - Inserimento dati anagrafici}
\begin{figure}[h]
	\includegraphics[width=5cm]{res/images/UC2-3Registrazione-cliente.png}
	\centering
	\caption{UC2.3 - Registrazione cittadino}
\end{figure}
\begin{itemize}
	\item \textbf{Attori Primari}: utente non autenticato;
	\item \textbf{Descrizione}: l'utente compila i campi relativi alla registrazione alla piattaforma come cittadino;
	\item \textbf{Scenario principale}: l'utente ha cliccato sul pulsante di registrazione di un cittadino, il sistema rende disponibile il form di registrazione relativo e l'utente compila tutti i campi necessari. In particolare, oltre ai dati account:
	\begin{enumerate}[label=\alph*.]
		\item l'utente inserisce il proprio nome [UC2.3.1];
		\item l'utente inserisce il proprio cognome [UC2.3.2];
		\item l'utente inserisce il proprio codice fiscale [UC2.3.3].
	\end{enumerate}
	\item \textbf{Precondizione}: l'utente ha espresso la volontà di iscriversi alla piattaforma cliccando il link per la registrazione come cittadino. Il sistema ha rilevato e salvato una chiave\glosp valida attraverso il plug-in MetaMask\glo;
	\item \textbf{Postcondizione}: l'utente ha compilato i campi relativi alla registrazione di un cittadino alla piattaforma.
\end{itemize}
\subsubsection{UC2.3.1 - Inserimento nome}
\begin{itemize}
	\item \textbf{Attori Primari}: utente non autenticato;
	\item \textbf{Descrizione}: al fine di portare a termine il processo di registrazione di un nuovo cittadino, l'utente deve inserire il proprio nome;
	\item \textbf{Scenario principale}: l'utente compila il campo relativo al nome;
	\item \textbf{Precondizione}: il sistema ha reso disponibile il campo per l'inserimento del nome;
	\item \textbf{Postcondizione}: l'utente ha compilato il campo con il proprio nome.
\end{itemize}
\subsubsection{UC2.3.2 - Inserimento cognome}
\begin{itemize}
	\item \textbf{Attori Primari}: utente non autenticato;
	\item \textbf{Descrizione}: al fine di portare a termine il processo di registrazione di un nuovo cittadino, l'utente deve inserire il proprio cognome;
	\item \textbf{Scenario principale}: l'utente compila il campo relativo al cognome;
	\item \textbf{Precondizione}: il sistema ha reso disponibile il campo per l'inserimento del cognome;
	\item \textbf{Postcondizione}: l'utente ha compilato il campo con il proprio cognome.
\end{itemize}
\subsubsection{UC2.4 - Inserimento dati aziendali}
\begin{figure}[h]
	\includegraphics[width=6cm]{res/images/UC2-4RegistrazioneAzienda.png}
	\centering
	\caption{UC2 - Registrazione di un'azienda}
\end{figure}
\begin{itemize}
	\item \textbf{Attori Primari}: proprietario d'azienda;
	\item \textbf{Attori Secondari}: azienda;
	\item \textbf{Descrizione}: l'utente compila i campi relativi alla registrazione della propria azienda sulla piattaforma;
	\item \textbf{Scenario principale}: l'utente ha cliccato sul pulsante di registrazione di un'azienda, il sistema rende disponibile il form di registrazione relativo e l'utente compila tutti i campi necessari. In particolare, oltre ai dati account:
	\begin{enumerate}[label=\alph*.]
		\item l'utente inserisce la partita IVA dell'azienda [UC2.4.1];
		\item l'utente inserisce il nome dell'azienda [UC2.3.2];
	\end{enumerate}
	\item \textbf{Precondizione}: l'utente ha espresso la volontà di iscriversi alla piattaforma cliccando il link per la registrazione di un'azienda. Il sistema ha rilevato e salvato una chiave\glosp valida attraverso il plug-in MetaMask\glo;
	\item \textbf{Postcondizione}: l'utente ha compilato i campi relativi alla registrazione di un'azienda alla piattaforma.
\end{itemize}
\subsubsection{UC2.4.1 - Inserimento partita IVA}
\begin{itemize}
	\item \textbf{Attori Primari}: proprietario d'azienda;
	\item \textbf{Descrizione}: al fine di portare a termine il processo di registrazione della propria azienda, l'utente deve inserire la relativa partita IVA;
	\item \textbf{Scenario principale}: l'utente compila il campo relativo alla partita IVA;
	\item \textbf{Precondizione}: il sistema ha reso disponibile il campo per l'inserimento della partita IVA;
	\item \textbf{Postcondizione}: l'utente ha compilato il campo con la partita IVA dell'azienda che intende registrare.
\end{itemize}
\subsubsection{UC2.4.2 - Inserimento nome azienda}
\begin{itemize}
	\item \textbf{Attori Primari}: proprietario d'azienda;
	\item \textbf{Descrizione}: al fine di portare a termine il processo di registrazione della propria azienda, l'utente deve inserire il relativo nome aziendale;
	\item \textbf{Scenario principale}: l'utente compila il campo relativo al nome aziendale;
	\item \textbf{Precondizione}: il sistema ha reso disponibile il campo per l'inserimento del nome aziendale;
	\item \textbf{Postcondizione}: l'utente ha compilato il campo con il nome dell'azienda che intende registrare.
\end{itemize}



\subsubsection{UC2.5 - Visualizzazione errore MetaMask\glosp non presente}
\begin{itemize}
	\item \textbf{Attori Primari}: utente non autenticato
	\item \textbf{Descrizione}: l'utente visualizza un errore relativo al fatto che non il plug-in MetaMask\glosp non risulta installato o attualmente disabilitato;
	\item \textbf{Scenario principale}: l'utente non ancora identificato dal sistema tenta di accedere a sezioni che necessitano la presenza di MetaMask\glosp, e quest'ultimo non è installato o attualmente disabilitato;
	\item \textbf{Precondizione}: MetaMask\glosp non è presente nel browser dell'utente o è attualmente disabilitato;
	\item \textbf{Postcondizione}: l'utente è a conoscenza che è necessario attivare o installare MetaMask\glosp per proseguire.
	
\end{itemize}

\subsubsection{UC2.6 - Visualizzazione errore chiave non 
	presente}
\begin{itemize}
	\item \textbf{Attori Primari}: utente non autenticato;
	\item \textbf{Attori Secondari}: MetaMask\glo;
	\item \textbf{Descrizione}:
	l'utente visualizza un messaggio di errore relativo al fatto che non è stata rilevata nessuna chiave\glosp all'interno del plug-in MetaMask\glo;
	\item \textbf{Scenario principale}: l'utente tenta di accedere ad una sezione del sito che necessita l'identificazione di una chiave\glosp attraverso il plug-in MetaMask\glo, e questo non contiene almeno una chiave\glo;
	\item \textbf{Precondizione}: il plug-in MetaMask\glosp è correttamente configurato, ma non è presente nessuna chiave\glo;
	\item \textbf{Postcondizione}:
	l'utente è consapevole che il plug-in MetaMask\glosp non contiene almeno una chiave\glo.
	
\end{itemize}




\subsubsection{UC2.7 - Visualizzazione errore chiave già presente nel sistema}
\begin{itemize}
	\item \textbf{Attori Primari}: utente non autenticato;
	\item \textbf{Attori Secondari}: MetaMask\glo;
	\item \textbf{Descrizione}:
	l'utente visualizza un messaggio di errore relativo al fatto che la chiave\glosp reperita dal plug-in MetaMask\glosp risulta già presente nella piattaforma;
	\item \textbf{Scenario principale}: l'utente tenta di registrarsi al sito utilizzando una chiave\glosp già presente nel sistema;
	\item \textbf{Precondizione}: MetaMask\glosp è correttamente configurato, ma la chiave\glosp selezionata nel plug-in è già stata utilizzata nella piattaforma;
	\item \textbf{Postcondizione}:
	l'utente è consapevole che la chiave\glosp selezionata nel plug-in MetaMask\glosp è già stata utilizzata nella piattaforma, e quindi che necessita di un'altra chiave per effettuare la nuova registrazione.
\end{itemize}

\subsubsection{UC2.8 - Conferma ed invio dei dati}
\begin{itemize}
	\item \textbf{Attori Primari}: utente non autenticato, eventualmente proprietario d'azienda;
	\item \textbf{Descrizione}:
	l'utente preme il pulsante per la conferma e l'invio dei dati. A schermo viene mostrato un messaggio che conferma il successo dell'operazione e spiega, nel caso la registrazione sia relativa ad un'azienda, di attendere la verifica da parte del governo\glosp prima di poter effettivamente accedere alla piattaforma.
	\item \textbf{Scenario principale}: l'utente preme il pulsante di verifica ed invio dei dati dopo aver compilato i campi del form;
	\item \textbf{Estensioni}: 
	\begin{itemize}
		\item \textbf{UC2.9}: l'utente preme il pulsante di verifica senza aver compilato almeno uno dei campi del form, viene visualizzato il relativo errore;
	\end{itemize}
	\item \textbf{Precondizione}: il sistema permette all'utente di compilare il form di registrazione. \`E presente il pulsante per la conferma dei dati;
	\item \textbf{Postcondizione}:
	l'utente è consapevole che la richiesta di registrazione è avvenuta con successo. Nel caso della registrazione di un'azienda, il proprietario è consapevole del fatto che dovrà attendere la verifica dell'account da parte del governo\glosp.
\end{itemize}

\subsubsection{UC2.9 - Visualizzazione errore campo non compilato}
\begin{itemize}
	\item \textbf{Attori Primari}: utente non autenticato;
	\item \textbf{Descrizione}:
	l'utente visualizza un messaggio di errore relativo al fatto che almeno uno dei campi del form considerato risulta non compilato;
	\item \textbf{Scenario principale}: l'utente tenta di confermare ed inviare i dati di un form senza aver compilato tutti i campi;
	\item \textbf{Precondizione}: il sistema permette all'utente di compilare il form. \`E presente il pulsante per la conferma dei dati;
	\item \textbf{Postcondizione}:
	l'utente è consapevole che per inviare i dati e terminare la procedura che sta seguendo deve compilare tutti i campi presenti nel form.
\end{itemize}

\subsubsection{UC3 - Login}
\begin{figure}[h]
	\includegraphics[width=12cm]{res/images/UC3Login.png} %da adattare in larghezza
	\centering
	\caption{UC3 - Login}
	
\end{figure}
\begin{itemize}
	\item \textbf{Attori Primari}:
	utente non autenticato;
	\item \textbf{Attori Secondari}:
	MetaMask\glo;
	\item \textbf{Descrizione}:
	l'utente tenta di autenticarsi alla piattaforma attraverso l'utilizzo di MetaMask\glo;
	\item \textbf{Scenario principale}:
	l'utente non è ancora autenticato al sito ed esegue il login;
	\item \textbf{Precondizione}:
	l'utente non è autenticato alla piattaforma;
	\item \textbf{Postcondizione}:
	l'utente si è autenticato con successo, ed è stato identificato dal sistema nel ruolo di cittadino, azienda o utente governativo. A seconda della tipologia di utente vengono rese disponili diverse pagine e funzionalità.
\end{itemize}
\subsubsection{UC3.1 - Login automatico}
\begin{itemize}
	\item \textbf{Attori Primari}:
	utente non autenticato;
	\item \textbf{Attori Secondari}:
	MetaMask\glo;
	\item \textbf{Descrizione}:
	in modo automatico, il sistema procede all'identificazione dell'utente;
	\item \textbf{Scenario principale}:
	l'utente non ancora autenticato richiede il login;
	\item \textbf{Estensioni}:
	\begin{itemize}
		\item \textbf{UC2.5}: se l'utente non dispone di MetaMask\glosp o ha disabilitato l'estensione, viene visualizzato un messaggio di errore a riguardo;
		\item \textbf{UC2.6}: se l'utente non possiede una chiave\glosp su MetaMask\glo, esso viene avvisato tramite l'apposito messaggio di errore;
		\item \textbf{UC3.2}: se l'utente tenta di accedere al sito tramite MetaMask\glosp senza aver mai provveduto a registrarsi, riceverà un messaggio di errore a riguardo;
		
		\item \textbf{UC3.3}: se l'utente si è registrato ma il suo account è stato disabilitato, riceverà un messaggio di errore a riguardo.
	\end{itemize}
	\item \textbf{Precondizione}:
	l'utente tenta di autenticarsi alla piattaforma;
	\item \textbf{Postcondizione}:
	l'utente viene individuato attraverso l'utilizzo del plug-in MetaMask\glo.
\end{itemize}
\subsubsection{UC3.2 - Visualizzazione messaggio di errore relativo a chiave non registrata}
\begin{itemize}
	\item \textbf{Attori Primari}:
	utente non autenticato;
	\item \textbf{Descrizione}:
	l'utente visualizza un messaggio di errore dovuto al fatto che ha tentato il login senza essersi registrato in precedenza;
	\item \textbf{Scenario principale}:
	l'utente tenta di eseguire la procedura di login alla piattaforma senza essere registrato;
	\item \textbf{Precondizione}:
	l'utente tenta di autenticarsi nella piattaforma;
	\item \textbf{Postcondizione}: viene visualizzato un messaggio d'errore per informare l'utente del fatto che è necessario registrarsi alla piattaforma prima di poter poi effettuare la procedura di login.
\end{itemize}
\subsubsection{UC3.3 - Visualizzazione schermata relativa a utente non abilitato}
\begin{itemize}
	\item \textbf{Attori Primari}: utente non autenticato;
	\item \textbf{Descrizione}:
	l'utente tenta di autenticarsi alla piattaforma, tuttavia, a causa della disabilitazione del suo account, il login viene interrotto, e l'utente visualizza il messaggio personalizzato lasciato dal governo che illustra la causa della disabilitazione dell'account;
	\item \textbf{Scenario principale}:
	l'utente non autenticato con account disabilitato tenta di autenticarsi. La procedura di autenticazione viene bloccata a causa dello stato dell'account;
	\item \textbf{Precondizione}:
	un utente non autenticato, registrato alla piattaforma e con account disabilitato tenta di effettuare il login automatico;
	\item \textbf{Postcondizione}:  viene visualizzato un messaggio d'errore per informare l'utente del fatto che il proprio account è stato disabilitato dal governo. Se quest'ultimo, durante la procedura di disabilitazione, ha inserito un messaggio contenente la causa di tale azione, allora tale messaggio viene visualizzato.
\end{itemize}
\subsubsection{UC4 - Logout}
\begin{itemize}
	\item \textbf{Attori Primari}:
	utente autenticato;
	\item \textbf{Attori Secondari}:
	MetaMask\glo;
	\item \textbf{Descrizione}: l'utente richiede il logout dalla piattaforma web. Vengono visualizzate le informazioni necessarie per procedere alla procedura di logout, che deve essere effettuata attraverso l'utilizzo del plug-in MetaMask\glo;
	\item \textbf{Scenario principale}: l'utente è autenticato dal sito e richiede di effettuare il logout, premendo sull'apposito pulsante;
	\item \textbf{Precondizione}: l'utente ha effettuato il login alla piattaforma web e richiede di essere disconnesso dal sito;
	\item \textbf{Postcondizione}: vengono visualizzate le istruzioni necessarie per eseguire il logout tramite MetaMask\glo. 
\end{itemize}


 \subsubsection{UC5 - Visualizzazione beni e servizi}
  \begin{figure}[H]
 	\includegraphics[width=6cm]{res/images/UC5-Generale.png}
 	\centering
 	\caption{UC5 - Visualizzazione beni e servizi}
 \end{figure}
 \begin{itemize}
 	\item \textbf{Attori Primari}: cittadino, azienda;
 	\item \textbf{Descrizione}: l'utente visualizza una lista contenente tutti i beni/servizi in vendita sulla piattaforma. Per ogni prodotto vengono visualizzate le seguenti informazioni:
 	\begin{itemize}
 		\item nome;
 		\item prezzo lordo\glosp (in Cubit\glo);
 		\item descrizione;
 		\item quantità di prodotto selezionata.
 	\end{itemize}
 	L'utente può filtrare i risultati/cercare dei prodotti inserendo una parola chiave nella barra di ricerca apposita;
 	\item \textbf{Scenario principale}: l'utente si trova all'interno della pagina per la ricerca dei prodotti e sta visualizzando dei beni/servizi;
	\item \textbf{Inclusioni}:
	\begin{itemize}
		\item \textbf{UC16}: l'utente può filtrare i risultati visualizzati inserendo una parola chiave.
	\end{itemize}
 	\item \textbf{Precondizione}: l'utente accede alla pagina del sito dedicata alla visualizzazione dei prodotti in vendita;
 	\item \textbf{Postcondizione}: l'utente visualizza le informazioni relative ai prodotti, con le eventuali operazioni disponibili su ognuno di essi.
 \end{itemize}
 \subsubsection{UC5.1 - Modifica della quantità selezionata}
 \begin{itemize}
 	\item \textbf{Attori Primari}: cittadino, azienda\glo;
 	\item \textbf{Descrizione}: l'utente può modificare la quantità selezionata di un prodotto utilizzando il campo apposito;
 	\item \textbf{Scenario principale}: l'utente modifica la quantità di un prodotto utilizzando l'apposito campo;
 	\item \textbf{Precondizione}: l'utente sta visualizzando un prodotto ed intende modificarne la quantità selezionata;
 	\item \textbf{Postcondizione}: la quantità del prodotto è stata aggiornata al nuovo valore selezionato.
 \end{itemize}
\subsubsection{UC6 - Gestione Carrello}
 \begin{figure}[h]
	\includegraphics[width=9cm]{res/images/UC6GestioneCarrello.png}
	\centering
	\caption{UC6 - Gestione carrello}
\end{figure}
\begin{itemize}
	\item \textbf{Attori Primari}: cittadino, azienda\glo;
	\item \textbf{Descrizione}: l'utente, che sia esso cittadino o azienda, ha il totale controllo del proprio carrello, con possibilità di aggiunta e rimozione di prodotti e/o servizi, con possibilità di trarne un riepilogo;
	\item \textbf{Scenario}: l'utente in questione intende comprare un bene e/o un servizio:
	 \begin{enumerate}[label=\alph*.]
		\item inserimento prodotto [UC6.1];
		\item visualizzazione prodotti nel carrello [UC6.2];
		\item rimozione prodotto [UC6.3];
	\end{enumerate}
	\item \textbf{Precondizione}: l'utente necessita di acquistare un prodotto o un servizio;
	\item \textbf{Postcondizione}: l'utente può procedere all'acquisto di tutti i beni e/o servizi presenti nel carrello.
\end{itemize} 
 \subsubsection{UC6.1 - Inserimento prodotto}
\begin{itemize}
	\item \textbf{Attori Primari}: cittadino, azienda\glo;
	\item \textbf{Descrizione}: l'utente inserisce il prodotto selezionato all'interno del proprio carrello;
	\item \textbf{Scenario}: l'utente si trova all'interno della pagina di un bene o servizio e clicca sul pulsante dedicato all'aggiunta dello stesso nel carrello virtuale;
	\item \textbf{Precondizione}: l'utente deve essere all'interno della pagina di un bene o un servizio;
	\item \textbf{Postcondizione}: nel carrello dell'utente è presente il bene o il servizio in quantità definite dall'utente stesso nella fase di visualizzazione della pagina del bene o del servizio in questione [UC5].
\end{itemize}
 \subsubsection{UC6.2 - Visualizzazione prodotti del carrello}
  \begin{figure}[h]
 	\includegraphics[width=7cm]{res/images/UC6-2VisualizzazioneProdottiCarrello.png}
 	\centering
 	\caption{UC6.2 - Visualizzazione prodotti del carrello}
 \end{figure}
\begin{itemize}
	\item \textbf{Attori Primari}: cittadino, azienda\glo;
	\item \textbf{Descrizione}: l'utente visualizza un riepilogo di tutti i prodotti presenti all'interno el proprio carrello;
	\item \textbf{Scenario}: l'utente si trova all'interno della pagina del proprio carrello virtuale e può apprezzare una lista di prodotti se essi sono stati aggiunti al carrello, in particolare per ognino di essi può trarre le informazioni:
	 \begin{enumerate}[label=\alph*.]
		\item visualizzazione nome [UC6.2.1];
		\item visualizzazione prezzo [UC6.2.2];
	\end{enumerate} 
	\item \textbf{Precondizione}: l'utente deve essere all'interno della pagina ddel proprio carrello;
	\item \textbf{Postcondizione}: l'utente è consapevole di tutti i prodotti all'interno del proprio carrello.
\end{itemize}
 \subsubsection{UC6.2.1 - Visualizzazione nome}
\begin{itemize}
	\item \textbf{Attori Primari}: cittadino, azienda\glo;
	\item \textbf{Descrizione}: l'utente visualizza il nome del bene o del servizio presente nel carrello;
	\item \textbf{Scenario}: l'utente si trova all'interno della pagina del proprio carrello;
	\item \textbf{Precondizione}: l'utente deve aver inserito almeno un bene o un servizio nel proprio carrello;
	\item \textbf{Postcondizione}: l'utente conosce il nome del prodotto presente nel carrello.
\end{itemize}
 \subsubsection{UC6.2.2 - Visualizzazione prezzo}
\begin{itemize}
	\item \textbf{Attori Primari}: cittadino, azienda\glo;
	\item \textbf{Descrizione}: l'utente visualizza il prezzo del bene o del servizio presente nel carrello;
	\item \textbf{Scenario}: l'utente si trova all'interno della pagina del proprio carrello;
	\item \textbf{Precondizione}: l'utente deve aver inserito almeno un bene o un servizio nel proprio carrello;
	\item \textbf{Postcondizione}: l'utente conosce il prezzo del prodotto presente nel carrello.
\end{itemize}
 \subsubsection{UC6.3 - Rimozione prodotto}
\begin{itemize}
	\item \textbf{Attori Primari}: cittadino, azienda\glo;
	\item \textbf{Descrizione}: l'utente rimuove il prodotto selezionato dal proprio carrello;
	\item \textbf{Scenario}: l'utente si trova all'interno del carrello e clicca sul pulsante dedicato alla rimozione dello stesso dal carrello virtuale;
	\item \textbf{Precondizione}: l'utente deve essere all'interno della pagina del proprio carrello;
	\item \textbf{Postcondizione}: nel carrello dell'utente non vi è più presente il bene o il servizio rimosso.
\end{itemize}
\subsubsection{UC7 - Acquisto beni}
\begin{figure}[h]
	\includegraphics[width=14cm]{res/images/UC7-Generale.png}
	\centering
	\caption{UC7 - Acquisto beni}
\end{figure}
\begin{itemize}
	\item \textbf{Attori Primari}: cittadino e azienda-cliente;
	\item \textbf{Attori Secondari}: azienda-venditrice, MetaMask\glo;
	\item \textbf{Descrizione}: i cittadini e le aziende possono acquistare i prodotti/servizi che avevano precedentemente aggiunto nel carrello;
	\item \textbf{Scenario principale}: l'utente innanzitutto inserisce dei prodotti nel carrello [UC6.1]. Dunque procede al checkout [UC7.1] dei prodotti nel carrello:
	\begin{enumerate}[label=\alph*.]
		\item i cittadini procedono con la conferma dell'ordine ed il pagamento immediato [UC7.2];
		\item le aziende invece dovono scegliere un metodo di pagamento [UC7.3.1] tra il pagamento immediato [UC7.3.2] ed il pagamento dilazionato [UC7.3.3]. A differenza dei cittadini, devono successivamente approvare la proposta d'ordine dal proprio account per ottenere la fatturazione.
	\end{enumerate}
	
	\item \textbf{Precondizione}: il sistema ha reso disponibile all'utente l'utilizzo del carrello per poter effettuare degli acquisti. L'utente ha inserito degli oggetti nel carrello ed ha espresso la volontà di acquistare tali prodotti;
	\item \textbf{Postcondizione}: è stato effettuato l'acquisto dei beni presenti nel carrello. Se l'utente è cittadino, allora l'ordine è concluso e viene effettuato il pagamento all'azienda venditrice. Viceversa, nel caso il cliente fosse un'azienda, viene inviata una proposta d'ordine nella pagina dedicata, che dovrà successivamente essere confermata. Inoltre, nel caso questo fosse un ordine con pagamento immediato, il sistema provvederà a trattenere i soldi dovuti all'azienda venditrice finché non verrà confermata la correttezza della proposta della fattura.
\end{itemize} 
\subsubsection{UC7.1 - Checkout}
\begin{itemize}
	\item \textbf{Attori Primari}: cittadino, azienda;
	\item \textbf{Descrizione}: l'utente effettua il checkout per poter poi acquistare i prodotti inseriti nel carrello;
	\item \textbf{Scenario principale}: l'utente preme il pulsante per effettuare il checkout;
	\item \textbf{Inclusioni}: 
	\begin{itemize}
		\item \textbf{UC7.6}: l'utente inserisce l'indirizzo di spedizione;
	\end{itemize}
	\item \textbf{Estensioni}: 
	\begin{itemize}
		\item \textbf{UC7.4}: l'utente preme il pulsante di checkout senza aver ancora inserito almeno un prodotto/servizio nel carrello;
	\end{itemize}
	\item \textbf{Precondizione}: il sistema ha reso disponibile il carrello all'utente, identificandolo quindi come cittadino o azienda. L'utente ha espresso la volontà di procedere con il checkout premendo l'apposito pulsante;
	\item \textbf{Postcondizione}: l'utente accede alla pagina dove poter confermare l'acquisto e procedere con il pagamento.
\end{itemize}

\subsubsection{UC7.2 - Conferma ordine e pagamento cittadino}
\begin{itemize}
	\item \textbf{Attori Primari}: cittadino;
	\item \textbf{Attori Secondari}: azienda-venditrice, MetaMask\glo;
	\item \textbf{Descrizione}: l'utente visualizza il contenuto del carrello ed ha la possibilità di confermare l'ordine e procedere al pagamento che avviene attraverso l'utilizzo del plugin Metamask\glo;
	\item \textbf{Scenario principale}: l'utente preme il pulsante per effettuare il checkout;
	\item \textbf{Inclusioni}: 
	\begin{itemize}
		\item \textbf{UC6.2}: l'utente visualizza un riepilogo di tutti i prodotti presenti all'interno del carrello.
	\end{itemize}
	\item \textbf{Estensioni}: 
	\begin{itemize}
		\item \textbf{UC7.5}: l'utente tenta il pagamento ma l'esito non va a buon fine. La causa del fallimento dell'operazione è gestito dal plugin stesso.
	\end{itemize}
	\item \textbf{Precondizione}: il sistema ha reso disponibile il riepilogo dei prodotti con la possibilità di effettuare la conferma dell'ordine ed il pagamento. Almeno un elemento era stato inserito nel carrello; 
	\item \textbf{Postcondizione}: l'utente ha portato a termine la procedura di acquisto dei prodotti/servizi presenti nel carrello. L'azienda venditrice ha ricevuto il pagamento dal cliente e l'ordine è aggiunto alla lista delle fatture dell'azienda. Al cliente viene segnalato il successo dell'operazione.
\end{itemize}
\subsubsection{UC7.3 - Pagamento aziendale}
\begin{figure}[H]
	\includegraphics[width=10cm]{res/images/UC7-PagamentoAzienda.png}
	\centering
	\caption{UC7.3 - Procedura di pagamento da parte delle aziende}
\end{figure}
\begin{itemize}
	\item \textbf{Attori Primari}: azienda-cliente;
	\item \textbf{Attori Secondari}: azienda-venditrice, MetaMask\glo;
	\item \textbf{Descrizione}: le aziende hanno un processo di pagamento differente dai cittadini, possono scegliere fra pagamento immediato e dilazionato\glo;
	\item \textbf{Scenario principale}: l'azienda sceglie un metodo di pagamento [UC7.3.1] tra il pagamento immediato [UC7.3.2] ed il pagamento dilazionato [UC7.3.3]. Successivamente dovrà approvare la proposta d'ordine dal proprio account per ottenere la fatturazione;
	\item \textbf{Inclusioni}: 
	\begin{itemize}
		\item \textbf{UC6.2}: l'utente visualizza un riepilogo di tutti i prodotti presenti all'interno del carrello.
	\end{itemize}
	\item \textbf{Precondizione}: il sistema ha reso disponibile il riepilogo dei prodotti con la possibilità di scegliere il metodo di pagamento ed effettuare la conferma dell'ordine;
	\item \textbf{Postcondizione}: è stato scelto il metodo di pagamento e l'ordine è stato effettuato con successo. All'azienda cliente viene inviata una proposta d'ordine e vengono spiegati i passi successivi da seguire per accettare tale proposta, in maniera da validare l'ordine e ricevere la fattura.
\end{itemize} 

\subsubsection{UC7.3.1 - Selezione metodo di pagamento}
\begin{itemize}
	\item \textbf{Attori Primari}: azienda;
	\item \textbf{Descrizione}: l'utente visualizza le possibili modalità di pagamento, immediato e dilazionato\glo, seleziona uno dei due metodi e conferma la scelta;
	\item \textbf{Scenario principale}: l'utente seleziona una modalità di pagamento e conferma l'ordine, per poi procedere al pagamento;
	\item \textbf{Precondizione}: il sistema ha reso disponibile la scelta del metodo di pagamento e la conferma dell'ordine;
	\item \textbf{Postcondizione}: l'utente ha selezionato la modalità di pagamento per l'ordine corrente e l'ordine è stato confermato, il sistema procede con la procedura di pagamento selezionata.
\end{itemize}

\subsubsection{UC7.3.2 - Pagamento immediato}
\begin{itemize}
	\item \textbf{Attori Primari}: azienda-cliente;
	\item \textbf{Attori Secondari}: azienda-venditrice, MetaMask\glo;
	\item \textbf{Descrizione}: l'utente procede con il pagamento immediato: l'azienda-cliente versa l'importo dell'ordine nel sistema attraverso l'utilizzo del plugin MetaMask\glo. Viene inviata automaticamente la proposta d'ordine dall'azienda-venditrice all'azienda-cliente. Al cliente viene spiegato come procedere per confermare l'ordine e ricevere la fattura IVA;
	\item \textbf{Scenario principale}: l'utente procede con il pagamento attraverso l'ausilio del plugin MetaMask\glo;
	\item \textbf{Estensioni}: 
	\begin{itemize}
		\item \textbf{UC7.5}: l'utente tenta il pagamento ma l'esito non va a buon fine. La causa del fallimento dell'operazione è gestito dal plugin stesso.
	\end{itemize}
	\item \textbf{Precondizione}: il sistema ha reso disponibile il pagamento dell'ordine;
	\item \textbf{Postcondizione}: l'utente ha portato a termine la procedura di acquisto. Il sistema ha ricevuto il pagamento dal cliente, il cliente riceve, nella pagina dedicata, la proposta d'ordine. Al cliente viene comunicato il successo dell'operazione e le istruzioni per procedere all'accettazione della proposta d'ordine. Nella pagina dedicata dell'azienda venditrice compare l'acquisto con lo stato di attesa di approvazione.
\end{itemize}

\subsubsection{UC7.3.3 - Pagamento dilazionato}
\begin{itemize}
	\item \textbf{Attori Primari}: azienda-cliente;
	\item \textbf{Attori Secondari}: azienda-venditrice, MetaMask\glo;
	\item \textbf{Descrizione}: l'utente procede con il pagamento dilazionato: l'azienda-cliente sceglie di quanto dilazionare il pagamento, e conferma l'operazione con MetaMask\glo. Viene inviata automaticamente la proposta d'ordine dall'azienda-venditrice all'azienda-cliente. Al cliente viene spiegato come procedere per confermare l'ordine;
	\item \textbf{Scenario principale}: l'utente procede con la conferma dell'operazione attraverso l'ausilio del plugin MetaMask\glo;
	\item \textbf{Inclusioni}: 
	\begin{itemize}
		\item \textbf{UC 7.3.4}: l'utente deve selezionare il periodo per la dilazione del pagamento;
	\end{itemize}
	\item \textbf{Precondizione}: il sistema ha reso disponibile il pagamento dell'ordine;
	\item \textbf{Postcondizione}: l'utente ha portato a termine la procedura di acquisto con pagamento dilazionato. Il cliente riceve, nella pagina dedicata, la proposta d'ordine. Al cliente viene comunicato il successo dell'operazione e le istruzioni per procedere all'accettazione della proposta d'ordine. Nella pagina dedicata dell'azienda venditrice compare l'acquisto con lo stato di attesa di approvazione con pagamento dilazionato.
\end{itemize}
\subsubsection{UC7.3.4 - Selezione mesi per la dilazione}
\begin{itemize}
	\item \textbf{Attori Primari}: azienda-cliente;
	\item \textbf{Attori Secondari}: azienda-venditrice, MetaMask\glo;
	\item \textbf{Descrizione}: l'utente seleziona il numero di mesi dei quali deve essere dilazionato il pagamento;
	\item \textbf{Scenario principale}: l'utente seleziona la durata della dilazione durante la procedura di pagamento dilazionato\glo;
	\item \textbf{Precondizione}: l'utente sta eseguendo la procedura per il pagamento dilazionato;
	\item \textbf{Postcondizione}: l'utente ha selezionato la durata in mesi per la dilazione del pagamento.
\end{itemize}

\subsubsection{UC7.4 - Visualizzazione errore carrello vuoto}
\begin{itemize}
	\item \textbf{Attori Primari}: azienda;
	\item \textbf{Descrizione}:
	l'utente visualizza un messaggio di errore relativo al fatto che non è presente alcun prodotto nel proprio carrello e che quindi non è possibile procedere con la procedura di checkout;
	\item \textbf{Scenario}: l'utente tenta di procedere con il checkout senza aver inserito alcun prodotto/servizio nel carrello;
	\item \textbf{Precondizione}: il sistema ha reso disponibile all'utente il carrello ed il pulsante di checkout. L'utente ha premuto sul pulsante di checkout ed il carrello risulta vuoto; 
	\item \textbf{Postcondizione}:
	l'utente è consapevole che, per procedere con il checkout, il carrello deve contenere almeno un prodotto/servizio. 
\end{itemize}
\subsubsection{UC7.5 - Visualizzazione errore pagamento fallito}
\begin{itemize}
	\item \textbf{Attori Primari}: azienda;
	\item \textbf{Attori Secondari}: MetaMask\glo;
	\item \textbf{Descrizione}:
	l'utente visualizza un messaggio di errore relativo al fatto che il tentativo di pagamento non è andato a buon fine, e che quindi l'ordine è stato annullato. L'utente viene invitato ad informarsi sulla causa del fallimento dell'operazione all'interno del plugin.
	\item \textbf{Scenario principale}: l'utente tenta pagare attraverso il plugin MetaMask\glosp la somma dovuta al venditore per l'acquisto corrente;
	\item \textbf{Precondizione}: il sistema permette all'utente di procedere con il pagamento, ovvero l'ordine è stato confermato da parte dell'utente;
	\item \textbf{Postcondizione}:
	l'utente è consapevole che l'acquisto non è andato a buon fine, e che ottenere informazioni più precise dovrà riferirsi al messaggio di errore riportato dal plugin. 
\end{itemize}

\subsubsection{UC7.6 - Inserimento indirizzo spedizione}
\begin{itemize}
	\item \textbf{Attori Primari}: azienda, cittadino;
	\item \textbf{Descrizione}:
	l'utente visualizza un form dove inserire l'indirizzo di spedizione.
	\item \textbf{Scenario principale}: l'utente sta eseguendo la fase di checkout e per continuare deve inserire tale indirizzo;
	\item \textbf{Precondizione}: l'utente sta eseguendo la procedura di checkout;
	\item \textbf{Postcondizione}:
	l'utente ha inserito l'indirizzo di spedizione e può continuare concludendo il checkout. Tale indirizzo potrà essere utilizzato per la creazione della relativa fattura.
\end{itemize}
\subsubsection{UC8 - Gestione Beni Venduti}
\begin{figure}[h]
	\includegraphics[width=10cm]{res/images/UC8-Generale.png}
	\centering
	\caption{Operazioni per la gestione dei beni venduti}
\end{figure}
\begin{itemize}
	\item \textbf{Attori Primari}: azienda;
	\item \textbf{Attori Secondari}: MetaMask\glo;
	\item \textbf{Descrizione}: le aziende hanno la possibilità di gestire i beni venduti sulla piattaforma;
	\item \textbf{Scenario principale}: l'azienda accede alla pagina per la gestione dei prodotti/servizi venduti e può:
	\begin{itemize}
		\item inserire un nuovo prodotto/servizio da vendere [UC8.1];
		\item visualizzare i propri prodotti attualmente in vendita [UC8.2];
		\item modificare un prodotto/servizio già presente nella piattaforma [UC8.3];
		\item rimuovere un prodotto/servizio presente [UC8.4]; 
	\end{itemize}
	\item \textbf{Precondizione}: il sistema ha identificato l'utente come azienda, l'azienda ha espresso la volontà di gestire i propri prodotti/servizi;
	\item \textbf{Postcondizione}: il sistema fornisce all'azienda le operazioni che possono essere svolte sui propri prodotti.	
\end{itemize}

\subsubsection{UC8.1 - Inserimento prodotto}
\begin{figure}[H]
	\includegraphics[width=10cm]{res/images/UC8-Inserimento.png}
	\centering
	\caption{Inserimento prodotto}
\end{figure}
\begin{itemize}
	\item \textbf{Attori Primari}: azienda;
	\item \textbf{Attori Secondari}: MetaMask\glo;
	\item \textbf{Descrizione}: le aziende possono inserire dei nuovi prodotti nella piattaforma;
	\item \textbf{Scenario principale}: l'azienda accede alla pagina per inserire un nuovo prodotto e deve:
	\begin{itemize}
		\item inserire il nome del prodotto/servizio [UC8.1.1];
		\item inserire la descrizione del prodotto/servizio [UC8.1.2];
		\item inserire il prezzo unitario del prodotto/servizio [UC8.1.3];
		\item inserire la quantità disponibile del prodotto/servizio [UC8.1.4];
		\item confermare i dati inseriti [UC8.1.5];
	\end{itemize}
	\item \textbf{Precondizione}: il sistema ha identificato l'utente come azienda, l'azienda ha espresso la volontà di inserire un nuovo prodotto/servizio;
	\item \textbf{Postcondizione}: l'azienda ha inserito correttamente i dati relativi al nuovo prodotto ed è riuscita ad inserire il nuovo prodotto/servizio sulla piattaforma.	
\end{itemize}
\subsubsection{UC8.1.1 - Inserimento nome}
\begin{itemize}
	\item \textbf{Attori Primari}: azienda;
	\item \textbf{Descrizione}: al fine di portare a termine il processo di inserimento di un nuovo prodotto l'utente deve inserire il nome del prodotto, campo ritenuto obbligatorio;
	\item \textbf{Scenario}: l'utente compila il campo relativo al nome del nuovo prodotto/servizio da inserire;
	\item \textbf{Precondizione}: il sistema ha reso disponibile il form per l'inserimento di un nuovo prodotto/servizio, in particolare è presente il campo per l'inserimento del nome;
	\item \textbf{Postcondizione}: l'utente ha compilato il campo relativo al nome del nuovo prodotto da inserire.
\end{itemize}
\subsubsection{UC8.1.2 - Inserimento descrizione}
\begin{itemize}
	\item \textbf{Attori Primari}: azienda;
	\item \textbf{Descrizione}: al fine di portare a termine il processo di inserimento di un nuovo prodotto l'utente deve inserire la descrizione del prodotto, campo ritenuto obbligatorio;
	\item \textbf{Scenario}: l'utente compila il campo relativo alla descrizione del nuovo prodotto/servizio da inserire;
	\item \textbf{Precondizione}: il sistema ha reso disponibile il form per l'inserimento di un nuovo prodotto/servizio, in particolare è presente il campo per l'inserimento della descrizione;
	\item \textbf{Postcondizione}: l'utente ha compilato il campo relativo alla descrizione del nuovo prodotto da inserire.
\end{itemize}
\subsubsection{UC8.1.3 - Inserimento prezzo}
\begin{itemize}
	\item \textbf{Attori Primari}: azienda;
	\item \textbf{Descrizione}: al fine di portare a termine il processo di inserimento di un nuovo prodotto l'utente deve inserire il prezzo unitario del prodotto, campo ritenuto obbligatorio;
	\item \textbf{Scenario}: l'utente compila il campo relativo al prezzo del nuovo prodotto/servizio da inserire;
	\item \textbf{Precondizione}: il sistema ha reso disponibile il form per l'inserimento di un nuovo prodotto/servizio, in particolare è presente il campo per l'inserimento del prezzo;
	\item \textbf{Postcondizione}: l'utente ha compilato il campo relativo al prezzo del nuovo prodotto da inserire.
\end{itemize}
\subsubsection{UC8.1.4 - Inserimento quantità}
\begin{itemize}
	\item \textbf{Attori Primari}: azienda;
	\item \textbf{Descrizione}: al fine di portare a termine il processo di inserimento di un nuovo prodotto l'utente deve inserire il numero di pezzi disponibili del prodotto, campo ritenuto obbligatorio;
	\item \textbf{Scenario}: l'utente compila il campo relativo al numero di pezzi del nuovo prodotto/servizio da inserire;
	\item \textbf{Precondizione}: il sistema ha reso disponibile il form per l'inserimento di un nuovo prodotto/servizio, in particolare è presente il campo per l'inserimento alla quantità di pezzi;
	\item \textbf{Postcondizione}: l'utente ha compilato il campo relativo alla quantità di pezzi del nuovo prodotto da inserire.
\end{itemize}
\subsubsection{UC8.1.5 - Conferma dati}
\begin{itemize}
	\item \textbf{Attori Primari}: azienda;
	\item \textbf{Attori Primari}: MetaMask\glo;
	\item \textbf{Descrizione}: al fine di portare a termine il processo di registrazione l'utente deve confermare i dati inseriti tramite l'approvazione della transazione, che verrà eseguita attraverso il plugin MetaMask\glo;
	\item \textbf{Scenario}: l'utente preme il pulsante di conferma dei dati inseriti e valida la transazione con MetaMask\glo;
	\item \textbf{Estensioni}:
	\begin{itemize}
		\item \textbf{UC2.9}: l'utente tenta di confermare i dati senza aver compilato tutti i campi richiesti;
	\end{itemize}
	\item \textbf{Precondizione}: il sistema ha reso disponibile il form per l'inserimento dei dati riguardanti il nuovo prodotto, l'utente ha compilato tutti i campi ed ha premuto il pulsante per la conferma.
	\item \textbf{Postcondizione}: il nuovo prodotto è stato inserito nella piattaforma e l'utente ottiene la conferma della riuscita dell'operazione.
\end{itemize}

\subsubsection{UC8.2 - Visualizzazione prodotto}
\begin{figure}[h]
	\includegraphics[width=6cm]{res/images/UC8-Visualizzazione.png}
	\centering
	\caption{Inserimento prodotto}
\end{figure}
\begin{itemize}
	\item \textbf{Attori Primari}: azienda;
	\item \textbf{Descrizione}: le aziende possono visualizzare i loro prodotti inseriti nella piattaforma;
	\item \textbf{Scenario principale}: l'azienda accede alla pagina per visualizzare i loro prodotti. Per ogni prodotto potrà: 
	\begin{itemize}
		\item visualizzare il nome [UC8.2.1];
		\item visualizzare la descrizione [UC8.2.2];
		\item visualizzare il prezzo [UC8.2.3];
		\item visualizzare la quantità [UC8.2.4];
	\end{itemize}
	\item \textbf{Precondizione}: il sistema ha identificato l'utente come azienda, l'azienda ha espresso la volontà di visualizzare i propri prodotti inseriti nella piattaforma;
	\item \textbf{Postcondizione}: l'azienda ha ottenuto la lista dei propri prodotti assieme alle operazioni eseguibili su di essi.	
\end{itemize}
\subsubsection{UC8.2.1 - Visualizzazione nome}
\begin{itemize}
	\item \textbf{Attori Primari}: azienda;
	\item \textbf{Descrizione}: l'azienda visualizza il nome del bene o del servizio;
	\item \textbf{Scenario}: l'utente visualizza le informazioni relative ai propri prodotti/servizi;
	\item \textbf{Precondizione}: l'utente ha espresso la volontà di visualizzare i propri prodotti/servizi;
	\item \textbf{Postcondizione}: l'utente può visualizzare il nome del prodotto.
\end{itemize}
\subsubsection{UC8.2.2 - Visualizzazione quantità}
\begin{itemize}
	\item \textbf{Attori Primari}: azienda;
	\item \textbf{Descrizione}: l'azienda visualizza la quantità rimanente del bene o del servizio;
	\item \textbf{Scenario}: l'utente visualizza le informazioni relative ai propri prodotti/servizi;
	\item \textbf{Precondizione}: l'utente ha espresso la volontà di visualizzare i propri prodotti/servizi;
	\item \textbf{Postcondizione}: l'utente può visualizzare la quantità rimanente di ciascuno dei prodotti.
\end{itemize}
\subsubsection{UC8.2.3 - Visualizzazione prezzo}
\begin{itemize}
	\item \textbf{Attori Primari}: azienda;
	\item \textbf{Descrizione}: l'azienda visualizza il prezzo del bene o del servizio;
	\item \textbf{Scenario}: l'utente visualizza le informazioni relative ai propri prodotti/servizi;
	\item \textbf{Precondizione}: l'utente ha espresso la volontà di visualizzare i propri prodotti/servizi;
	\item \textbf{Postcondizione}: l'utente può visualizzare il prezzo del prodotto.
\end{itemize}
\subsubsection{UC8.2.4 - Visualizzazione descrizione}
\begin{itemize}
	\item \textbf{Attori Primari}: azienda;
	\item \textbf{Descrizione}: l'azienda visualizza la descrizione del bene o del servizio;
	\item \textbf{Scenario}: l'utente visualizza le informazioni relative ai propri prodotti/servizi;
	\item \textbf{Precondizione}: l'utente ha espresso la volontà di visualizzare i propri prodotti/servizi;
	\item \textbf{Postcondizione}: l'utente può visualizzare la descrizione del prodotto.
\end{itemize}

\subsubsection{UC8.3 - Modifica prodotto}
\begin{figure}[H]
	\includegraphics[width=10cm]{res/images/UC8-Modifica.png}
	\centering
	\caption{Inserimento prodotto}
\end{figure}
\begin{itemize}
	\item \textbf{Attori Primari}: azienda;
	\item \textbf{Attori Secondari}: MetaMask\glo;
	\item \textbf{Descrizione}: le aziende possono modificare i loro prodotti inseriti nella piattaforma;
	\item \textbf{Scenario principale}: l'azienda accede alla modifica di un prodotto attraverso l'apposito pulsante mostrato assieme alle informazioni riguardanti l'oggetto stesso[UC8.2]. Per ogni prodotto potrà: 
	\begin{itemize}
		\item modificarne il nome [UC8.3.1];
		\item modificarne la descrizione [UC8.3.2];
		\item modificarne il prezzo [UC8.3.3];
		\item modificarne la quantità [UC8.3.4];
	\end{itemize}
	Dunque dovrà confermare l'operazione attraverso l'utilizzo di MetaMask\glo.
	\item \textbf{Inclusione}:
	\begin{itemize}
		\item \textbf{UC8.2}: per poter modificare i prodotti l'azienda deve prima individuarli nella lista dei propri prodotti;
	\end{itemize}
	\item \textbf{Precondizione}: il sistema ha identificato l'utente come azienda, l'azienda ha espresso la volontà modificare un prodotto;
	\item \textbf{Postcondizione}: l'azienda ha modificato le caratteristiche del prodotto ed ha ottenuto un messaggio che conferma il successo dell'operazione.	
\end{itemize}

\subsubsection{UC8.3.1 - Modifica nome}
\begin{itemize}
	\item \textbf{Attori Primari}: azienda;
	\item \textbf{Descrizione}: l'azienda inserisce nell'apposito campo il nuovo nome del prodotto/servizio;
	\item \textbf{Scenario}: l'utente visualizza il vecchio nome con affianco un campo dati per inserire il nuovo nome, ed inserisce il nuovo valore;
	\item \textbf{Precondizione}: l'utente ha espresso la volontà di modificare i dati di un prodotto/servizio;
	\item \textbf{Postcondizione}: l'utente ha inserito il nuovo nome nel campo dedicato del form.
\end{itemize}

\subsubsection{UC8.3.2 - Modifica descrizione}
\begin{itemize}
	\item \textbf{Attori Primari}: azienda;
	\item \textbf{Descrizione}: l'azienda inserisce nell'apposito campo la nuova descrizione del prodotto/servizio;
	\item \textbf{Scenario}: l'utente visualizza la vecchia descrizione con affianco un campo dati per inserire la nuova descrizione, ed inserisce il nuovo valore;
	\item \textbf{Precondizione}: l'utente ha espresso la volontà di modificare i dati di un prodotto/servizio;
	\item \textbf{Postcondizione}: l'utente ha inserito la nuova descrizione nel campo dedicato del form.
\end{itemize}

\subsubsection{UC8.3.3 - Modifica prezzo}
\begin{itemize}
	\item \textbf{Attori Primari}: azienda;
	\item \textbf{Descrizione}: l'azienda inserisce nell'apposito campo il nuovo prezzo del prodotto/servizio;
	\item \textbf{Scenario}: l'utente visualizza il vecchio prezzo con affianco un campo dati per inserire il nuovo prezzo, ed inserisce il nuovo valore;
	\item \textbf{Precondizione}: l'utente ha espresso la volontà di modificare i dati di un prodotto/servizio;
	\item \textbf{Postcondizione}: l'utente ha inserito il nuovo prezzo nel campo dedicato del form.
\end{itemize}

\subsubsection{UC8.3.4 - Modifica quantità}
\begin{itemize}
	\item \textbf{Attori Primari}: azienda;
	\item \textbf{Descrizione}: l'azienda inserisce nell'apposito campo la nuova quantità del prodotto/servizio;
	\item \textbf{Scenario}: l'utente visualizza la vecchia quantità con affianco un campo dati per inserire la nuova quantità, ed inserisce il nuovo valore;
	\item \textbf{Precondizione}: l'utente ha espresso la volontà di modificare i dati di un prodotto/servizio;
	\item \textbf{Postcondizione}: l'utente ha inserito la nuova quantità nel campo dedicato del form.
\end{itemize}

\subsubsection{UC8.3.5 - Conferma modifiche}
\begin{itemize}
	\item \textbf{Attori Primari}: azienda;
	\item \textbf{Attori Primari}: MetaMask\glo;
	\item \textbf{Descrizione}: al fine di portare a termine il processo di modifica dei dati di un prodotto/servizio, l'utente deve confermare i dati inseriti tramite l'approvazione della transazione, che verrà eseguita attraverso il plugin MetaMask\glo;
	\item \textbf{Scenario}: l'utente preme il pulsante di conferma dei dati inseriti e valida l'operazione con MetaMask\glo;
	\item \textbf{Estensioni}:
	\begin{itemize}
		\item \textbf{UC8.3.6}: l'utente tenta di confermare i dati senza aver compilato almeno uno dei campi;
	\end{itemize}
	\item \textbf{Precondizione}: il sistema ha reso disponibile il form per la modifica dei dati riguardanti un prodotto. L'utente ha compilato almeno un campo ed ha premuto il pulsante per la conferma.
	\item \textbf{Postcondizione}: il prodotto è stato aggiornato con i nuovi dati e l'utente ottiene la conferma della riuscita dell'operazione.
\end{itemize}

\subsubsection{UC8.3.6 - Visualizzazione errore almeno una modifica}
\begin{itemize}
	\item \textbf{Attori Primari}: azienda;
	\item \textbf{Descrizione}:
	l'utente visualizza un messaggio di errore relativo al fatto nessuno dei campi per la modifica è stato compilato, e che quindi non è possibile attuare alcuna modifica;
	\item \textbf{Scenario}: l'utente tenta di confermare ed inviare le modifiche ai dati senza aver compilato almeno uno dei campi del form;
	\item \textbf{Precondizione}: il sistema permette all'utente di compilare il form per le modifiche. L'utente ha premuto il pulsante di conferma senza aver modificato almeno uno dei campi; 
	\item \textbf{Postcondizione}:
	l'utente è consapevole che per effettuare una modifica, almeno uno dei dati presenti deve essere stato modificato, compilando almeno uno dei campi del form.
\end{itemize}

\subsubsection{UC8.4 - Eliminazione prodotto}
\begin{itemize}
	\item \textbf{Attori Primari}: azienda;
	\item \textbf{Attori Primari}: MetaMask\glo;
	\item \textbf{Descrizione}:
	l'utente elimina uno dei propri prodotti/servizi presenti sulla piattaforma;
	\item \textbf{Scenario}: l'utente clicca sul pulsante di eliminazione prodotto, mostrato assieme alle informazioni riguardanti l'oggetto stesso [UC8.2]. Dunque dovrà confermare l'operazione attraverso l'utilizzo di MetaMask\glo;
	\item \textbf{Inclusione}:
	\begin{itemize}
		\item \textbf{UC8.2}: per poter eliminare i prodotti l'azienda deve prima individuarli nella lista dei propri prodotti;
	\end{itemize}
	\item \textbf{Precondizione}: il sistema ha identificato l'utente come azienda, l'azienda ha espresso la volontà eliminare un prodotto;
	\item \textbf{Postcondizione}: l'azienda ha eliminato il prodotto ed ha ottenuto un messaggio che conferma il successo dell'operazione.
\end{itemize}



\pagebreak
\subsubsection*{Operazioni governative}
Di seguito sono riportati tutti i casi d'uso che coinvolgono come attore primario il governo.

\begin{figure}[H]
	\includegraphics[width=8cm]{res/images/UseCaseGoverno.png}
	\centering
	\caption{Use cases che interessano il governo}
\end{figure}
\subsubsection{UC9 - Coniazione Cubit}
\begin{itemize}
	\item \textbf{Attori Primari}: governo;
	\item \textbf{Attori Secondari}: MetaMask\glo;
	\item \textbf{Descrizione}: viene coniata una quantità definita di Cubit\glo;
	
	\item \textbf{Scenario principale}: il governo ritiene necessario coniare ulteriori Cubit rispetto a quelli attualmente presenti sul mercato. Per fare ciò deve accedere alla pagina dedicata in cui, attraverso un form dovrà:
	 \begin{enumerate}[label=\alph*.]
		\item inserire la quantità $x$ di Cubit da coniare;
		\item confermare tale operazione attraverso l'utilizzo di MetaMask\glo.
	\end{enumerate}
	\item \textbf{Precondizione}: sia $x$ l'ammontare di Cubit che il governo 
	vuole coniare e $n$ l'ammontare di Cubit attualmente in circolo. L'utente 
	governativo ha acceduto alla pagina per la coniazione ed ha compilato e 
	confermato il form;
	\item \textbf{Postcondizione}: i Cubit in circolo sono $x+n$.
\end{itemize}
\subsubsection{UC10 - Distribuzione Cubit}
%non capisco perchè la figura è prima del titolo sul PDF
\begin{figure}[h]
	\includegraphics[width=13.5cm]{res/images/UC10Distribuzione.png} %da adattare in larghezza
	\centering
	\caption{UC10 - Distribuzione Cubit}
	
\end{figure}
\begin{itemize}
	\item \textbf{Attori Primari}: governo;
	\item \textbf{Attori Secondari}: MetaMask\glo, cittadino, azienda\glo;
	\item \textbf{Descrizione}: il governo trasferisce una somma di Cubit\glosp sull'account di uno o più  utenti, che siano essi cittadini o aziende;
	\item \textbf{Scenario principale}: il governo deve:
	 \begin{enumerate}[label=\alph*.]
		\item determinare l'ammontare di Cubit pro capite da trasferire [UC10.1];
		\item  selezionare la lista dei destinatari del trasferimento dalla lista degli utenti [UC10.2];
		\item confermare l'operazione [UC10.3].
	\end{enumerate}

	\item \textbf{Precondizione}: il governo accede alla pagina contenente il form per la distribuzione di Cubit agli utenti e compila il suddetto form;
	\item \textbf{Postcondizione}: tali utenti ricevono i Cubit da parte del governo.
\end{itemize}
\subsubsection{UC10.1 - Inserimento quantità}
\begin{itemize}
	\item \textbf{Attori Primari}: governo;
	\item \textbf{Descrizione}: il governo inserisce nell'apposito campo del form la quantità di Cubit\glosp da distribuire, ritenendo la quantità inserita come quantità da inviare ad ogni singolo utente;
	\item \textbf{Scenario principale}: il governo inserisce l'ammontare $x$ di 
	Cubit da inviare;
	\item \textbf{Precondizione}: il governo si trova alla pagina contenente il form per la distribuzione di Cubit agli utenti;
	\item \textbf{Postcondizione}: il campo relativo all'ammontare di Cubit da 
	distribuire pro capite è stato compilato. 
\end{itemize}
\subsubsection{UC10.2 - Inserimento destinatari}
\begin{itemize}
	\item \textbf{Attori Primari}: governo;
	\item \textbf{Attori Secondari}: cittadino, azienda;
	\item \textbf{Descrizione}: il governo inserisce nell'apposito form la lista dei destinatari della quantità di Cubit\glosp precedentemente inserita;
	\item \textbf{Scenario principale}: il governo:
	\begin{enumerate}[label=\alph*.]
		\item visualizza delle liste contenenti i cittadini e le aziende registrate al sito, che sono accompagnati da una checkbox;
		\item l'utente può usare una barra di ricerca per individuare dei particolari utenti;
		\item l'utente governativo spunta le caselle relative agli utenti ai quali intende trasferire la quantità di Cubit precedentemente definita.
	\end{enumerate}
	\item \textbf{Precondizione}: il governo si trova nella pagina atta alla distribuzione dei Cubit e ha selezionato la quantità di Cubit da distribuire;
	\item \textbf{Postcondizione}: il governo ha selezionato uno o più destinatari dalle liste degli utenti, e può procedere con l'approvazione della transazione.
\end{itemize}
\subsubsection{UC10.3 - Approvazione transazione}
\begin{itemize}
	\item \textbf{Attori Primari}: governo;
	\item \textbf{Attori Secondari}: MetaMask\glo, cittadino, azienda\glo;
	\item \textbf{Descrizione}: il governo decide se approvare o rifiutare la transazione;
	\item \textbf{Scenario principale}: il governo ha specificato la somma 
	$x$ di Cubit\glosp da distribuire ad ognuno degli $y$ destinatari;
	\item \textbf{Estensioni}:
	\begin{itemize}
		\item \textbf{UC10.4}: verrà mostrato un messaggio di errore nel caso 
		l'ammontare totale $x\cdot y$ superi la quantità di Cubit disponibili 
		alla distribuzione.
	\end{itemize}
	\item \textbf{Precondizione}: è necessario aver selezionato una lista di destinatari ed una quantità da distribuire ad ogni singolo utente;
	\item \textbf{Postcondizione}: ogni utente riceverà nel proprio wallet\glosp la quantità di Cubit trasferita dal governo.
\end{itemize}
\subsubsection{UC10.4 - Visualizzazione messaggio di errore relativo al raggiungimento della soglia massima di Cubit distribuiti}
\begin{itemize}
	\item \textbf{Attori Primari}: governo;
	\item \textbf{Descrizione}: il governo riceve un messaggio di errore relativo al fatto che ha selezionato una quantità di Cubit\glosp superiore a quella disponibile alla distribuzione;
	\item \textbf{Scenario principale}: il governo clicca sull'apposito pulsante per confermare la transazione di distribuzione, ma non ha sufficienti fondi per completare tale operazione;
	\item \textbf{Precondizione}: il governo deve aver inserito una quantità di Cubit superiore a quella disponibile e cerca di effettuare la transazione;
	\item \textbf{Postcondizione}: viene visualizzato un messaggio d'errore per informare l'utente del fatto che attualmente non dispone dei fondi necessari per effettuare l'operazione di distribuzione.
	
\end{itemize} 

 \subsubsection{UC11 - Visualizzazione delle informazioni utente}
 \begin{figure}[h]
 	\includegraphics[width=9cm]{res/images/UC11.png}
 	\centering
 	\caption{UC11 - Visualizzazione delle informazioni utente}
 	
 \end{figure}
 \begin{itemize}
 	\item \textbf{Attori Primari}: governo;
 	\item \textbf{Descrizione}: il governo può accedere alle informazioni riguardanti le aziende ed i cittadini che hanno avanzato una richiesta di iscrizione, o che sono già iscritti alla piattaforma. Offre inoltre la possibilità di eseguire delle operazioni su di essi;
 	\item \textbf{Scenario principale}: il governo cerca delle informazioni riguardanti gli utenti della piattaforma e/o esegue delle operazioni su di essi;
 	
 	\item \textbf{Precondizione}: il sistema riconosce che l'utente è autenticato con privilegi governativi e mostra le pagine utili alla ricerca di informazioni sugli utenti;
 	
 	\item \textbf{Postcondizione}: il governo ottiene dal sistema la lista degli utenti dei quali cercava delle informazioni, assieme al set di operazioni che possono essere effettuate su quella tipologia di utente.
 \end{itemize}
 \subsubsection{UC11.1 - Visualizzazione lista aziende}
 \begin{itemize}
 	\item \textbf{Attori Primari}: governo;
 	\item \textbf{Descrizione}: il governo accede ad una lista di aziende, nella quale vengono mostrate tutte le informazioni utili ed operazioni disponibili. In particolare avviene la:
 	\begin{itemize}
 		\item visualizzazione della chiave\glosp Ethereum;
 		\item visualizzazione del nome;
 		\item visualizzazione della partita IVA;
 		\item visualizzazione dell'indirizzo di residenza;
 	\end{itemize}
 	\item \textbf{Scenario principale}: il governo può voler ottenere la lista di tutte le aziende:
 	\begin{enumerate}[label=\alph*.]
 		\item iscritte alla piattaforma con le relative informazioni riguardanti l'IVA [UC11.2];
 		\item che attualmente hanno fatto richiesta di registrazione alla piattaforma, ed eventualmente gestirne la richiesta [UC11.3];
 	\end{enumerate}
	\item \textbf{Specializzazione}:
	\begin{itemize}
	 	\item \textbf{UC11.2}: il governo vuole ottenere le informazioni relative alle aziende che sono già registrate alla piattaforma. In questo caso saranno presenti anche delle informazioni riguardati la situazione IVA delle aziende;
	 	\item \textbf{UC11.3}: il governo vuole ottenere le informazioni relative alle aziende che sono in attesa di essere verificate, per poter poi usufruire della piattaforma. Da questa vista verranno offerte al governo le operazioni di approvazioni/rigetto della richiesta;
	\end{itemize}
 	\item \textbf{Precondizione}: il sistema riconosce che l'utente è autenticato con privilegi governativi ed ha richiesto le informazioni relative alle aziende;
 	\item \textbf{Postcondizione}: il governo ottiene dal sistema la lista delle aziende cercate con le relative operazioni disponibili.
\end{itemize}
\subsubsection{UC11.2 - Visualizzazione lista aziende verificate}
 \begin{itemize}
	\item \textbf{Attori Primari}: governo;
	\item \textbf{Descrizione}: il governo ottiene la lista delle aziende registrate alla piattaforma e già verificate. Per ogni azienda ottiene, oltre alle informazioni di base [UC11.1], lo stato IVA dell'azienda, ovvero può sapere se l'azienda si trova in stato di credito o di debito con lo stato;
	\item \textbf{Scenario principale}: il governo richiede la lista delle aziende registrate e  già verificate;
	\item \textbf{Precondizione}: il sistema riconosce che l'utente è autenticato con privilegi governativi ed ha richiesto di ottenere la lista di tutte aziende già verificate;
	\item \textbf{Postcondizione}: il governo ottiene dal sistema la lista delle aziende registrate e verificate, con associate le operazioni che può effettuare su di esse.
\end{itemize}
\subsubsection{UC11.3 - Visualizzazione lista aziende in attesa di verifica}

\begin{itemize}
	\item \textbf{Attori Primari}: governo;
	\item \textbf{Descrizione}: il governo ottiene la lista delle aziende che sono in attesa di essere verificate per poter poi usufruire della piattaforma. Per ognuna di esse avrà la possibilità di accettare o rifiutare la richiesta;
	\item \textbf{Scenario principale}: il governo richiede la lista delle aziende che sono in attesa di essere verificate;
	\item \textbf{Precondizione}: il sistema riconosce che l'utente è autenticato con privilegi governativi ed ha richiesto di ottenere la lista di tutte le aziende in attesa di verifica;
	\item \textbf{Postcondizione}: il governo ottiene dal sistema la lista delle aziende in attesa di verifica, con associate le operazioni che può effettuare su di esse.
\end{itemize}
\subsubsection{UC11.4 - Visualizzazione lista cittadini in attesa di verifica}
\begin{itemize}
	\item \textbf{Attori Primari}: governo;
	\item \textbf{Descrizione}: il governo ottiene la lista dei cittadini che sono in attesa di essere verificati per poter poi usufruire della piattaforma. Per ognuno di essi può:
	\begin{itemize}
		\item visualizzazione della chiave\glosp Ethereum;
		\item visualizzazione del nome;
		\item visualizzazione del cognome;
		\item visualizzazione della codice fiscale;
		\item visualizzazione dell'indirizzo di residenza;
	\end{itemize}
	\item \textbf{Scenario principale}: il governo richiede la lista dei cittadini che sono in attesa di essere verificati. Per ognuno di essi visualizza alcune informazioni, assieme alla possibilità di accettare o rifiutare la proposta di iscrizione;
	\item \textbf{Precondizione}: il sistema riconosce che l'utente è autenticato con privilegi governativi ed ha richiesto di ottenere la lista di tutti i cittadini in attesa di verifica;
	\item \textbf{Postcondizione}: il governo ottiene dal sistema la lista dei cittadini in attesa di verifica, con associate le operazioni che può effettuare su di essi.
\end{itemize}

\subsubsection{UC12 - Abilitazione/disabilitazione account}
\begin{figure}[h]
	\includegraphics[width=8cm]{res/images/UC12-Generale.png}
	\centering
	\caption{UC12 - Abilitazione/disabilitazione account}
\end{figure}
\begin{itemize}
	\item \textbf{Attori Primari}:
	governo;
	\item \textbf{Attori Secondari}:
	azienda, cittadino;
	\item \textbf{Descrizione}: il governo può gestire gli account degli utenti registrati alla piattaforma, in particolare può:
	\begin{itemize}
		\item abilitare un account;
		\item disabilitare un account, lasciando un messaggio relativo alla causa di tale provvedimento.
	\end{itemize}
	\item \textbf{Scenario principale}: il governo visualizza la lista degli utenti, aziende [UC11.1] o cittadini [UC11.2]. Per ognuna di esse ha la possibilità di abilitare l'account o disabilitarlo premendo l'apposito pulsante;
	\item \textbf{Precondizione}: l'utente governativo sta visualizzando una lista di utenti registrati al sistema, e può quindi accedere al pulsante per l'abilitazione o disabilitazione dell'account di un cittadino;
	\item \textbf{Postcondizione}: l'account dell'utente selezionato è stato abilitato o disabilitato. Nel caso sia stato disabilitato, durante i prossimi tentativi di autenticazione, tale utente riceverà l'errore di "account disabilitato" [UC3.3].
\end{itemize} 

\subsubsection{UC12.1 - Abilitazione account}
\begin{itemize}
	\item \textbf{Attori Primari}:
	governo;
	\item \textbf{Attori Secondari}:
	azienda, cittadino;
	\item \textbf{Descrizione}: il governo può abilitare l'account di un utente;
	\item \textbf{Scenario principale}: il governo visualizza la lista degli utenti, aziende [UC11.1] o cittadini [UC11.2]. Per ogni utente il cui stato dell'account risulta "disabilitato", può effettuare l'abilitazione premendo sul pulsante dedicato;

	\item \textbf{Precondizione}: l'utente governativo sta visualizzando una lista di utenti registrati al sistema, preme il pulsante di abilitazione dell'account relativo ad un utente il quale stato dell'account risulta "disabilitato";
	\item \textbf{Postcondizione}: lo stato  dell'account utente sopra menzionato risulta "abilitato".
\end{itemize} 


\subsubsection{UC12.2 - Disabilitazione account}
\begin{itemize}
	\item \textbf{Attori Primari}:
	governo;
	\item \textbf{Attori Secondari}:
	azienda, cittadino;
	\item \textbf{Descrizione}: il governo può disabilitare l'account di un utente;
	\item \textbf{Scenario principale}: il governo visualizza la lista degli utenti, aziende [UC11.1] o cittadini [UC11.2]. Per ogni utente il cui stato dell'account risulta "abilitato", può effettuare la disabilitazione premendo sul pulsante dedicato;
	\item \textbf{Inclusioni}: 
	\begin{itemize}
		\item \textbf{UC12.3}: durante la disabilitazione l'utente governativo è richiesto di inserire un messaggio personalizzato per spiegare all'utente il motivo della disabilitazione dell'account.
	\end{itemize}
	\item \textbf{Precondizione}: l'utente governativo sta visualizzando una lista di utenti registrati al sistema, preme il pulsante di disabilitazione dell'account relativo ad un utente il quale stato dell'account risulta "abilitato";
	\item \textbf{Postcondizione}: lo stato  dell'account utente sopra menzionato risulta "disabilitato". Durante i prossimi tentativi di autenticazione, tale utente riceverà l'errore di "account disabilitato" [UC3.3].
\end{itemize} 

\subsubsection{UC12.3 - Scrittura messaggio di disabilitazione}
\begin{itemize}
	\item \textbf{Attori Primari}:
	governo;
	\item \textbf{Attori Secondari}:
	azienda, cittadino;
	\item \textbf{Descrizione}: il governo può inserire un messaggio personalizzato all'utente al quale sta disabilitando l'account riferendo la causa di tale disabilitazione;
	\item \textbf{Scenario principale}: il governo visualizza una lista di utenti registrati [UC11]:
	\begin{enumerate}[label=\alph*.]
		\item l'utente governativo disabilita l'account di un utente il cui stato dell'account risulta "abilitato" [UC12.2];
		\item il sistema propone una casella di testo nella quale l'utente governativo può inserire la causa della disabilitazione dell'account;
		\item l'utente governativo conferma e salva tale messaggio. Durante i prossimi tentativi di autenticazione, l'utente al quale è stato disabilitato l'account, riceverà l'errore di "account disabilitato" [UC3.3].
	\end{enumerate}
	 
	\item \textbf{Precondizione}: l'utente governativo ha richiesto la disabilitazione di un account utente premendo sull'apposito pulsante, il sistema visualizza la casella di testo per inserire il messaggio di errore contenente la causa di tale azione;
	\item \textbf{Postcondizione}: l'utente governativo ha inserito un messaggio contenente la causa della disabilitazione dell'account ed ha salvato e confermato tale messaggio.
\end{itemize}


\subsubsection{UC13 - Rimborso IVA}
\begin{itemize}
	\item \textbf{Attori Primari}:
	governo;
	\item \textbf{Attori Secondari}:
	azienda, MetaMask\glo;
	\item \textbf{Descrizione}: il governo effettua il rimborso IVA ad un'azienda che risulta in stato di credito al momento del saldo trimestrale;
	\item \textbf{Scenario principale}: dopo aver visualizzato lo stato IVA di un'azienda [UC11.1] il governo decide di effettuare il rimborso cliccando sul pulsante dedicato. Dovrà confermare la transazione attraverso il plugin MetaMask\glo;
	\item \textbf{Inclusioni}: 
	\begin{itemize}
		\item \textbf{UC11.2}: l'utente per poter accedere al pulsante per il rimborso dell'azienda deve visualizzare la lista delle aziende verificate.
	\end{itemize}
	\item \textbf{Precondizione}: il sistema ha mostrato la lista delle aziende già verificate, con il relativo stato IVA. L'utente ha premuto il pulsante per il rimborso e confermato l'operazione;
	\item \textbf{Postcondizione}: il sistema avvisa l'utente che l'operazione è avvenuta con successo. Il rimborso è stato versato all'azienda.
\end{itemize} 

\subsubsection*{Operazioni di gestione degli ordini e delle fatture}
Di seguito sono riportate tutti i casi d'uso che coinvolgono la gestione degli ordine e delle fatture.

\begin{figure}[H]
	\includegraphics[width=8cm]{res/images/UseCaseFinali.png}
	\centering
	\caption{Use Cases riguardanti la gestione degli ordini e delle fatture}
\end{figure}
\subsubsection{UC14 - Gestione ordini}
\begin{figure}[h]
	\includegraphics[width=10cm]{res/images/UC14-GestioneOrdini.png}
	\centering
	\caption{Gestione degli ordini}
\end{figure}
\begin{itemize}
	\item \textbf{Attori Primari}: azienda, cittadino;
	\item \textbf{Descrizione}: agli utenti sono messe a disposizione diverse operazione per visualizzare e gestire gli ordini;
	\item \textbf{Scenario principale}: l'utente visualizza e svolge alcune operazioni per gestire gli ordini dei quali ne è partecipe;
	\item \textbf{Precondizione}: il sistema ha riconosciuto l'utente autenticato come azienda o cittadino e mette a disposizione tutte le pagine necessarie alla visualizzazione e gestione degli ordini;
	\item \textbf{Postcondizione}: l'utente ha visualizzato e/o gestito i propri ordini.
\end{itemize} 
\subsubsection{UC14.1 - Visualizzazione Ordini}
\begin{itemize}
	\item \textbf{Attori Primari}: azienda, cittadino;
	\item \textbf{Descrizione}: alle aziende ed ai cittadini sono messe a disposizione diverse operazione per visualizzare e gestire gli ordini all'interno della piattaforma. In tale maniera vi è possibile avere un elenco dettagliato di tutti gli ordini, in particolare:
	\begin{itemize}
		\item visualizzazione data dell'ordine;
		\item visualizzazione numero dell'ordine;
		\item visualizzazione prezzo netto;
		\item visualizzazione prodotti inclusi nell'ordine;
		\item visualizzazione totale IVA;
		\item visualizzazione prezzo lordo;
	\end{itemize}
	\item \textbf{Scenario principale}: l'utente visualizza e svolge alcune operazioni per gestiregli ordini;
	\item \textbf{Precondizione}: il sistema ha riconosciuto l'utente autenticato come azienda o cittadino e mette a disposizione tutte le pagine necessarie alla visualizzazione e gestione degli ordini;
	\item \textbf{Postcondizione}: l'utente ha visualizzato e/o gestito i propri ordini.
\end{itemize} 

\subsubsection{UC14.2 - Visualizzazione vendite in attesa di approvazione}
\begin{itemize}
	\item \textbf{Attori Primari}: azienda;
	\item \textbf{Descrizione}: all'azienda è messa a disposizione la possibilità di visualizzare e gestire le vendite in attesa di approvazione;
	\item \textbf{Scenario principale}: l'utente visualizza le vendite in attesa di approvazione tramite una vista dettagliata che include: 
		\begin{itemize}
		\item visualizzazione data ultima per la conferma;
	\end{itemize}
		\item \textbf{Inclusioni}:
	\begin{itemize}
		\item \textbf{UC14.1}: Visualizzazione degli ordini;
	\end{itemize}
	\item \textbf{Precondizione}: il sistema ha riconosciuto l'utente autenticato come azienda e mette a disposizione tutte le pagine necessarie alla visualizzazione di questo tipo di ordini;
	\item \textbf{Postcondizione}: l'utente ha visualizzato e/o gestito le rpoprie vendite in attesa di approvazione.
\end{itemize} 

\subsubsection{UC14.3 - Visualizzazione acquisti conclusi}
\begin{itemize}
	\item \textbf{Attori Primari}: cittadino, azienda;
	\item \textbf{Descrizione}: l'utente può visualizzare la lista degli acquisti conclusi;
	\item \textbf{Scenario principale}: l'utente visualizza la lista degli acquisti conclusi;
	\item \textbf{Inclusioni}:
	\begin{itemize}
		\item \textbf{UC14.1}: Visualizzazione degli ordini;
	\end{itemize}
	\item \textbf{Precondizione}: il sistema ha riconosciuto l'utente autenticato come azienda o cittadino e questo ha espresso la volontà di visualizzare la lista degli acquisti conclusi;
	\item \textbf{Postcondizione}: l'utente visualizza tale lista.
\end{itemize}

\subsubsection{UC14.4 - Visualizzazione vendite concluse}
\begin{itemize}
	\item \textbf{Attori Primari}: azienda;
	\item \textbf{Descrizione}: l'utente può visualizzare la lista delle proprie vendite ritenute concluse;
	\item \textbf{Scenario principale}: l'utente visualizza la lista delle vendite concluse;
	\item \textbf{Inclusioni}:
	\begin{itemize}
		\item \textbf{UC14.1}: Visualizzazione degli ordini;
	\end{itemize}
	\item \textbf{Precondizione}: il sistema ha riconosciuto l'utente autenticato come azienda e questo ha espresso la volontà di visualizzare la lista delle vendite concluse;
	\item \textbf{Postcondizione}: l'utente visualizza tale lista.
\end{itemize}


\subsubsection{UC14.5 - Rifiuto di una proposta di acquisto}
\begin{itemize}
	\item \textbf{Attori Primari}: azienda;
	\item \textbf{Attori Secondari}: MetaMask\glo, azienda;
	\item \textbf{Descrizione}: l'azienda può rifiutare una proposta d'acquisto. Per confermare l'operazione verrà utilizzato MetaMask\glo;
	\item \textbf{Scenario principale}: l'utente visualizza una proposta d'acquisto che necessitano di conferma e decide di rifiutarla;
	\item \textbf{Inclusioni}: 
	\begin{itemize}
		\item \textbf{UC14.7}: visualizzazione delle proposte d'acquisto in attesa di approvazione;
	\end{itemize}
	\item \textbf{Precondizione}: il sistema ha riconosciuto l'utente autenticato come azienda ed ha mostrato la lista delle proposte d'acquisto che necessitano di conferma;
	\item \textbf{Postcondizione}: l'azienda ha rifiutato la proposta d'acquisto. La proposta non sarà più presente nella lista di attesa per la conferma. L'acquisto comparirà come acquisto concluso, con esito negativo, nella lista delle vendite dell'azienda-venditrice [UC14.4]. Il sistema ritorna l'ammontare trattenuto per l'ordine all'azienda-cliente.
\end{itemize}
\subsubsection{UC14.6 - Conferma di una proposta di acquisto}
\begin{itemize}
	\item \textbf{Attori Primari}: azienda;
	\item \textbf{Attori Secondari}: MetaMask\glo, azienda;
	\item \textbf{Descrizione}: l'azienda accetta la proposta d'acquisto ricevuta;
	\item \textbf{Scenario principale}: l'azienda decide di approvare una proposta d'acquisto da parte di un acquirente. L'utente acquirente ha la possibilità di effettuare il versamento all'azienda-venditrice;
		\item \textbf{Inclusioni}: 
	\begin{itemize}
		\item \textbf{UC14.7}: visualizzazione delle proposte d'acquisto in attesa di approvazione;
	\end{itemize}
	\item \textbf{Precondizione}: il sistema ha riconosciuto l'utente autenticato come azienda, e questo ha espresso la volontà di approvare una richiesta d'acquisto;
	\item \textbf{Postcondizione}: l'azienda approva tale proposta.
\end{itemize}


\subsubsection{UC14.7 - Visualizzazione proposte d'acquisto in attesa di approvazione}
\begin{itemize}
	\item \textbf{Attori Primari}: azienda;
	\item \textbf{Descrizione}: l'azienda può visualizzare delle proposte d'acquisto che necessitano di conferma;
	\item \textbf{Scenario principale}: l'utente visualizza la lista delle proposte d'acquisto che necessitano di conferma. Per ognuna di esse ha a possibilità di confermare la proposta [UC14.6] o di rifiutarla [UC14.5];
		\item \textbf{Inclusioni}:
	\begin{itemize}
		\item \textbf{UC14.1}: Visualizzazione degli ordini;
	\end{itemize}
	\item \textbf{Precondizione}: il sistema ha riconosciuto l'utente autenticato come azienda e questo ha espresso la volontà di visualizzare la lista delle proposte d'acquisto che necessitano di conferma;
	\item \textbf{Postcondizione}: l'azienda visualizza tale lista.
\end{itemize}






\subsubsection{UC15 - Gestione transazioni ed IVA}
\begin{figure}[H]
	\includegraphics[width=8cm]{res/images/UC15-Generale.png}
	\centering
	\caption{UC15 - Operazioni riguardanti la gestione dell'IVA}
\end{figure}
\begin{itemize}
	\item \textbf{Attori Primari}: azienda;
	\item \textbf{Descrizione}: alle aziende sono messe a disposizione diverse operazioni per gestire le transazioni e l'IVA:
	\begin{itemize}
		\item può visulizzare i saldi IVA di un periodo, comprendendo la visualizzazione di tutte le transazioni, rappresentate dalla rispettiva fattura, e del saldo finale [15.3 \& 15.4];
		\item per ogni fattura è disponibile la visualizzazione dettagliata [15.8];
		\item può scaricare le informazioni riguardanti un trimestre come sopra descritto sotto forma di documento PDF [15.5]; 
		\item in caso il saldo di un semestre risulti negativo (situazione di debito), l'azienda può procedere ad effettuare il pagamento al governo [15.6] o può dilazionarlo\glosp [15.7];
	\end{itemize}
	\item \textbf{Scenario principale}: l'utente visualizza e svolge alcune operazioni per gestire l'IVA, gli ordini e le fatture;
	\item \textbf{Precondizione}: il sistema ha riconosciuto l'utente autenticato come azienda, e mette a disposizione tutte le pagine necessarie alla visualizzazione e gestione delle fatture e dell'IVA;
	\item \textbf{Postcondizione}: l'azienda ha visualizzato e/o svolto delle operazioni riguardanti fatture ed IVA.
\end{itemize} 
\subsubsection{UC15.1 - Visualizzazione saldo IVA}
\begin{itemize}
	\item \textbf{Attori Primari}: azienda;
	\item \textbf{Descrizione}: l'azienda può visualizzare un saldo parziale relativo ad un trimestre IVA. In particolare può visualizzare il saldo:
	\begin{itemize}
		\item del trimestre corrente [15.3];
		\item di un trimestre concluso [15.4];
	\end{itemize}
	\item \textbf{Scenario principale}: l'utente visualizza il saldo parziale relativo ad un trimestre. In particolare:
	\begin{enumerate}[label=\alph*.]
		\item seleziona da un menù a tendina uno dei semestri disponibili;
		\item di questo semestre scelto viene visualizzata la relativa lista delle fatture; 
	\end{enumerate}
	\item \textbf{Inclusioni}: 
	\begin{itemize}
		\item \textbf{UC15.2}: visualizzazione generale delle fatture, ovvero il mostrare la lista delle fatture nel saldo;
	\end{itemize}
	\item \textbf{Specializzazioni}: 
	\begin{itemize}
		\item \textbf{UC15.3}: visualizzazione saldo IVA corrente;
		\item \textbf{UC15.4}:  visualizzazione saldo IVA di un trimestre concluso;
	\end{itemize}
	\item \textbf{Precondizione}: il sistema ha riconosciuto l'utente autenticato come azienda, e mette a disposizione le pagine per visualizzazione dei saldi dei semestri IVA;
	\item \textbf{Postcondizione}: l'azienda ha visualizzato il saldo IVA parziale riguardante il trimestre scelto ed è a conoscenza dalla situazione di debito o credito verso il governo.
\end{itemize} 
\subsubsection{UC15.2 - Visualizzazione generale fattura}
\begin{itemize}
	\item \textbf{Attori Primari}: azienda;
	\item \textbf{Descrizione}: per ogni fattura l'utente può visualizzare i seguenti campi:
	\begin{itemize}
		\item numero identificativo;
		\item data;
		\item nome dell'azienda venditrice/cliente;
		\item partita IVA dell'azienda venditrice/cliente;
		\item importo totale dell'ordine;
		\item IVA a credito/debito derivante dalla transazione;
	\end{itemize}
	\item \textbf{Scenario principale}: l'utente visualizza una fattura, riguardante i propri  acquisti o le vendite
	\item \textbf{Precondizione}: l'azienda sta visualizzando una pagina che richiede la visualizzazione dei dati generali di una fattura;
	\item \textbf{Postcondizione}: l'azienda ha visualizzato i dati relativi alla fattura.
\end{itemize}


\subsubsection{UC15.3 - Visualizzazione saldo IVA corrente}
\begin{itemize}
	\item \textbf{Attori Primari}: azienda;
	\item \textbf{Descrizione}: l'azienda può visualizzare il saldo parziale relativo al trimestre corrente, non ancora concluso. Non sono previste particolari operazioni per questa visualizzazione;
	\item \textbf{Scenario principale}: l'utente visualizza il saldo parziale relativo al trimestre corrente;
	\item \textbf{Precondizione}: l'utente ha selezionato la visualizzazione del saldo relativo al trimestre corrente;
	\item \textbf{Postcondizione}: l'azienda ha visualizzato il saldo IVA parziale riguardante il trimestre corrente ed è a conoscenza dall'attuale parziale situazione di debito o credito verso il governo.
\end{itemize} 

\subsubsection{UC15.4 - Visualizzazione saldo IVA trimestri conclusi}
\begin{itemize}
	\item \textbf{Attori Primari}: azienda;
	\item \textbf{Descrizione}: l'azienda può visualizzare la lista delle operazioni  relative ad un trimestre IVA già concluso. Può inoltre leggerne lo stato, ovvero sapere se i pagamenti a debito o credito con il governo sono saldati oppure no. In caso di stato di debito relativo ad un trimestre, è resa disponibile la possibilità di effettuare il pagamento, o di dilazionarlo;
	\item \textbf{Scenario principale}: l'utente visualizza la lista delle operazioni riguardanti  un saldo IVA relativo ad un trimestre concluso. Per ognuno di essi ottiene l'informazione sullo stato del pagamento dell'IVA, assieme alle operazioni che possono essere effettuate su di esso;
	\item \textbf{Precondizione}: l'utente ha selezionato la visualizzazione del saldo relativo ad un semestre concluso;
	\item \textbf{Postcondizione}: l'azienda ha visualizzato il saldo IVA riguardante un trimestre concluso, assieme alle possibili operazioni da effettuare.
\end{itemize} 

\subsubsection{UC15.5 - Download di un saldo IVA comprensivo delle operazioni in formato PDF}
\begin{itemize}
	\item \textbf{Attori Primari}: azienda;
	\item \textbf{Descrizione}: alle aziende è offerta la possibilità di scaricare le operazioni riguardanti un saldo trimestrale IVA nel formato PDF;
	\item \textbf{Scenario principale}: l'utente, dopo aver individuato il saldo desiderato, scarica, in formato PDF, l'elenco delle operazioni riguardanti il periodo trimestrale IVA selezionato;
	\item \textbf{Inclusioni}:
	\begin{itemize}
		\item \textbf{UC15.2}: l'azienda, per poter scaricare le informazioni riguardanti le operazioni di un trimestre IVA concluso, deve aver prima visualizzato un semestre IVA concluso [UC15.4];
	\end{itemize}
	\item \textbf{Precondizione}: l'azienda ha selezionato un saldi IVA riguardante un semestre concluso, ed ha espresso la volontà di scaricare le operazioni riguardanti un periodo cliccando l'apposito pulsante;
	\item \textbf{Postcondizione}: l'azienda ha scaricato il documento PDF contenente tutte le operazioni riguardanti il periodo trimestrale IVA selezionato.
\end{itemize} 

\subsubsection{UC15.6 - Pagamento IVA di un trimestre}
\begin{itemize}
	\item \textbf{Attori Primari}: azienda;
	\item \textbf{Attori Secondari}: governo;
	\item \textbf{Descrizione}: l'azienda può versare l'ammontare dovuto al governo, relativo ad un saldo trimestre IVA concluso che risultasse in stato di debito verso il governo;
	\item \textbf{Scenario principale}: l'utente, dopo aver individuato il saldo desiderato, effettua il versamento dell'ammontare dovuto allo stato premendo sull'apposito pulsante. Per effettuare il versamento viene utilizzato MetaMask\glo;
	\item \textbf{Inclusioni}:
	\begin{itemize}
		\item \textbf{UC15.4}: l'azienda per poter saldare il debito riguardante un semestre IVA concluso deve averlo prima visualizzato;
	\end{itemize}
	\item \textbf{Precondizione}: l'utente ha visualizzato un particolare saldo trimestrale IVA concluso. L'utente è in stato di debito verso il governo relativamente al saldo IVA trimestrale considerato. L'utente desidera saldare il debito relativo al suddetto trimestre; 
	\item \textbf{Postcondizione}: l'azienda ha effettuato il pagamento verso il governo. Viene aggiornato lo stato del trimestre IVA, che ora risulta saldato, sia nella visualizzazione da parte dell'azienda che dalla parte del governo.
\end{itemize} 

\subsubsection{UC15.7 - Dilazione pagamento IVA}
\begin{itemize}
	\item \textbf{Attori Primari}: azienda;
	\item \textbf{Attori Secondari}: governo;
	\item \textbf{Descrizione}: l'azienda può dilazionare\glosp il pagamento dovuto al governo, relativo ad un saldo trimestre IVA concluso che risultasse in stato di debito verso il governo;
	\item \textbf{Scenario principale}: l'utente, dopo aver individuato il saldo desiderato:
	\begin{enumerate}[label=\alph*.]
		\item sceglie da un menù a tendina di quanti mesi dilazionare\glosp il pagamento, se tale opzione risulta disponibile;
		\item conferma la dilazione del versamento dell'ammontare dovuto allo stato premendo sull'apposito pulsante. Per effettuare il versamento viene utilizzato MetaMask\glo;
	\end{enumerate}
	 
	\item \textbf{Inclusioni}:
	\begin{itemize}
		\item \textbf{UC15.4}: l'azienda per poter dilazionare\glosp il pagamento del debito riguardante un semestre IVA concluso deve averlo prima visualizzato;
	\end{itemize}
	\item \textbf{Precondizione}: l'utente ha visualizzato un particolare saldo trimestrale IVA concluso. L'utente è in stato di debito verso il governo relativamente al saldo IVA trimestrale considerato. L'utente desidera dilazionare\glosp il debito relativo al suddetto trimestre; 
	\item \textbf{Postcondizione}: l'azienda ha effettuato la dilazione\glosp del pagamento verso il governo.
\end{itemize} 

\subsubsection{UC15.8 - Visualizzazione fattura in dettaglio}

\begin{itemize}
	\item \textbf{Attori Primari}: azienda;
	\item \textbf{Descrizione}: l'azienda può leggere tutti i dettagli di una fattura. In essa devono essere presenti tutti i seguenti campi:
	\begin{itemize}
		\item data della fattura;
		\item numero identificativo della fattura;
		\item la data dell'ordine relativo alla fattura;
		\item il numero identificativo dell'ordine relativo alla fattura;
		\item visualizzazione dei beni in formato fattura, ovvero UC5 con due campi aggiuntivi per ogni prodotto, ovvero il prezzo netto e la percentuale IVA imposta;
		\item l'importo totale dell'ordine;
		\item il nome dell'azienda emittente;
		\item la partita IVA dell'azienda emittente;
		\item il nome dell'azienda destinataria;
		\item la partita IVA dell'azienda destinataria;
		\item indirizzo di spedizione dell'ordine;
	\end{itemize}
	\item \textbf{Scenario principale}: l'azienda seleziona una fattura da una lista e può leggere tutti i dettagli di tale fattura;
	\item \textbf{Inclusioni}:
	\begin{itemize}
		\item \textbf{UC15.2}: per poter accedere ai dettagli di una particolare fattura è necessario selezionarla da una delle liste disponibili nei saldi;
	\end{itemize}
	\item \textbf{Precondizione}: il sistema ha riconosciuto l'utente autenticato come azienda, l'utente ha espresso la volontà di visualizzare i dettagli di una specifica fattura;
	\item \textbf{Postcondizione}: l'azienda ha visualizzato i dettagli della fattura selezionata.
\end{itemize} 



\subsubsection{UC16 - Ricerca}
\begin{itemize}
	\item \textbf{Attori Primari}: utente autenticato;
	\item \textbf{Descrizione}: l'utente può filtrare i risultati delle informazioni che sta visualizzando inserendo una parola per la ricerca;
	\item \textbf{Scenario principale}: l'utente sta visualizzando una lista di informazioni, digita una parola chiave per la ricerca nella barra apposita per filtrare i risultati e poter trovare le informazioni desiderate;
	\item \textbf{Precondizione}: il sistema ha reso disponibile la barra di ricerca per filtrare i risultati;
	\item \textbf{Postcondizione}: i risultati sono stati filtrati. Vengono mostrati solo i risultati che contengono la parola inserita nella barra di ricerca.
\end{itemize} 

