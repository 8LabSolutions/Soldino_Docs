\subsection*{\quad$D\quad$}
\subsubsection*{DApp}
\index{DApp}
Anche scritta ÐApp, è un'applicazione decentralizzata che gestisce gli smart contracta\glo, arricchendo le transazioni con regole user-defined, garantendo maggiore sicurezza e controllo. Solitamente una ÐApp è open-source e  gestisce operazioni con dati criptati.

\subsubsection*{Dashboard}
\index{Dashboard}
È una postazione o a una pagina web basata su una tecnologia che visualizza informazioni relative a un business, raccolte in tempo reale da varie fonti nel settore. I dati vengono visualizzati in tempo reale nella pagina usando grafici, riepiloghi e liste.

\subsubsection*{Design pattern}
\index{Design pattern}
Soluzione progettuale generale ad un problema ricorrente. Una descrizione o un modello da applicare per risolvere un problema che può presentarsi in diverse situazioni durante la progettazione e lo sviluppo del software.

\subsubsection*{DevOps}
\index{DevOps}
Metodologia di sviluppo software, che combina lo sviluppo software con la tecnologia delle informazioni. Nasce dalla sinergia tra cultura aziendale, pratiche e strumenti, e fornisce a un'organizzazione l'abilità di sviluppare applicazioni e servizi con un breve ciclo di vita del software e in allineamento con gli obiettivi aziendali.

\subsubsection*{Dilazione}
\index{Dilazione}
L'atto di rimandare ad un secondo momento l'esecuzione di un'operazione, posticipando il termine normalmente fissato.

\subsubsection*{Docker}
\index{Docker}
Software per la creazione, sviluppo ed esecuzione di applicazione attraverso l’uso di container. I container
permettono di incorporare un’applicazione e tutte le parti necessarie ad essa in un unico pacchetto.

\subsubsection*{Documento esterno}
\index{Documento esterno}
Documento a uso di team, committente e proponente (RedBabel).

\subsubsection*{Documento interno}
\index{Documento interno}
Documento a uso del team di sviluppo e del committente (Prof. Vardanega Tullio e Prof. Cardin Riccardo).

\subsubsection*{Driver }
\index{Driver }
Nel test di unità, il driver funge da main e permette l'esecuzione di unità chiamate dal main.

