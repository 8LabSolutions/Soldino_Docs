\subsection{Capitolato C4 - MegAlexa}
\subsubsection{Informazioni sul capitolato}
\begin{itemize}
    \item \textbf{Nome}: MegAlexa: arricchitore di skill di Amazon Alexa;
	\item \textbf{Proponente}: ZERO12; 
	\item \textbf{Committente}: Prof. Tullio Vardanega e Prof. Riccardo Cardin.
\end{itemize}
\subsubsection{Descrizione}
La sfida proposta dall'azienda proponente consiste nel progettare una skill per 
Alexa di Amazon
in grado di avviare dei workflow creati dagli utenti tramite interfaccia web o
mobile app per iOS e Android.

\subsubsection{Studio del dominio}
\paragraph{Dominio tecnologico} \mbox{}\\
Le tecnologie da trattare per svolgere questo progetto sono:
\begin{itemize}
    \item \textbf{Amazon Alexa}: l'assistente digitale di Amazon;
    \item \textbf{Lambda (AWS)}: servizio di elaborazione serverless per 
l'esecuzione del proprio codice;
    \item \textbf{API gateway (AWS)}: servizio API per la comunicazione con 
Lambda;
    \item \textbf{Aurora Serverless (AWS)}: offre capacità di database senza 
dover allocare e gestire il server;
    \item \textbf{Node.js}: piattaforma event-driven per esecuzione di codice 
JavaScript server-side;
    \item \textbf{HTML5}, \textbf{CSS3} e \textbf{Javascript}: linguaggi da 
utilizzare per l'implementazione
    dell'interfaccia web;
    \item \textbf{Bootstrap}: Framework per front-end più utilizzato, l'azienda 
lo consiglia solamente;
	\item \textbf{Android} e \textbf{iOS}: studio di questi sistemi operativi e 
	dei relativi framework per lo sviluppo dell'applicazione. 

\end{itemize}
\subsubsection{Aspetti positivi}
Gli aspetti positivi affiorati sono:
\begin{itemize}
    \item Il proponente offre delle lezioni al fine di introdurre il gruppo alle 
nuove tecnologie da utilizzare nello sviluppo del progetto e dirigere lo studio 
autonomo;
    \item La massiccia presenza nel web di documentazione dettagliata, esempi e 
strumenti può semplificare l'apprendimento di tali tecnologie, in 
particolare Amazon fornisce Alexa Skills Kit (raccolta di API self-service, 
strumenti, documentazioni ed esempi);
    \item Amazon ed il mercato in generale sembrano, al momento, molto interessati 
agli assistenti vocali, quindi la conoscenza di tali tecnologie può essere una 
nota rilevante a livello curricolare.
\end{itemize}
\subsubsection{Criticità}
Gli elementi negativi riscontrati sono invece:
\begin{itemize}
    \item É obbligatoro che le shortcut siano multilingua. Echo al momento 
supporta: Inglese, Francese, Tedesco, Italiano, Giapponese e Spagnolo; tuttavia 
le nostre conoscenze in ambito linguistico ci permettono di realizzare in modo 
esaustivo solamente le versione italiana e inglese;
    \item Sono già presenti nel web tecnologie per realizzare delle skill in 
    grado di avviare dei workflow personalizzati, anche se in modo piuttosto grezzo.
    Infatti la stessa applicazione di Alexa permette di creare sequenze di azioni precedentemente 
selezionate;
    \item Il proponente offre l'opportunità solamente a due gruppi di 
aggiudicarsi il capitolato.
\end{itemize}
\subsubsection{Conclusioni}
Nonostante tale capitolato abbia destato particolare interesse all'interno del 
team di lavoro, sia a livello tecnologico che di competenze curricolari, il 
gruppo si è mostrato più stimolato verso un altro progetto non meno allettante.


